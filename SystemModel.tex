\section{System Model}
\label{SystemModel}

The model will be a public anonymous blackboard.
We will use this blackboard to publish locations proofs after the protest.

We want the following properties:
\begin{description}
  \item[Correctness] We want to count everyone once and only once.
  \item[Unforgeability] Participation should not be possible to forge.
    I.e.\ non-participants should not be able to say they were present.
    This together with correctness will provide authenticity to the protest 
    verification.
  \item[Anonymity] Given the data published on the blackboard, the adversary 
    should not be able to tell two participants apart.
  \item[Decentralized] There should be no central authority.
\end{description}

Can the adversary publish a public-key looking random string to sabotage the 
blackboard?

Basically the properties that we are interested in are similar to those of 
electronic voting, so we would like to reuse as many definitions from there as 
possible.

\subsection{Decentralization}

In \ac{PROPS} the prover must have a secret signed by \iac{CA}.
This \ac{CA} can be implemented as a secret-sharing scheme.
Usually these have a fixed threshold \((t, n)\).
This can be achieved by commitments, then the shared secret is computed.
However, it would be interesting to have the threshold dynamically adapt to the 
number of participants, i.e.\ participants can join later too.

Can the adversary then \enquote{join} enough many to throw off the threshold?

\subsection{Correctness and Unforgeability}

Each prover must reach a threshold of witnesses \(k\).
We must be able to prevent a prover to collect \(2k\) witnesses and then 
publishing two proofs.
We therefore must achieve conditional linkability between the \acp{LP} 
published on the blackboard.
If a second \ac{LP} for a secret is published, then those should be linkable.
This might be possible to achieve with unique group 
signatures~\cite{UniqueGroupSignatures,UniqueRingSignatures,ListSignatures}.


