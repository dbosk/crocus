\mode*

\section{System model}
\label{SystemModel}

\begin{frame}
\begin{question}
  What is participating in a protest?
  We need a definition.
\end{question}
\mode<presentation>{
\begin{remark}
  \begin{itemize}
    \item This depends on the unique ring/list signature schemes.
    \item I.e.\ when we can no longer guarantee uniqueness.
    \item The one setting up that signature scheme decides.
    \item We don't know how this will be done, nor by whom.
  \end{itemize}
\end{remark}
}
\end{frame}
This sort of depends on the unique ring/list signature scheme that we use.
As long as we can provide linkability between a user's proofs, we can consider 
it the same protest.
When we cannot, it must be a new protest.
So this is supposedly decided by the one setting up the unique ring/list 
signature scheme --- which we do not know yet how it will be done.

\begin{frame}
\begin{remark}
  We provide a lower bound, since there might be participants without phones.
\end{remark}
\begin{remark}
  I.e.\ provided we can prevent Sybil.
\end{remark}
\end{frame}

\begin{frame}
\begin{question}
  What assumptions do we need?
\end{question}

We assume that
\begin{itemize}
  \item Everyone has a smartphone.
  \item Can use ad-hoc networks during the event.
\end{itemize}
\end{frame}

\subsection{Adversary model}

We have two types of adversaries:
\begin{frame}
\begin{itemize}
  \item Eve tries to lower the count.
  \item Eve tries to deanonymize protesters.
  \item Rusuk, who heads another nation state, want to affect the outcome to 
    his advantage.
\end{itemize}
\end{frame}


\mode<all>\endinput

\section{Old stuff}

We can divide the problem into three parts:
\begin{itemize}
  \item Register participants in the system.
  \item Collect the necessary data.
  \item Perform the counting and verification.
\end{itemize}

To register the participants we might be able to piggy-back on existing 
\acp{PKI} and blind signatures --- similarly as for voting systems.

There are several issues that must be treated:
\begin{itemize}
  \item Each location proof must be bound to one individual's participation.
    One protester must not be able to create two unique location proofs and be 
    able to increase the number of participants, that would violate 
    \cref{EligibilityVerif} above.
    In other words, we must prevent the Sybil attack.

  \item A group of people should not be able to generate \acp{LP} to make it 
    look as if they participated in the protest but actually stayed at home.
    This means that all \acp{LP} must be linked to each other.
    If a protest has hundreds of thousands of participants, then this linking 
    must be achieved efficiently.
\end{itemize}

The model will be a public anonymous blackboard.
We will use this blackboard to publish locations proofs after the protest.

We want the following properties:
\begin{description}
  \item[Correctness] We want to count everyone once and only once.
  \item[Unforgeability] Participation should not be possible to forge.
    I.e.\ non-participants should not be able to say they were present.
    This together with correctness will provide authenticity to the protest 
    verification.
  \item[Anonymity] Given the data published on the blackboard, the adversary 
    should not be able to tell two participants apart.
  \item[Decentralized] There should be no central authority.
\end{description}

Can the adversary publish a public-key looking random string to sabotage the 
blackboard?

Basically the properties that we are interested in are similar to those of 
electronic voting, so we would like to reuse as many definitions from there as 
possible.

\subsection{Decentralization}

In \ac{PROPS} the prover must have a secret signed by \iac{CA}.
This \ac{CA} can be implemented as a secret-sharing scheme.
Usually these have a fixed threshold \((t, n)\).
This can be achieved by commitments, then the shared secret is computed.
However, it would be interesting to have the threshold dynamically adapt to the 
number of participants, i.e.\ participants can join later too.

Can the adversary then \enquote{join} enough many to throw off the threshold?

\subsection{Correctness and Unforgeability}

Each prover must reach a threshold of witnesses \(k\).
We must be able to prevent a prover to collect \(2k\) witnesses and then 
publishing two proofs.
We therefore must achieve conditional linkability between the \acp{LP} 
published on the blackboard.
If a second \ac{LP} for a secret is published, then those should be linkable.
This might be possible to achieve with unique group 
signatures~\cite{UniqueGroupSignatures,UniqueRingSignatures,ListSignatures}.


