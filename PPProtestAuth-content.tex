\title{Authenticating Protests}
\author{%
  Daniel Bosk\inst{1}
  \and
  Sébastien Gambs\inst{2}
  \and
  Sonja Buchegger\inst{1}
}
\institute{%
  School of Computer Science and Communication\\
  KTH Royal Institute of Technology, Stockholm\\
  \texttt{\{dbosk,buc\}@kth.se}
  \and
  Université de Rennes 1 and Inria, Rennes\\
  \texttt{sgambs@irisa.fr}
}

\maketitle

\begin{abstract}
  \dots

  \keywords{\dots}
\end{abstract}

\mode*


\section{Introduction}
\label{Introduction}

There are different possibilities for a protest: e.g.\ one performed online or 
one performed in the physical reality.
Purely online protests are essentially petitions.
This case can use digital signatures to verify the authenticity of a list of 
names.
It can also be compared to electronic voting --- to vote for or against 
something.
In this case all participants and third-parties can verify the authenticity of 
the result, but cannot verify the vote of an individual.

We are interested in real-world protests, for these we need to bind all 
participants to the same physical location at a reasonably similar time (within 
the duration of the protest).
\citeauthor{PROPS} developed a decentralized location-proof system, \ac{PROPS}, 
which provides a participant with a proof of being at the location.
However, we would like that any participant can prove the demonstration's 
authenticity to a third-party, e.g.\ the UN or a human-rights 
organization.
The reason this is an interesting property is the following scenario:
Imagine there is a demonstration, the police shows up and tries to apprehend 
every participant.
However, one demonstrator survives and manages to flee the country, here the 
demonstrator wants to prove that there was a protest of a certain size.

There are at least two reasons we are interested in authenticating a protest in 
this way:
\begin{enumerate}
\item for a third-party can verify the authenticity of the protest, e.g.\ that 
  there were \(n\) participants;
\item for the organizers to use the participation as feedback into a reputation 
  system.
\end{enumerate}
We want to do this in a privacy-preserving way:
We want to verify the participation of a protest, but not identifying 
individuals who participated.
If a demonstrator flees the country, that demonstrator should be able to prove 
to the e.g.\ the UN that there were \(n\) participants.
But if the demonstrator is caught by the secret police before leaving, then the 
secret police should not learn the identities of the other participants.


\section{Adapting PROPS}

\subsection{The \acs*{CA} Problems}

One issue is the \ac{CA}.
One direction we can take to solve this is by threshold signatures, e.g.\ 
\cite{FSThresholdSignatures}.
This way the participants can jointly issue new certificates, by signing them 
using the threshold signature scheme.
This should solve part of the problems related to Sybil attacks too.
But, if \(k > t\) participants collude, they can create multiple identities for 
themselves.
The main problem to solve, however, is that we want \(n\), and possibly \(t\) 
too, to change with the number of participants.
I'm not sure, but in what I've read about threshold schemes, they generally 
keep \(n\) and \(t\) fixed.

Another direction on solving the \ac{CA} problem, is if we can use already 
existing \acp{PKI} instead.
E.g.\ Sweden has a deployed national \ac{eID} system; usually these are RSA or 
DSA keys --- \((pq, e), (p, q, d): ed = 1\pmod{(p-1)(q-1)}\) --- signed by 
running everything through a hash function and the sign that value.
Due to the use of the hash function we cannot prove in zero knowledge that we 
have a signed certificate.
But maybe we can use the keys \((pq, e)\) and \((p, q, d)\) in such a way that 
they do not reveal the identity, but that we can if needed --- similar to 
a commitment which we reveal only in emergencies.

It would be interesting to solve the W--P collusion attack, where the witness 
reports false \acp{LPS}, e.g.\ we would need to prevent that a group of 
individuals fake this information at home.
One possibility in our scenario is that e.g.\ an external journalist on 
location can create \iac{LPS}, because a journalist is generally trusted.
This would be useful as occasionally journalists are also captured, and anyone 
that escaped should be able to show a convincing proof.
However, one problem with this is that the journalist will not necessarily have 
a key signed by the \ac{CA} of our scheme, but rather a \ac{CA} of say a UN 
\ac{PKI}.

In relation to the previous, it would be interesting to be able to tie the 
\acp{LP} (or \acp{LPS}) to a reputation system.
Then, even if no external journalist is on location, the participants can rely 
on their online reputation.
One option would be to have a reputation system in a global location-proof 
system, another would be to tie it to a reputation system already used in e.g.\ 
\iac{OSN}.
One possible solution would be to include a commitment in the \ac{LPS} and when 
necessary use it to tie it to the online identity.

\subsection{One Proof for All}

The naïve approach would be that everyone generates \acp{LPS} for everyone 
else, but this would be computationally and communicatively expensive 
(\(O(n^2)\)).
A better approach would be \iac{SMC} scheme where several witnesses participate 
to compute a joint \ac{LPS} for one prover.
This would reduce the need for pairwise communication.
In our scenario each witness will also want to participate as a prover, if the 
jointly computed \ac{LPS} could further be used by all participants in the 
computation, this would reduce the complexity further.

To make it even more efficient, it would be desirable to make the proofs 
\enquote{transitive}.
I.e.\ if Alice the prover has \acp{LPS} from a set of witnesses, then she can 
act as a witness and issue \iac{LPS} which includes her previous \acp{LPS}.
Speaking in graphs, we want to be able to join sub-graphs instead of having to 
join node-by-node.


\subsubsection*{\ackname}

This work was funded by the Swedish Foundation for Strategic Research grant SSF 
FFL09-0086 and the Swedish Research Council grant VR 2009-3793.


\printbibliography{}
