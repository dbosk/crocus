\Textcite{2016DemonstrationsInSeoul} discusses various approaches to crowd 
counting and the problems they face:

\blockcquote{2016DemonstrationsInSeoul}{%
  The rally renewed what has become a weekly back-and-forth between police and 
  protest organizers, whose crowd estimates have differed widely. Police said 
  about 270,000 people turned out on Saturday, making it the largest anti-Park 
  rally yet. Organizers estimated the crowd at 1.5 million.%
}

\blockcquote{2016DemonstrationsInSeoul}{%
  The rallies stretch from midday to late night --- some people stay for 
  several hours, others just several minutes.%
}

\blockcquote{2016DemonstrationsInSeoul}{%
  The demonstrators not only gather in open space, but also small alleys and 
  between buildings.%
}

\blockcquote{2016DemonstrationsInSeoul}{%
  For the police, the aim is to measure the maximum crowd occupying a certain 
  space at any given time so that they could determine the size of police 
  personnel and resources to deploy%
}

\blockcquote{2016DemonstrationsInSeoul}{%
  Police presume that, when sitting, six people would fill a space of 3.3 
  square meters
  [\dots]
  The same area would hold nine or 10 people when standing%
}

\blockcquote{2016DemonstrationsInSeoul}{%
  Based on this, police consider the 32,100-square-meter (345,525-square-foot) 
  boulevard in front of the Gwanghwamun palace gate, which has been the center 
  of the protests, as accommodating close to 600,000 people when fully packed.%
}

\blockcquote{2016DemonstrationsInSeoul}{%
  Organizers seek to track the entire flow of people from the protest’s start 
  to its finish. They collect estimates given by counters deployed at different 
  locations to size up the crowds in each area and update the numbers 
  throughout the duration of the protest.%
}

\blockcquote{2016DemonstrationsInSeoul}{%
  One South Korean retail analytics company measured the number of protesters 
  by counting the number of smartphone signals.
  [\dots]
  Zoyi Corp. deployed about 60 people, each carrying a backpack with the 
  company’s Wi-Fi signal-detecting sensor, to the rally on Nov. 19.
  [\dots]
  They collected Wi-Fi and Bluetooth signals from smartphones for about seven 
  hours from 2 p.m. at 53 spots and found about 369,372 smartphones that had 
  their wireless signals on during the rally.%
}

\blockcquote{2016DemonstrationsInSeoul}{%
  The company presumed that about half of smartphone users usually leave their 
  Wi-Fi feature on and the other half switch it off, based on a separate survey 
  on smartphone usage.%
}

\blockcquote{2016DemonstrationsInSeoul}{%
  They collected Wi-Fi and Bluetooth signals from smartphones for about seven 
  hours from 2 p.m. at 53 spots and found about 369,372 smartphones that had 
  their wireless signals on during the rally. The company presumed that about 
  half of smartphone users usually leave their Wi-Fi feature on and the other 
  half switch it off, based on a separate survey on smartphone usage. It also 
  assumed that about 20 percent of the smartphone signals were repetition from 
  the same device. Based on this, Zoyi concluded that about 738,700 people 
  joined the rally.%
}

\blockcquote{TheCrowdNumbersGame}{%
  square footage divided by occupation density%
}

\blockcquote{TheCrowdNumbersGame}{%
  Clark McPhail, professor emeritus of sociology at the University of Illinois at 
  Urbana-Champaign, is one of America’s preeminent authorities on protest crowds.
  [\dots]
  studying D.C. demonstrations since 1967, using a fairly simple mathematical 
  formula --- square footage divided by occupation density --- to count 
  everything from antiwar gatherings to the annual March for Life to the 
  Million Man March to Promise Keepers%
}

\blockcquote{TheCrowdNumbersGame}{%
  The crowd-counting method he uses, he says, was devised by Berkeley 
  journalism professor Herbert Jacobs in the 1960s during the Free Speech 
  movement.%
}

\blockcquote{TheCrowdNumbersGame}{%
  Farouk El-Baz, director of the Center for Remote Sensing at Boston 
  University, says that the methodology is well known and well respected by 
  scientists.%
}

\blockcquote{TheCrowdNumbersGame}{%
  McPhail took the measurements leading to that conclusion at 1 p.m., two hours 
  after the rally began. While he was measuring, a few last busloads of people 
  were still coming in --- as much as 10 percent of the rally.
  [\dots]
  So to be on the safe side, he added 20 percent to his 1 p.m. total%
}

\blockcquote{TheCrowdNumbersGame}{%
  According to El-Baz, who has worked with the Park Service, aerial photographs 
  would have given a more accurate estimate.%
}

\blockcquote{TheCrowdNumbersGame}{%
  After viewing aerial photographs of the crowds on the streets leading into the 
  plaza, the police upped their estimate to between 50,000 and 100,000, and 
  then said it could have been as large as 150,000.%
}

\blockcquote{TheCrowdNumbersGame}{%
  In fact, he says, he thinks they do themselves a disservice by exaggerating. 
  “While hyperbole is to be expected from organizers — it has always been higher 
  than the estimates of the police — this can work to their disadvantage,”
  [\dots]
  in the antiwar demonstrations of the ’60s and early ’70s, the protests became 
  effective once Middle America started turning out. At that point, the size of 
  the rallies mattered less than who attended. Once the mainstream turned against 
  the war, it had to end.
  [\dots]
  “In the Vietnam era, it was when middle-aged folks and middle-class folks, 
  doctors and lawyers began coming out and participating that public opinion 
  began to turn,”%
}

\Textcite{HowToEstimateCrowdSize} \dots

\blockcquote[Interview with Stephen Doig, ASU]{HowToEstimateCrowdSize}{%
  I would say the most accurate way to estimate the number of people in a large 
  outdoor crowd is to get good aerial imagery, break it into regions of similar 
  density, measure the square footage of each region, apply a reasonable value 
  for the density in each region%
}
\blockcquote{HowToEstimateCrowdSize}{%
  Stephen Doig, a census expert at Arizona State University.%
}

\blockcquote{HowToEstimateCrowdSize}{%
  Doig uses his crowd-counting experience and techniques to help companies like 
  AirPhotosLive.com (which covered President Obama's 2009 Inauguration ceremony
  [\dots]
  the company uses a 360-degree, spherical panoramic camera that can capture 
  shots of the crowd from every direction all at once during the \enquote{peak} 
  time of an event, or when attendance is at its highest%
}

\blockcquote{HowToEstimateCrowdSize}{%
  Crowd size is also needed for media news reports and to historically record 
  the event%
}

\blockcquote{HowToEstimateCrowdSize}{%
  Boston University Center for Remote Sensing.%
}

\blockcquote{HowToEstimateCrowdSize}{%
  the more widely that a crowd is spread out, the more challenging it can be to 
  get an accurate estimate of those attending, as it can become difficult to 
  determine whether people near the event's perimeter are there for the event 
  or are in the area for other reasons,%
}

