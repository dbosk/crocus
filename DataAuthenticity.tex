\section{Data authenticity}
\label{DataAuthenticity}

The problem of authentically associating data with a physical event is 
difficult.
We can essentially divide it into the following requirements:
\begin{requirements}[A]
  \item\label{CreatedAfterStart} Prove that the data was created after the 
    start of the event.
  \item\label{CreatedBeforeEnd} Prove that the data was created before the end 
    of the event.
  \item\label{SpatiallyRelated} Prove that the data is spatially related to the 
    physical location of the event.
\end{requirements} %TODO: A4 accurate/not fabricated?
\Cref{CreatedBeforeEnd,CreatedAfterStart} together bind the data to the time of 
the event whereas \cref{SpatiallyRelated} binds the data spatially to the 
event.

Now, consider the scenario of Alice taking a photo during the demonstration and 
posting it online.
What can we say about this photo?
First, we can say that it was created before we viewed it, so 
\cref{CreatedBeforeEnd} is fulfilled if we view it in relation to the event.
If it was submitted to a service that we trust, e.g.\ a blockchain, then we can 
also trust the time-stamp of the service for fulfilling 
\cref{CreatedBeforeEnd}.
Furthermore, we can also consider it spatially related to the physical location
(\cref{SpatiallyRelated}) if we can convince ourselves that the photo is 
depicting the physical location and not any kind of \enquote{reconstruction}, 
e.g.\ it is not computer generated or a photo of a similarly looking location.

\Cref{CreatedAfterStart} is more difficult to achieve.
In the above scenario, there is nothing that prevents Alice from submitting an 
older photo which is spatially related to the event --- i.e.\ to ensure 
\cref{CreatedAfterStart} we must ensure some sort of freshness.
% XXX Verify that this can be done using Fiat-Shamir heuristic
This can (probably) be solved using the Fiat--Shamir 
heuristic~\cite{FiatShamirHeuristic} (or something similar).
The main idea is that the operation binding the unpredictable value to the data 
must be difficult to redo later to change the value.

Instead of a photo, we will use \acp{LP}, more specifically, we will use 
\ac{PROPS}.
This will allow us to fulfil \cref{SpatiallyRelated}.
Since a location proof is a cryptographic value, it is easier to adapt it to 
fulfil \cref{CreatedAfterStart} too (compared to a photo).
