\mode*

\section{Security}

\subsection{Reusing participation proofs}

We consider two ways of reusing participation proofs:
\begin{enumerate}
  \item Alice and Bob can reuse old proofs by modifying the identifiers.
  \item Eve can reuse proofs by creating a manifesto which yields the same 
    identifier as Alice's protest, then try to create new nodes in the 
    \ac{tposet} that matches the ones used in the old proofs.
\end{enumerate}
We will now estimate the difficulty of both types of attacks.

\subsubsection{Forging new proofs}

\dots

\subsubsection{Finding second preimages}

The Eve (possibly with Rusuk's aid) wants to arrange a counter-protest in Eve's 
favour.
One way to get a higher participation count is to use the same identifier as 
Alice's protest.
Since the identifier for the protest is the hash value of the manifest, Eve 
must find a second preimage for the hash function.

\mode<presentation>{%
\begin{frame}
  \begin{itemize}
    \item Eve wants to arrange a counter-protest.
    \item She wants to steal some of Alice's participants' proofs.
    \item To do this she must have an identifier that matches.
    \item I.e.\ she must find a second preimage for the hash function.
  \end{itemize}
\end{frame}
}

% XXX Rephrase identifier problem as a game
\begin{proposition}
  Given \(x\) and \(H(x) = y\), the probability of finding an \(x'\) such that 
  \(H(x') = y\) is \(\frac{1}{\lambda-1}\).
\end{proposition}

\begin{question}
  What if we use a Merkle tree instead of only a hash, will that change the 
  second preimage property?
\end{question}
