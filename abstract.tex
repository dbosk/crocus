% What's the problem?
Say Alice organizes a protest against the current regime leader Eve.
The protest is held in some location(s) and Alice wants to estimate the number 
of participants to prove a certain support to her cause.
% Why is it a problem?
Current methods have wide margins for error.
In addition, Alice may be tempted to cheat by estimating the number of 
participants as high as possible whereas the government's incentive is to make 
this estimation as low as possible to support Eve.
                                                                                  
% What's the approach?
In this work, we are interested in combining location proofs with electronic 
voting to solve this issue.
Since the parties have incentives to cheat, we require transparency: in 
particular we are aiming at a decentralized scheme that provides authenticity, 
verifiability and privacy.
More specifically, the scheme should be distributed and not depend on any 
authority.
We must also ensure that the data is time-wise related to the event (i.e.\ 
authenticity), which can be provided by blockchain-like structures.
The data must also be related to the location, something a location proof can 
achieve.
We need verifiability, which means that every participant can verify that their 
participation has been included --- i.e.\ individual verifiability, to prevent 
Eve from dropping data --- and that anyone can verify the correctness of the 
result --- i.e.\ universal and eligibility verifiability, to prevent any 
malicious actor (e.g. Sybil) from affecting the result.
Finally, as all things policital, privacy is critical.
Similar to e-voting schemes, we need to guarantee properties close to vote 
privacy and receipt freeness --- Alice does not want Eve's agents to be able to 
infer that she is critical of Eve.
Finally, the scheme must be scalable to handle large protests (millions of 
participants).

\keywords{%
  crowd counting;
  protesting;
  electronic voting;
  location proof;
  privacy
}
