Say Alice organizes a protest against the current regime leader Eve.
The protest is held in some location(s).
Alice wants to estimate the number of participants to prove certain support for
her cause.
Current methods have wide margins for error.
In addition, Alice's incentive is to estimate the number of participants as 
high as possible whereas the government's incentive is to estimate the number 
of participants as low as possible to support Eve.

We are interested in combining location proofs with electronic voting.
Since the parties' have incentives to cheat, we require transparency: we desire 
a decentralized scheme that provides authenticity, verifiability and privacy.
More specifically:
The scheme must be decentralized to not depend on any authority.
We must ensure that the data is time-wise related to the event (authenticity), 
which can be provided by blockchain-like structures.
The data must also be related to the location, something a location proof can 
solve.
As all things policital, privacy is critical.
Similar to e-voting schemes, we need what corresponds to vote privacy and 
receipt freeness --- Alice does not want Eve's agents to verify that she is a 
critic of Eve.
Furthermore, we need verifiability: that every participant can verify that 
their participation has been included --- individual verifiability, to prevent 
Eve from dropping data --- and that anyone can verify the correctness of the 
result --- universal and eligibility verifiability, to prevent any malicious 
actor (e.g.\ Sybil) from affecting the result.
Finally, the scheme must be scalable to handle large protests (millions of 
participants).

In this talk we will discuss this work in progress and elicit input from the 
participants.

\keywords{%
  crowd counting;
  protesting;
  electronic voting;
  location proof;
  privacy
}
