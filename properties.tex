\mode*
\section[Verification and privacy]{Authenticity, verification and privacy}

Alice is an activist who organizes a demonstration for some cause.
Her main goal after the protest is to provide verifiable data.
There are several problems:
\begin{frame}
\begin{itemize}
  \item Alice must prove that the data is related to the event, that it is not 
    reused data from previous demonstrations.
  \item Alice must provide data which can be used for the desired verification 
    of the event.
  \item Alice must also consider the privacy of the participating individuals.
\end{itemize}
\end{frame}
Since many of the current techniques are based on photos, we will give an 
example of Alice's problems in that setting.
How can she ensure that photos from a demonstration are authentic?
We can probably recognize the place the photo is portraying, however, this 
might just as well be a reconstruction or entirely computer generated.
We cannot trust the meta-data of the photo, such as time-stamps of the file, 
because these can easily be manipulated.
So the only thing we can say for sure is that the photo was taken at the latest 
at the time of publication.
Now, if we cannot trust these photos, how can we determine the number of 
participants of a demonstration?
This is discussed further in \cref{DataAuthenticity}.

Assume that Alice can provide this authenticated data.
Alice and Bob do not want that this data ties them to the demonstration, 
because then the autocrat Eve can use this data to find out about Alice and Bob 
and arrest them.
Thus, what Alice and Bob need is a system which can provide data authenticity 
and user privacy, i.e.\ that the data can be correctly tied to the 
demonstration without outing Alice and Bob as supporters but still ensuring 
that Alice is Alice and Bob is Bob.
As a protest can be compared to voting (either voting for or against the 
cause), we will adapt the security and privacy properties of electronic 
voting.
We will discuss this further in \cref{Verification}.

It can also be compared to electronic voting --- to vote for or against the 
cause.
In this case all participants and third-parties can verify the authenticity of 
the result, this would inherit all the desirable properties of the electronic 
voting system (these properties are discussed in \cref{Verification}).
We are interested in real-world protests, i.e.\ demonstrations where 
participants gather in a physical location.
For these we need to bind all participants to the same physical location at 
a reasonably similar time (within the duration of the protest).

\subsection{Data authenticity}
\label{DataAuthenticity}

The problem of authentically associating data with a physical event is 
difficult.
We can essentially divide it into the following requirements:
\begin{frame}
\begin{requirements}[A]
  \item\label{CreatedAfterStart} Prove that the data was created after the 
    start of the event.
  \item\label{CreatedBeforeEnd} Prove that the data was created before the end 
    of the event.
  \item\label{SpatiallyRelated} Prove that the data is spatially related to the 
    physical location of the event.
\end{requirements} %TODO: A4 accurate/not fabricated?

\pause

\mode<presentation>{
\begin{idea}
  \begin{itemize}
    \item There is a mechanism that provides A\ref{CreatedAfterStart} and 
      A\ref{CreatedBeforeEnd}: a blockchain.

    \item There are also other structures, e.g.\ a directed 
      graph~\cite{BlockchainFreeCryptocurrencies}.

    \item Add a consensus protocol and Eve must subvert a majority to change 
      the outcome.

      \pause

    \item A\ref{SpatiallyRelated} can be provided by \iacf{LP}.
  \end{itemize}
\end{idea}
}
\end{frame}
\Cref{CreatedBeforeEnd,CreatedAfterStart} requires a \emph{strict partial 
  order}\footnote{%
  A relation \(\prec\) which is irreflexive, antisymmetric and transitive.
}.
If some objects in the set relate to known points in time, then the partial 
order allows to relate the data to the time of the event.
\Cref{SpatiallyRelated} binds the data spatially to the event.

One primitive that fulfils \cref{CreatedAfterStart,CreatedBeforeEnd} is a 
blockchain.
There are also other structures, e.g.\ a directed graph that 
converges~\cite{BlockchainFreeCryptocurrencies}.
For generality, we will use the terminology of 
\textcite{BlockchainFreeCryptocurrencies} and simply call this storage 
structure \iac{tposet}.

\begin{frame}
  \begin{question}
    What about distributed voting/polling protocols?
    A tree should give a strict partial order?
  \end{question}
  \begin{question}
    Can linear blockchains also fit in the framework of 
    \textcite{BlockchainFreeCryptocurrencies}?
    Can we generalize it?
  \end{question}
\end{frame}

\Cref{CreatedBeforeEnd} is straight-forward, Alice and Bob simply add their 
proofs to the \ac{tposet} \emph{in (time-wise) relation to the event}.
Then we know that they cannot have been created after the event.

\begin{frame}<presentation>
  \begin{solution}[A\ref{CreatedBeforeEnd} Created before end]
    \begin{itemize}
      \item Make the \ac{LP} part of the \ac{tposet}.
      \item The proof must've been created before it was included (duh!).
      \item (Also related to V\ref{EligibilityVerif}.)
    \end{itemize}
  \end{solution}
\end{frame}

For \cref{CreatedAfterStart} Alice and Bob must include an unpredictable value 
in their proofs, e.g.\ the \enquote{head} of the \ac{tposet}.
This means that they cannot create a proof and keep it for a future protest.

They can use an older value (pick one from the history of the \ac{tposet}) when 
they create their proofs.
This will only expand the interval of possible creation into the past.
This will not help their cause, as anyone will recognize that the proof could 
be created at any time in that interval.

\begin{frame}
  \mode<presentation>{
  \begin{idea}[A\ref{CreatedAfterStart} Created after start]
    \begin{itemize}
      \item Include an unpredictable value in the \ac{LP}.
      \item This could be e.g.\ the hash of the \enquote{head} of the 
        \ac{tposet}.
      \item This prevents creating proofs for future protests.
    \end{itemize}
  \end{idea}
  }

  \pause

  \begin{question}
    How hard is it to reuse old proofs?
    Forge all signatures due to changed values?
  \end{question}
\end{frame}

If we include \iac{LP}, this will tie the proof to the physical location and, 
thus, solve \cref{SpatiallyRelated}.
The \ac{LP} is \enquote{witnessed} (signed) by other participants.
The \ac{LP} contains coarse coordinates of the location, which must be within 
the protest area to be valid.
This allows Bob the protester to move within the protest area to collect 
witnesses (signatures).
It also provides privacy since the location cannot be used to deanonymize Bob's 
proof --- e.g.\ to correlate the location in his proof with captured 
surveillance footage.

\begin{frame}
  \mode<presentation>{
  \begin{idea}[A\ref{SpatiallyRelated} Correct location]
    \begin{itemize}
      \item The \ac{LP} is \enquote{witnessed} (signed) by other participants.
      \item We have a coarse location, just within protest area.
      \item This allows protesters to move around to collect signatures.
      \item This provides privacy for the location, e.g.\ cannot map to 
        surveillance cameras.
    \end{itemize}
  \end{idea}
  }

  \pause

  \begin{question}
    How can we do \acp{LP} with 1000s of witnesses each?
    And then 1000s of such \acp{LP}?
    Ad-hoc networks, broadcast \acp{LP}?
  \end{question}
  \begin{question}
    Why should anyone trust these witnesses?
    What are the assumptions?
    Signed over a short period of time?\footnote{%
      In a sense, if they collude they have shown support for the cause.
      As long as they are not Sybil it should be fine?
    }
  \end{question}
\end{frame}

\subsection{Verification}
\label{Verification}

We have an adversarial setting between Alice the activist and Eve the evil 
regime: Alice wants to show large support against Eve, Eve wants to show little
support for Alice.
In this case we have two options:
either we trust Alice or Eve, or we must verify their claims.
We aim for verifiable claims.

In general, protesting is very similar to voting: both are many individuals 
expressing their opinion.
Hence we desire to have similar properties of verification and privacy for 
participation in a protest as there is for voting.
In the context of (electronic) voting protocols, there are three requirements 
for verification~\cite{VerifyingPrivacyPropertiesOfVotingProtocols}:
\begin{frame}
\begin{requirements}[V]
\item\label{EligibilityVerif} Eligibility: anyone can verify that each vote 
  cast is legitimate.
\item\label{UniversalVerif} Universal verifiability: anyone can verify that the 
  result is according to the cast votes.
\item\label{IndividualVerif} Individual verifiability: every voter can verify 
  that their vote is included in the result.
\end{requirements}
\end{frame}
We can translate these to the case of participation in a demonstration, then 
each vote would be replaced by a proof of participation.
To have any verifiability, we need the storage (i.e.\ blockchain) to be 
immutable.
So we need a consensus protocol which prevents malicious changes.

\begin{frame}
  \mode<presentation>{
  \begin{remark}
    \begin{itemize}
      \item For valid verification we must have an immutable blockchain.
      \item We need something like a consensus protocol to prevent malicious 
        changes.
    \end{itemize}
  \end{remark}
  }

  \begin{question}
    What can we achieve with a consensus protocol?
    Majority must be honest?
  \end{question}
  \begin{question}
    What is a suitable proof-of-work?
    See e.g.~\cite{FairProofOfWork}.
  \end{question}
\end{frame}

If the proofs are included in a public blockchain (as suggested in 
\cref{DataAuthenticity}), then an individual can check that their proof has 
been included in the chain.
Thus \cref{IndividualVerif} is fulfilled.

\begin{frame}
  \mode<presentation>{
  \begin{idea}[V\ref{IndividualVerif} Individual verifiability]
    \begin{itemize}
      \item Proofs are stored publicly (blockchain).
      \item Each participant can check that their proof is indeed included.
    \end{itemize}
  \end{idea}
  }

  \pause

  \begin{question}
    Malicious nodes want to provide Alice with a view that her proof is 
    included.
    Everyone else is provided a view without Alice.
    How can we ensure a consistent view for everyone?
    (Probably similar to the solution in Joakim's thesis.)
  \end{question}
\end{frame}

\Cref{EligibilityVerif} would in this case mean that anyone can verify that 
each participation proof belongs to a unique individual, i.e.\ to prevent Sybil 
attacks.
They must also be able to detect if the proof is reused from a previous protest 
(i.e.\ related to \cref{CreatedAfterStart,CreatedBeforeEnd}).

\begin{frame}
  \mode<presentation>{
  \begin{idea}[V\ref{EligibilityVerif} Eligibility]
    \begin{itemize}
      \item Must verify that the proof is not reused from before 
        (A\ref{CreatedAfterStart} and A\ref{CreatedBeforeEnd}).
      \item Must verify that each proof belongs to a unique individual (not 
        only Sybil).
    \end{itemize}
  \end{idea}
  }

  \pause

  \begin{question}
    How to solve the Sybil problem?
    Can we piggyback on existing \acp{PKI}?
    E.g.~\cite{Cinderella}.
  \end{question}
  \begin{question}
    This is similar to the double-spending problem for cryptocurrencies, can 
    we reuse any such techniques?
  \end{question}
\end{frame}

If the blockchain is public, then anyone can download all the proofs, verify 
them (eligibility) and count them.
Thus anyone can verify the result, which fulfils \cref{UniversalVerif}.

\begin{frame}
  \mode<presentation>{
  \begin{idea}[V\ref{UniversalVerif} Universal verifiability]
    \begin{itemize}
      \item Proofs are stored publicly.
      \item Anyone can download all proofs, verify them and then count them.
    \end{itemize}
  \end{idea}
  }

  \pause

  \begin{question}
    Similar as before: how can we ensure consistent views for everyone?
  \end{question}
\end{frame}

\subsection{Privacy}
\label{Privacy}

We also need privacy in addition to the verification requirements.
As we indirectly pointed out earlier, we focus on the privacy provided to Alice 
and Bob by the data.
So as long as Alice and Bob can conceal their identities at the demonstration 
and escape without arrest, their support is recorded in the data while their 
privacy is not violated.
(Following this line of thinking, it can actually be beneficial for the privacy 
of the demonstrators to mix with the participants from any 
counter-demonstrations --- since the counts will still be correct.)

In voting, we have the following requirements:
\begin{frame}
\begin{requirements}[P]
\item\label{VotePrivacy} Vote privacy: the voting does not reveal any 
  individual vote.
\item\label{ReceiptFreeness} Receipt freeness: the voting system does not 
  provide any data that can be used as a proof of how the voter voted.
\item\label{CoercionResistance} Coercion resistance: a voter cannot cooperate 
  with a coercer to prove the vote was cast in any particular way.
\end{requirements}
\pause
\mode<presentation>{
  \begin{remark}
    P\ref{CoercionResistance} \(\implies\)
    P\ref{ReceiptFreeness} \(\implies\)
    P\ref{VotePrivacy}
  \end{remark}
}
\end{frame}
\Textcite{VerifyingPrivacyPropertiesOfVotingProtocols} showed that 
\cref{CoercionResistance} implies \cref{ReceiptFreeness}, which in turn implies
\cref{VotePrivacy}.
\Cref{CoercionResistance} is probably not possible to achieve for protests:
e.g.\ Eve can simply physically bring Alice to a protest against her will.
This leaves us with \cref{ReceiptFreeness,VotePrivacy}.

\begin{frame}<presentation>
  \begin{remark}
    \begin{itemize}
      \item Coercion resistance is difficult to achieve for voting.
      \item It doesn't make much sense for protests.
      \item E.g.\ Eve physically brings Alice to a protest she doesn't want to 
        participate in.
      \item That leaves us with receipt freeness and vote privacy.
    \end{itemize}
  \end{remark}
\end{frame}

A demonstration is very different from voting in one sense: Alice must be 
physically present and that very presence shows her support for the cause.
In voting, on the other hand, Alice has multiple options which are not revealed
by her mere presence.
% XXX Check if unlinkable is the correct term
Hence, if Alice submits a proof to the blockchain, the proof must be 
unlinkable to Alice, yet, if Alice submits another proof, those two proofs must 
be linkable (due to eligibility verification, \cref{EligibilityVerif}).
E.g.\ Eve compromises Alice's device (and private keys).
Anyone can verify any proof.
With the inputs that Eve can extract from the published proof and the keys she 
can find on Alice's device, she should not be able to reproduce the same 
signatures on the proof.

\begin{frame}<presentation>
  \begin{question}
    How do we submit the proofs to storage?
    Voting normally uses mix-nets.
  \end{question}

  \begin{remark}
    Without anonymous submission we don't get vote pricacy, thus no receipt 
    freeness either.
  \end{remark}
\end{frame}

\begin{frame}<presentation>
  \begin{remark}
    \begin{itemize}
      \item Voting: different alternatives when participating.
      \item Protesting: participation implies the alternative.
    \end{itemize}
  \end{remark}

  \pause

  \begin{idea}[P\ref{ReceiptFreeness} Receipt freeness]
    \begin{itemize}
      \item If Eve compromises Alice's device, she should not be able to use 
        it to verify that she participated.

      \item But Alice should not be able to participate twice.

      \item We need a signature scheme that cannot reproduce the same 
        signature when run on the same inputs.
    \end{itemize}
  \end{idea}
\end{frame}

\begin{frame}
  \begin{question}
    Will such a signature scheme solve both problems?
  \end{question}
  \begin{question}
    Is there such a signature scheme?
    Can we construct it?
  \end{question}
  \begin{question}
    What Alice submits her proof, stores the hash of the block, removes proof 
    and signature key?
  \end{question}
\end{frame}

