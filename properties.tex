\mode*
\section{Desired properties}%
\label{Properties}

We categorize the desired properties into verifiability and privacy.

\subsection{Verification}%
\label{Verification}

We desire three verifiability requirements:
\begin{requirements}[V]
\item\label{EligibilityVerif} Eligibility: anyone can verify that each 
  participation proof provides temporal and spatial eligibility and that it has 
  not been counted before.
\item\label{UniversalVerif} Universal verifiability: anyone can verify that the 
  result is according to the submitted participation proofs.
\item\label{IndividualVerif} Individual verifiability: every participant can 
  verify that their participation proof is included in the global count.
\end{requirements}
We can translate these to the case of participation in a demonstration, then 
each vote would be replaced by a proof of participation.

\Cref{EligibilityVerif} would in this case mean that anyone can verify that 
each participation proof belongs to a unique individual, i.e.\ to prevent Sybil 
attacks.
They must also be able to verify that each proof is indeed related to the 
event.
We must associate the proof to the event both spatially (the correct location) 
and temporally (it is related to the time-interval of the event).
We can essentially divide it into the following requirements:
\begin{requirements}[\ref*{EligibilityVerif}.]
  \item Temporal egligibility:%
    \label{CreatedAfterStart} prove that the data was created after the start of 
    the event;%
    \label{CreatedBeforeEnd} prove that the data was created before the end of 
    the event.
  \item Spatial eligibility:%
    \label{SpatiallyRelated} prove that the data is spatially related to the 
    physical location of the event.
  \item One-proof-per-person:%
    \label{CountOnce} prove that no individual can be counted more than once.
  \item Designated event:%
    \label{DesignatedEvent} prove that the data is designated for the event.
\end{requirements}

\mode<presentation>{%
\begin{frame}
  \begin{requirements}[V\ref*{EligibilityVerif}.]
  \item\label{CreatedAfterStart} Prove that the data was created after the 
    start of the event.
  \item\label{CreatedBeforeEnd} Prove that the data was created before the end 
    of the event.
  \item\label{SpatiallyRelated} Prove that the data is spatially related to the 
    physical location of the event.
  \item\label{CountOnce} Prove that no individual can be counted more than 
    once.
  \item\label{DesignatedEvent} Prove that the data is designated for the event.
\end{requirements}
\end{frame}
}

\subsubsection{Temporal eligibility}

\begin{frame}
\begin{definition}[Forging temporal eligibility]
  \dots
\end{definition}
\end{frame}

\subsubsection{Spatial eligibility}

\begin{frame}
\begin{definition}[Forging spatial eligibility]
  \dots
\end{definition}
\end{frame}

\subsubsection{Linkability and designated protest}

\dots

\subsubsection{Individual and universal verifiability}

\dots

\subsection{Privacy}%
\label{Privacy}

We also need privacy in addition to the verification requirements.
As we indirectly pointed out earlier, we focus on the privacy provided to Alice 
and Bob by the data.
So as long as Alice and Bob can conceal their identities at the demonstration 
and escape without arrest, their support is recorded in the data while their 
privacy is not violated.
(Following this line of thinking, it can actually be beneficial for the privacy 
of the demonstrators to mix with the participants of any counter-demonstrations 
--- since the counts will still be correct.)

In voting, we have the following requirements:
\begin{frame}
\begin{requirements}[P]
\item\label{VotePrivacy} Vote privacy: the voting does not reveal any 
  individual vote.
\item\label{ReceiptFreeness} Receipt freeness: the voting system does not 
  provide any data that can be used as a proof of how the voter voted.
\item\label{CoercionResistance} Coercion resistance: a voter cannot cooperate 
  with a coercer to prove the vote was cast in any particular way.
\end{requirements}
\pause{}
\mode<presentation>{%
  \begin{remark}
    P\ref{CoercionResistance} \(\implies\)
    P\ref{ReceiptFreeness} \(\implies\)
    P\ref{VotePrivacy}
  \end{remark}
}
\end{frame}
\Textcite{VerifyingPrivacyPropertiesOfVotingProtocols} showed that 
\cref{CoercionResistance} implies \cref{ReceiptFreeness}, which in turn implies
\cref{VotePrivacy}.
\Cref{CoercionResistance} is probably not possible to achieve for protests:
e.g.\ Eve can simply physically bring Alice to a protest against her will.
This leaves us with \cref{ReceiptFreeness,VotePrivacy}.

\mode<none>{%
\begin{frame}
  \begin{remark}
    \begin{itemize}
      \item Coercion resistance is difficult to achieve for voting.
      \item It doesn't make much sense for protests.
      \item E.g.\ Eve physically brings Alice to a protest she doesn't want to 
        participate in.
      \item That leaves us with receipt freeness and vote privacy.
    \end{itemize}
  \end{remark}
\end{frame}
}

\subsubsection{Participation-proof privacy}

A demonstration is very different from voting in one sense: at a demonstration, 
Alice must be physically present and that very presence shows her support for 
the cause.
In voting, on the other hand, everyone is present and Alice has multiple 
options which are not revealed by her mere presence.
% XXX Check if unlinkable is the correct term
Hence, if Alice submits a proof of participation, the proof must be unlinkable 
to Alice, yet, if Alice submits another proof, those two proofs must be 
linkable (due to eligibility verification, \cref{EligibilityVerif}) so that 
Alice is not counted twice.
This is fine, since we do not want to catch any cheater, we just do not want to
count them more than once.

\mode<presentation>{%
\begin{frame}
  \begin{remark}
    \begin{itemize}
      \item Voting: different alternatives when participating.
      \item Protesting: participation implies the alternative.
    \end{itemize}
  \end{remark}

  \pause{}

  \begin{idea}[Proof privacy]
    \begin{itemize}
      \item Eve shouldn't be able to link Alice's published proof back to 
        Alice.
      \item But if Alice publishes two proofs, those two are linkable.

        \pause{}

      \item We're not interested in unmasking trolls, just not count them more 
        than once.
    \end{itemize}
  \end{idea}
\end{frame}
}

\begin{frame}
  \begin{remark}
    Unique ring signatures~\cite{UniqueRingSignatures} has this property: two 
    signatures are linked with high probability.
  \end{remark}

  \begin{question}
    Can we link a verified key to a unique ring signature key?
  \end{question}

  \pause{}

  \begin{question}
    How to submit the proofs to the system?
  \end{question}
\end{frame}

\subsubsection{Receipt freeness}

Receipt freeness implies that Alice enjoys deniability against Eve:
Eve should not be able to use the published proof to tie it to Alice, even if 
Eve has access to Alice's device (i.e.\ all her private keys).
E.g.\ Eve should not be able to reproduce the same proof and thus verify that 
Alice has created one of the proofs.

\mode<presentation>{%
\begin{frame}
  \begin{remark}
    \begin{itemize}
      \item The proof itself is a sort of receipt.
        
        \pause{}

      \item If Eve compromises Alice's device, she can use it to verify that 
        she participated.

        \pause{}

      \item We need a signature scheme that cannot reproduce the same signature 
        when run on the same inputs.
    \end{itemize}
  \end{remark}
\end{frame}
}

\begin{frame}
\mode<presentation>{%
  \begin{idea}[Receipt freeness (deniability)]
    \begin{itemize}
      \item Alice submits her proof.
      \item She stores the hash of the head of the \ac{tposet}.
      \item She removes the proof and signature key.

        \pause{}

      \item E.g.\ a short-term and a long-term credential, then Alice can 
        drop the short-term credential.

      \item But Alice should still not be able to participate twice.
    \end{itemize}
  \end{idea}
}

  \pause{}

  \begin{question}
    Is there such a signature scheme?
    Can we construct it?
  \end{question}
\end{frame}

Say that Alice submits her proof and stores the hash of the head of the 
\ac{tposet} after her proof is included.
Now she can remove her proof and just keep the hash to verify that her proof is 
still there.
She can also remove her signature key, so that Eve cannot use it to reproduce 
the signature using the same inputs.

\subsubsection{Architecture}

We also have an architectural problem.
If the system is only run by volunteering activists, then Eve can simply 
analyse network traffic and look for people running nodes in the system.

\begin{question}
  Shall we offer a platform that is run by a variety of institutions who 
  offers to verify protests?
  Or go strictly peer-to-peer?
\end{question}

\begin{question}
  If strictly peer-to-peer, and only run by protesters, one can infer that 
  they are protesters by running such a node.
  Can this be mitigated?
\end{question}

Similarly, we have the problem of how protesters submit their proofs to the 
system.
This must be done anonymously.
Otherwise Eve can analyse the network traffic to find out who submitted proofs 
to the system.
The main problem here is that the participants must be indistinguishable from 
non-participants.
This means that we cannot use special-purpose anonymizing 
technologies.

\begin{question}
  How do we submit the proofs to storage?
  Voting normally uses mix-nets (provable shuffles).
\end{question}

\begin{remark}
  We must mix participants with non-participants.
\end{remark}

