\mode*
\section{Authenticity, verification and privacy}

Alice is an activist who organizes a demonstration for some cause.
Her main goal after the protest is to provide verifiable data.
There are several problems:
\begin{frame}
\begin{itemize}
  \item Alice must prove that the data is related to the event, that it is not 
    reused data from previous demonstrations.
  \item Alice must provide data which can be used for the desired verification 
    of the event.
  \item Alice must also consider the privacy of the participating individuals.
\end{itemize}
\end{frame}
Since many of the current techniques are based on photos, we will give an 
example of Alice's problems in that setting.
How can she ensure that photos from a demonstration are authentic?
We can probably recognize the place the photo is portraying, however, this 
might just as well be a reconstruction or entirely computer generated.
We cannot trust the meta-data of the photo, such as time-stamps of the file, 
because these can easily be manipulated.
So the only thing we can say for sure is that the photo was taken at the latest 
at the time of publication.
Now, if we cannot trust these photos, how can we determine the number of 
participants of a demonstration?
This is discussed further in \cref{DataAuthenticity}.

Assume that Alice can provide this authenticated data.
Alice and Bob do not want that this data ties them to the demonstration, 
because then the autocrat Eve can use this data to find out about Alice and Bob 
and arrest them.
Thus, what Alice and Bob need is a system which can provide data authenticity 
and user privacy, i.e.\ that the data can be correctly tied to the 
demonstration without outing Alice and Bob as supporters but still ensuring 
that Alice is Alice and Bob is Bob.
As a protest can be compared to voting (either voting for or against the 
cause), we will adapt the security and privacy properties of electronic 
voting.
We will discuss this further in \cref{Verification}.

It can also be compared to electronic voting --- to vote for or against the 
cause.
In this case all participants and third-parties can verify the authenticity of 
the result, this would inherit all the desirable properties of the electronic 
voting system (these properties are discussed in \cref{Verification}).
We are interested in real-world protests, i.e.\ demonstrations where 
participants gather in a physical location.
For these we need to bind all participants to the same physical location at 
a reasonably similar time (within the duration of the protest).

\subsection{Data authenticity}
\label{DataAuthenticity}

The problem of authentically associating data with a physical event is 
difficult.
We can essentially divide it into the following requirements:
\begin{frame}
\begin{requirements}[A]
  \item\label{CreatedAfterStart} Prove that the data was created after the 
    start of the event.
  \item\label{CreatedBeforeEnd} Prove that the data was created before the end 
    of the event.
  \item\label{SpatiallyRelated} Prove that the data is spatially related to the 
    physical location of the event.
\end{requirements} %TODO: A4 accurate/not fabricated?
\end{frame}
\Cref{CreatedBeforeEnd,CreatedAfterStart} together bind the data to the time of 
the event whereas \cref{SpatiallyRelated} binds the data spatially to the 
event.

\begin{frame}<presentation>
  \begin{solution}[A\ref{CreatedAfterStart}, after start]
    \begin{itemize}
      \item Include an unpredictable value in the proof.
      \item This could be e.g.\ the (hash of the) head of block chain.
      \item This prevents creating proofs for future protests.
    \end{itemize}
  \end{solution}

  \pause

  \begin{question}
    \begin{itemize}
      \item Is this hard?
      \item I think we can show that it is.
      \item What assumptions do we need for this?
    \end{itemize}
  \end{question}
\end{frame}

\begin{frame}<presentation>
  \begin{solution}[A\ref{CreatedBeforeEnd}, before end]
    \begin{itemize}
      \item Make the proof part of the block chain.
      \item The proof must've been created before it was included.
    \end{itemize}
  \end{solution}
\end{frame}

\begin{frame}<presentation>
  \begin{solution}[A\ref{SpatiallyRelated}, correct location]
    \begin{itemize}
      \item The proof includes \iac{LP}.
      \item The proof is \enquote{witnessed} (signed) by other participants.
      \item We have a coarse location, just within protest area.
    \end{itemize}
  \end{solution}

  \pause

  \begin{question}
    \begin{itemize}
      \item How can we do 1000s of \acp{LP} with 1000s of witnesses each?
    \end{itemize}
  \end{question}
\end{frame}

One primitive that provides the properties 
\cref{CreatedAfterStart,CreatedBeforeEnd} is a block chain with a distributed 
consensus protocol.
\Cref{CreatedBeforeEnd} is straight-forward, we simply add the proof to the 
block chain in relation to the event.
For \cref{CreatedAfterStart} we must include an unpredictable value in the 
proof, e.g.\ the head of the block chain.

If we include \iac{LP}, this will tie the proof to the physical location and, 
thus, solve \cref{SpatiallyRelated}.

\subsection{Verification}
\label{Verification}

We have an adversarial setting between Alice the activist and Eve the evil 
regime: Alice wants to show large support against Eve, Eve wants to show little
support for Alice.
In this case we have two options:
either we trust Alice or Eve, or we must verify their claims.
We aim for verifiable claims.

In general, protesting is very similar to voting: both are many individuals 
expressing their opinion.
Hence we desire to have similar properties of verification and privacy for 
participation in a protest as there is for voting.
In the context of (electronic) voting protocols, there are three requirements 
for verification~\cite{VerifyingPrivacyPropertiesOfVotingProtocols}:
\begin{frame}
\begin{requirements}[V]
\item\label{EligibilityVerif} Eligibility: anyone can verify that each vote 
  cast is legitimate.
\item\label{UniversalVerif} Universal verifiability: anyone can verify that the 
  result is according to the cast votes.
\item\label{IndividualVerif} Individual verifiability: every voter can verify 
  that their vote is included in the result.
\end{requirements}
\end{frame}
We can translate these to the case of participation in a demonstration, then 
each vote would be replaced by a proof of participation.

If the proofs are included in a public block chain (as suggested in 
\cref{DataAuthenticity}), then an individual can check that their proof has 
been included in the chain.
Thus \cref{IndividualVerif} is fulfilled.

Anyone can count the proofs included in the block chain.
They can also verify the proofs and thus verify the result, which fulfils 
\cref{UniversalVerif}.

\Cref{EligibilityVerif} would in this case mean that anyone can verify that 
each participation proof belongs to a unique individual, i.e.\ to prevent Sybil 
attacks.

\subsection{Privacy}
\label{Privacy}

We also need privacy in addition to the verification requirements.
In voting, we have the following requirements:
\begin{frame}
\begin{requirements}[P]
\item\label{VotePrivacy} Vote privacy: the vote does not reveal any individual 
  vote.
\item\label{ReceiptFreeness} Receipt freeness: the voting system does not 
  provide any data that can be used as a proof of how the voter voted.
\item\label{CoercionResistance} Coercion resistance: a voter cannot cooperate 
  with a coercer to prove the vote was cast in any particular way.
\end{requirements}
\mode<presentation>{
\pause
\begin{remark}
  \begin{itemize}
    \item Coercion resistance is difficult to achieve for voting.
    \item It doesn't make much sense for protests.
    \item E.g.\ Eve physically brings Alice to a protest she doesn't want to 
      participate in.
  \end{itemize}
\end{remark}
}
\end{frame}
\Textcite{VerifyingPrivacyPropertiesOfVotingProtocols} showed that 
\cref{CoercionResistance} implies \cref{ReceiptFreeness}, which in turn implies
\cref{VotePrivacy}.
We can see that \cref{VotePrivacy,ReceiptFreeness} are desirable in Alice's 
situation: there should not be any proof which binds Alice to the participation 
of the demonstration (\cref{ReceiptFreeness}), because she does not want to 
explicitly reveal her opinions to the government (\cref{VotePrivacy}) due to 
the risk of reprimands.
\Cref{ReceiptFreeness} implies that even if Alice is arrested, the data should 
not provide any proof to the regime's agents that can reveal Alice's opinion.

A demonstration is very different from voting in one sense: Alice must be 
physically present and that very precense shows her support for the cause.
In voting, on the other hand, Alice has multiple options which are not revealed
by her mere precense.
As we indirectly pointed out earlier, we focus on the privacy provided to Alice 
and Bob by the data.
So as long as Alice and Bob can conceal their identities at the demonstration 
and escape without arrest, their support is recorded in the data while their 
privacy is not violated.
(Following this line of thinking, it can actually be beneficial for the privacy of 
the demonstrators to mix with the participants from any counter-demonstrations 
--- as long as the correct counts for each demonstration can still be ensured, 
which is the idea.)

The main problem is how to ensure the connection between the physical location 
and the data used for verification, i.e.\ fulfil all the requirements for 
authenticity, verification and privacy given above.
We must bind participants to the same physical location at a reasonably similar 
time, i.e.\ within the area and duration of the demonstration.
In the case of the Korean demonstrations~\cite{2016DemonstrationsInSeoul}, this 
was in one place during an entire day and then repeated for several 
weekends.
In the case of the Women's Marches in the US~\cite{2017WomensMarchesInUS}, they
were in several locations at the same time.

%This means that for a system like this to work, we also need the requirements 
%from \cref{DataAuthenticity}: proof that the data was created after the start 
%of the protest, proof that it was created before the end, and proof of spatial 
%relation to the location.

\textcite{PROPS} developed a decentralized \ac{LPS} which provides 
a participant with a verifiable proof of having been at a location at a certain 
time, something we call \iac{LP}.
It is decentralized because there is no central authority that vouches for the 
location, instead peers act as witnesses.
Then a third-party can verify the authenticity of the \ac{LP}, by verifying the 
witnesses' signatures, and can thus be sure that the person has indeed been in 
the location.
Bosk, Gambs and Buchegger are currently exploring (work in progress) the 
possibility of combining such \iac{LPS} with the verifiability and privacy 
requirements discussed above.
The overall idea is that each participant generates \iac{LP} during the 
demonstration, where (some of) the other protesters act as witnesses, then the 
\acp{LP} can be used to compute the participation count with all the 
authenticity, verifiability and privacy requirements above.
%TODO: reviewer: what about cellphones to count attendance
% This is actually done in \cite{2016DemonstrationsInSeoul}, it requires some 
% assumptions --- but they use wifi signals instead of the mobile network.
% In either case, we pointed out that one wants the phone in flight mode due to
% surveillance.
