\mode*
\section{Desired properties}%
\label{Properties}

We categorize the desired properties into verifiability and privacy.

\subsection{Verification}%
\label{Verification}

We desire three verifiability requirements:
\begin{requirements}[V]
\item\label{EligibilityVerif} Eligibility: anyone can verify that each 
  participation proof provides temporal and spatial eligibility and that it has 
  not been counted before.
\item\label{UniversalVerif} Universal verifiability: anyone can verify that the 
  result is according to the submitted participation proofs.
\item\label{IndividualVerif} Individual verifiability: every participant can 
  verify that their participation proof is included in the global count.
\end{requirements}
We can translate these to the case of participation in a demonstration, then 
each vote would be replaced by a proof of participation.

\Cref{EligibilityVerif} would in this case mean that anyone can verify that 
each participation proof belongs to a unique individual, i.e.\ to prevent Sybil 
attacks.
They must also be able to verify that each proof is indeed related to the 
event.
We must associate the proof to the event both spatially (the correct location) 
and temporally (it is related to the time-interval of the event).
We can essentially divide it into the following requirements:
\begin{requirements}[\ref*{EligibilityVerif}.]
  \item Temporal egligibility:%
    \label{CreatedAfterStart} prove that the data was created after the start of 
    the event;%
    \label{CreatedBeforeEnd} prove that the data was created before the end of 
    the event.
  \item Spatial eligibility:%
    \label{SpatiallyRelated} prove that the data is spatially related to the 
    physical location of the event.
  \item One-proof-per-person:%
    \label{CountOnce} prove that no individual can be counted more than once.
  \item Designated event:%
    \label{DesignatedEvent} prove that the data is designated for the event.
\end{requirements}

\mode<presentation>{%
\begin{frame}
  \begin{requirements}[V\ref*{EligibilityVerif}.]
  \item\label{CreatedAfterStart} Prove that the data was created after the 
    start of the event.
  \item\label{CreatedBeforeEnd} Prove that the data was created before the end 
    of the event.
  \item\label{SpatiallyRelated} Prove that the data is spatially related to the 
    physical location of the event.
  \item\label{CountOnce} Prove that no individual can be counted more than 
    once.
  \item\label{DesignatedEvent} Prove that the data is designated for the event.
\end{requirements}
\end{frame}
}

\subsubsection{Temporal eligibility}

\begin{frame}
\begin{definition}[Forging temporal eligibility]
  \dots
\end{definition}
\end{frame}

\subsubsection{Spatial eligibility}

\begin{frame}
\begin{definition}[Forging spatial eligibility]
  \dots
\end{definition}
\end{frame}

\subsubsection{Linkability and designated protest}

\dots

\subsubsection{Individual and universal verifiability}

\dots

\subsection{Privacy}%
\label{Privacy}

We also need privacy in addition to the verification requirements.
As we indirectly pointed out earlier, we focus on the privacy provided to Alice 
and Bob by the data.
So as long as Alice and Bob can conceal their identities at the demonstration 
and escape without arrest, their support is recorded in the data while their 
privacy is not violated.
(Following this line of thinking, it can actually be beneficial for the privacy 
of the demonstrators to mix with the participants of any counter-demonstrations 
--- since the counts will still be correct.)

In voting, we have the following requirements:
\begin{frame}
\begin{requirements}[P]
\item\label{VotePrivacy} Vote privacy: the voting does not reveal any 
  individual vote.
\item\label{ReceiptFreeness} Receipt freeness: the voting system does not 
  provide any data that can be used as a proof of how the voter voted.
\item\label{CoercionResistance} Coercion resistance: a voter cannot cooperate 
  with a coercer to prove the vote was cast in any particular way.
\end{requirements}
\pause{}
\mode<presentation>{%
  \begin{remark}
    P\ref{CoercionResistance} \(\implies\)
    P\ref{ReceiptFreeness} \(\implies\)
    P\ref{VotePrivacy}
  \end{remark}
}
\end{frame}
\Textcite{VerifyingPrivacyPropertiesOfVotingProtocols} showed that 
\cref{CoercionResistance} implies \cref{ReceiptFreeness}, which in turn implies
\cref{VotePrivacy}.
\Cref{CoercionResistance} is probably not possible to achieve for protests:
e.g.\ Eve can simply physically bring Alice to a protest against her will.
This leaves us with \cref{ReceiptFreeness,VotePrivacy}.

\mode<none>{%
\begin{frame}
  \begin{remark}
    \begin{itemize}
      \item Coercion resistance is difficult to achieve for voting.
      \item It doesn't make much sense for protests.
      \item E.g.\ Eve physically brings Alice to a protest she doesn't want to 
        participate in.
      \item That leaves us with receipt freeness and vote privacy.
    \end{itemize}
  \end{remark}
\end{frame}
}

\subsubsection{Participation-proof privacy}

\begin{frame}
\begin{definition}[Proof indistinguishability]
  \dots
\end{definition}
\end{frame}

\subsubsection{Receipt freeness}

\begin{frame}
\begin{definition}[Receipt freeness/deniability]
  \dots
\end{definition}
\end{frame}

