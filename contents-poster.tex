\begin{columns}
  \begin{column}{0.17\linewidth}
    \includegraphics[width=0.5\linewidth]{fig/kth_cmyk.eps}
    \hfill
  \end{column}
  \hfill
  \begin{column}{0.64\linewidth}
    \begin{center}
      \Huge\bfseries
      Securely and Privately Verifiable Protests
    \end{center}
    \begin{center}
      \large
      Daniel Bosk <dbosk@kth.se>
      $\bullet$
      Sonja Buchegger
      $\bullet$
      Sébastien Gambs
    \end{center}
    \vspace{1.5em}
  \end{column}
  \hfill
  \begin{column}{0.17\linewidth}
    \hfill
    \includegraphics[width=0.8\linewidth]{fig/uqam.pdf}
  \end{column}
\end{columns}

\begin{columns}[t]

  \begin{column}{0.32\linewidth}

    \begin{blueblock}{Scenario: The crowd-counting problem}
      \begin{itemize}
        \item Alice organizes a protest against Eve's regime.
        \item Bob, Carol and others show up.
        \item Must know how many showed up to support Alice's cause.
      \end{itemize}
    \end{blueblock}

    \begin{figure}
      \centering
      \includegraphics[width=0.9\linewidth]{fig/Jacobs-method.jpg}
      \caption{%
        Jacob's method. Image: popularmechanics.com
      }\label{JacobsMethod}
    \end{figure}

    \begin{whiteblock}{Jacob's method, most used method}
      \begin{itemize}
        \item Most used method, Jacob's method (\cref{JacobsMethod}).
        \item Divide the area into regions, estimate the density in each 
          region, sum them up.
        \item \color{red} Cannot handle cumulative counts.
      \end{itemize}
    \end{whiteblock}

    \begin{whiteblock}{Computer vision methods}
      \begin{itemize}
        \item Computer vision does object recognition.
        \item Requires photos/video that cover the entire location, all the time.
        \item \color{red} This will count people twice.
      \end{itemize}
    \end{whiteblock}

    \begin{whiteblock}{Other methods}
      \begin{itemize}
        \item Scan active mobile phones in the area.
        \item This requires some extra equipment.
        \item \color{red} This catches bystanders who are not protesting.
      \end{itemize}
    \end{whiteblock}

    \begin{redblock}{Verifying protest participation}
      \begin{itemize}
        \item Alice wants to show that many support her cause.
        \item Eve wants to show that few support Alice's cause.
        \item \color{red} Adversarial setting, requires verifiable results!
      \end{itemize}
    \end{redblock}

    \begin{blueblock}{Requirements for verifiability}
      \begin{itemize}
        \item\label{EligibilityVerif} Eligibility: anyone can verify that each 
          participation proof provides temporal and spatial eligibility and that 
          it has not been counted before.

        \item\label{UniversalVerif} Universal verifiability: anyone can verify 
          that the result is according to the submitted participation proofs.

        \item\label{IndividualVerif} Individual verifiability: every participant 
          can verify that their participation proof is included in the global 
          count.
      \end{itemize}
    \end{blueblock}

    \begin{blueblock}{Requirements for privacy}
      \begin{itemize}
        \item The verifiability protocol should not increase the risk already 
          incurred by attending.
      \end{itemize}
    \end{blueblock}

    \begin{greenblock}{Our approach: user's perspective}
      \begin{itemize}
        \item The organizer publishes the protest's manifesto, e.g.\ QR-code on 
          placard.
        \item Each protester has a smartphone with a protest app.
        \item Each protester scans the QR code with the app.
        \item The app communicates with other apps locally, e.g.\ ad-hoc network 
          or Bluetooth.
        \item The apps issues participation proofs to each other.
        \item Once there is an Internet connection, the proofs are submitted to 
          a blockchain.
        \item Anyone can verify the proofs on the blockchain and count the valid 
          ones.
      \end{itemize}
    \end{greenblock}

  \end{column}

  \hfill

  \begin{column}{0.32\linewidth}

    \begin{greenblock}{Our approach: during protest}
      \begin{itemize}
        \item The organizer publishes the protest's manifesto.
        \item Each protester reads it, approves it, computes a protest (cause) 
          identifier (\(cid\)) to designate which protest they participate in.
        \item Each protester computes a personal identifier (\(pid\)), 
          unlinkable between protests,
        \item Each protester acts as witness (unlinkable between protesters) for 
          other protesters by creating participation-proof shares.
        \item Proof shares vouches for temporal and spatial eligibility.
        \item A threshold-based number of valid shares is a valid proof.
      \end{itemize}
    \end{greenblock}

    \begin{whiteblock}{Designated protest}
      \begin{itemize}
        \item A protest is identified by its cause, captured in a 
          manifesto.
        \item The hash of the manifesto provides an unpredictable identifier 
          (\(cid\) in \cref{ProofShare,Protocol}).
        \item It's difficult to create a second manifesto that disagrees but 
          gets the same identifier.
      \end{itemize}
    \end{whiteblock}

    \begin{whiteblock}{Privacy-preserving linkability}
      \begin{itemize}
        \item Need linkability during a protest to prevent Sybil attacks.
        \item Also need privacy.
        \item Uses techniques from anonymous credentials to create identifiers 
          (\(pid, wid\) in \cref{ProofShare,Protocol}).
        \item The identifiers are random but unique per protest, i.e.\ depends 
          on \(cid\).
        \item The accompanying \ac{NIZK} proof prevents cheating (used during 
          verification).
      \end{itemize}
    \end{whiteblock}

    \begin{whiteblock}{Temporal eligibility}
      \begin{itemize}
        \item We use a blockchain for commitments, freshness.
        \item Freshness: a protester must include the head (hash value) of the 
          blockchain in their proof shares (\(t_s\) in 
          \cref{ProofShare,Protocol}).
        \item This value is difficult to guess and prevents preparing proofs too 
          long in advance.
        \item Commitments: when a proof share is prepared, it is committed to 
          the blockchain for timestamping.
      \end{itemize}
    \end{whiteblock}

    \begin{whiteblock}{Location proofs and spatial eligibility}
      \begin{itemize}
        \item Each proof must be bound to the location (\(l\) in 
          \cref{ProofShare,Protocol}).
        \item This is done using witnesses.
        \item A witness runs a distance-bounding protocol to ensure proximity.
        \item If in proximity, the witness will issue a witness identifier and 
          signature (\(wid, wsig\)).
      \end{itemize}
    \end{whiteblock}

    \begin{greenblock}{Our approach: after the protest}
      \begin{itemize}
        \item Proof shares are committed to a blockchain as soon as possible.
        \item \ac{NIZK} proofs are submitted whenever convenient.
        \item Anyone can verify the proofs: eligibility, universal and 
          individual verifiability.
        \item Must be done for all shares and their \ac{NIZK} proofs.
        \item Count all \(pid\)s with valid proofs.
      \end{itemize}
    \end{greenblock}

    \begin{whiteblock}{Eligibility and universal verifiability}
      \begin{itemize}
        \item Verify that \(pid\), \(wid\) and \(wsig\) are correct, i.e.\ verify 
          their \ac{NIZK} proofs.
        \item \ac{NIZK} proofs must show \(pid = PRF_{k_P}(id)\) and \(k_P\) is 
          signed by \iac{CA}.
        \item \ac{NIZK} proofs must show \(wid = PRF_{k_W}(pid), wsig = 
            PRF_{k_W}(wid, t_s, l)\) and \(k_W\) is signed by \iac{CA}.
      \end{itemize}
    \end{whiteblock}

    \begin{whiteblock}{Individual verifiability and receipt freeness}
      \begin{itemize}
        \item Each protester can verify that their proof is available on the 
          blockchain.
        \item Protester no longer needs to keep the proof, just the hash.
        \item Eve can use \(k_P\) to redo the computations to verify 
          participation:
          \[pid'\gets PRF_{k_P}(cid) \text{ and compare } pid = pid'.\]
        \item However, requires arrest and access to key.
      \end{itemize}
    \end{whiteblock}

  \end{column}

  \hfill

  \begin{column}{0.32\linewidth}

    \begin{figure}
      \centering
      \begin{tikzpicture}[%
        -Latex,
        item/.style={rectangle,draw},
        edge from parent/.style={},
        ]
        \tikzset{%
          %grow'=left,%
          %level distance=5em%
        }
        \node[item] (proof) {Proof share}
        child {%
          node[item] (pid) {$pid$}
          child {%
            node[item] (cid) {$cid$}
            child {%
              node[item] (manifesto) {Manifesto}
            }
          }
        }
        child {%
          node[item] (wid) {$wid$}
        }
        child {%
          node[item] (ts) {$t_s$}
        }
        child {%
          node[item] (l) {$l$}
        }
        child {%
          node[item] (wsig) {$wsig$}
        }
        ;

        \path[every node/.style={font=\small}]
        (pid) edge node [anchor=south east] {$\in$} (proof)
        (wid) edge node [anchor=east] {$\in$} (proof)
        (ts) edge node [anchor=east] {$\in$} (proof)
        (l) edge node [anchor=west] {$\in$} (proof)
        (wsig) edge node [anchor=south west] {$\in$} (proof)
        ;

        \path[every node/.style={font=\small}]
        (manifesto) edge node [anchor=east] {$H(\cdot)$} (cid)
        (cid) edge node [anchor=east] {$PRF_{k_P}(\cdot)$} (pid)
        (pid) edge[bend right] node [anchor=north west] {$PRF_{k_W}(\cdot)$} (wid)
        % wsig
        (l) edge[bend right] (wsig)
        (ts) edge[bend right] (wsig)
        (wid) edge[bend right] node [anchor=north west] {$PRF_{k_W}(\cdot, \cdot, 
          \cdot)$} (wsig)
        ;

      \end{tikzpicture}
      \caption{%
        Structure of a proof share.
        The protester \(P\)'s identifier \(pid\) is computed using the protester's 
        key \(k_P\).
        The witness \(W\)'s identifier \(wid\) is computed using the witness's key 
        \(k_W\).
        \(t_s\) is a time interval and \(l\) is the coordinates of an area.
        The protest (cause) identifier \(cid\) is the hash value of the manifesto.
      }%
      \label{ProofShare}
    \end{figure}%

    \begin{figure}
      \centering
      \begin{minipage}{\linewidth}
        \begin{align*}
          O\to \text{all}\colon & \text{manifesto} \\
          P\colon & cid\gets H(\text{manifesto}), \\
          & pid\gets PRF_{k_P}(cid) \\
          P\to W\colon & pid \\
          W\leftrightarrow P\colon & \text{perform distance bounding} \\
          W\colon & wid\gets PRF_{k_W}(pid), \\
          & wsig\gets PRF_{k_W}(wid, t_s, l) \\
          W\to P\colon & (wid, t_s, l, wsig) \\
          W\to S\colon & H(pid, wid, t_s, l, wsig) \\
          P\to S\colon & H(pid, wid, t_s, l, wsig) \\
          W\to S\colon & (pid, wid, t_s, l, wsig),\\
          & NIZK(wid = PRF_{k_W}(pid), \\
            & wsig = PRF_{k_W}(wid, t_s, l), \\
            & \exists sign(k_W)) \\
          P\to S\colon & (pid, wid, t_s, l, wsig),\\
          & NIZK(pid = PRF_{k_P}(cid), \exists sign(k_P))
        \end{align*}
      \end{minipage}
      \caption{%
        An overview of message exchanges.
        The organizer \(O\) broadcasts the manifesto.
        \(P\), \(W\) and their computations are as in \cref{fig:ProofFig}.
        Finally, both \(P\) and \(W\) submits the proof share to the storage \(S\).
      }%
      \label{Protocol}
    \end{figure}

    \begin{purpleblock}{Conclusions}
      \begin{itemize}
        \item Provides a \dots
      \end{itemize}
    \end{purpleblock}

    \printbibliography[heading=none]

    %\vspace{10cm}
    \begin{center}
      \includegraphics[width=0.5\linewidth]{fig/qr.eps}
    \end{center}

  \end{column}

\end{columns}

