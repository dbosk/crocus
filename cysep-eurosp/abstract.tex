% describe the problem being addressed
% describe in detail the approach to solve the problem
% preliminary results is encouraged but not required
% describe the envisioned contribution,
% motivate how it relates to the event theme and
% why it could spark discussions.

% It should also provide information about the presenters and their organizations 
% and any special needs for the setup/presentation.



\textbf{Problem:}
Crowd counts vary according to who is doing the counting and what
methods and criteria they use. Current methods have wide margins for
error, are difficult to verify, and can be
privacy-invasive. Historically, there are many examples of events such
as protests in which the count estimated by police and that of the
organizers differ significantly, sometimes by hundreds of
thousands. Even without foul play, the difference is quite natural as
both parties have different objectives and metrics. More precisely,
the organizers want to count everyone who participated while the
police want to estimate the count at the peak of participation, due to
crowd control. We
recognize the potential adversarial setting for crowd counts and thus
aim for both transparency (in the form of verifiability) and
privacy. Since the crowd count could be done at political protests,
any solution needs to not endanger participants more than what they
already face by their physical presence.

\textbf{Approach:}
In this paper we propose CROCUS, a decentralized system based on
smartphones that combines anonymous credentials, witnesses of
proximity, and storage of participation proofs on a public ledger
(e.g., blockchain).
CROCUS achieves properties similar to those needed for
electronic voting, where, put simply, it is important to count
everyone, but only once, not leak private information, and be able to
prove that the count was done correctly. 


\textbf{More detailed approach:} 
By running a distance bounding scheme with anonymous credentials,
physically present persons (such as protest participants) create proof
shares. This is done using smartphones with (assumed)
distance-bounding chips, without Internet connection but using any
available local communications. Before and after the event (protest), participants (and
witnesses) interact with a public ledger such as a blockchain (which does require
Internet access, and anonymization by e.g. Tor). Before the event to get
a timestamp and after the event to upload proof shares, again with a
time stamp. The proof shares form verifiable yet privacy-preserving
participation proofs according to counting eligibility criteria. These criteria, including
time interval, location range, number or kind (e.g., anyone, trusted
journalists) of unique proof shares
needed for a proof, enable verification by recount. Counting is done
by downloading the proof shares from the ledger, checking the validity
and eligibility, and summing up. Any published counts need to also
publish the criteria used to be verified, but anyone could decide on
their own criteria and count accordingly. 

\textbf{Preliminary results:} We find that, with some
assumptions about the availability of future technology such as
distance-bounding chips on smartphones, CROCUS can provide the
desired properties at acceptable performance, except against a truly
global passive adversary.

\textbf{Relation to theme:} This work is concerned with both security
and privacy: accountability and verifiability (individual, universal, eligibility) and
privacy (unlinkability to individual, unlinkability between events,
unlinkability between witnessing instances). The building blocks are
anonymous credentials, distance bounding, blockchain (or other public ledger)


\textbf{Why it could spark discussions:} There are some practical
problems that might also be ideological problems: to ensure
verifiability (specifically, eligibitlity of counting only once, in
analogy to votes or double spending in e-currencies) despite
unlinkability/privacy properties, we need sybil-proof identities for
anonymous credentials.

\textbf{Information about the presenters:} Daniel Bosk is a PhD
student and Sonja Buchegger a Computer Science professor, both at the
division of Theoretical Computer Science at KTH Royal Institute of
Technology, working on privacy-enhancing technologies.
