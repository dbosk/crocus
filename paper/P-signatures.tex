\subsection{Non-interactive anonymous credentials}%
\label{NIZK-anon-cred}

\NewCryptoScheme{\Psig}{Psig}

\Citet{Psignatures} presents a class of signatures, commitments and 
non-interactive proof protocols named P-signatures (\Psig), which yields for 
creating non-interactive anonymous credentials.
P-signatures provides a signature scheme and a commitment scheme with a protocol 
to obtain a signature on a committed value and two non-interactive proof 
systems, one for proving that a committed value has been signed and the other 
for proving that a pair of commitments are commitments to the same value.

\NewAlgorithm{\PSetup}{\Psig.\!Setup}

\(params\gets \PSetup[1^\lambda]\): generates the global parameters \(params\), 
which contains parameters for the underlying algorithms.
For ease of notation we will omit \(params\) from all functions.

\NewAlgorithm{\PCommit}{\Psig.\!Commit}

\(comm\gets \PCommit[m, open]\): takes a message \(m\) and an opening value 
\(open\) and returns a commitment.
The commitment scheme must be \emph{perfectly binding} and \emph{strongly 
  computationally hiding}.
Perfectly binding means that for every bit string \(comm\) there exists at most 
one value \(x\) such that there exists a value \(open\) with \(comm = 
  \PCommit[x, open]\).
Strongly computationally hiding means that there exists an alternate setup 
algorithm which generates indistinguishable parameters which yields 
information-theoretically hiding commitments.
I.e.\ the adversary cannot tell which setup algorithm has been used.

\NewAlgorithm{\PObtainSig}{\Psig.\!ObtainSig}
\NewAlgorithm{\PIssueSig}{\Psig.\!IssueSig}

\(\sigma\gets \PObtainSig[pk, m, comm, open]\leftrightarrow
  \PIssueSig[sk, comm]\): obtains a signature \(\sigma\) on the message 
\(m\) hiding it in the commitment \(comm\) with the opening value \(open\).
The function \(\PObtainSig\) is run by the user and \(\PIssueSig\) is run by the 
signature issuer.

\NewAlgorithm{\PProve}{\Psig.\!Prove}

\((comm, \pi, open)\gets \PProve[pk, m, \sigma]\): outputs a commitment 
\(comm\) on the message \(m\) and can be opened using \(open\).
\(\pi\) is a proof of knowledge of the signature \(\sigma\) on the message \(m\) 
made by the owner of \(pk\).

\NewAlgorithm{\PVerifyProof}{\Psig.\!VerifyProof}

\(\bin\ni \PVerifyProof[pk, comm, \pi]\): verifies the proof \(\pi\) that \(m\) 
has been signed by the owner of \(pk\).
(\(comm\) is a commitment to \(m\).)
It outputs \(1\) (accept) if \(\pi\) is a proof of knowledge of \(F(m)\) and a 
signature on \(m\).

\NewAlgorithm{\PEqCommProve}{\Psig.\!EqCommProve}

\(\pi\gets \PEqCommProve[m, open, open']\): outputs a proof \(\pi\) saying that 
\(comm = \PCommit[m, open]\) is a commitment to the same value as \(comm' = 
  \PCommit[m, open']\).
This can be used to tie ephemeral commitments to long-term commitments.

\NewAlgorithm{\PVerEqComm}{\Psig.\!VerEqComm}

\(\bin\ni \PVerEqComm[comm, comm', \pi]\): outputs \(1\) (accept) if the proof 
\(pi\) is a proof saying that \(comm, comm'\) are commitments to the same value.
