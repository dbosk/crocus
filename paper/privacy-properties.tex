\subsection{Privacy requirements}%
\label{privacy-properties}


\begin{requirements}[P]
\item\label{PseudonymUnlink} \emph{Pseudonym unlinkability}: given a protest 
  (identifier \cid), protesters Alice and Bob, and a pseudonym \(\pid_b\), the 
  adversary cannot tell if \(\pid_b = \pid_{\text{Alice}}\) or \(\pid_b = 
    \pid_{\text{Bob}}\), except with negligible probability. And similarly with $wid$ if Alice and Bob act as witnesses.
\item\label{ProtestUnlink} \emph{Protest unlinkability}: protesters' pseudonyms 
  (\(\pid_\cid, \pid_{\cid'}\)) must be unlinkable between protests (\(\cid, 
    \cid'\)) from the adversary's perspective.
\item\label{WitnessUnlink}\emph{Witness unlinkability}:  witnesses' pseudonyms 
  (\(\wid_\pid, \wid_{\pid'}\)) must be unlinkable between protesters (\(\pid, 
    \pid'\)) from the adversary's perspective.
\end{requirements}

What these properties say is that pseudonyms must look random 
(\cref{PseudonymUnlink}) and that each psdeudonym must not be reused
more than strictly necessary for the verifiability properties 
(\cref{ProtestUnlink,WitnessUnlink}). In Pfitzmann-Hansen
terminology~\cite{pfitzmann-hansen}, they are role-relationship
pseudonyms, but with increasingly narrowed notions of roles and
relationships (participant of particular protest, witness of a
particular participant at a particular protest).
