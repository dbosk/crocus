\subsection{Privacy requirements}%
\label{privacy-properties}

In addition to the verification requirements, we also need to define the 
privacy properties.
In voting protocols, there are mainly three levels of 
privacy~\cite{VerifyingPrivacyPropertiesOfVotingProtocols}:
\begin{description}
  \item[Vote privacy:] the voting does not reveal any individual vote.
  \item[Receipt freeness:] the voting system does not provide any data that can 
    be used as a proof of how the voter voted.
  \item[Coercion resistance:] a voter cannot cooperate with a coercer to prove 
    the vote was cast in any particular way.
\end{description}

Coercion resistance is not possible to achieve for protests.
For instance, someone could simply physically bring Alice to a protest against 
her will.
Receipt freeness, while desirable, is not \emph{always} desirable.
For instance, the location proof might be useful to prove presence to a third 
party, \eg for strengthening the authenticity of witness information.

Thus privacy remains.
For the protester we want unlinkability between a protester's real identity 
\(P\) and the cause identifier \(\cid\) against the adversary.
Given a participation proof, Grace will not be able to tell if it belongs to 
Alice or Bob.
Furthermore, if Grace has managed to link one proof to Alice's, she should not 
be able to link another proof (\eg, from a different protest) as a consequence.

For the witness, we want unlinkability between a witness's identity \(W\) and a 
location \(l\) against an adversary who is not present in the location.
Due to the limitations of physical space, two individuals present in the same 
location will inevitably learn of each other's presence.
Thus we will only try to prevent individuals who are not present to learn of 
this presence.

We can thus summarize the desired privacy properties as follows:
\begin{requirements}[P]
\item\label{ProofUnlink} \emph{Anonymity}: participants' real (long-term) 
  identities must be unlinkable to their proofs.
\item\label{ProtestUnlink} \emph{Unlinkability}: protesters' proofs must be 
  unlinkable between protests and witnesses' proofs must be unlinkable between 
  protesters.
\end{requirements}

As a result of these properties, if there are two protests in parallel (\eg, one against Grace's regime and one in favor) Alice can always argue to be part of the protest in Grace's favor instead of any protest against Grace.
This is, in fact, is a slight increase in privacy over ordinary protests.

