\subsection{Privacy requirements}%
\label{privacy-properties}

In addition to the verification requirements, we also need to define the privacy properties.
In voting protocols, there are mainly three levels of privacy~\cite{VerifyingPrivacyPropertiesOfVotingProtocols}:
\begin{description}
  \item[Vote privacy:] the voting does not reveal any individual vote.
  \item[Receipt freeness:] the voting system does not provide any data that can 
    be used as a proof of how the voter voted.
  \item[Coercion resistance:] a voter cannot cooperate with a coercer to prove 
    his vote was cast in any particular way.
\end{description}

Coercion resistance is not possible to achieve for protests.
For instance, someone could simply physically bring Alice to a protest against her will.
Receipt freeness, while appealing at first glance, is not desirable in our context, as in contrast to voting, receipt freeness for \emph{how} the voter voted (\ie the cause of the protest) implies receipt freeness for \emph{that} the voter voted (\ie the protester was there) which conflicts with verifiability.  

Thus, in our context the crucial property is therefore privacy.
More precisely for the protester, we want unlinkability between a protester's real identity \(P\) and the cause identifier \(\cid\) against the adversary.
Given a participation proof, Grace should not be able to distinguish if it belongs to Alice or Bob.
Furthermore, if Grace has managed to link one proof to Alice's due to some auxiliary knowledge, she should not be able to link it to another proof (\eg, from a different protest).

%Seb: I have put it in comments to save on space
%\sonja{removed for now: For the witness, we want unlinkability between a witness's identity \(W\) and a 
%location \(l\) against an adversary who is not present in the location.
%Due to the limitations of physical space, two individuals present in the same 
%location will inevitably learn of each other's presence.
%Thus we will only try to prevent individuals who are not present to learn of 
%this presence.}

% \sonja{start of older thoughts}
% Difference to e-voting:
% physical presence means support for a cause if there is one protest. If there is a counterprotest, physical presence with that group means support for the counterprotest. In our system, it gets more ambiguous as the participants are counted if all of the following is true
% \begin{enumerate*}
% \item they participate in the protocol,
% \item they use the protest identifier (cause) for which the count is done,
% \item they meet the eligibility criteria, and
% \item they receive enough proof shares,
% \item that are uploaded to the ledger.
% \end{enumerate*}
% Thus one could participate in a protest and be counted even when physically 
% present at a counter-protest if the two protests are co-located and the 
% participant chooses the corresponding cause/identifier of the original protest.

% Given that all conditions for being counted in a protest are met, individual 
% verifiability implies a receipt \daniel{no, I know my proof is included in 
%  block 0xdeadbeef in the blockchain, so as long as 0xdeadbeef is there I'm 
%  sure my block is counted (individual verifiability); but if I give 0xdeadbeef 
%  to you, you cannot know for sure if 0xdeadbeef says anything or I made it up, 
%  hence receipt-freeness} because being counted implies support for a cause. 
% This also means that besides physical force, coercion can happen by forcing the 
% participate to show a receipt. \daniel{The only receipt we have is the private 
%  key of the protester, with that key the adversary can compute \(\pid = 
%    \ACprf[_k][\cid]\) and check that \(\pid\) is indeed in the ledger.}
% We thus have to decide between individual verifiability and the privacy 
% properties for e-voting or define privacy properties suitable for the scenario. 
% The ultimate privacy requirement is thus stronger here: in contrast to voting, 
% where it is the content of the vote itself that needs to be unlinkable to the 
% voter, here the fact of having cast a vote at all needs to be unlinkable.

% The privacy properties we need in this context are the following:
% \begin{description}
%  \item[Involuntary participant unlinkability:] \daniel{the name has a 
%      double-negative ring to me} unless the participant discloses the receipt 
%    (by revealing her identity), there is no link between the protest data on 
%    the ledger and the identity of the participant.

%  \item[Involuntary witness unlinkability:] \daniel{and double-negative here 
%      too of course} unless the witness discloses the receipt (by revealing her 
%    identity), there is no link between the protest data on the ledger and the 
%    identity of the witness.
% \end{description}

% In essence, receipt freeness means that upon completing the protocol, Grace cannot link Alice to Alice's participation proof --- even if she were to compromise Alice's device or Alice accepts to collaborate with Grace.
% As we will see, if Grace gets Alice's device, she can re-do the computations on the same inputs.
% Since, in our case, the algorithms are deterministic and collisions are negligible, she will produce the same output and can thus conclude that Alice is indeed the one she is looking for.
% Thus, we cannot provide receipt freeness (see \cref{Conclusion} for further discussions on this issue).

% As a consequence, this leaves us with with what would correspond to vote privacy.
% In particular, we need unlinkability between Alice's long-term identity and Alice's participation proof.
% \sonja{end older thoughts}
Finally, any temporary identifiers such as pseudonyms should only be reused when strictly necessary for the fulfillment of verifiability properties listed in \label{verifiability-properties} to prevent inference on long-term identities by repeated samples and side information.

We can thus summarize the desired privacy properties as follows:
\begin{requirements}[P]
\item\label{ProofUnlink} \emph{Proof Unlinkability (Anonymity)}: every
  participant's real (long-term) identity (\(P\)) must be unlinkable to their participation proofs \(\prf_{\pid, \prtst}\)
\item\label{ProtestUnlink} \emph{Protest Unlinkability}: protesters'
  participation proofs \(\prf_{\pid, \prtst}\) must be unlinkable between protests (\(\cid, \cid' \))
\item\label{WitnessUnlink}\emph{Witness Unlinkability}:  witnesses'
  participation proof shares \(\psh = (\cid, t, l, \pid,\wid)\) must be unlinkable between 
  protesters. (\(\pid, \pid' \))
\end{requirements}

\sonja{removed for now: As a result of these properties, if there are two protests in parallel (\eg, one against Grace's regime and one in favor) Alice can always argue to be part of the protest in Grace's favor instead of any protest against Grace.
This is, in fact, is a slight increase in privacy over ordinary protests.}

