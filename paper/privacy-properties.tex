\subsection{Privacy requirements}%
\label{privacy-properties}

In addition to the verification requirements, we also need to define the privacy properties.
In voting protocols, there are mainly three levels of privacy~\cite{VerifyingPrivacyPropertiesOfVotingProtocols}:
\begin{description}
  \item[Vote privacy:] the voting does not reveal any individual vote.
  \item[Receipt freeness:] the voting system does not provide any data that can be used as a proof of how the voter voted.
  \item[Coercion resistance:] a voter cannot cooperate with a coercer to prove his vote was cast in any particular way.
\end{description}

Coercion resistance is not possible to achieve for protests.
For instance, someone could simply physically bring Alice to a protest against her will.
Receipt freeness, while appealing at first glance, is not desirable\footnote{%
  Or not possible without violating some of the verifiability 
  requirements.
} in our context since this will conflict with verifiability: in contrast to 
voting, receipt freeness for \emph{how} the voter voted (\ie the cause of the 
protest) implies receipt freeness for \emph{that} the voter voted (\ie the 
protester was there), which conflicts with verifiability.

Thus, in our context the crucial property is therefore privacy.
More precisely, for the protester we want unlinkability (from the adversary's 
perspective) between a protester's real identity \(P\) and the cause identifier 
\(\cid\).
Phrased differently, given a participation proof, Grace should not be able to 
distinguish if it belongs to Alice or Bob.
Furthermore, if Grace has managed to link one proof to Alice due to some 
auxiliary knowledge, she should not be able to link it to another proof (\eg 
from a different protest).

Finally, any temporary identifiers, such as pseudonyms, should only be reused 
when strictly necessary for the fulfillment of verifiability properties (in 
\cref{verifiability-properties}) to prevent inference on long-term identities 
by repeated samples and side information.

We can thus summarize the desired privacy properties as follows:
\begin{requirements}[P]
\item\label{ProofUnlink} \emph{Proof Unlinkability (Anonymity)}: every 
  participant's (protester and witness) real (long-term) identity (\(P\)) must 
  be unlinkable to their participation proofs \(\prf_{\pid, \prtst}\)
\item\label{ProtestUnlink} \emph{Protest Unlinkability}: protesters' participation proofs \(\prf_{\pid, \prtst}\) must be unlinkable between protests (\(\cid, \cid' \))
\item\label{WitnessUnlink}\emph{Witness Unlinkability}:  witnesses' participation proof shares \(\psh = (\cid, t, l, \pid,\wid)\) must be unlinkable between protesters. (\(\pid, \pid' \))
\end{requirements}

In light of these definitions, we note that \cref{CountOnce} can be rephrased 
in terms of pseudonyms as follows:
\begin{enumerate}
  \item[V1.3.] \emph{Counted only once}:
    A protester can create \emph{only one} pseudonym per protest, this 
    pseudonym is unique except with negligible probability.
\end{enumerate}

%Seb: ok for me, I have put it in comment
%\sonja{removed for now: As a result of these properties, if there are two protests in parallel (\eg, one against Grace's regime and one in favor) Alice can always argue to be part of the protest in Grace's favor instead of any protest against Grace.
%This is, in fact, is a slight increase in privacy over ordinary protests.}

