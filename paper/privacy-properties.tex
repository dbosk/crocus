\subsection{Privacy requirements}%
\label{privacy-properties}

Finally, any temporary identifiers, such as pseudonyms, should only be reused 
when strictly necessary for the fulfillment of verifiability properties (in 
\cref{verifiability-properties}) to prevent inference on long-term identities 
by repeated samples and side information.

We can thus summarize the desired privacy properties as follows:
\begin{requirements}[P]
\item\label{ProofUnlink} \emph{Proof Unlinkability (Anonymity)}: every 
  participant's (protester and witness) real (long-term) identity (\(P\)) must 
  be unlinkable to their participation proofs \(\prf_{\pid, \prtst}\)
\item\label{ProtestUnlink} \emph{Protest Unlinkability}: protesters' participation proofs \(\prf_{\pid, \prtst}\) must be unlinkable between protests (\(\cid, \cid' \))
\item\label{WitnessUnlink}\emph{Witness Unlinkability}:  witnesses' participation proof shares \(\psh = (\cid, t, l, \pid,\wid)\) must be unlinkable between protesters. (\(\pid, \pid' \))
\end{requirements}

In light of these definitions, we note that \cref{CountOnce} can be rephrased 
in terms of pseudonyms as follows:
\begin{enumerate}
  \item[V1.3.] \emph{Counted only once}:
    A protester can create \emph{only one} pseudonym per protest, this 
    pseudonym is unique except with negligible probability.
\end{enumerate}

%Seb: ok for me, I have put it in comment
%\sonja{removed for now: As a result of these properties, if there are two protests in parallel (\eg, one against Grace's regime and one in favor) Alice can always argue to be part of the protest in Grace's favor instead of any protest against Grace.
%This is, in fact, is a slight increase in privacy over ordinary protests.}

