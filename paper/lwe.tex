\mode*

\section{The \acl*{LWE} problem}

The l-bit problem is related to the \ac{LWE} problem.
In the \ac{LWE} problem we have a secret vector \(\vec s \in \ZZ_q^n\).
Given
\begin{align*}
  \langle \vec s, \vec a_i\rangle + \epsilon_i \pmod q,
\end{align*}
where \(\vec a_i \in_R \ZZ_q^n\) is a set of random vectors and \(\epsilon_i 
\in \ZZ_q\) are errors according to some distribution (usually Gaussian), the 
adversary must find \(\vec s\).

The l-bit problem is the case of \(\ZZ_q^1\), \ie with \(n = 1\).
All hardness results depends on \(n\).
This means that the problem becomes easier.
However, the size of the set of equations also depends on \(n\), which brings 
some balance.

However, the major difference is the blinding factor.
The format of the l-bit equations is \(\rho_i + c_i\alpha\) and \(\rho_i + 
c_i'\alpha\), where \(\alpha\) is the secret and \(\rho_i\) is uniformly 
randomly chosen in \(\ZZ_q\) and \(c_i, c_i'\) take the role of \(\vec a\) 
above.
These equations come in pairs with an error that is related between the pair of 
equations.

Considering this, it's hard to map the \ac{LWE} problem to the l-bit problem.
We must essentially map the blinding factor \(\rho\), the \(q+l\) learned bits 
and the error together.
It's probably easier to just use the information theoretic aspects of the case.
