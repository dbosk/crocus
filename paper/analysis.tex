\section{Security and privacy analysis}%
\label{SecurityAnalysis}

We note that Alice and Bob do not trust each other, for all they know one or the 
other could be an agent of Grace the autocrat.
\begin{itemize}
  \item Grace the evil autocrat,
  \item Mallory the foreign nation state, and finally
\end{itemize}


\subsection{Forging new proofs}

It follows from the digital signatures used that proofs are difficult to forge.

\subsection{Reusing participation proofs}

We consider two ways of reusing participation proofs:
\begin{enumerate}
  \item Alice and Bob can reuse old proofs by modifying the identifiers.
  \item Grace can reuse proofs by creating a manifesto which yields the same 
    identifier as Alice's protest, then try to create new nodes in the 
    \ac{tposet} that matches the ones used in the old proofs.
\end{enumerate}
We will now estimate the difficulty of both types of attacks.

\subsubsection{Finding second preimages}

Grace (possibly with Mallory's aid) wants to arrange a counter-protest.
One way to get a higher participation count is to use the same identifier as 
Alice's protest.
Since the identifier for the protest is the hash value of the manifest, Grace 
must find a second preimage for the hash function.

% XXX Rephrase identifier problem as a game
\begin{proposition}
  Given \(x\) and \(H(x) = y\), the probability of finding an \(x'\) such that 
  \(H(x') = y\) is \(\frac{1}{\lambda-1}\).
\end{proposition}

\begin{question}
  What if we use a Merkle tree instead of only a hash, will that change the 
  second preimage property?
\end{question}

Alice wants to arrange another protest against Grace.
Her manifesto can remain the same, thus the protest identifier remains the 
same.
She wants to reuse some old proofs.
The head value that are included in the proofs extend them too far back into 
time.
The digital signatures prevents her from changing them.
So an alternative is to create nodes in the \ac{tposet} which has the same 
hashes.

\subsection{Spoofing locations}

\subsubsection{Trust propagation for witnesses}

\dots

\subsection{Individual verifiability}

Grace and Mallory have an incentive to lower the confidence in the result.
One way they can do this is to say that the result does not contain all 
participation proofs.
Statements such as this should be proved: provide a valid but missing proof.

\section{Performance and implementation}

\subsection{Implementation on smartphones or smart-cards}
Technological progress has enabled more advanced cryptography on Smart-cards and Smartphones which could be used to implement our solution : Benchmarks~\cite{Benchmarking} have shown that Android devices are now fast enough to efficiently implement Privacy Enhancing Technologies (PETs), with a Samsung Galaxy S i9000 being able to execute Idemix in  \SI{153}{\milli\second}. However, those benchmarks also demonstrate that SmartCards remain slow to process complex protocols such as Idemix or U-Prove (taking between \SI{4}{\second} and \SI{8}{\second} to process them). While the limited processing power of many embarked systems has been a challenge, Idemix has been succesfully implemented to prove the posession of credentials on Java Cards by Bichsel et al. in 2009~\cite{Bichsel} and the IRMA project, released in 2014, aimed to achieve an implementation « suitable for real life transactions »~\cite{IRMA} while maintaining security and privacy for its users. 

\subsection{Threshold of witnesses}

Assume that every person has 5000 contacts in their contact 
book~\cite{DifficultyOfPrivateContactDiscovery}.
Then it would be reasonable to set the threshold of at least 6000 witnesses.
According to the Anon-Pass performance measures~\cite{AnonPass} each witness 
signature would require \SI{8}{\milli\second} per core on a quad-core Intel 
\SI{2.66}{\giga\hertz} Core 2 processor.
This yields \(
  \SI{8}{\milli\second}\times 6000 = \SI{48000}{\milli\second} = 
  \SI{48}{\second}.
\) of processing.
