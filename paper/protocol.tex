%\section{\CROCUS:\@ A protocol for crowd counting estimation}%
\section{The \CROCUS Protocol}%
\label{Protocol}


We now present \CROCUS, a protocol for privately verifiable crowd
counts. A prerequisite for using \CROCUS is a one-to-one mapping of a person's
sybil-proof identity and a cryptographic key%
%, more specifically, an anonymous 
%identity credential that provides unlinkability between contexts but 
%linkability for reuse within the same context
%\sonja{this is actually achieved by our use of prfs for pid and wid,
%not inherent in the credential}
% \footnote{%
%   Given any \ac{CA} that prevents Sybil 
%   attacks~\cite{SybilAttack}, systems such as~\cite{clonewars} could
%   be used to make sure that for each identity one-and-only-one
%   cryptographic key is issued.
%   In terms of~\cite{clonewars}, this can be seen as an unclonable subscription 
%   to one's identity.
% }
. Since we assume these properties, we
present only the result of the process of getting anonymous
credentials from a \ac{CA}, in \cref{ProtocolSetup}.

The core of the \CROCUS protocol consists of generating participation
proofs (joining a protest, participating in witnessing, submitting
proof shares to a ledger, all in \cref{ProtocolDuring}) with an overview in \cref{fig:ProtocolOverview}, and then counting and
verification of that count (in \cref{ProtocolVerification}). 

% The protocol consists of five phases: setup (in \cref{ProtocolSetup}),
% join, participation, submission (all in in \cref{ProtocolDuring}) and
% verification (in \cref{ProtocolVerification}).

The entities involved in our protocol are participants and (count) verifiers.
A participant can assume three different roles:
\begin{enumerate}
\item The \emph{organizer} has written a manifesto for the protest and 
  disseminated it to others.
  Anyone can do this, but in general, there is only one organizer.
\item A \emph{protester} is attending the protest and asks witnesses to vouch 
  for their presence.
\item A \emph{witness} provides proofs to protesters which state that they indeed participated, constructed such that the proofs are verifiable by 
  third parties.
\end{enumerate}
After the protest, anyone can be a verifier (universal verifiability, 
\cref{UniversalVerif} in \cref{verifiability-properties}), \ie count the number 
of participants (of the protest and the protocol) or verify anyone else's count 
given the relevant meta data (which protest, location, time, and witness 
parameters used in the
count). We refer to anyone counting or verifying a count as the
verifier; the process is the same. 

Any participant can verify their inclusion (individual verifiability, 
\cref{IndividualVerif} in \cref{verifiability-properties}) by repeating the 
count (the verification process) but also checking that their proof is included 
while counting.

\subsection{Prerequisite: anonymous credentials}%
\label{ProtocolSetup}

Before Alice can have her participation in any event counted by
\CROCUS, she needs to get an anonymous credential and corresponding
keys. This only needs to be repeated when the credential expires or is
lost, in analogy to a passport in terms of expected intervals. The
keys can be reused for an arbitrary number of protests or, given
careful choices in the PRF used for deriving identifiers, other
services that work with anonymous credentials. 

We use the setup and registration phases of Anon-Pass~\cite{AnonPass} for
getting anonymous credentials, adapting only the notation to fit ours. 
%More precisely, we have simply adapted their description to our notation, but otherwise kept the setting as similar as possible.


\emph{Setup: \((\spk, \ssk)\gets \CROCUSsetup\).}
During the setup phase, the \ac{CA} creates all the needed keys.
The \ac{CA} generates a service public-private key-pair \((\spk, \ssk)\gets 
  \ACsetup\) (see \cref{ACsignAlg}).

\emph{Registration: \(\sk\gets 
    \Proto{\CROCUSreg[_P][\spk]}{\CROCUSreg[_{\CA}][\ssk]}\).}
During the registration phase, each participant generates a secret key~\((k, 
  r)\) and obtains a signature on it by the \ac{CA} \emph{but without revealing 
  it} to the \ac{CA} (or to any part thereof in a multi-party scenario).
At the end, each participant will have a signed secret key while the \ac{CA} 
will issue only one signature per participant but without knowing the 
association between a particular key and the identity of the participant.
The participant chooses \(k, r\rgets \ZZ_q\) uniformly randomly and runs 
\(\sigma \gets \Proto{\ACgetSig[\spk, k, r]}{\ACissueSig[\spk, \ssk]}\) (see 
\cref{ACacAlg}).
Upon success, the participant sets \(\sk = (\sigma, k, r)\).



\subsection{Participation}%
\label{ProtocolDuring}


%The join, participation and submission phases are as follows.
% illustrated in \cref{fig:ProtocolOverview}.
The goal of \CROCUS is to generate and collect privacy-preserving
participation proofs that can be counted and verified. These proofs
consist of proof shares that are constructed as depicted in
\cref{fig:ProofFig}. The protocol phases are given in
\cref{fig:ProtocolOverview} and described below.

\begin{figure}
  \centering
  \small
  \includegraphics{proofshare.tikz}
  \caption{%
    Structure of a proof share.
    The protest (cause) identifier \(\cid\) is the hash value of the manifesto.
    The protester \(P\)'s identifier \(\pid\) is computed using the protester's key \(\sk_P\) and \(\cid\).
    The witness \(W\)'s protester-specific identifier \(\wid\) is computed using the
    witness's key \(\sk_W\) and the protester's \(\pid\).
    \(t_s, t_s'\) are the hashes of the head blocks in the ledger seen by the 
    protester and witness, respectively, and \(l\) is an area.
    All values are signed by the witness (signature \(\prf_W = \SPK[\sk_W][\wid 
      = \dotsb][\cid, \pid, \wid, t_s, t_s', l]\)) while also proving the 
    correctness of \(\wid\) and knowledge of a signature on \(\sk_W\).
    The protester constructs \(\prf_P\) analogously.
  }%
  \label{fig:ProofFig}
\end{figure}%


\emph{Creation of a protest: the manifesto.}
The organizer writes a manifesto for the protest, which describes its cause.
This manifesto could take the form of any intelligible text, in
practice at minimum a name.
The organizer then distributes this manifesto to people through
any suitable means  (\eg on the Web, on placards, \etc).
If they agree with the cause, they will use the knowledge of the
manifesto to join the protest. 


\emph{Joining as a protester: \((\pid, t_s)\gets 
    \CROCUSjoin[_P][\text{manifesto}]\).}
A protester who wants to join the protest uses the manifesto to
compute an identifier for the cause by hashing the manifesto,
\(\cid\gets \Hash[\mfst]\) (and comparing the result to that received
from the organizer, we omit this in the protocol for readibility).
Afterwards, this identifier is used to create the protest-specific identifier 
for the protester, \(\pid\gets \ACprf[_{\sk_P}][\cid]\)%
% (see \cref{fig:ProofFig} and \cref{ACprfAlg} in the appendix for details of 
%the algorithms)%
.
The protester also fetches a time-correlated random value, \(t_s\), from 
\(\TS\), \(t_s\gets \TSget\).


\emph{Joining as a witness: \(t_s'\gets \CROCUSjoin_W\).}
The witness simply gets a time-correlated random value from the time-stamping service, \(t_s'\gets \TSget\).
Note that we do this for redundancy, the newest of \(t_s\) and \(t_s'\) will 
set the start of the time interval of creation for the proof share.


\emph{Participation: \(\pi\gets
    \Proto{\CROCUSparticipate[\cid, \sk_P]}{\CROCUSwitness[\sk_W, \spk]}\),}
In the participation phase, the protester and 
the witness construct the proof share of the protester (\cref{fig:ProofFig}).

The protester sends \(\pid\) and \(t_s\) to the witness.
Then they run the protocol \[
  \Proto{\ACproveSig[\spk, k, r, \sigma]}{\ACverifySig[\spk, \ssk]}
\] (see \cref{ACacAlg}), \(k\) and \(r\) are part of \(\sk_P\).
Note that the \acf{PK} in \cref{ACacAlg} must be
run as a \iacf{PPK}, which we do by distance bounding.
If the protocol succeeds, the witness will compute \(\wid \gets 
  \ACprf[_{\sk_W}][\pid]\) and send \((\wid, t_s', l)\) to the protester.


\emph{Submission: \(\psh_P\gets \CROCUSsubmit[_P][\cid, \pid, \wid, t_s, t_s',  l]\).}
The protester commits the proof-share data to the ledger~\(L\) and receives the 
proof of commitment, \(t_e\gets \TSsubmit[\Hash[\cid, \pid, \wid, t_s, t_s', 
  l]]\).
The sooner this is done, the higher the precision for the time-dependent 
eligibility criterion will be for later counting.
The remaining operations are not time critical.

The protester computes \iac{NIZK} proof \(\corr_{\pid}\), which shows the 
correctness of \(\pid\).
More specifically,
\begin{multline*}
  \corr_{\pid}\gets \SPK\left\{ (\sk_P) : \right. \\
    \begin{aligned}
      \pid &= \ACprf[_{\sk_P}][\cid] \quad \land \\
      \sigma_P' &= \left. \ACblind[\ACsign[_{\ssk}][\sk_P]] \right\}
    \end{aligned} \\
      (\cid, \pid, \wid, t_s, t_s', l).
\end{multline*}
Finally, the protester uploads the tuple \[
  \psh_P = (\cid, \pid, \wid, t_s, t_s', t_e, l, \corr_{\pid})
\] for permanent storage, \(\TSsubmit[\psh_P]\).

\emph{Submission: \(\psh_W\gets \CROCUSsubmit[_W][\cid, \pid, \wid, t_s, t_s', 
    l]\).}
The witness, like the protester, commits the proof-share data to the 
ledger, \(t_e\gets \TSsubmit[\Hash[\cid, \pid, \wid, t_s, t_s', l]]\).
%(This is to make the time interval as early as possible, whoever is the faster 
%will submit it.) \sonja{but both do}
Then, without any time requirements, the witness computes \iac{NIZK} proof 
\(\corr_{\wid}\) as follows:
\begin{multline*}
  \corr_{\wid}\gets \SPK\left\{ (\sk_W) : \right. \\
    \begin{aligned}
      \wid &= \ACprf[_{\sk_W}][\pid] \quad \land \\
      \sigma_W' &= \left. \ACblind[\ACsign[_{\ssk}][\sk_W]] \right\}
    \end{aligned} \\
      (\cid, \pid, \wid, t_s, t_s', l).
\end{multline*}
Finally, the witness uploads the tuple \[
  \psh_W = (\cid, \pid, \wid, t_s, t_s', t_e', l, \corr_{\wid})
\] for permanent storage on the ledger, \(\TSsubmit[\psh_W]\).


\begin{figure*}
  \centering
  \tiny
  \begin{subfigure}[b]{0.49\columnwidth}
    \begin{align*}
      O\to \text{all}\colon & \text{manifesto} \\
      P\colon & t_s\gets \TSget \\
        & \cid\gets \Hash[\text{manifesto}], \\
        & \pid\gets \ACprf[_{\sk_P}][\cid] \\
      W\colon & t_s'\gets \TSget
      \\[-1em]
      \noalign{\hfill Join}
      \midrule
      \noalign{\hfill Participation}
      \\[-3em]
      P\to W\colon & \pid \\
      P\leftrightarrow W\colon &
        \PPK\mleft\{ (\sk_P) : \mright. \\
        & \qquad \pid = \ACprf[_{\sk_P}][\cid], \\
        & \qquad \mleft. \sigma_P' = \ACblind[\ACsign[_{\ssk}][\sk_P]] \mright\} 
        \\
      W\colon & \wid\gets \ACprf[_{\sk_W}][\pid] \\
      W\to P\colon & (\wid, t_s', l)
    \end{align*}
    \caption{Join and participation.}
  \end{subfigure}
  \begin{subfigure}[b]{0.49\columnwidth}
    \begin{align*}
      P\colon & t_e\gets \TSsubmit[\Hash[\pid, \wid, t_s, t_s', l]] \\
      W\colon & t_e'\gets \TSsubmit[\Hash[\pid, \wid, t_s, t_s', l]] \\
      W\colon & \TSsubmit[(\cid, \pid, \wid, t_s, t_s', t_e, l, 
      \pi_{\wid})],\quad \text{where} \\
        & \pi_{\wid} = \SPK\mleft\{ (\sk_W) : \mright. \\
        & \qquad \wid = \ACprf[_{\sk_W}][\pid], \\
        & \qquad \mleft. \sigma_W' = \ACblind[\ACsign[_{\ssk}][\sk_W]]\mright\} 
        \\
        & \qquad\qquad (\cid, \pid, \wid, t_s, t_s', l) \\
      P\colon & \TSsubmit[(\cid, \pid, \wid, t_s, t_s', t_e, l, 
      \pi_{\pid})],\quad \text{where}\\
        & \pi_{\pid} = \SPK\mleft\{ (\sk_P) : \mright. \\
        & \qquad \pid = \ACprf[_{\sk_P}][\cid], \\
        & \qquad \mleft. \sigma_P' = \ACblind[\ACsign[_{\ssk}][\sk_P]] \mright\} 
        \\
        & \qquad\qquad (\cid, \pid, \wid, t_s, t_s', l)
    \end{align*}
    \caption{Submission.}
  \end{subfigure}
  \caption{%
    An overview of \CROCUS participation.\@
    The organizer \(O\) broadcasts the manifesto.
    The protester \(P\), witness \(W\) and their computations are as in \cref{fig:ProofFig}.
    Finally, both \(P\) and \(W\) submit the proof shares to a
   public ledger for permanent storage \(S\). Note that \pid  always refers to the
    protester whose presence is being witnessed.
  }%
  \label{fig:ProtocolOverview}
\end{figure*}
%\normalsize


\subsection{Count and Verification}%
\label{ProtocolVerification}

% While there are various ways for verifying the participation count, hereafter, 
% we will detail the two suggested just after \cref{DefParticipationCount}.
% In the first approach, we do not trust individual witnesses, rather we \emph{assume} that it is difficult for Alice to find more than \(\theta\) witnesses willing to collude.
% Thus, the strength comes from the number of witnesses and we require at least \(\theta\) witnesses to accept a participation proof as valid.
% In the second approach, we trust specific witnesses, but no others.
% In this case, to accept a participation proof as valid, we require at least one trusted witness, the independent journalist Jane.
% It is the strength function \(\str\) of \cref{DefParticipationCount} that 
% differ in the two cases.
% We will first give the procedure and then how to construct the two different 
% strength functions.


To count or verify the participation count for a protest \(\prtst\) with 
identifier \(\cid_0\), a verifier must download the set \(\pshs_{\cid_0}\) of 
all \(s_P\) and \(s_W\) tuples containing \(\cid_0\) from the ledger~\(\TS\).
Then from \(\pshs_{\cid_0}\), a verifier can build, in succession,
\begin{enumerate*}
\item the valid proof shares \(s_j^{(i)}\) for all matching pairs \((s_P, 
    s_W)\) corresponding to a witness \(i\) and a protester \(j\),
\item the participation proof \(\prf_{j}\) for each protester \(j\),
\item the set \(\prfs_{\prtst}^{\str,\theta}\) of eligible participation proofs 
  for all protesters in \(\prtst\), and finally,
\item the participation count, \ie the cardinality of 
  \(\prfs_{\prtst}^{\str,\theta}\).
\end{enumerate*}

More precisely, given \[
  \pshs_{\cid_0} = \{ (\cid, \pid, \wid, l, t_s, t_s', t_c, \corr) \in \pshs 
  \mid \cid = \cid_0 \}
\] and a matching pair \((s_P, s_W) \in {\pshs_{\cid_0}}^2\) for a witness 
\(i\) and a protester \(j\) with
\begin{align*}
  s_P &= (\cid_0, \pid_j, \wid_i, l, t_s, t_s', t_c, \corr_i) &\text{and} \\
  s_W &= (\cid_0, \pid_j, \wid_i, l, t_s, t_s', t_c', \corr_j),
\end{align*}
%, with matching values for \(cid_0, pid_j, wid_i, l, t_s, t_s'\),
the verifier can build a valid proof share \(s_j^{(i)}\) certified by \(i\) for 
\(j\) as follows:
verify \(\corr_i\) and \(\corr_j\),
let
\begin{align*}
  t &= \interval{\max(t_s, t_s')}{\min(t_c, t_c')} &\text{and} \\
  s_j^{(i)} &= (\cid_0, \pid_j, \wid_i, l, t),
\end{align*}
as in \cref{DefProofShare},
check that \(s_j^{(i)}\) is valid (\ie happened during and at the location of 
the protest), as in \cref{DefProofShare}.

Then the set of all valid proof shares for a protester \(j\) constitutes its 
participation proof \(\prf_{j}\), as in \cref{DefParticipationProof},
and the verifier thus can derive the set of \((\str,\theta)\)-eligible participation proofs \(\prfs_{\prtst}^{\str,\theta}\) for all protesters for the protest \(\prtst\), as in \cref{DefParticipationCount}.
Finally, the  participation count \(|\prfs_{\prtst}^{\str,\theta}|\) is the cardinality of this set by  \cref{DefParticipationCount}.


% MOST PROBABLY obsolete below v 

%To verify the participation count for a protest \(\prtst\)  with identifier $\cid$
%(see \cref{DefProtest}), a verifier must download all the proof shares \[
%  \psh_i =   (\cid, \pid_j, \wid_i, t_s^{(i)}, t_s^{\prime (i)}, t_e^{(i)}, 
%  t_e^{\prime   (i)}, l_i, \corr_{\pid_j}^{(i)}, \corr_{\wid_i})
%\] for each protester \(j\) from the ledger, verify \(\corr_{\pid_j}^{(i)}\), 
%\(\corr_{\wid_i}\) and that the interval \(\interval{\max(t_s^{(i)}, 
%    t_s^{\prime (i)})}{\min(t_e^{(i)}, t_e^{\prime (i)})}\subseteq t\) and that 
%\(l_i\subseteq  l\).
%Any proof share that does not verify correctly will be discarded.
%At this point, the verifier has constructed the set \(S\) from 
%\cref{DefProofShares} and can thus construct any participation proof 
%\(\prf_{\pid_j, P}\) as in \cref{DefParticipationProof}.
%Now the verifier can compute the participation count \(|\prfs_P^{\str, 
%    \theta}|\) as in \cref{DefParticipationCount}.


%__________________________

% In the case \emph{without} trusted witnesses, all the weights equal to 1 is equivalent to counting the elements in the set, 
% \(\str[\prf_{\pid_j, P}] = |\prf_{\pid_j, P}|\).

In the case of trusted witnesses, each such trusted witness must
publish or otherwise inform the verifier of which proof shares they
have signed, \eg by giving a list of all such proof shares or
digitally signing each proof share\footnote{%
  To achieve witness privacy in this situation, one could employ a
  group or ring signature scheme for a set of potentially trusted witnesses, \eg
  members of an independent journalist association.  Then one learns
  that at least one member of this set must have
  been there.
}.

Note that, thanks to the \((\str,\theta)\)-eligibility criterion
(\cref{DefParticipationCount}), the method of counting is extremely
generic, and each (counting) verifier can make an independent choice to regulate their trust in the final result, based on their initial trust in the witnesses. In other words, anyone who does the counting can choose the eligibility
criteria (time interval, location, number of regular or trusted
witnesses, who is considered to be a trusted witness) for their own count
and as long as these are published along with the result, anyone can
verify the correctness of the count under those criteria, and potentially question the validity of this choice. Biased or partisan verifiers may be tempted to make extreme choices, but they will have to publish those choices and lose credibility. Reasonable verifiers on the other hand will try to find a good middle-ground that counts all legitimate protesters while being resistant to isolated malicious agents.

% Then the verifier can define \[
%   \str[\prf_{\pid_j, P}] = \begin{cases}
%     1 & \text{if \(\exists \psh_i\in \prf_{\pid_j, P}\) that is such a proof 
%       share} \\
%     0 & \text{otherwise}
%   \end{cases}
% \] and sets \(\theta = 1\).
