\section{\PRIVO:\@ A protocol for protest outcome verification}%
\label{Protocol}

\NewScheme{\PRIVOscheme}{PRIVO}

We now present \PRIVO, a protocol for securely and PRIvately Verifying PROtest 
Outcomes.
The protocol consists of five phases: the setup, join, participation, 
submission and verification phases.
The setup is described in \cref{ProtocolSetup}.
The join, participation and submission phases are described in 
\cref{ProtocolDuring}.
Finally, the verification phase is described in \cref{ProtocolVerification}.

We have several roles.
We have \iac{CA} who is responsible for a one-to-one mapping between a person's 
identity and a cryptographic key.
The \ac{CA} is only involved in the setup phase.
We also have the protesters who want to participate in a given protest.
A protester can assume three different roles:
\begin{enumerate*}
\item We have the \emph{organizer} who will simply write a manifesto for the 
  protest and spread to others.
\item A \emph{participant} participates in the protest (and asks witnesses to 
  vouch for his presence).
\item A \emph{witness} provides proofs to participants that state the 
  participant was indeed participating and the proofs are verifiable by third 
  parties.
\end{enumerate*}
In general, there is one organizer and every protester will act as both 
participant and witness.

\subsection{The setup phase}

The setup phase is only run once and is the same as in 
Anon-Pass~\cite{AnonPass}.
We have simply adapted their description for our notation, but otherwise tried 
to keep them as similar as possible.

\NewAlgorithm{\PRIVOsetup}{Setup}
\NewVariable{\sk}{sk}
\NewVariable{\spk}{spk}
\NewVariable{\ssk}{ssk}

\emph{Setup: \((\spk, \ssk)\gets \PRIVOsetup\).}
The setup phase is for the \ac{CA} to set up all the needed keys.
The \ac{CA} generates a service public-private key-pair \((\spk, \ssk)\) as 
follows.
Let \(G = \gen g\) be a group with generator \(g\) and prime order \(q\).
Let \(G_T\) be a group such that there is a bilinear pairing \(e\colon G\times 
  G\to G_T\)\footnote{%
  \Ie \(e(g_1^a, g_2^b) = e(g_1, g_2)^{ab}\) and if \(G = \gen{g_1} = 
    \gen{g_2}\) then \(G_T = \gen{e(g_1, g_2)}\).
}.
Choose \(x,y,z\rgets \ZZ_q\) uniformly at random and set \(X = g^x, Y = g^y, Z = 
  g^z\).
Then the service public key \(\spk = (q, G, G_T, g, X, Y, Z)\) and the service 
private key \(\ssk = (x, y, z)\).

\NewAlgorithm{\PRIVOreg}{Reg}

\emph{Registration: \(\sk\gets 
    \Proto{\PRIVOreg[_P][\spk]}{\PRIVOreg[_{CA}][\ssk]}\).}
During the registration phase, each protester will generate a secret key and 
obtain a signature on it by the \ac{CA} --- without revealing it to the \ac{CA}.
Thus each protester will have a signed secret key, the \ac{CA} will issue only 
one signature per protester but will not know who has which key.
The protester chooses \(k, r\rgets \ZZ_q\) uniformly randomly and sets \(M = g^k 
  Z^r\) and sends \(M\) to the server.
The protester acts as prover and the \ac{CA} as verifier while they perform the 
following \ac{ZKPK} protocol:
\[
  \PK[k,r][M = g^k Z^r].
\]
If the proof succeeds, the \ac{CA} chooses \(a\rgets \ZZ_q^*\) and sets \(A = 
  g^a\).
Then it forms the signature \(\sigma = (A, B = A^y, Z_B = B^z = Z^{ay}, C = A^x 
  M^{axy})\) and sends it to the protester.
The protester verifies that the following equations hold:
\begin{align*}
  A &\neq 1 \\
  e(g, B) &= e(Y, A) \\
  e(g, Z_B) &= e(Z, B) \\
  e(g, C) &= e(X, A) e(X, B)^k e(X, Z_B)^r.
\end{align*}
Upon success, the protester uses \(\sk = (\sigma, k, r)\).

The setup and registration phases can be done once and the keys reused for an 
arbitrary number of protests.
Thus it would be suitable to have the registration phase as part of the issuance 
of \eg national identity cards, passports or \acp{eID}.

\subsection{Creating, joining, participating in a protest and submitting 
  proofs}

The join, participation and submission phases are illustrated in 
\cref{fig:ProtocolOverview}.
The participation proof share that will be constructed in the participation 
phase is illustrated in \cref{fig:ProofFig}.

\emph{Creating a protest: the manifesto.}
The organizer will write a manifesto for the protest, \ie to capture the cause.
This manifesto is simply any intelligible text.
He will distribute this manifesto to people: on the Web, on placards, \etc.
If they agree, they will use it to join the protest.

\NewAlgorithm{\PRIVOjoin}{Join}
\NewVariable{\mfst}{manifesto}
\NewVariable{\cid}{cid}
\NewVariable{\pid}{pid}

\emph{Joining participant: \((\pid, t_s)\gets 
    \PRIVOjoin[_P][\text{manifesto}]\).}
A participant who wants to join the protest will use the manifesto to compute an 
identifier for the cause, \(\cid\gets \Hash[\mfst]\).
Then this identifier is used to create the participant's protest-specific 
identifier, \(\pid\gets \ACprf[_{\sk}][\cid]\) (in 
\cref{fig:ProofFig,fig:ProtocolOverview}).
More specifically, \(\pid = g_T^{1/(k+\cid)}\) (remember that \(\sk = (\sigma, 
  k, r)\)).
The participant should also get a time-correlated random value from the 
time-stamping service, \(t_s\gets \TSget\).

\NewVariable{\wid}{wid}

\begin{figure}
  \centering
  %\footnotesize
  \includegraphics{proofshare.tikz}
  \caption{%
    Structure of a proof share.
    The protest (cause) identifier \(\cid\) is the hash value of the manifesto.
    The protester \(P\)'s identifier \(\pid\) is computed using the protester's 
    key \(\sk_P\) and \(\cid\).
    The witness \(W\)'s identifier \(\wid\) is computed using the witness's key 
    \(\sk_W\) and \(\pid\).
    \(t_s, t_s'\) are the last head of the blockchain seen by the protester and 
    witness, respectively, and \(l\) is an area.
    All values are signed by the witness while also proving the correctness of 
    \(\wid\) and knowledge of a signature on \(\sk_W\).
  }%
  \label{fig:ProofFig}
\end{figure}%

\begin{figure}
  \centering
  %\footnotesize
  \begin{minipage}{\linewidth}
    \begin{align*}
      O\to \text{all}\colon & \text{manifesto} \\
      P\colon & t_s\gets \TSget \\
        & \cid\gets \Hash[\text{manifesto}], \\
        & \pid\gets \ACprf[_{\sk_P}][\cid] \\
      W\colon & t_s'\gets \TSget
      \\[-1em]
      \noalign{\hfill Join}
      \midrule
      \noalign{\hfill Participation}
      \\[-3em]
      P\to W\colon & \pid \\
      P\leftrightarrow W\colon &
        \DBPK\mleft\{ (\sk_P) : \mright. \\
        & \qquad \pid = \ACprf[_{\sk_P}][\cid], \\
        & \qquad \mleft. \sigma_P' = \ACblind[\ACsign[_{\ssk}][\sk_P]] \mright\} 
        \\
      W\colon & \wid\gets \ACprf[_{\sk_W}][\pid] \\
      W\to P\colon & (\wid, t_s', l)
      \\[-1em]
      \noalign{\hfill Participation}
      \midrule
      \noalign{\hfill Submission}
      \\[-2em]
      P\colon & t_e\gets \TSstamp[\Hash[\pid, \wid, t_s, t_s', l]] \\
      W\colon & t_e'\gets \TSstamp[\Hash[\pid, \wid, t_s, t_s', l]] \\
      W\to S\colon & (\pid, \wid, t_s, t_s', t_e, l, \pi_{\wid}),\quad 
      \text{where} \\
        & \pi_{\wid} = \SPK\mleft\{ (\sk_W) : \mright. \\
        & \qquad \wid = \ACprf[_{\sk_W}][\pid], \\
        & \qquad \mleft. \sigma_W' = \ACblind[\ACsign[_{\ssk}][\sk_W]]\mright\} 
        \\
        & \qquad\qquad (\pid, \wid, t_s, t_s', l) \\
      P\to S\colon & (\cid, \pid, \wid, t_s, t_s', t_e, l, \pi_{\pid}),\quad 
      \text{where}\\
        & \pi_{\pid} = \SPK\mleft\{ (\sk_P) : \mright. \\
        & \qquad \pid = \ACprf[_{\sk_P}][\cid], \\
        & \qquad \mleft. \sigma_P' = \ACblind[\ACsign[_{\ssk}][\sk_P]] \mright\} 
        \\
        & \qquad\qquad (\cid, \pid, \wid, t_s, t_s', l)
    \end{align*}
  \end{minipage}
  \caption{%
    An overview of of the Join, Participation and Submission phases of \PRIVO.\@
    The organizer \(O\) broadcasts the manifesto.
    The protester \(P\), witness \(W\) and their computations are as in 
    \cref{fig:ProofFig}.
    Finally, both \(P\) and \(W\) submits the proof share to some permanent 
    storage \(S\).
  }%
  \label{fig:ProtocolOverview}
\end{figure}

\emph{Joining witness: \(t_s'\gets \PRIVOjoin_W\).}
The witness should simply get a time-correlated random value from the 
time-stamping service, \(t_s'\gets \TSget\).
(We do this for redundancy, the newest of \(t_s\) and \(t_s'\) will set the 
start of the time interval of creation for the proof share.).

\NewAlgorithm{\PRIVOparticipate}{Prticip}
\NewAlgorithm{\PRIVOwitness}{Witness}

\emph{Participating: \(\pi\gets
    \Proto{\PRIVOparticipate[\cid, \sk_P]}{\PRIVOwitness[\sk_W, \spk]}\),}
In the participating phase the participant and the witness constructs the proof 
share of the participant (\cref{fig:ProofFig}).
The participant reblinds its signature \(\sigma\) on its secret key: choose 
\(r_1, r_2\rgets \ZZ_q^*\) and let \(
  \tilde\sigma = (\tilde A = A^{r_1},
                  \tilde B = B^{r_1},
                  \tilde Z_B = Z_B^{r_1},
                  \hat C = C^{r_1 r_2}).
\) (It is not necessary to compute a new \(\tilde\sigma\) more than once per 
\(\pid\).)
The participant sends \((\pid, \tilde\sigma)\) to the witness.
The witness verifies that
\begin{align*}
  \tilde A          &\neq 1 \\
  e(g, \tilde B)    &= e(Y, \tilde A) \\
  e(g, \tilde Z_B)  &= e(Z, \tilde B).
\end{align*}
Both the participant and the witness compute
\begin{align*}
  v &= e(g, \hat C) \\
  v_x &= e(X, \tilde A) \\
  v_{xy} &= e(X, \tilde B) \\
  v_{xy}' &= e(X, \tilde Z_B).
\end{align*}
Then they run the following \ac{DBPK} with the participant as the prover and the 
witness as the verifier:
\begin{multline*}
  \mleftright
  \DBPK\left\{(k, r, r') : v^{r'} = v_x v_{xy}^k v_{xy}^{\prime r} \land
    \pid = g_T^{1/(k+\cid)} \right\}.
\end{multline*}
where \(r' = 1/r_2\).
We can rewrite \(\pid = g_T^{1/(k+\cid)}\) as \(\pid^k = g_T g_T^{-\cid}\).
Then the \ac{PK} can be designed as follows (remember that this \ac{PK} should 
be run as \iac{DBPK} and be repeated multiple times to achieve security).
The participant chooses \(r_k, r_r, r_{r'}\rgets \ZZ_q\) uniformly randomly, 
sets \(R_1 = v^{r_{r'}} v_{xy}^{r_k} v_{xy}^{\prime r_r}\) and \(R_2 = 
  \pid^{r_k}\) and sends \(R_1, R_2\) to the witness.
The witness replies with a challenge \(c\).
The participant then computes
\begin{align*}
s_k &= -ck + r_k, \\
s_r &= -cr + r_r, \\
s_{r'} &= cr' + r_{r'}
\end{align*}
and sends them to the witness.
The witness checks whether
\begin{align*}
  v_x^c R_1 &\overset{\mathclap{?}}{=} v^{s_{r'}} v_{xy}^{s_k} v_{xy}^{\prime 
    s_r}, \\
  \mleft(g_T \pid^{-\cid}\mright)^{-c} R_2 &\overset{\mathclap{?}}{=} 
  \pid^{s_k}.
\end{align*}
If the proof succeeds the witness will compute \(\wid\gets 
  \ACprf[_{\sk_W}][\pid] = g_T^{1/(k_W+\pid)}\) (as above).
Then the witness returns \((\wid, t_s', l)\) to the participant.
Now both the participant and the witness have the tuple \(\pi = (\pid, \wid, 
  t_s, t_s', l)\) which constitutes a proof share of the participant 
(\cref{fig:ProofFig}).

\NewAlgorithm{\PRIVOsubmit}{Submit}

\emph{Submission: \(\pi'\gets \PRIVOsubmit[_P][\cid, \pid, \wid, t_s, t_s', 
    l]\).}
The participant should as soon as possible commit the proof-share data to the 
time-stamping service and receive the proof of commitment, \(t_e\gets 
  \TSstamp[\Hash[\pid, \wid, t_s, t_s', l]]\).
The remaining operations are not time critical.
The participant computes \iac{NIZK} proof \(\pi_{\pid}\) which shows the 
correctness of \(\pid\).
More specifically,
\begin{multline*}
  \pi_{\pid}\gets \SPK\left\{ (\sk_P) : \right. \\
    \begin{aligned}
      \pid &= \ACprf[_{\sk_P}][\cid] \quad \land \\
      \sigma_P' &= \left. \ACblind[\ACsign[_{\ssk}][\sk_P]] \right\}
    \end{aligned} \\
      (\cid, \pid, \wid, t_s, t_s', l).
\end{multline*}
Finally, the participant uploads the tuple \[
  (\cid, \pid, \wid, t_s, t_s', t_e, l, \pi_{\pid})
\] for permanent storage.
\daniel{Return \(\pi'\), as specified as output.}

\emph{Submission: \(\pi''\gets \PRIVOsubmit[_W][\pid, \wid, t_s, t_s', l]\).}
The witness should, just as the participant, commit the proof-share data to the 
time-stamping service, \(t_e'\gets \TSstamp[\Hash[\pid, \wid, t_s, t_s', l]]\).
(This is for redundancy only.)
Then, without any time requirements, the witness computes \iac{NIZK} proof 
\(\pi_{\wid}\) as follows:
\begin{multline*}
  \pi_{\wid}\gets \SPK\left\{ (\sk_W) : \right. \\
    \begin{aligned}
      \wid &= \ACprf[_{\sk_W}][\pid] \quad \land \\
      \sigma_W' &= \left. \ACblind[\ACsign[_{\ssk}][\sk_W]] \right\}
    \end{aligned} \\
      (\pid, \wid, t_s, t_s', l).
\end{multline*}
Finally, the witness uploads the tuple \[
  (\pid, \wid, t_s, t_s', t_e', l, \pi_{\wid})
\] for permanent storage.
\daniel{Return \(\pi''\), as specified as output.}

\subsection{Verifying the participation count}

There are various ways for verifying the participation count, we will cover the 
two suggested just after \cref{DefParticipationCount}.
In the first approach, we do not trust individual witnesses, instead we 
\emph{assume} that it is difficult for Alice to find more than \(\theta\) 
witnesses willing to collude.
Thus we require at least \(\theta\) witnesses to accept a participation proof 
as valid.
In the second approach, we trust certain witnesses but not others.
Thus to accept a participation proof as valid, we require at least one trusted 
witness.

\paragraph{Without trusted witnesses}

To verify the participation count for a protest \(P = (\cid, t, l)\) (see 
\cref{DefProtest}), a verifier must download all the proof shares \(\psh_i = 
  (\cid, \pid_i, \wid_i, t_s^{(i)}, t_s^{\prime (i)}, t_e^{(i)}, t_e^{\prime 
    (i)}, l_i, \prf_{\pid_i}, \prf_{\wid_i})\), from the blockchain and verify 
\(\prf_{\pid_i}\) and \(\prf_{\wid_i}\) and that the interval starting at the 
latest of \(t_s^{(i)}, t_s^{\prime (i)}\) and ending at the earliest of 
\(t_e^{(i)}, t_e^{\prime (i)}\) is a subset of \(t\) and that \(l_i\subseteq 
  l\).
Any proof share that does not verify correctly will be discarded.
At this point, the verifier has constructed the set \(S\) from 
\cref{DefProofShares} and can thus construct any participation proof 
\(\prf_{\pid_i, P}\) as in \cref{DefParticipationProof}.

Now, we simply define \(\str[\prf_{\pid_i, P}] = |\prf_{\pid_i, P}|\) and then 
the verifier can compute the participation count \(|\prfs_P^{\str, \theta}|\) 
as in \cref{DefParticipationCount}.

\paragraph{With trusted witnesses}

For the this approach, we must adapt the selection of proof shares from the 
blockchain by selecting only proof shares issued by witnesses that the verifier 
trusts.
Thus the verifier must be able to recognize proof shares issued by trusted 
witnesses, but as \(\prf_\wid\) is currently constructed, this is not possible.
So trusted witnesses must complement their proofs by a digital signature that 
the verifier can recognize.

When \(S\) only contains proof shares issued by witnesses the verifier trusts, 
we let \(\str[\prf_{\pid_i, P}] = 1\) and let \(\theta = 1\).
Then the verifier can compute the participation count \(|\prfs_P^{\str, 
    \theta}|\) as in \cref{DefParticipationCount}.
