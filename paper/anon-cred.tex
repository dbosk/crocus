\subsection{Anonymous credentials}%
\label{ZK-anon-cred}

%For an anonymous credential system (\(\ACscheme\)), it is sufficient to have a 
%commitment scheme (\(\ACcommit\)), a signature scheme and (efficient) protocols 
%for
%\begin{enumerate*}
%  \item proving equality of two committed values (\(\ACeq\));
%  \item getting a signature on a committed value (\(\ACreq, \ACissue\)); and
%  \item proving knowledge of a signature on a committed value 
%    (\(\ACprove\))~\cite{CLsignatures}.
%\end{enumerate*}

\NewCryptoScheme{\AC}{AC}

We need an anonymous credential system, \(\AC\), which provides the following algorithms and properties.

\NewAlgorithm{\ACcommit}{\AC.\!Commit}

\(\AC\) must provide a commitment scheme, \(\ACcommit\), and algorithms such that the prover can convince a verifier that he knows the commitment, which means that:
\begin{equation*}
  \PK[k, o][c = \ACcommit[k,o]].
\end{equation*}
%and that two commitments are commitments to the same value:
%\begin{multline*}
%  \mleftright
%  \PK\left\{ \left(k, o, o'\right) : \right. \\
%    \left. c = \ACcommit[k,o]\land c' = \ACcommit[k,o'] \right\}.
%\end{multline*}
% XXX Review requirements of AC.commit
We will require that the commitment scheme is \emph{perfectly hiding} and computationally binding, rather than the other way around\footnote{We are more concerned with long-term privacy, and thus we are looking for information-theoretic hiding.}.

\NewAlgorithm{\ACsign}{\AC.\!Sign}
\NewAlgorithm{\ACblind}{\AC.\!BlindSig}
\NewAlgorithm{\ACverifySig}{\AC.\!VerifySig}
\NewAlgorithm{\ACgetSig}{\AC.\!GetSig}
\NewAlgorithm{\ACissueSig}{\AC.\!IssueSig}
\NewAlgorithm{\ACproveSig}{\AC.\!ProveSig}

\(\AC\) must also provide a (blindable) signature scheme with protocols to get a signature on a committed value (\(\ACgetSig\leftrightarrow \ACissueSig\)) and to prove knowledge of a signature on a committed value (\(\ACproveSig\leftrightarrow \ACverifySig\)).

The prover commits to a value \(k\) with commitment \(c\gets \ACcommit[k, o]\) and opening \(o\).
Afterwards, he uses \(\sigma\gets \ACgetSig\leftrightarrow \ACissueSig\) to obtain a signature \(\sigma = \ACsign[_{sk}][k]\) on the value \(k\), in which \(sk\) is the signing key of the signer.

At a later point, the prover wants to prove to a verifier that he knows \(k\) and a signature \(\sigma\) on \(k\) made by the owner of \(pk\) (corresponding to \(sk\)), \ie without revealing \(k\) nor \(\sigma\).
The prover and verifier will run the protocol \(\ACproveSig\leftrightarrow 
  \ACverifySig\) to prove the following:
\begin{equation*}
  \PK[k][\sigma' = \ACblind[\ACsign[_{sk}][k]]].
\end{equation*}

\NewAlgorithm{\ACprf}{\AC.\!PRF}
\NewAlgorithm{\ACprovePRF}{\AC.\!ProvePRF}
\NewAlgorithm{\ACverifyPRF}{\AC.\!VerifyPRF}

Finally we need \iac{PRF}, \(\ACprf\), such that there exists a protocol \(\ACprovePRF\leftrightarrow \ACverifyPRF\) which implements the following 
\ac{PK}:
\begin{equation*}
  \PK[k][y = \ACprf[_k][x]].
\end{equation*}
This means that the prover can convince the verifier that \(y = \ACprf[_k][x]\) without revealing \(k\).


\emph{Instantiations of the anonymous credentials.} The \(\AC\) scheme can be instantiated using the Pedersen 
commitment~\cite{PedersenCommitment} for \(\ACcommit\), CL-signatures~\cite{CLsignatures} (or adapted CL-signatures as in~\cite{AnonPass}) for \(\ACsign, \ACblind, \ACgetSig, \ACissueSig\) and the \ac{VRF} by \citet{DY-VRF} as \(\ACprf\).
These primitives are used together in Anon-Pass~\cite{AnonPass} and by  \citet{HowToWinTheCloneWars} to form unclonable anonymous credentials for subscriptions.
However in the next subsections, we will modify the \ac{ZKPK} protocols slightly, by making them \acl{DB}.
