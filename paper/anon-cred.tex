\subsection{Anonymous credentials}%
\label{ZK-anon-cred}

%For an anonymous credential system (\(\ACscheme\)), it is sufficient to have a 
%commitment scheme (\(\ACcommit\)), a signature scheme and (efficient) protocols 
%for
%\begin{enumerate*}
%  \item proving equality of two committed values (\(\ACeq\));
%  \item getting a signature on a committed value (\(\ACreq, \ACissue\)); and
%  \item proving knowledge of a signature on a committed value 
%    (\(\ACprove\))~\cite{CLsignatures}.
%\end{enumerate*}

We need an anonymous credential system, \(\AC\), which provides the following algorithms and properties.

\(\AC\) must provide a commitment scheme, \(\ACcommit\), and algorithms such that the prover can convince a verifier that he knows the value inside a commitment, which means that:
\begin{equation*}
  \PK[k, o][c = \ACcommit[k,o]].
\end{equation*}
%and that two commitments are commitments to the same value:
%\begin{multline*}
%  \mleftright
%  \PK\left\{ \left(k, o, o'\right) : \right. \\
%    \left. c = \ACcommit[k,o]\land c' = \ACcommit[k,o'] \right\}.
%\end{multline*}
% XXX Review requirements of AC.commit
We require the commitment scheme to be \emph{perfectly hiding} and computationally binding, rather than the other way around.
Indeed, we are more concerned with long-term privacy, which means that we are looking for information-theoretic security with respect to confidentiality.
\(\ACcommit\) can be instantiated with the Pedersen commitment 
scheme~\cite{PedersenCommitment}, see \cref{ACcommitAlg}.

% figure moved to anon-cred-figures.tex

\(\AC\) must also contain a (blindable) signature scheme with the associated 
protocols enabling one to get a signature on a committed value 
(\(\Proto{\ACgetSig}{\ACissueSig}\)) and to prove knowledge of a signature on a 
committed value (\(\Proto{\ACproveSig}{\ACverifySig}\)).
This can be instantiated using CL-signatures~\cite{CLsignatures}, see 
\cref{ACsignAlg,ACacAlg}.

% figures moved to anon-cred-figures.tex

The prover commits to a value \(k\) with commitment \(c\gets \ACcommit[k, o]\) 
and opening \(o\).
Afterwards, he uses \(\sigma\gets \Proto{\ACgetSig}{\ACissueSig}\) to obtain a 
signature \(\sigma = \ACsign[\pk, \sk, k, r]\) on the value \(k\) and some 
random value \(r\).
(\(\pk\) and \(\sk\) are the public verification key and the private signing 
key, respectively.)

At a later point, the prover wants to prove to a verifier that he knows \(k\) and a signature \(\sigma\) on \(k\) made by the owner of \(pk\) (corresponding to \(sk\)) without revealing \(k\) nor \(\sigma\) (\ie, in a zero-knowledge manner).
The prover and verifier run the protocol \(\Proto{\ACproveSig}{\ACverifySig}\) 
to prove the following:
\begin{equation*}
  \PK[k, r][\sigma' = \ACblind[\ACsign[\pk, \sk, k, r]]].
\end{equation*}

Finally, we need \iac{PRF}, \(\ACprf\), such that there exists a protocol 
\(\Proto{\ACprovePRF}{\ACverifyPRF}\) implementing the following \ac{PK}:
\begin{equation*}
  \PK[k][y = \ACprf[k, x]].
\end{equation*}
This means that the prover can convince the verifier that \(y = \ACprf[k, x]\) 
without revealing \(k\).
This can be instantiated by the \textcite{DY-VRF} \ac{VRF}, see 
\cref{ACprfAlg}.

% figures moved to anon-cred-figures.tex

%\paragraph*{Possible instantiations}
%The \(\AC\) scheme can be instantiated for instance using the Pedersen commitment~\cite{PedersenCommitment} for \(\ACcommit\), CL-signatures~\cite{CLsignatures} (or adapted CL-signatures as in~\cite{AnonPass}) for \(\ACsign, \ACblind, \ACgetSig, \ACissueSig\) and the \ac{VRF} by \citet{DY-VRF} as \(\ACprf\).
%These are used together in Anon-Pass~\cite{AnonPass} and by \textcite{HowToWinTheCloneWars} to form unclonable anonymous credentials for subscriptions.
%We have adapted their mechanisms to change the basis of the authentication window (\ie epoch) from a time interval to a protest (\ie cause).
%However, we will use our \ac{DB} version of the Schnorr protocol (\cref{DBSchnorr}) for the \ac{ZKPK} protocols.
