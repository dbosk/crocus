\section{Related work}%
\label{related-work}

One protocol exists in the literature that provides \acp{PPK} with most of the desired properties described previously, ProProx~\cite{ProProx}\footnote{Note that~\cite{ProProx} uses the abbreviation PoPoK, we prefer \acs{PPK} for shorter notation.}.
ProProx is secure against a malicious verifier, but it provides \iac{PPK} protocol for quadratic residues (\ie protocols of the form \(\PPK[\alpha][a = 
  \alpha^2]\)), while in our context, we need \iac{PPK} protocol for discrete logarithms (\ie \(\PPK[\alpha][a = g^\alpha]\)) due to our choice of anonymous credentials whose desirable properties we list in \cref{ZK-anon-cred}.%, which in turn is due to their \ac{PRF} for deriving identifiers from a long-term ID with unlinkability to the original ID but with proof of validity of this derivation. 

  

% One protocol exists in the literature that provides \acp{PPK} with most of the 
% desired properties described previously, ProProx~\cite{ProProx}\footnote{Note 
%   that~\cite{ProProx} uses the abbreviation PoPoK, we prefer \acs{PPK} for 
%   shorter notation.}.
% ProProx is secure against a malicious verifier, but it provides \iac{PPK} protocol for quadratic residues (\ie protocols of the form \(\PPK[\alpha][a = 
%   \alpha^2]\)), while in our context, we need \iac{PPK} protocol for discrete 
% logarithms (\ie \(\PPK[\alpha][a = g^\alpha]\), see \cref{ZK-anon-cred} for 
% details).


