\section{Related work}

Our scheme relates to several other works: the crowd counting methods already 
described in \cref{current-crowd-counting}, location-proof systems and 
Sybil-free pseudonym systems.

\paragraph*{Location-proof systems}%
\label{db-and-lp}

In a nutshell, \iac{LP} is a digital certificate attesting that someone was at 
a particular location at a specific moment in time.
\Iac{LPS} is an architecture by which users can obtain \acp{LP} from 
neighboring witnesses (\eg trusted access points or other users) that can later 
be shown to verifiers who can check the validity of a particular 
proof~\cite{luo2010veriplace,zhu2011applaus}.
Most of the existing approaches to \acp{LP} require the prover and the 
witnesses to disclose their identities, thus raising many privacy issues such 
as the possibility of tracing the movements of users of the \ac{LPS}.
However, some \acp{LPS}, such as PROPS~\cite{PROPS}, exist that provide strong 
privacy guarantees along with the possibility of verifying the claim of the 
location.

\CROCUS shares some similarities with PROPS\@.
The main difference is that \CROCUS operates in a more adverse environment, and 
can thus not provide any guarantees using the decentralized, untrusted 
witnesses that PROPS can.
The incentives to cheat are also bigger and consequently the thresholds for 
collusion are much higher.
CROCUS must also tie the cause to the location proof, to designate the proof to 
the protest in which the protester participated.

The same can be said in comparison to Platin.io\footnote{%
  URL: \url{https://platin.io}.
}.

\paragraph*{Sybil-free pseudonym systems}

Our Sybil-free pseudonym system is very similar to that of 
\textcite{SybilFreePseudonyms}.
Both are based on the work of \textcite{HowToWinTheCloneWars}.
We make the same simplification of~\cite{HowToWinTheCloneWars} as 
\textcite{SybilFreePseudonyms}, but we also require an interactive version.
Furthermore, our \acp{PK} must be distance bounding.
