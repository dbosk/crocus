\paragraph*{Related work}%
\label{related-work}

\textcite{DistanceBounding} were the first to adapt the Schnorr
protocol to achieve distance-bounding.
Unfortunately, their approach has been shown to be vulnerable to 
\ac{DBDH}~\cite{DistanceBounding,TamarinDB}.
Furthermore, it is also vulnerable to \ac{DBTF}.

The Bussard-Bagga protocol also uses Schnorr, but slightly
differently. It generates a sesson key, encrypts the private key
with this session key. Then commits to each bit of the session key
and each bit of the ciphertext. During the fast phase the verifier requests some bits from the session key and some from the ciphertext. Terrorist fraud is based on that knowing both session key and ciphertext one can compute the private key.
The prover then opens the commits, the verifier verifies them.
The prover and verifier run a PK that ties the commits to the private key.
However, a malicious prover can reduce all the N challenges in the fast phase
to 1. This allows both distance and terrorist
fraud~\cite{Bussard-Bagga-attack}.

One protocol exists in the literature that provides \acp{PPK} with most of the desired properties described previously, ProProx~\cite{ProProx}\footnote{Note that~\cite{ProProx} uses the abbreviation PoPoK, we prefer \acs{PPK} for shorter notation.}.
ProProx is secure against a malicious verifier, but it provides \iac{PPK} 
protocol for quadratic residues (\ie protocols of the form \(
  \PPK[\alpha][a = \alpha^2]
\)), while in our context, we need \iac{PPK} protocol for discrete logarithms 
(\ie \(\PPK[\alpha][a = g^\alpha]\)).
Also, ProProx's proofs of knowledge are only for \emph{bit commitments} of the 
above form.

