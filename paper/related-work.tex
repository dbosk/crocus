\section{Related work}%
\label{related-work}

\subsection{Distance bounding}

One protocol exists in the literature that provides \acp{PPK} with most of the 
desired properties described previously, ProProx~\cite{ProProx}\footnote{Note 
  that~\cite{ProProx} uses the abbreviation PoPoK, we prefer \acs{PPK} for 
  shorter notation.}.
ProProx is secure against a malicious verifier, but it provides \iac{PPK} protocol for quadratic residues (\ie protocols of the form \(\PPK[\alpha][a = 
  \alpha^2]\)), while in our context, we need \iac{PPK} protocol for discrete 
logarithms (\ie \(\PPK[\alpha][a = g^\alpha]\), see \cref{ZK-anon-cred} for 
details).
To realize this, we will now introduce such a protocol to enable us to have 
\ac{DB} anonymous credentials.

\subsection{Crowd counting methods}
The seemingly most commonly used method for counting crowds at protests is \emph{Jacobs's method}~\cite{2016DemonstrationsInSeoul,BBCHowToCountProtestNumbers,HowWillWeKnowTrumpInauguralCrowdSize,TheXManMarch,TheCrowdNumbersGame}.
This manual method devised in the 1960s relies on aerial pictures of the event.
The verifier divides the protest venue into regions and then estimates the density of the crowd in the different regions before summing them up to get an estimate of the global count.
Clearly, this method is prone to errors as it is based on estimates.
It provides universal verifiability (\cref{UniversalVerif}) since anyone can redo the counting using the same pictures.
However, it is difficult to achieve individual verifiability (\cref{IndividualVerif}) using this method since it is hard to verify that oneself is indeed in any of the photos\footnote{It is of course possible with very detailed photos, but not currently realistic to do with aerial pictures.}.
We can argue that it also achieves eligibility (\cref{EligibilityVerif}) if all the photos are taken at the same point in time with no overlap, since then no one can be counted twice.
Unfortunately, we can only get the peak participation with this method.
When it comes to spatial and temporal eligibility, we have to rely on pictures (\eg by determining from the pictures themselves when and where the they were taken).
However, this is prone to cheating --- and subject to a variety of manipulation techniques --- and the best we can do is to employ forensic methods to try to detect any modifications or other attempts of fraud.
From a privacy perspective, all techniques based on photo or video material are necessarily not privacy-preserving unless the protesters take actions to protect themselves.
For instance, there is a risk that participants are recognizable from the material if they do not wear masks.

Among more recent methods, there is an application called CrowdSize~\cite{CrowdSize} that can estimate the size of a crowd on a selected area.
Once the user has specified the zone, he selects one of three pre-set density estimates: light, medium or dense.
This is similar to the method usually used by police (\eg during the protests in Seoul):
\blockcquote{2016DemonstrationsInSeoul}{%
  \textins*{p}olice presume\textins*{d} that, when sitting, six people would fill a space of 3.3 square meters
  \textelp{}
  The same area would hold nine or 10 people when standing%
}.
This method is not verifiable unless the user takes a picture of the crowd, in which case it inherits the verifiability properties of the previous method.

In the computer vision community, there is a body of work on estimating the number of persons in a picture (\eg the work of \citet{NNCrowdCounting}).
This class of methods requires pictures or video surveillance of the protest location during the entire protest and are thus highly privacy-invasive.
They are generally based on machine learning techniques, and thus also require a training dataset to work.
In the work by \cite{NNCrowdCounting}, they actually train and evaluate their algorithm on different scenes, which might make this method easier to use for protesting.
Universal verifiability (\cref{UniversalVerif}) can be provided with this method, since someone can always recount the participants using the recorded video material.
Some degree of individual verifiability could also be provided (\cref{IndividualVerif}), if Alice can recognize her own face in the video, but still this might be difficult. 
Remark also that while automatic face recognition can help verifiability, it has a high impact on privacy.
In addition, it is difficult (if not impossible) to avoid to count some people twice, thus we cannot argue for eligibility (\cref{EligibilityVerif}).
In theory, this class of methods will provide an upper bound, whereas our method outputs a lower bound.
Furthermore, it is also difficult to capture the entire location on video surveillance, which diminishes the argument for an upper bound in practice.
The spatial and temporal verifiability properties are reduced to those of the video, which should be similar as in our discussion about the verifiability of photos (\ie, using forensic methods).

Another problem for all the above methods is exemplified by the demonstrations in Seoul:
\blockcquote{2016DemonstrationsInSeoul}{%
  \textins*{t}he demonstrators not only gather in open space, but also small alleys and between buildings%
}.
In this situation it is very difficult to faithfully capture the situation.
Indeed, even if the scene is taken from different angles, there is the additional problem of not counting people twice.

During the protests in Seoul~\cite{2016DemonstrationsInSeoul}, one physical analytics company tried to estimate the number of participants using their physical analytics technology.
In a nutshell, this technology scans the MAC addresses emitted from the participants' smartphones during Wi-Fi probe requests.
However, in order to work this method makes many additional assumptions:
\blockcquote{2016DemonstrationsInSeoul}{%
The company presumed that about half of smartphone users usually leave on their Wi-Fi feature on and the other half switch it off, based on a separate survey on smartphone usage. 
It also assumed that about 20\% of the smartphone signals were repetition from the same device.
}
This method cannot provide any verifiability, as we must trust the company on performing the measurements correctly and to be honest about when and where they conducted them.
In addition, as MAC address randomization is now getting a wider adoption precisely to fight physical tracking, this method will not work any more, although some tracking of smartphones could still be possible~\cite{WhyMACRandomizationIsNotEnough}.

A better proposal would be to use \emph{IMSI catchers} (or the real cell towers of the mobile network) to count unique phones at a location.
However, there are several problems with this approach.
First, it will be difficult to register only the participants' phones, as many bystanders will also be counted.
Thus, it will not be possible to distinguish clearly between the protesters and the counter-protesters.
Second, since the phone has a unique identifier, participants might be uncomfortable to be registered in association with the event and might thus turn the device off, even though, at least with 5G, the IMSI number will not be transmitted anymore in plaintext. %need to cite?
Finally, there will be limited verifiability, as the data recorder must be trusted to record all data in relation to the protest.

An approach that relies on a trusted infrastructure was recently deployed by a collection of media outlets to count protesters passing the line defined by a trusted sensor on marches~\cite{LeMondeProtestingSolution}. 
This solution does not offer strong verifiability guarantees and thus is complemented by micro-counts made by humans to estimate their margin of error.

In general for all of the above methods, the more a crowd spreads out, the more difficult it is to determine its size.
In particular, one of the challenges is determining whether people near the event's perimeter are participants or simply bystanders~\cite{HowToEstimateCrowdSize}.
These methods also have difficulties capturing the actual attendance of an event (\ie, the cumulative participation, not just the count at a snapshot around the peak).

One of the most closely related work to our scheme is \citet{CrowdCount}, which is a web service that lets Alice create an event such that anyone can submit his location to register that he is in Alice's event.
This method has the benefit of counting everyone who has declared his presence, not just the count at the snapshot of the pictures.
However, there is no verification as the service must be trusted to behave honestly, but even then, nothing prevents Bob from submitting more than once (violating eligibility, \cref{EligibilityVerif}).
Another downside is that the service also requires an Internet connection during the event to register as a participant.
This makes it prone to some form of denial-of-service attack.
For instance, if Alice has organized a protest against Grace's regime, who shuts down the cellular network or Internet backbone as a means to censor the protest.

Another related approach based on devices is UrbanCount~\cite{UrbanCount}, which relies on epidemic spreading of crowd-size estimates by device-to-device communication to count crowds in dense urban environments with high node-mobility and churn.
However, there is no consideration of a potentially adversarial setting and thus no verification or checks on eligibility.  DiVote~\cite{DiVote}, a prior work by the same authors for polling in dense areas, avoids double counting, but again only works with honest participants.



\subsection{Location proofs}
%\subsection{Location proofs}
\section{Distance bounding and location proofs}
\label{db-and-lp}
\Ac{DB} protocols were first suggested by \citet{DistanceBounding} to prevent relay attacks in contactless communications in which the adversary forwards a communication between a prover and a possibly far-away verifier to authenticate. 
These attacks cannot be prevented by cryptographic means as they are independent of the semantics of the messages exchanged.
As a consequence, mechanisms ensuring the physical proximity between a verifier and a prover should be used instead.
\Ac{DB} protocols enable the verifier to estimate an upper bound on their distance to the prover by measuring the time-of-flight of short challenge-response messages (or rounds) exchanged during time-critical phases. 
%Time critical phases are complemented by slow phases during which the time is not taking into account. 
Slow phases, during which the time is not taken into account, complement the time critical phases.
At the end of a \Ac{DB} protocol, the verifier should be able to determine if the prover is legitimate \emph{and} in their vicinity.
In this sense, \Ac{DB} protocols combine the classical properties of authentication protocols with the possibility of verifying the physical proximity.

The main attacks against \ac{DB} protocols can be summarized as follows:
\begin{itemize}
  \item \Acf{DBDF}: a legitimate but malicious prover wants to fool the verifier on the distance between them.
  \item \Acf{DBMF}: the adversary illegitimately authenticates, possibly using an honest prover far away from the verifier.
  \item \Acf{DBTF}: a legitimate but malicious prover helps an accomplice close to the verifier to authenticate.
  \item \Acf{DBDH}: similar to \ac{DBDF}, the malicious prover is far away but uses an unsuspecting honest prover close to the verifier to pass as being close.
\end{itemize}
There are two lines of attempts at formalizing the above properties: one by \citet{DB-BMV} and another by \citet{DB-DFKO}.

The majority of the existing \ac{DB} protocols are symmetric and thus require an honest verifier.
Indeed, in this context it does not make sense to protect against the verifier as they can easily impersonate the prover as they have knowledge of their secret key.
There has been less work done in the domain of asymmetric (or public-key) \ac{DB} protocols.

Some location-based services only grant access to resources to users located at a particular location, thus raising the issue of verifying the position claimed by a particular user. 
In most of the existing schemes, the location of a user (or their device) is determined by the device itself (\eg through GPS) and forwarded to the location-based service provider. 
One of the main drawbacks of this approach is that a user can cheat by having their device transmit a false location. 
Therefore, it is possible for a user to be inappropriately granted access to a particular resource while being thousands of kilometers away.

One possible way to counter this threat is by having the requesting device formally prove that it really is at the claimed location, which gives rise to the concept of \emph{location proof}. 
In a nutshell, a location proof is a digital certificate attesting that someone was at a particular location at a specific moment in time. 
A location proof system is an architecture by which users can obtain location proofs from neighboring witnesses (\eg trusted access points or other users) that can later be shown to verifiers who can check the validity of a particular proof \cite{luo2010veriplace, zhu2011applaus}.
Most of the existing approaches to location proofs require the prover and the witnesses to disclose their identities, thus raising many privacy issues such as the possibility of tracing the movements of users of the location-proof architecture.
However, some location proofs systems, such as PROPS~\cite{PROPS}, exist that provide strong privacy guarantees along with the possibility of verifying the claim of the location.
\CROCUS shares some similarities with PROPS, although their objective is quite different as it aims at verifying a global property of the population (\ie crowd estimation) in contrast to checking the location claim made by a user, which is an individual property.
\sonja{any more differences to ours, relevance of this one?}
\simon{Yeah, there is a problem here. As it currently reads, I see no reason to not use PROPS directly as a building block for CROCUS, instead of devising our own distance-bounding. Maybe there is no authentication, it's ONLY a proof of location? I don't know, but some detail is missing.}

\sonja{add something on platin.io, details unknown but roughly relying on witnesses and graph theory (unique big cluster, assumption of honest majority)}


