\subsection{Individual and universal verifiability}%
\label{analysis-individual}%
\label{analysis-universal}

\Cref{IndividualVerif} requires that Alice and Bob, as participants, can verify 
that their participation proofs (\ie proof shares) are indeed included in the 
computed count.
All proof shares (\ie \(\prf_{\pid, \prtst} = (\cid, \pid, \wid, t_s, t_s', 
  t_e, t_e', l, \pi_\pid, \pi_{\wid})\)) are committed to the ledger and 
available from a public and permanent storage (see 
\cref{fig:ProtocolOverview}).
Thus, Alice and Bob can simply check that all of their proof shares are indeed 
there and the security of individual verifiability depends on the properties of 
the ledger and storage (\(\TS\), \cref{timestamp}).
(As these are the main properties of distributed ledgers, we do not elaborate 
further.)

\Cref{UniversalVerif} includes that anyone can check the result and that all 
participation proofs counted are legitimate.
As the proof shares are committed and stored publicly, anyone can
download them, verify eligibility (\ie verify \(\pi_\pid,
\pi_{\wid}\)) of the proofs and count them.
As for individual verifiability, the security of universal verifiability is 
reduced to the properties of the ledger and storage; but universal 
verifiability also depends on the eligibility verification, which we will 
discuss next.

\subsection{Eligibility verifiability}%
\label{analysis-eligibility}

\Cref{EligibilityVerif} states that anyone must be able to determine the authenticity of the relevant attributes of the data.
In \CROCUS, we have several attributes that must be verifiable: the time of 
creation (\ie, temporal eligibility, \cref{TemporallyRelated}), the physical 
location of \(\sk_P\) at creation (\ie, spatial eligibility, 
\cref{SpatiallyRelated}), recognition of two proofs originating from the same 
person (\ie, one-proof-per-person eligibility, \cref{CountOnce}) and that the 
proof is indeed designated for the event (\ie, designated-event eligibility, 
\cref{DesignatedEvent}).
We will now analyze these properties.

\paragraph{Temporal eligibility}%
\label{analysis-temporal}

\Cref{TemporallyRelated} ensures freshness, as Alice cannot simply resubmit an 
old proof as a new one or create a proof in advance.

In general, to prevent replays, Alice must respond to an unpredictable 
challenge.
The challenge here is the hash value of the head block of the ledger at the 
time of the proof's creation (\(\TSget\)).
The response is simply the challenge itself and is included as \(t_s\) and 
\(t_s'\) in the proof share (see \cref{fig:ProofFig}).

As it is the case in the construction of ledgers, the hash value of the head 
block depends on the previous blocks.
Thus, the predictability of \(t_s\) depends on the predictability of the 
blocks, which depends on the predictability of the transactions.
Assume that Alice can control all transactions going into a block.
This means that she must find a pre-image of the form of a valid block that is 
accepted by all nodes in the distributed ledger.
\Ie it depends on the pre-image resistance of the hash function used.
Thus, with a collision resistant hash function, Alice can predict \(t_s\) only 
with negligible probability; and it follows that, with high probability, 
\(t_s\) was published on the ledger before being included in her proof share.

The correctness of the challenge must be verifiable by any verifier 
(\(\TStime\)).
This can be done if the ledger is continuously extended at a fixed rate, \eg 
one new block every 10 minutes, as was the case for Bitcoin.

According to \cref{TemporallyRelated}, we must also prove that a proof share has not been created after a certain time.
Otherwise, Grace could argue that the proof share was created after the protest, thus defeating the purpose of our protocol.
The hash values of the proof shares are committed to the ledger, which means 
that there is a negligible probability that they were created after that.
Alice is in the same situation as above:
she would have to choose a value \(y\) in the domain of the hash function 
\(\Hash\) and then find a pre-image \(x\) such that \(y = \Hash[x]\) and \(x\) 
is a valid proof for the desired protest, at the desired time.
If \(\Hash\) is collision resistant, finding \emph{any} pre-image is hard.

\paragraph{Linkability}%
\label{analysis-linkability}

\Cref{CountOnce} is required to prevent Sybil attacks, in the sense
that Alice cannot provide two (or more) participation proofs for a
specific protest and thus be counted more than once.
This possibility is prevented by the use of \(\pid\).
Indeed to be counted twice, Alice must produce a \(\pid'\neq \pid\).
Due to the deterministic property of \(\ACprf\), Alice must produce a new key 
\(\sk_P'\) such that the verifier\footnote{%
  Here the verifier is either the witness during the distance bounding or the 
  verifier who tries to verify the count.
} accepts the proof
\begin{multline*}
\PK\mleft\{ (\sk_P') : \pid' = \ACprf[_{\sk_P'}][\cid] \quad \land \mright. \\
    \sigma_P'' = \mleft. \ACblind[\ACsign[_{\ssk}][\sk_P']] \mright\}
\end{multline*}
while she does not know a valid signature on \(\sk_P'\).
As a consequence, this is reduced to the security of the \(\AC\) scheme and how 
often Alice can get a valid signature on a secret key from the \ac{CA} --- 
either by requesting a new key, \eg request a new passport or ID-card, or 
illicitly use someone else's.

\paragraph{Spatial eligibility}%
\label{analysis-spatial}

\Cref{SpatiallyRelated} is achieved by having a witness vouch anonymously that Alice was indeed on the location when the proof share was created.
To realize this, we must ensure that Alice cannot forge witness signatures, 
which is realized by using the same linkability mechanism as above.
To forge a witness signature, Alice must produce a \(\wid'\) such that
\begin{multline*}
  \pi_{\wid'}\gets \SPK\left\{ (\sk_W') : \right. \\
    \begin{aligned}
      \wid' &= \ACprf[_{\sk_W'}][\pid] \quad \land \\
      \sigma_W'' &= \left. \ACblind[\ACsign[_{\ssk}][\sk_W']] \right\}
    \end{aligned} \\
      (\pid, \wid', t_s, t_s', l),
\end{multline*}
is accepted by the verifier while not knowing a signature on the secret key 
\(\sk_W'\).
Thus, she must break the \(\AC\) scheme to succeed or get a valid signature on \(\sk_W'\) from the \ac{CA}.
As above, the latter could be done if she can find people willing to collude, 
people unwilling to collude but she can illicitly use their keys anyway or 
simply request new keys from the \ac{CA}.

However, due to the assumptions made in \cref{ProtocolVerification}, we either assume that we can trust (dedicated) witnesses or that Alice cannot collude with more than a threshold \(\theta\) of witnesses (of course these two approaches could be combined).
Note that Alice can produce \emph{one} witness signature for herself, since she 
has access to her own \(\sk_P\), but no more.
Thus, \(\theta > 1\) is a requirement in the trust-free case.

\paragraph{Designated protest}%
\label{analysis-designated}

\Cref{DesignatedEvent} is to prevent Alice (or someone else) from reusing the same proof (or proof share) for another event.
This possibility is prevented through the use of \(\cid\) in the proof shares.
To reuse the proof share for another protest, with a different manifesto, one 
must find a second pre-image \(\mfst'\) such that \(\cid = \Hash[\mfst] = 
  \Hash[\mfst']\).

There exists another case of collision that we must prevent.
Consider the situation in which Alice computes \(\pid = \ACprf[_\sk][\cid]\) for some cause identifier \(\cid\) and some witnesses computes \(\wid_1, \dotsc, \wid_n\), with \(\wid_i = \ACprf[_{\sk_i}][\pid]\).
Now, if Alice constructs a manifesto \(m\) such that \(\Hash[m] = \pid\), then \(\wid_1, \dotsc, \wid_n\) would be valid participant identifiers for the protest with manifesto \(m\).
This means that we must separate the mechanisms used for participating and witnessing.
The protocol achieves this by the fact that \[
  \prf_{\wid} = \SPK[\sk_i][\dotsc][\pid, \wid, t_s, t_s', l]
\]
does not include \(\cid\) whereas \[
  \prf_{\pid} = \SPK[\sk_i][\dotsc][\cid, \pid, \wid, t_s, t_s', l]
\]
does.
Thus, the verification process differs for the two types of proofs.
(An alternative approach would have been to compute them differently, for 
instance \(\pid = \AC[PRF]_{\sk}(x)\) but \(\wid = \AC*[PRF]_{\sk}(x)\), where 
\(\AC \neq \AC*\) are separate anonymous-credential systems.)

%\paragraph{Forging proof shares}
%
%\daniel{This is work-in-progress.}
%Alice can attempt to forge the full proof directly, \ie
%\begin{multline*}
%\pi_{\pid'} = \SPK\mleft\{ (\sk_P') : \mright. \\
%  \begin{aligned}
%    \pid' &= \ACprf[_{\sk_P'}][\cid], \\
%    \sigma_P'' &= \mleft. \ACblind[\ACsign[_{\ssk}][\sk_P']] \mright\}
%  \end{aligned} \\
%    (\cid, \pid', \wid', t_s, t_s', l)
%\end{multline*}
%and consequently
%and thus solve same problem twice (once for \(pid'\) and once for \(\wid'\)).
%

