
\subsection{Eligibility verifiability}%
\label{analysis-eligibility}

\Cref{EligibilityVerif} states that anyone must be able to determine the authenticity of the relevant attributes of the data.
In \CROCUS, we have several attributes that must be verifiable: the time of 
creation (\ie, temporal eligibility, \cref{TemporallyRelated}), the physical 
location of \(\sk_P\) at creation (\ie, spatial eligibility, 
\cref{SpatiallyRelated}), recognition of two proofs originating from the same 
person (\ie, one-proof-per-person eligibility, \cref{CountOnce}) and that the 
proof is indeed designated for the event (\ie, designated-event eligibility, 
\cref{DesignatedEvent}).

As we will show, it follows from \cref{analysis-individual} that the adversary 
cannot drop submitted proof shares and thus cannot decrease the count.
As indicated in the adversary model (\cref{formal-adversary-model}), the 
adversary can submit proof shares as everyone else, so the adversary's only 
option is to increase the count.
We will thus let Alice pose as the adversary (malicious protester) in this 
section, as she naturally has an incentive to increase the count, as the 
organizer and a participant.

\subsubsection{Temporal eligibility}%
\label{analysis-temporal}

\Cref{TemporallyRelated} ensures freshness, as Alice cannot simply resubmit an 
old proof as a new one or create a proof in advance.

In general, to prevent replays, Alice must respond to an unpredictable 
challenge.
The challenge here is \enquote{what did \(\TSget\) return at the time of the 
  proof's creation?}.
The response is included as \(t_s\) and \(t_s'\) in the proof share (see 
\cref{fig:ProofFig}).
The unpredictability of \(\TSget\) ensures that a proof cannot have been 
created before \(\max\{t_s, t_s'\}\).
The correctness of the response must be verifiable by any verifier, which is 
the case with \(\TStime\).

According to \cref{TemporallyRelated}, we must also prove that a proof share has not been created after a certain time.
Otherwise, Grace could argue that the proof share was created after the protest, thus defeating the purpose of our protocol.
The hash values of the proof shares are committed to the ledger 
(\(\TSsubmit\)), which means that there is a negligible probability that they 
were created after that:
Alice would have to choose a value \(y\) in the range of the hash function 
\(\Hash\) and then find a pre-image \(x\) such that \(y = \Hash[x]\) and \(x\) 
is a valid proof for the desired protest, at the desired time.
If \(\Hash\) is collision resistant, she will succeed with negligible 
probability.

\subsubsection{Counted only once}%
\label{analysis-linkability}

\Cref{CountOnce} aims to prevent Sybil attacks, in the sense
that Alice cannot provide two (or more) participation proofs for a
specific protest and thus be counted more than once.
To do this she must create more than one pseudonym, \(\pid\).
Indeed, to be counted twice, Alice must produce a \(\pid'\neq \pid\).
Due to the deterministic property of \(\ACprf\), Alice must produce a new key 
\(\sk_P'\) such that the verifier\footnote{%
  Here the verifier is either the witness during the distance bounding or the 
  verifier who tries to verify the count.
} accepts the proof
\begin{multline*}
\PK\mleft\{ (\sk_P') : \pid' = \ACprf[_{\sk_P'}][\cid] \quad \land \mright. \\
    \sigma_P'' = \mleft. \ACblind[\ACsign[_{\ssk}][\sk_P']] \mright\}
\end{multline*}
while she does not know a valid signature on \(\sk_P'\).
As a consequence, this is reduced to the security of the \(\AC\) scheme.
Remember, by assumption the \ac{CA} will issue only one signature for such a 
key, so Alice cannot ask for a second one.

\subsubsection{Spatial eligibility}%
\label{analysis-spatial}

\Cref{SpatiallyRelated} is achieved by having a witness vouch that Alice was 
indeed at the location when the proof share was created.
In essence, the witness performs distance bounding to ensure Alice is close to 
them, this is then propagated to the verifier through the signature.
This leaves Alice with three options to cheat:
\begin{enumerate*}
\item\label{spatial-DF} relay her communication with an honest witness through 
  a conspirator,
\item\label{spatial-forge} forge a witness signature for the proof,
\item corrupt a witness to issue a proof although neither might be present at 
  the location.
\end{enumerate*}

Relaying the communication is an attack against the \ac{DB} protocol.
Due to the construction of \textcite{DB-Schnorr}, Alice cannot succeed with 
more than negligible probability.

Forging a witness's signature on a proof is equivalent to breaking the 
counted-only-once property 
above (\cref{analysis-linkability}).
As above, this is reduced to the security of the \(\AC\) scheme.

Finally, Alice can corrupt witnesses.
In the case of trusted witnesses, Alice's chance of corrupting witnesses is 
reduced to the trustworthiness of the witnesses chosen by the verifier.
In the \(\theta\)-threshold case, with unknown witnesses, 
%(by assumption) 
Alice succeeds only if she can corrupt at least \(\theta\) witnesses.

We note that the strength function from \cref{DefParticipationCount} allows the 
verifier to take different approaches, each of which must be individually 
analyzed.
In all of these cases, it is up to the verifier to perform a risk analysis.

\subsubsection{Designated use}%
\label{analysis-designated}

\Cref{DesignatedEvent} aims to prevent Alice (or someone else) from reusing the 
same proof (or proof share) for another event.
This possibility is prevented through the use of \(\cid\) in the proof shares.
To reuse the proof share for another protest, with a different manifesto, one 
must find a second pre-image \(\mfst'\) such that \(\cid = \Hash[\mfst] = 
  \Hash[\mfst']\).

There exists another case of collision that we must prevent.
Consider the situation in which Alice computes \(\pid = \ACprf[_{\sk}][\cid]\) 
for some cause identifier \(\cid\) and some witnesses computes \(\wid_1, 
\dotsc, \wid_n\), with \(\wid_i = \ACprf[_{\sk_i}][\pid]\).
Now, if Alice constructs a manifesto \(m\) such that \(\Hash[m] = \pid\), then \(\wid_1, \dotsc, \wid_n\) would be valid participant identifiers for the protest with manifesto \(m\).
The protocol prevents such use by the fact that \(\pid\) and \(\wid\) are in 
fixed positions in both \[
  \prf_{\wid} = \SPK[\sk_i][\dotsc][\cid, \pid, \wid, t_s, t_s', l]
\] and \[
  \prf_{\pid} = \SPK[\sk_i][\dotsc][\cid, \pid, \wid, t_s, t_s', l],
\] and the two proofs can thus not be confused.
Thus, the verification process differs for the two types of proofs.

\subsection{Individual and universal verifiability}%
\label{analysis-individual}%
\label{analysis-universal}

\Cref{IndividualVerif} means that Alice and Bob, as participants, can verify 
that their participation proofs (\ie proof shares) are indeed included in the 
computed count.
All proof shares (\ie \(\prf_{\pid, \prtst} = (\cid, \pid, \wid, t_s, t_s', 
t_e, t_e', l, \corr_{\pid}, \corr_{\wid})\)) are committed to the 
ledger~\(\TS\) and available from a public and permanent storage.
Thus, Alice and Bob can simply check that all of their proof shares are indeed 
there and the security of individual verifiability depends on the properties of 
the ledger (\cf \cref{timestamp}).

We assumed an honest-but-curious adversary controlling \(\TS\)\footnote{%
  We note that, in general, distributed (decentralized) ledgers cannot 
  withstand a malicious \ac{ISP}.
  Such an adversary can partition the network and provide Alice and Bob with 
  different views of the ledger, thus breaking individual verifiability.
  However, this requires that the adversary \emph{can observe} Alice's and 
  Bob's channels to the ledger.
}.
This means that Alice can check that her proof share is indeed there.

\Cref{UniversalVerif} implies that anyone can check the result and that all 
participation proofs counted are legitimate.
As the proof shares are committed and stored publicly, anyone can
download them, verify eligibility (\ie verify \(\corr_{\pid},
\corr_{\wid}\)) of the proofs and count them.
Like for individual verifiability, the security of universal verifiability is 
reduced to the properties of the ledger; but universal verifiability also 
depends on the eligibility verification (each proof must be verified as 
eligible to count).
