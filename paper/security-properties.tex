\subsection{Verifiability and privacy requirements}%
\label{verifiability-properties}

We now try to make the properties from \cref{desired-properties} more 
specific.
We define three verifiability requirements, among which eligibility can be 
further broken up into four subproperties:
\begin{requirements}[V]
  \item\label{EligibilityVerif} \emph{Eligibility}: anyone can verify that each participation proof provides temporal and spatial eligibility and that only one participation proof is counted per individual.
    \begin{requirements}

    \item\label{TemporallyRelated} \emph{Temporal eligibility}: demonstrate that the proof was created after the start of the protest and before the end of the protest.
      \Ie \(t\subseteq t_\prtst\) in \cref{ValidProofShare}.

    \item\label{SpatiallyRelated} \emph{Spatial eligibility}: demonstrate that the proof is spatially related to the physical location or journey of the protest.
      \Ie \(l\subseteq l_\prtst\) in \cref{ValidProofShare}.

    \item\label{CountOnce} \emph{Counted only once}:
      A protester can create \emph{one and only one} pseudonym (\(\pid\) in 
      \cref{DefProofShares}) per protest (\(\cid\) in \cref{DefProofShares}), 
      this pseudonym is unique except with negligible probability.
      Analogously, a witness can create \emph{one and only one} pseudonym per 
      protester (\(\wid\) in \cref{DefProofShares}), this pseudonym is unique 
      except with negligible probability.

    \item\label{DesignatedEvent} \emph{Designated event}: prove that the proof 
      is designated for the particular protest.
      \Ie designated to \(\cid\) in \cref{ValidProofShare}.

    \end{requirements}

  \item\label{UniversalVerif} \emph{Universal verifiability}: anyone can verify that the result obtained matches the submitted participation proofs.

  \item\label{IndividualVerif} \emph{Individual verifiability}: each participant can verify that their participation proof is included in the global count.
\end{requirements}

For privacy, we require a set of unlinkability properties:%
\label{privacy-properties}
\begin{requirements}[P]
\item\label{PseudonymUnlink} \emph{Pseudonym unlinkability}: given a protest 
  (identifier \cid), protesters Alice and Bob, and a pseudonym \(\pid_b\), the 
  adversary cannot tell if \(\pid_b = \pid_{\text{Alice}}\) or \(\pid_b = 
    \pid_{\text{Bob}}\), except with negligible probability. And similarly with $wid$ if Alice and Bob act as witnesses.
\item\label{ProtestUnlink} \emph{Protest unlinkability}: protesters' pseudonyms 
  (\(\pid_\cid, \pid_{\cid'}\)) must be unlinkable between protests (\(\cid, 
    \cid'\)) from the adversary's perspective.
\item\label{WitnessUnlink}\emph{Witness unlinkability}:  witnesses' pseudonyms 
  (\(\wid_\pid, \wid_{\pid'}\)) must be unlinkable between protesters (\(\pid, 
    \pid'\)) from the adversary's perspective.
\end{requirements}

What these properties say is that pseudonyms must look random 
(\cref{PseudonymUnlink}) and that each psdeudonym must not be reused
more than strictly necessary for the verifiability properties 
(\cref{ProtestUnlink,WitnessUnlink}). In Pfitzmann-Hansen
terminology~\cite{pfitzmann-hansen}, they are role-relationship
pseudonyms, but with increasingly narrowed notions of roles and
relationships (participant of particular protest, witness of a
particular participant at a particular protest).
In the terminology of \textcite{SybilFreePseudonyms}; for each protester a 
protest is a context or identity domain, whereas for each witness every 
protester is a context (or identity domain).
