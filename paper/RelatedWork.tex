\section{Related work}%
\label{RelatedWork}

The seemingly most commonly used method for counting crowds at protests is 
Jacob's method~\cite[c.f.][]{%
  2016DemonstrationsInSeoul,%
  BBCHowToCountProtestNumbers,%
  HowWillWeKnowTrumpInauguralCrowdSize,%
  TheXManMarch,%
  TheCrowdNumbersGame,%
}.
It is a manual method devised in the 1960s which relies on aerial photos of the 
event.
The verifier divides the protest venue into regions, then estimates the crowd 
density in the different regions and finally sums them up to get an estimate of 
the total.
This method is prone to errors, since it is based on estimates.
However, it allows for verification, since anyone can redo the counting using 
the same photos.
This provides universal verifiability (\cref{UniversalVerif}).
It is difficult to achieve individual verifiability (\cref{IndividualVerif}) 
since it is hard to verify that oneself is indeed in any of the photos.
(It is of course possible with very detailed photos, but not currently realistic 
to do with aerial photos.)
We can argue eligibility verifiability (\cref{EligibilityVerif}) if all the 
photos are taken at the same point in time with no overlap, since then no one 
can be counted twice.
Unfortunately, we can only get the peak participation with this method.

When it comes to spatial and temporal verifiability, we have to rely on the 
photos.
We can try to determine from the photos themselves when and where the they were 
taken.
However, the best we can do is to employ forensic methods to try to detect any 
modifications or other attempts at fraud.

There is an app CrowdSize~\cite{CrowdSize} which estimates the size of a crowd 
based on a selected space.
Then the user selects one of three pre-set density estimates: light, medium or 
dense.
This is similar to the method usually used by police, \eg for the protests in
Seoul
\blockcquote{2016DemonstrationsInSeoul}{%
  \textins*{p}olice presume\textins*{d} that, when sitting, six people would 
  fill a space of 
  3.3 square meters
  \textelp{}
  The same area would hold nine or 10 people when standing%
}.
This method is not verifiable unless the user takes a picture of the crowd, 
then it inherits the verifiability properties of the previous method.

There is a body of work in the computer vision community, \eg the work of 
\citet{NNCrowdCounting}.
This class of methods requires photos or video surveillance of the protest 
location during the entire protest.
They generally include machine learning, and thus also require a training 
dataset to train the algorithms.
In the work by \citet{NNCrowdCounting}, they actually train and evaluate their 
algorithm on different scenes, which might make this method easier to use for 
protesting.
We can provide universal verifiability (\cref{UniversalVerif}) with this 
method, since someone can always recount the participants using the recorded 
video material.
We might be able to provide some degree of individual verifiability 
(\cref{IndividualVerif}), if Alice can recognize her own face in the video, but 
this might still be difficult. Automatic face recognition can help
verifiability but at the price of privacy.
Also, it is difficult (if not impossible) to not count people twice, thus we 
cannot argue for eligibility verifiability (\cref{EligibilityVerif}).
In theory, this class of methods will provide an upper bound, whereas our method 
can provide a lower bound.
However, it is still difficult to capture the entire location on surveillance 
video, which diminishes the argument for an upper bound.
The spatial and temporal verifiability properties are reduced to those of the 
video, which should be similar as in our discussion about the verifiability of 
photos above, \ie forensic methods.

Another problem for all the above methods is exemplified by the demonstrations 
in Seoul:
\blockcquote{2016DemonstrationsInSeoul}{%
  \textins*{t}he demonstrators not only gather in open space, but also small 
  alleys and between buildings%
}.
This is difficult to fully capture.
If photo material is taken from angles we will additionally have problems not 
counting people twice.

During the protests in Seoul~\cite{2016DemonstrationsInSeoul} there was one 
retail analytics company that tried to estimate the number of participants using 
their technology.
They scanned for MAC addresses emitted from the Wi-Fi of participants' 
smartphones.
However, this method required many assumptions:
\blockcquote{2016DemonstrationsInSeoul}{%
  The company presumed that about half of smartphone users usually leave their 
  Wi-Fi feature on and the other half switch it off, based on a separate survey 
  on smartphone usage. It also assumed that about 20 percent of the smartphone 
  signals were repetition from the same device.
}
This method cannot provide any verifiability, we must trust the company to do 
the measurements correctly and to be honest about when and where they did them.
As MAC address randomization get wider adoption this will be even more 
difficult, although some tracking of smartphones will still be 
possible~\cite{WhyMACRandomizationIsNotEnough}.

A better proposal would be to use \emph{IMSI catchers} (or the real cell 
towers of the mobile network) to count unique phones at the location.
However, there are several problems with this approach.
First, it will be difficult to register only the participants' phones, many 
bystanders will also be counted.
It is also difficult to differentiate between the protesters and the 
counter-protesters.
Second, since the phone is a unique identifier, participants might be 
uncomfortable to be registered in association with the event and might thus 
turn the device off, even though, at least with 5G, the IMSI will not be
transmitted in plaintext anymore. %need to cite?
Finally, there will be limited verifiability, as the data recorder must be 
trusted to record all data and do it in the relation to the event.

\Citet{WifiCrowdCounting} also relies of Wi-Fi, but they instead use the Wi-Fi 
signal as a sonar to detect people.
However this method is designed for indoor use and does not scale to more than 
a crowd of 20 people.
In either case, this type of data cannot provide verifiability.

In general for all of the above methods, the more a crowd spreads out, the more
difficult it will be to determine its size:
the problem is determining whether people near the event's perimeter are 
participants or simply bystanders~\cite{HowToEstimateCrowdSize}.
These methods also have difficulty capturing the actual attendance of an event,
\ie the cumulative participation, not just the count at a snapshot around the 
peak.

One of the most closely related work is \citet{CrowdCount}.
This is a web service which lets Alice create an event and Bob can submit his 
location to register that he is in Alice's event.
This solves the problem that it counts everyone (who submitted), not just the 
count at the snapshot of the photos.
However, there is no verification, \ie the service must be trusted to behave 
honestly, but even then, nothing prevents Bob from submitting twice (violating 
eligibility, \cref{EligibilityVerif}).
Another downside is that the service also requires an Internet connection 
during the event to register as a participant.
This can make it difficult to use if \eg Alice has organized a protest 
against the government, who shuts down the cellular network or Internet 
backbone as a means to censor the protest.

Another related approach based on devices is UrbanCount~\cite{UrbanCount} which
uses epidemic spreading of crowd-size estimates by device-to-device
communication to count crowds in dense urban environments with high
node mobility and churn. There is, however, no consideration of a potentially
adverserial setting and thus no verification or checks on
eligibility. DiVote~\cite{DiVote}, a prior work by the same authors for
polling in dense areas, avoids double counting but again only for
honest participants.

