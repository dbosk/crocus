\subsection{Location proofs}

Some Location-Based Services (\emph{LBS}) only grant access to resources to users located at a particular location,  thus raising the issue of verifying the position claimed by a particular user. 
In most current schemes, the location of a user/device is determined by the device itself (\emph{e.g.}, through GPS) and forwarded to the LBS provider. 
One of the main drawback of this approach is that a user can cheat by having his device transmitting a false location. 
Therefore, it is possible for a user to be inappropriately granted access to a particular resource while being thousands of kilometers away.

To counter this threat, a LBS should ask the requesting device to formally prove that it really is at the claimed location. This notion can be formalized through the concept of \emph{location proof}. 
In a nutshell, a location proof is a digital certificate attesting that someone was at a particular location at a specific moment in time. A location proof architecture is a system by which users can obtain location proofs from neighboring witnesses (\emph{e.g.}, trusted access points or other users) that can later be shown to verifiers who can check the validity of a particular proof. 
Most of the existing approaches to location proofs require the prover and the witnesses to disclose their identities, thus raising many privacy issues such as the possibility of tracing the movements of users of the location proof architecture.
%Seb: to do add the corresponding reference
However, some location proofs systems such as PROPS exists that provides strong privacy guarantees along with the possibility of verifying the claim of the location. 
\PRIVO share some similarities with PROPS although their objective is quite different as we aim at verifying a global property of the population (\emph{i.e.}, crowd estimation) in contrast to checking the location claim made a user, which is an individual property. 
