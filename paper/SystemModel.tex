\section{Defining and formalizing protest and crowd estimation}%
\label{SystemModel}

%\subsection{What is \protect\emph{a} protest?}%
%\label{WhatIsAProtest}

\emph{Defining the concept of protest.} To be able to estimate the participation count for a protest, we first need to define this concept and what should be counted.
Let us start by considering some examples.
During the demonstrations against the South Korean president in Seoul
\blockcquote{2016DemonstrationsInSeoul}{%
  \textins*{t}he rallies stretch\textins{ed} from midday to late night --- some people stay\textins{ed} for several hours, others just several minutes%
}.
These rallies were all in the same location in the capital and repeated every 
weekend for the duration of a few weeks.
The Women's Marches~\cite{2017WomensMarchesInUS}, on the other hand, occurred in parallel in many locations.
We also have the Venezuelan demonstrations in which
\blockcquote{2017VenezuelaProtestFrequency}{%
  anti-government demonstrators have staged daily protests across Venezuela%
} while
\blockcquote{AlJazeeraOnVenezuela2017}{%
  pro-government workers sang and danced as they staged a rival march to show their support for the president's controversial plan to rewrite the constitution%
}.
Judging from these examples, the minimal common part is the cause,%
\label{CauseIsTheCommonDenominator} but there is also a location (or area) that varies over time.

For the rest of the paper, we will refer to the organizer as Alice.
Assume that the objective of Alice is to count everyone who participated at any time and in any of the locations~\cite{2016DemonstrationsInSeoul}\footnote{Note that the objective does not necessarily align with the definition of the police whose interest could be to determine the maximum crowd at any point in time, to deploy enough personnel for crowd-control.}.
Formally, we define a protest as an event that is uniquely identified by its cause (\(id\) below), its time interval (\(t\)) and its location (area \(l\)).
More specifically, we will use the following definition.

\begin{definition}[Protest]\label{DefProtest}
  A \emph{subprotest} is a tuple \((cid, t, l)\) in which \(cid\in \Z_{2^\lambda}\) is the identifier of the cause of the protest,
  \(t = \interval{t_s}{t_e}\subseteq \R\) is a time interval and \(l = N_\epsilon(x, y)\subseteq \R^2\) is the location (an \(\epsilon\)-neighbourhood of \((x,y)\)).
  A \emph{protest} is a set of subprotests sharing the same \(id\).
\end{definition}

The protests described in the previous examples can be captured using this definition by splitting 
them up into subprotests.
Each subprotest will then be captured by our definition and to estimate the total participation to the protest we can just sum up the estimates obtained.
Similarly for march, the marching path can be divided into subprotests with locations (or areas) that slightly overlap.

\emph{Formalizing the notion of crowd estimation.} Each participant who wants to be counted must submit a participation proof.
The proof must be associated with the protest (\emph{i.e.}, its identifier \(cid\)), time and location must coincide with any of its subprotests.

We use other participants, that we called \emph{witnesses} to associate the proof to the location by creating a proof 
share.
A witness is only allowed to create one proof share per protester to avoid the risk of inflation.
The participation proof and its proof shares are stored in a set, as there should be only unique proof shares.

\begin{definition}[Participation-proof share]
  A \emph{participation-proof share} \(w = (pid, cid, t, l, wid)\) is a tuple 
  where
  \(cid, t, l\) are as in \cref{DefProtest},
  \(pid\) is unique for a protester--\(cid\) pair, and
  \(wid\) is unique for a witness--protester pair.
  We let \(W\) be the set of all proof shares.
\end{definition}
%Sonja asks: do the same pair of people (witness-protester) get the
%same wid on a different protest? The pid is unique for protester-cid
%pair thus the id of a protester at a particular protest. Pid is thus
%derived from the protester id (which has no term) and the protest id
%(cid)? In analogy, wid is derived from pid (and thus cid) of both
%participants, protesters i and j, say, where i witnesses j at cid and
%simultaneously symmetrically j witnesses i?
%Using wid and w is confusing since we changed the terminology to use
%proof shares instead of the previous witness (as in testimony, not
%person). wid sounds like it would be the id of the witness person,
%pid sounds like it would be the id of the protester being witnessed,
%we have defined neither and they might not be needed, but it is
%confusing to read.

A subset of the proof shares forms a participation proof for a protester.

\begin{definition}[Participation proof]
  A \emph{participation proof} of protester \(i\) participating in protest \(j\) 
  is the set
  \begin{multline}
    \nonumber
    p_{i, j} =
    \left\{ (pid, cid, t, l, wid)\in W \mid \right. \\
    \left. pid = i, cid = j,
    t \subseteq t_j, l\subseteq l_j \right\},
  \end{multline}
  of all proof shares with the same protester and protest identifiers, witness 
  time interval \(t\) and location \(l\) is within those of the protest \(j\) 
  (\ie \(t_j\) and \(l_j\)).
\end{definition}

We can now define the participation count as follows.
\begin{definition}[Participation count]
  We define the \emph{participation count} of a protest \(cid\) as the cardinality \(|P_{cid}^{s,\theta}|\) of the set of participation proofs \[
    P_{cid}^{s,\theta} = \left\{ p_{i,j} \mid
      j = cid, s(p_{i,j})\geq \theta \right\}
  \] for some strength function \(s\colon \powerset(W)\to \R_+\) and a threshold 
  \(\theta\).
\end{definition}
The strength function \(s\) can be used to regulate trust in witnesses.
One example would be for \(s\) to return the number of proof shares (\emph{i.e.}, witnesses) and thus let \(\theta\) to be the threshold of the number of required witnesses.

\emph{System model and physical assumptions.}
In practical terms, participants witness each other's particpation using their smartphones running the protocol described in \cref{protocol} and uploading their testimony to a blockchain after the protest. 
Thus, they need connectivity at least to each other during the protest and Internet connectivity after. 

\include*{MANETs}
