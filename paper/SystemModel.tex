\section{Defining and formalizing protest and crowd estimation}%
\label{SystemModel}

%\subsection{What is \protect\emph{a} protest?}%
%\label{WhatIsAProtest}

\emph{Defining the concept of protest.} To be able to estimate the participation count for a protest, we first need to define this concept and what should be counted.
Let us start by considering some examples.
During the demonstrations against the South Korean president in Seoul
\blockcquote{2016DemonstrationsInSeoul}{%
  \textins*{t}he rallies stretch\textins{ed} from midday to late night --- some people stay\textins{ed} for several hours, others just several minutes%
}.
These rallies were all in the same location in the capital and repeated every 
weekend for the duration of a few weeks.
The Women's Marches~\cite{2017WomensMarchesInUS}, on the other hand, occurred in parallel in many locations.
We also have the Venezuelan demonstrations in which
\blockcquote{2017VenezuelaProtestFrequency}{%
  anti-government demonstrators have staged daily protests across Venezuela%
} while
\blockcquote{AlJazeeraOnVenezuela2017}{%
  pro-government workers sang and danced as they staged a rival march to show their support for the president's controversial plan to rewrite the constitution%
}.
Judging from these examples, the minimal common part is the cause,%
\label{CauseIsTheCommonDenominator} but there is also a location (or area) that varies over time.

For the rest of the paper, we will refer to the organizer as Alice.
Assume that the objective of Alice is to count everyone who participated at any time and in any of the locations~\cite{2016DemonstrationsInSeoul}\footnote{Note that the objective does not necessarily align with the definition of the police whose interest could be to determine the maximum crowd at any point in time, to deploy enough personnel for crowd-control.}.

% cause identifier
\NewVariable{\cid}{cid}

Formally, we define a protest as an event that is uniquely identified by its 
cause (\(\cid\) below), its time interval (\(t\)) and its location (area \(l\)).
More specifically, we will use the following definition.

% protest
\NewVariable{\prtst}{P}
% subprotest
\NewVariable{\sprtst}{p}

\begin{definition}[Protest]\label{DefProtest}
  A \emph{subprotest} is a tuple \((\cid, t, l)\) in which \(\cid\in 
    \ZZ_{2^\lambda}\) is the identifier of the cause of the protest,
  \(t \subseteq \RR\) is a time period and \(l \subseteq \RR^2\) is the location 
  (topological connectedness is not necessary).
  A \emph{protest} is a set of subprotests sharing the same \(\cid\).
\end{definition}

The protests described in the previous examples can be captured using this definition by splitting 
them up into subprotests.
Each subprotest will then be captured by our definition and to estimate the total participation to the protest we can just sum up the estimates obtained.
Similarly for marches, the marching path can be divided into subprotests with 
locations (or areas) that slightly overlap.

\emph{Formalizing the notion of crowd estimation.} Each participant who wants to 
be counted must submit a \emph{participation proof}.
The proof must be associated with the protest (\ie its identifier \(cid\)), time and location must coincide with any of its subprotests.

We use \emph{witnesses} to associate the proof to the location by creating a 
\emph{proof share}.
A witness is only allowed to create one proof share per protester to avoid the 
risk of inflation.
The participation proof and its proof shares are stored in a set, as there should be only unique proof shares.

% proof share
\NewVariable{\psh}{s}
% proof shares
\NewVariable{\pshs}{S}
% protester identifier
\NewVariable{\pid}{pid}
% witness identifier
\NewVariable{\wid}{wid}

\begin{definition}[Participation-proof share]
  A \emph{participation-proof share} \(\psh = (\cid, t, l, \pid, \wid)\) is a 
  tuple where:
  \(\cid, t, l\) are as in \cref{DefProtest};
  \(\pid\) is a protester's pseudonym for the protest identified by \(\cid\); 
  and
  \(\wid\) is a witness's pseudonym for a protester with pseudonym \(\pid\).
  We say that \(\psh\) is part of a subprotest \(p = (\cid', t', l')\) iff 
  \(\cid = \cid', t\subseteq t', l\subseteq l'\) and denote this by \(\psh 
    \sqsubseteq p\).
  We let \(\pshs\) be the set of all proof shares.
\end{definition}

A subset of the proof shares forms a participation proof for a protester.

% proof
\NewVariable{\prf}{\pi}
% proofs
\NewVariable{\prfs}{\Pi}

\begin{definition}[Participation proof]%
  \label{DefParticipationProof}
  A \emph{participation proof} of protester with pseudonym \(\pid\) 
  participating in a protest \(\prtst = \{ (\cid, t_i, l_i) \}_i\) is the set
  \begin{multline}
    \nonumber
    \prf_{\pid, \prtst} =
    \left\{ \psh = (\cid, t, l, \pid, \wid)\in \pshs \mid
      \exists \sprtst\in \prtst\colon \psh\sqsubseteq \sprtst \right\}
  \end{multline}
  of all proof shares with the same protester and protest identifiers, 
  witnessing time interval \(t\) and location \(l\) is within those of the 
  protest.
  We denote by \(\prfs\) the set of all proofs.
\end{definition}

% strength
\NewFunction{\str}{\varsigma}

We can now define the participation count as follows.
\begin{definition}[Participation count]
  We define the \emph{participation count} of a protest \(\prtst\) (as in 
  \cref{DefParticipationProof}) as the cardinality \(|\prfs_{P}^{\str,\theta}|\) 
  of the set of participation proofs \[
    \prfs_{\prtst}^{\str,\theta} = \left\{ \prf_{i,P} \in \prfs \mid
      \str(\prf_{i,j})\geq \theta \right\}
  \] for some strength function \(\str\colon \powerset(\pshs)\to \RR_+\) and a 
  threshold \(\theta\).
\end{definition}
The strength function \(\str\) can be used to regulate trust in witnesses.
One example would be for \(\str\) to return the number of unique witnesses and 
thus let \(\theta\) to be the threshold of the number of required witnesses.
Another example would be to return the number of proof shares issued by trusted 
witnesses and set \(\theta = 1\) to require at least one proof share issued by a 
trusted witness.

\emph{System model and physical assumptions.}\label{assumptions}
Throughout this text, when we refer to a participant, say Alice, we actually 
mean an agent which can perform cryptographic operations and communicate with 
other local devices on Alice's behalf. We assume that every participant has a digital certificate signed by 
some logically centralized certificate authority.
\Eg that we can use the cryptographic keys of any national electronic identity 
system, identity card or passport.
We need this to prevent Sybil attacks~\cite{SybilAttack}.

In practical terms, participants witness each other's participation
using their smartphones (or similar devices) running the protocol described in
\cref{Protocol} and uploading their testimony (\ie proof shares) to a blockchain 
after the protest. During the protest the devices are computationally limited by 
their batteries and need local connectivity to each other but no connection to
any global network such as the Internet is necessary. Before and
after the protest, we assume that the devices have global connectivity, \ie 
Internet  connections, and are not computationally limited by any battery.

\include*{MANETs}




%assumptions about smartphones:
%(This means that we essentially provide a lower bound for the participation 
%count, since some participants might not have such a device.)
%Sonja says: is this still true? Given mafia fraud we can't guarantee
%that, right?





