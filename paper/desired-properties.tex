\subsection{Desired properties}%
\label{desired-properties}

We note that, in general terms, protests are petitions with a given time and 
location.
Protests, petitions and elections share that in all three many individuals 
express their opinion.
To support democracy, we need strong requirements for verification and privacy 
for elections, it follows that we should have similar properties for protests
and petitions.

We draw inspiration from properties for voting systems:
\begin{description}
  \item[Eligibility:] anyone can verify that each cast vote is legitimate.
  \item[Universal verifiability:] anyone can verify that the result is according to the cast votes.
  \item[Individual verifiability:] each voter can verify that their vote is 
    included in the result.
\end{description}
In our context, votes are translated into \emph{participation proofs}.
Universal and individual verifiability remain the same, in the sense that anyone can verify the participation count by counting the proofs and a participant can verify their proof is included.
The eligibility requirement is slightly different as for protests it must also
include temporal %\footnote{%
%   Elections also have some temporal aspects, such that one cannot vote at any 
%   time and a vote in one election should not be reusable in the next election.
%   However, the temporal relation for a protest is not as strictly defined as for 
%   elections as a protest can start at any time and last for an arbitrary period 
%   of time.%
% }
and spatial eligibility (\ie the protester must have been present at the 
protest).
In essence, the proof must bind the person to the time and location of the protest.
(This is the difference to a petition.)

The main three privacy properties for voting protocols are given as:
\begin{description}
  \item[Vote privacy:] the voting does not reveal any individual vote.
  \item[Receipt freeness:] the voting system does not provide any data that can be used as a proof of how the voter voted.
  \item[Coercion resistance:] a voter cannot cooperate with a coercer to prove their vote was cast in any particular way.
\end{description}

%Coercion resistance is not possible to achieve for protests.
Coercion resistance in voting typically relies on physical isolation
(\eg private voting booths), including for digital systems, and that is by 
definition not possible in our context.
For instance, someone could simply physically bring Alice to a protest against her will.
As for receipt freeness, while
desirable \emph{in itself}, it implies a conflict with verifiability in our context:
%appealing at first glance, is not desirable (probably not possible)
%in our context since this will conflict with verifiability:
in contrast to voting, receipt freeness for \emph{how} the voter voted (\ie the 
cause of the protest) here implies receipt freeness for \emph{that} the voter voted 
(\ie the protester was there), which would make verifiability impossible.

Privacy remains as the crucial property.
More precisely, for the protester we want unlinkability (from the adversary's 
perspective) between a protester's real identity and the participation proof 
(and thus also the protest itself).

