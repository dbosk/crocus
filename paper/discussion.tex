\section{Discussion}%
\label{discussion}


\emph{System model and physical assumptions.}\label{assumptions}
Throughout this text, when we refer to a participant, say Alice, we actually 
mean an agent which can perform cryptographic operations and communicate with 
other local devices on Alice's behalf. We assume that every participant has a digital certificate signed by 
some logically centralized certificate authority.
\Eg that we can use the cryptographic keys of any national electronic identity 
system, identity card or passport.
We need this to prevent Sybil attacks~\cite{SybilAttack}.

In practical terms, participants witness each other's participation
using their smartphones (or similar devices) running the protocol described in
\cref{Protocol} and uploading their testimony (\ie proof shares) to a blockchain 
after the protest. During the protest the devices are computationally limited by 
their batteries and need local connectivity to each other but no connection to
any global network such as the Internet is necessary. Before and
after the protest, we assume that the devices have global connectivity, \ie 
Internet  connections, and are not computationally limited by any battery.

\include*{MANETs}

\sonja{will remove redundancy}
\emph{Limitations and assumptions.} Some of the assumptions that are required for implementing our proposition are not yet realized% (briefly, use of smartphones, unique identities, distance bounding - 
(see \cref{assumptions} for a more detailed description). 
In particular, not all (or even many) protesters, especially in countries with oppressive regimes, currently possess smartphones.
In addition, digital certificates signed by a central authority are not yet widely available and if we rely on government-issued credentials, we cannot prevent the government from performing Sybil attacks --- since it can issue arbitrarily many \emph{valid} credentials.
In this case, it does not make sense to verify any pro-government protests in which Grace's authoritarian regime wants to demonstrate its wide support, \eg~\cite{AlJazeeraOnVenezuela2017,VenezuelanStateWorkersCalledToParticipate}.
Finally, achieving distance-bounding protocols on smartphones is currently not feasible within a meaningful range of distance needed for our scenario.
Indeed, existing smartphones lack the required hardware to do the distance bounding fast enough.

However, we believe that some assumptions might become sufficiently realistic in a near future, due to the smartphone penetration being on the rise and the increased digitalization and government-issued credentials available in some countries already now (\eg Estonia, Germany and Sweden).
In addition, the distance-bounding chips\footnote{\url{https://www.3db-access.com}} currently available on the market can already enable proofs of proximity for any range of up to 200 meters.
One of the short-term objectives is to be able to integrate them in phones or smartcards in the near-future, and even phones with off-the-shelf hardware running in RFID-emulation mode have shown promising results~\cite{DBonSmartphones}.

Another limitation of \PRIVO is that it achieves privacy but not receipt freeness.
This means that there is a way for Grace to efficiently verify that Alice has participated if she is able to get a copy of Alice's secret key.
Indeed, in this case Grace can simply re-do the (deterministic)
computations on the same inputs (the probability of collisions being
negligible).

Currently, the only problem preventing wide deployment is the lacking hardware 
in smartphones.
However, due to the rapid development of \enquote{smart} solutions, we believe 
this is only a matter of time.
\Eg mobile payments and public transport tickets that can be used over \ac{NFC} 
will eventually require \iac{DB} chip.
Without \ac{DB} for these applications, the user must manually confirm on the 
screen to prevent attacks.
We believe that the drive for increased usability will replace this type of 
confirmation by distance bounding.

The deployment of the needed credentials should not be a problem either.
There are already countries where national \ac{eID} systems are widely deployed.
The speed of adoption will depend on the architecture of these systems.
For instance, the system used in Sweden, BankID, is implemented in software and 
can thus be easily upgraded to include the anonymous credentials we need.
As for coverage of the population: more than \SI{95}{\%} of people in the ages 
21--50 uses BankID, it is \SI{88}{\%} for ages 51--60 and \SI{76}{\%} for ages 
61--70\footnote{%
  Official statistics, in Swedish:
  \url{https://www.bankid.com/assets/bankid/stats/2018/statistik-2018-04.pdf}.
}.

While the deployment of the necessary technologies are not yet available in 
authoritarian regimes, they are in several democracies.
We believe that it as important to implement systems such as ours to prevent any 
democracy of slipping down the slope towards dictatorship.


%assumptions about smartphones:
%(This means that we essentially provide a lower bound for the participation 
%count, since some participants might not have such a device.)
%Sonja says: is this still true? Given mafia fraud we can't guarantee
%that, right?
 
\emph{Differences to e-voting}
\sonja{will rephrase receipt freeness, add more than one valid way to count}
When it comes to receipt freeness, our construction does not allow it.
The very property that prevents Alice from cheating by creating \(\pid, \pid'\), 
such that \(\pid \neq \pid'\), is the property that allows Grace to verify 
Alice's participation proof.
If Grace can perform computations using Alice's key, Grace can input the 
\(\cid\) of interest and see if the resulting \(\pid\) is available on the 
blockchain.
To mitigate this situation, Alice could delete her key, in which case she cannot 
participate in any future protest.

As implied by the last statement, there is the problem of renewing credentials.
As with all identity cards, passports \etc, they have a limited lifetime and 
must be renewed at regular intervals.
The rate at which Alice is allowed to renew her credentials affects to what 
extent she can perform Sybil attacks.

\emph{Other actors}

\sonja{will add Jane}

Finally, Mallory represents another nation state and has some interest in 
affecting the stability of Grace's regime, for Mallory's own gain, thus 
supporting either Grace or Alice as she see fits.
Thus, the objective of Mallory will also be to either increase or decrease the count.
\seb{we have to justify why Mallory is not captured either by Alice or Grace}
In addition, Mallory could also have simply as her objective to cause a 
denial-of-service attack on the architecture of \PRIVO.

Mallory cannot create keys valid in another nation state.
