\section{Discussion}%
\label{Discussion}

\subsection{Assumptions limitation} 

Some of the assumptions that are required for implementing our proposition are not yet realized, such as tying a unique identity to a smartphone or performing distance-bounding protocols on existing smartphones. 
In particular, not all (or even many) protesters, especially in countries with oppressive regimes, currently possess smartphones.
In addition, digital certificates signed by a central authority are not yet widely available and if we rely on government-issued credentials, we cannot prevent the government from performing Sybil attacks --- since it can issue arbitrarily many \emph{valid} credentials. \sonja{moved the following from system model, need to clean up this paragraph} For instance, we could use the cryptographic keys of any national electronic identity system, identity card or passport.
This assumption is needed to prevent Sybil attacks~\cite{SybilAttack}, which would have a disastrous effect on the crowd-counting estimation\footnote{
Of course, this has for consequence that the \ac{CA} can perform a Sybil attack on the system.
Thus, if the \ac{CA} is controlled by the government, no counts in favor of the government can be trusted.
}.

In this case, it does not make sense to verify any pro-government protests in which Grace's authoritarian regime wants to demonstrate its wide support~\cite{AlJazeeraOnVenezuela2017,VenezuelanStateWorkersCalledToParticipate}.
Finally, achieving distance-bounding protocols on smartphones is currently not feasible within a meaningful range of distance needed for our scenario.
Indeed, existing smartphones lack the required hardware to conduct the distance bounding fast enough.

However, we believe that some of these assumptions might become sufficiently realistic in the near future, due to the smartphone penetration being on the rise and the increased digitalization and government-issued credentials already available in some countries now (\eg Estonia, Germany, and Sweden). 
The speed of adoption will depend on the architecture of these systems.
For instance, the system used in Sweden, BankID, is implemented in software and can thus be easily upgraded to include the anonymous credentials that \CROCUS need.
As for coverage of the population: more than \SI{95}{\%} of people in the ages 21--50 uses BankID, it is \SI{88}{\%} for ages 51--60 and \SI{76}{\%} for ages 61--70\footnote{Official statistics, in Swedish:  \url{https://www.bankid.com/assets/bankid/stats/2018/statistik-2018-04.pdf}.}.

%Seb: I have commented this part because it is redondant with the performance and evaluation part
%Furthermore, recent technological progress has enabled more advanced cryptographic primitives to be used on smartcards and smartphones that could be used to implement our solution.
%For instance, benchmarks~\cite{Benchmarking} have shown that Android devices are now fast enough to efficiently implement Privacy Enhancing Technologies (PETs), with a Samsung Galaxy S i9000 being able to execute Idemix in \SI{153}{\milli\second}. 
%However, those benchmarks also demonstrate that smartcards remain slow to process complex protocols such as Idemix or U-Prove (taking between \SI{4}{\second} and \SI{8}{\second} to process them). %
%While the limited processing power of many embedded systems has been a challenge, Idemix has been successfully implemented to prove the possession of credentials on Java Cards by~\citet{Bichsel} and the IRMA project, released in 2014, aimed to achieve an implementation ``suitable for real life transactions''~\cite{IRMA} while maintaining security and privacy for its users. 

%Seb: I have moved the two following sentences to the performance analysis section
%In addition, the distance-bounding chips\footnote{\url{https://www.3db-access.com}} currently available on the market can already enable proofs of proximity for any range of up to 200 meters.
%One of the short-term objectives is to be able to integrate them in phones or smartcards in the near-future, and even phones with off-the-shelf hardware running in RFID-emulation mode have shown promising results~\cite{DBonSmartphones}.
% Currently, the only problem preventing wide deployment is the lacking hardware 
% in smartphones.
% However, due to the rapid development of \enquote{smart} solutions, we believe 
% this is only a matter of time.
% \Eg mobile payments and public transport tickets that can be used over \ac{NFC} 
% will eventually require \iac{DB} chip.
% Without \ac{DB} for these applications, the user must manually confirm on the 
% screen to prevent attacks.
% We believe that the drive for increased usability will replace this type of 
% confirmation by distance bounding.

% The deployment of the needed credentials should not be a problem either.
% There are already countries where national \ac{eID} systems are widely deployed.

\include*{MANETs}


%\subsection{Implementation on smartphones or smart-cards}



 
\subsection{Differences to e-voting}
\sonja{should we keep this now?}
In e-voting, receipt freeness means that one cannot prove to a third party how one voted. 
In the context of protests, receipts for mere participation can imply one's political leanings, as even with a counterprotest at the same location and time, side-channel information can be used to decide which one a person participated in. 
Individual verifiability and eligibility (especially one proof per person, prevention of Sybil attacks) means that a person can check whether her input was counted, thus sufficing as a receipt in the case of protests.  
Therefore, \CROCUS achieves privacy but not receipt freeness.
As a consequence, this means that there is a way for Grace to efficiently verify that Alice has participated if she is able to get a copy of Alice's secret key. 
The very property that prevents Alice from cheating by creating \(\pid, \pid'\), such that \(\pid \neq \pid'\), is the property that
allows Grace to verify Alice's proof of participation.  
Indeed, in this situation Grace can simply re-do the (deterministic) computations on the same inputs (the probability of collisions being negligible).  
More precisely, if Grace can perform computations using Alice's key, Grace can input the \(\cid\) of interest and see if the resulting \(\pid\) is available on the blockchain.  
To mitigate this situation, Alice could delete her key, in which case she cannot participate in any future protest.  

As implied by the last statement, this leads to a problem of credential renewing.  
Indeed as with all identity cards, passports \etc, the lifetime of the document is limited and this latter must be renewed at regular intervals.  
The rate at which Alice is allowed to renew her credentials affects to what extent she can perform Sybil attacks. 
Another facet concerning receipt freeness is traffic analysis. 
Indeed, if Grace can monitor all connections within the country, which seems a reasonable assumption, she can monitor Alice when she submits her proof to the blockchain. 
However, as the miner collects many transactions, Alice should have plausible deniability that she submitted one of the other transactions --- but again not deniability/receipt freeness as Grace can verify if she accesses Alice's device.

Counting votes is also more straightforward than counting protest participation. 
Depending on the value of interest (\eg peak participation, total participation, individual sub-protests or counting participation only once for the cause, etc.), different ways of counting are
possible and valid. 
Since anyone can read the data, they can verify the results given the parameters of choice.

\subsection{Other actors}

Without verifiable binding to a location, even distance-bounding protocols do not prevent collusions such as protesters meeting at a safe location away from the official site. 
To address that, one can use trusted infrastructure (for an example by media see~\cite{LeMondeProtestingSolution}) or people such as journalists. 
Given the conflict of interest of potentially state-provided infrastructure for political protests, we opt for the trusted journalist, Jane, who reveals her identity after the protest and thus can be used as a trusted witness. 
In practice, there can be any number of  Janes and, as with the different ways of counting, there does not need to be an agreement on who is trusted as anyone can set the eligibility criteria for their count.

Finally, Mallory represents another nation state and has some interest in affecting the stability of Grace's regime, for Mallory's own gain, thus supporting either Grace or Alice as she see fits.  
Thus, the objective of Mallory will also be to either increase or decrease the count. 
Unlike Alice, Mallory can create as many keys as she likes, but she cannot create keys valid in another nation state. 
Nonetheless, Mallory could also have simply as her objective to cause a denial-of-service attack on the architecture of \CROCUS.



