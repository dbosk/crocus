\section{Discussion: limitations of assumptions}%
\label{Discussion}

%\subsection{Assumptions limitations}
%\subsection{Equipment}
\subsection{Implementability}

Some of the assumptions that are required for implementing our proposition are 
not yet realized, however, we believe that they soon will be.

One major issue is that all protesters must have smartphones --- in many 
countries with oppressive regimes, far from all possess smartphones
currently, though the rate is on the rise.

On those smartphones we additionally require an identity credential
signed by some \ac{CA}.  This is also not yet widely available.
However, more and more nation states are starting to issue digital
certificates in identity cards and many already have crypto-enabled
RFID chips in their passports.  \Eg Estonia, Germany, and Sweden
already have the infrastructure and widely deployed electronic
identity systems, and the EU already has regulation in place (eIDAS).
%The speed of adoption will depend on the architecture of these systems.
%For instance, the system used in Sweden (BankID) is implemented in software 
%(most use it on smartphones) and can thus easily be upgraded to include the 
%anonymous credentials that \CROCUS needs. \sonja{this means one can
%have more than one more easily as well}
In Sweden, more than \SI{95}{\%} of people in the ages 21--50 use BankID, 
\SI{88}{\%} for ages 51--60 and \SI{76}{\%} for ages 
61--70~\cite{BankID-stats}.

National identities, however, are problematic when there is an
incentive for the government to create more identities, for example
when it comes to elections.  In functioning voting systems, there is
no more than one ballot (token) per physical person and the ballot is
not linked to the vote. Audits and other processes help ensure that
there are no ballots for non existing persons (Sybil-proof identities)
or extra ballots for voters. Similarly, here, we need
\begin{enumerate*}
\item\label{discussion-credential-map} a mapping between an identity and a 
  physical person (in the form of an anonymous identity credential, functioning 
  as a pseudonym for privacy
properties) and
\item\label{discussion-one-credential} only one such credential per identity 
  (token of physical personhood instead of token of being a voter in a 
  particular election).
\end{enumerate*}
With the same caveat as for identities for voting 
for~\ref{discussion-credential-map}, mechanisms such as collective 
signing~\cite{collective-signing} could ensure~\ref{discussion-one-credential}.
Again, this is not yet realized, but with the current pace
of adopting public ledgers and efforts for cross-national identity
credentials such as the EU's eIDAS, the prerequisites exist. Moreover,
the code of practice for European statistics~\cite{EuroStat} includes 
principles such as coordination and cooperation as well as
impartiality and objectivity and is an example of both efforts toward
and motivation for cross-national improvements of statistics.


We also need to run distance-bounding protocols on smartphones.
Achieving this is currently not feasible within a meaningful range as
existing smartphones lack the required hardware to conduct the
distance bounding fast enough.  However, thefts of luxury cars due to
relay attacks have driven the development of hardware for doing
distance bounding in car keys.  We believe that using smartphones for
contactless payment and electronic tickets will drive a similar
development for this hardware on smartphones.

If we remove the distance-bounding component, we can, given some
caveats on trust and setup, use \CROCUS for online petitions: a
trusted consortium could set up a petition site, ensure that only one
credential is issued per identity (given sybil-proof identities,
again, building on something like eIDAS), use \CROCUS without
witnesses and location and still benefit from the privacy and
verifiability properties from its pseudonyms and ledger setup.


\subsection{Communications}

%@sbuc: Can you write a few words about MANETs (file MANETs.tex)?
%More specifically, why we can assume there is a way for the
%protesters to communicate locally at the protest without relying on the
%Internet. Also mention potential privacy problems at this layer, e.g.\
%MAC-address tracking etc. which is outside of our protocol (this should
%cover part of issue #58).

Since the cellular network could be shut down to keep protesters from accessing the Internet, making phone calls or texting, we require a different means of communications between protesters. 
This could be accomplished by Bluetooth or WiFi communications, as demonstrated by Briar~\cite{Briar} and FireChat~\cite{FireChat}, two examples of applications for communication during protests via wireless mesh networking. The crowd-counting scenario likely has higher requirements on capacity and withstanding interference as the participants continuously run the protocol for witnessing each others presence; messages are presumably less frequent. 5G is intended to cope with billions of devices (IoT), and thus could help cope with the device density in crowds.
An alternative, although originally designed to work within 5G cellular networks, is device-to-device communication (D2D)~\cite{D2D}. 
Specifically, the out-of-band and autonomous version D2D fits our scenario thanks to using unlicensed spectrum and working without cellular coverage. 


\subsection{Adversaries}

Our adversary model considers only protocol data, no auxiliary data.
Against this adversary our scheme is secure.
The question is how this adversary model maps to real adversaries.

In any real implementation there are potential side-channels.  \Eg in
the communication layer: IP-addresses translate into identities,
devices' MAC-addresses can be used as persistent identifiers.  We do
not consider these aspects as there are entire fields dedicated to
some of them, we simply assume the tools developed in those fields
(\eg Tor~\cite{Tor} and randomized MAC-addresses) to prevent these
problems will be used.

Like other schemes involving Internet communication, we do not
consider a global passive adversary.  Such an adversary could use side
information to do \eg time-correlation attacks against people that
submit transactions concerning a particular \(\cid\) over Tor,
identify them, and link them to the \(\cid\). On a more realistic
scale, a national passive adversary can control all the nation's
\acp{ISP} but would not be able to observe all Tor exit nodes or
otherwise observe all input to the ledger needed to perform such
correlation attacks.

During a protest there are also other information channels available
to the adversary.  \Eg, one could argue that the adversary might be
able to map a face to a \(\pid\) by means of signal triangulation
during the protocol run, and then map the face to an identity through
face recognition. However, there are far easier tactics the adversary
could use: \eg, the adversary can take photos of the event and try to
capture as many faces as possible.  This is already possible today and
thus not a weakness introduced by our scheme.

Besides these privacy concerns, there are also verifiability
concerns. There, we have shown that the security is reduced to the
witnesses. In the case of using trusted witnesses --- which we
consider the canonical \CROCUS --- it simply reduces to their
trustworthiness, like in any system with trusted third parties.

We outlined a variant of counting and verifying for \CROCUS that uses
a threshold~\(\theta\) of untrusted witnesses and, trivially, resists
a collusion of malicious witnesses smaller than \(\theta\).  While
using this approach is technically possible, we do not know whether it
is of practical use. We have not found a principled way of determining
a secure value for \(\theta\), yet neither can we at this point
conclude that there is no such way. Informally, a higher \(\theta\)
increases the probability of a collusion that is large enough to break
verifiability being detected and made known to the verifier. 

Overall, while \CROCUS provides the mechanism for a privacy-preserving
and verifiable way of counting crowds, any verifier still needs to
consider their own trust, analogous to detecting bias in
science. There, if a paper espousing, say, the efficacy of X, comes
from research sponsored by the company that manufactures X, it might
be biased. Likewise, if a count of a pro-government protest is
published by that government, one needs to consider where identities
were issued and which witnesses were considered trusted. In contrast
to a layperson interpreting science, however, the results can easily
be reproduced by the verifier: knowing which set of witnesses (and
other criteria such as location and time) the counter used for the
count, the count can be verified (by a re-count).  Whether the result
can be trusted then depends on whether the verifier also trusts this
set of witnesses. Given that all proof shares are on the ledger,
however, a verifier can come up with their own criteria and make a
count themselves. This count then, given that they publish what
criteria they used, can in turn be verified by anyone else.





% \subsection{Sybil and issuing keys}

% We need an identity management system that provides unlinkability between 
% contexts (\eg protests) but linkability for reuse within the same
% context (\eg participation proof shares for counting). 




%\subsection{Other actors}
%
%Without verifiable binding to a location, even distance-bounding protocols do not prevent collusions such as protesters meeting at a safe location away from the official site. 
%To address that, one can use trusted infrastructure (for an example by media see~\cite{LeMondeProtestingSolution}) or people such as journalists. 
%Given the conflict of interest of potentially state-provided infrastructure for political protests, we opt for the trusted journalist, Jane, who reveals her identity after the protest and thus can be used as a trusted witness. 
%In practice, there can be any number of  Janes and, as with the different ways of counting, there does not need to be an agreement on who is trusted as anyone can set the eligibility criteria for their count.
%
%Finally, Mallory represents another nation state and has some interest in affecting the stability of Grace's regime, for Mallory's own gain, thus supporting either Grace or Alice as she see fits.  
%Thus, the objective of Mallory will also be to either increase or decrease the count. 
%Unlike Alice, Mallory can create as many keys as she likes, but she cannot create keys valid in another nation state. 
%Nonetheless, Mallory could also have simply as her objective to cause a denial-of-service attack on the architecture of \CROCUS.
%

%\subsection{Trust assumptions}
%
%The goal of CROCUS is to be able to count people in a crowd in a verifiable way, without leaking information about the individuals in the crowd.
%We believe one of the strongest adversaries in practice, and one of the most dangerous situation, is that of an authoritarian government wanting to learn the identities of protesters,
%so we chose a theoretical model that fits this situation well, but it is interesting to see what are the trust relations which are actually needed in various real world situations.
%
%\paragraph{ID issuer} We assumed a trusted third party issued certified IDs to every participant (protesters and witnesses). This hypothesis is important to avoid Sybil attacks,
%but only the side protesting (not the side being protested against) has an incentive to perform a Sybil attack.
%So in practice, the ID issuer doesn't need to be a trusted 3rd-party, it only needs to not be aligned with the protesters.
%In our authoritarian government scenario for instance, the protesters can use IDs issued by said government.
%On the other hand however, a pro-government protest cannot use the same IDs and offer a good level of trust in the final count, and it would be better suited to use IDs from a different source
%(\eg a supra-national issuer like the EU).
%
% \paragraph{Verifier's trust in the count} We proposed a generic way of counting, that relies on two parameters: the strength function and the threshold.
%%We then suggested two possible instantiations: every witness is equal and we use a relatively large threshold, or we only trust a minority of independent witnesses with a threshold of 1.
%An important point here is that these parameters can be chosen by each verifier independently, so each verifier can make choices that satisfy \emph{their} own level of trust, and can each obtain a different final count, and \CROCUS does not output one definitive answer. The trust in a final count depends only on the trust between the verifier and the witnesses.
%
%\simon{maybe delete that second half}
%In practice though, the verifier can be in either of three situations: for the protest, against the protest, or indifferent, which leads to three main categories of choices.
%\begin{itemize}
%	\item A verifier aligned with the protesters has an incentive to trust any witness and use as low a threshold as possible.
%	\item A verifier opposed to the protesters has the opposite incentive to exclude as many witnesses and use as high a threshold as possible.
%	\item An indifferent verifier who only cares about the correctness of the result can trust each witness based on their reputation and should use a threshold that is reasonably reachable in practice but high enough to avoid small-scale collusion.
%\end{itemize}
%
%\paragraph{Protesters and witnesses} Active participants in a protest do not need to trust anything other than that their app is behaving properly and that their smartphone is not compromised. Indeed, \CROCUS has been designed with their privacy and security as a central concern, and being able to use \CROCUS without any additional risk compared to simply come at the protest was one of our main requirement.
