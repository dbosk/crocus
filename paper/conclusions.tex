\section{Conclusions}%
\label{Conclusions}

We have introduced \PRIVO, a protocol for verifiably counting
participants at protests. Despite the verification, the privacy of the
participants is preserved to the extent possible by their physical
presence at the protest. To achieve that, we adapted the Schnorr
protocol to develop distance-bounding anonymous credentials. Except for receipt
freeness, which we cannot guarantee if the adversary gets access to
Alice's device and secret PRF key after her participation has
been witnessed and before she had a chance to
delete it, \PRIVO
fulfills all verification and privacy properties we derived from those
of electronic voting complemented by the specific constraints on time
and location required for protests. 
In general, none of the related works fulfills the eligibility verifiability 
\cref{%
  CountOnce,%
  TemporallyRelated,SpatiallyRelated,%
}.
To some extent, some techniques do fulfill \cref{DesignatedEvent}, since people 
usually use placards in the protests.
Also, some schemes provide universal verifiability (\cref{UniversalVerif}), but 
not individual verifiability (\cref{IndividualVerif}).
If they would provide individual verifiability, they would likely violate the 
privacy requirements.

Finally, most of the current methods are estimates, with quite some error 
margins. While \PRIVO is an actual count and not an estimate, its
accuracy for the total number of participants hinges on the
participants having the necessary equipment (a smartphone or similar
device), some sort of trustworthy credential, and the willingness to
run the protocol. If used today, \PRIVO would result in a
considerable undercount in most scenarios, \daniel{why?} but it presents a first
step toward accurate verifiable yet privacy-preserving crowd counting
by showing how it can be done in theory at least and in practice once
some assumptions become more realistic. 

When it comes to receipt freeness, our construction does not allow it.
The very property that prevents Alice from cheating by creating \(\pid, \pid'\), 
such that \(\pid \neq \pid'\), is the property that allows Grace to verify 
Alice's participation proof.
If Grace can perform computations using Alice's key, Grace can input the 
\(\cid\) of interest and see if the resulting \(\pid\) is available on the 
blockchain.
To mitigate this situation, Alice could delete her key, in which case she cannot 
participate in any future protest.

As implied by the last statement, there is the problem of renewing credentials.
As with all identity cards, passports \etc, they have a limited lifetime and 
must be renewed at regular intervals.
The rate at which Alice is allowed to renew her credentials affects to what 
extent she can perform Sybil attacks.

Currently, the only problem preventing wide deployment is the lacking hardware 
in smartphones.
However, due to the rapid development of \enquote{smart} solutions, we believe 
this is only a matter of time.
\Eg mobile payments and public transport tickets that can be used over \ac{NFC} 
will eventually require \iac{DB} chip.
Without \ac{DB} for these applications, the user must manually confirm on the 
screen to prevent attacks.
We believe that the drive for increased usability will replace this type of 
confirmation by distance bounding.

The deployment of the needed credentials should not be a problem either.
There are already countries where national \ac{eID} systems are widely deployed.
The speed of adoption will depend on the architecture of these systems.
For instance, the system used in Sweden, BankID, is implemented in software and 
can thus be easily upgraded to include the anonymous credentials we need.
As for coverage of the population: more than \SI{95}{\%} of people in the ages 
21--50 uses BankID, it is \SI{88}{\%} for ages 51--60 and \SI{76}{\%} for ages 
61--70\footnote{%
  Official statistics, in Swedish:
  \url{https://www.bankid.com/assets/bankid/stats/2018/statistik-2018-04.pdf}.
}.

While the deployment of the necessary technologies are not yet available in 
authoritarian regimes, they are in several democracies.
We believe that it as important to implement systems such as ours to prevent any 
democracy of slipping down the slope towards dictatorship.

Finally, Mallory represents another nation state and has some interest in 
affecting the stability of Grace's regime, for Mallory's own gain, thus 
supporting either Grace or Alice as she see fits.
Thus, the objective of Mallory will also be to either increase or decrease the count.
\seb{we have to justify why Mallory is not captured either by Alice or Grace}
In addition, Mallory could also have simply as her objective to cause a 
denial-of-service attack on the architecture of \PRIVO.

Mallory cannot create keys valid in another nation state.
