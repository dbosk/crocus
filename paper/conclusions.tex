\section{Conclusion}%
\label{Conclusion}

We have introduced \PRIVO, a protocol for verifiably counting participants at protests. Despite the verification, the privacy of the
participants is preserved to the extent possible by their physical
presence at the protest. To achieve that, we adapted the Schnorr
protocol to develop distance-bounding anonymous credentials. Except for receipt
freeness, which we cannot guarantee if the adversary gets access to
Alice's device and secret PRF key after her participation has
been witnessed and before she had a chance to
delete it, \PRIVO
fulfills all verification and privacy properties we derived from those
of electronic voting complemented by the specific constraints on time
and location required for protests. 
In general, none of the related works fulfills the eligibility verifiability 
\cref{%
  CountOnce,%
  TemporallyRelated,SpatiallyRelated,%
}.
To some extent, some techniques do fulfill \cref{DesignatedEvent}, since people 
usually use placards in the protests.
Also, some schemes provide universal verifiability (\cref{UniversalVerif}), but 
not individual verifiability (\cref{IndividualVerif}).
If they would provide individual verifiability, they would likely violate the 
privacy requirements.

Finally, most of the current methods are estimates, with quite some error 
margins. While \PRIVO is an actual count and not an estimate, its
accuracy for the total number of participants hinges on the
participants having the necessary equipment (a smartphone or similar
device), some sort of trustworthy credential, and the willingness to
run the protocol. If used today, \PRIVO would result in a
considerable undercount in most scenarios, \daniel{why?} but it presents a first
step toward accurate verifiable yet privacy-preserving crowd counting
by showing how it can be done in theory at least and in practice once
some assumptions become more realistic. 

