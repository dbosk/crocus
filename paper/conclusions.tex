\section{Conclusion}%
\label{Conclusion}

In this paper, we have adapted the Schnorr protocol to work as a
distance-bounding protocol.  It allows us to construct distance-bounding
general attribute-based authentication, with more general attributes than
provided by previous \ac{DB} protocols.  We made and formally verified two versions of the
protocol, one that is more efficient and resistant to the attacks
leveled at \ac{DB} protocols but that assumes that more than one
bit at a time can be sent in the fast phase (as demonstrated by
\textcite{UWBPR}). We also developed a PKI scheme to generalize~\cite{UWBPR} to work for scenarios such as ours where the prover and verifier do
not trust or even know each other. The second version is a classic bit-by-bit
protocol that is not fully \ac{DBTF} resistant but otherwise fulfills
all requirements. 

