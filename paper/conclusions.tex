\section{Conclusion}%
\label{Conclusion}

In this paper, we have introduced \CROCUS, a privacy-preserving protocol for 
verifiably counting participants at protests.
Despite the verification, the privacy of the participants is preserved to the 
extent possible by their physical presence at the protest.

We showed that \CROCUS preserves privacy and that the only privacy-threat it 
adds to protesting without it, is that Grace can use Alice's private key to 
verify her participation.
However, this requires that Grace already suspects Alice --- Grace cannot check 
everyone --- and that Alice has not renewed her digital certificate.

We also showed that \CROCUS provides universally verifiable data.
With \CROCUS, a journalist can easily count the participation, specify how they 
counted (time, location \etc) along with the result and everyone can 
independently verify that the result is correct.
The only results that cannot be trusted are results aligned with the interests 
of the \ac{CA}.

%To achieve \CROCUS, we adapted the Schnorr protocol as a distance-bounding 
%protocol.
%This is of separate interest, as it allows to construct distance-bounding 
%general attribute-based authentication, more general attributes than provided 
%by previous \ac{DB} ones.
%With our contribution, Alice can now prove more complex statements to Bob; \eg  
%that she is older than 18 but younger than 40 without revealing her actual 
%age, and at the same time Bob can be sure that Alice is not just relaying 
%messages from her older sister Carol.

While \CROCUS is an actual count and not an estimate, its accuracy for the 
total number of participants hinges on the participants having the necessary 
equipment (\ie a smartphone or similar device), some sort of trustworthy 
credential, and the willingness to run the protocol. Owing to the network 
effect, until used by most participants, \CROCUS would result in a considerable 
undercount. However, it represents a first step toward accurate verifiable yet 
privacy-preserving crowd counting by showing how it can be done in theory at 
least and in practice once some assumptions concerning hardware and 
e-identities become more realistic. While the necessary technologies are not 
yet available in most authoritarian regimes, some of them are in several 
democracies. We believe that it is important to implement systems such as ours 
also there to support the maintenance of democratic processes and counter fake 
news.

