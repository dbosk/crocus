\subsection{Defining and formalizing protest and crowd estimation}%
\label{protest-model}

%\subsection{What is \protect\emph{a} protest?}%
%\label{WhatIsAProtest}

%\emph{Defining the concept of protest.} 
To be able to estimate the participation count for a protest, we first need to define this concept and which quantity should be counted.
Let us start by considering some examples.
During the demonstrations against the South Korean president in Seoul in 2016
\blockcquote{2016DemonstrationsInSeoul}{%
  \textins*{t}he rallies stretch\textins{ed} from midday to late night --- some people stay\textins{ed} for several hours, others just several minutes%
}.
These rallies were all in the same location in the capital and repeated every weekend for a few weeks duration.
The Women's Marches in 2017~\cite{2017WomensMarchesInUS}, on the other hand, occurred in parallel in many locations.
We also have the Venezuelan demonstrations in 2017 in which \blockcquote{2017VenezuelaProtestFrequency}{%
  anti-government demonstrators have staged daily protests across Venezuela%
} while
\blockcquote{AlJazeeraOnVenezuela2017}{%
  pro-government workers sang and danced as they staged a rival march to show their support for the president's controversial plan to rewrite the constitution%
}.
Generalizing from these examples, the minimal common part is the cause,\label{CauseIsTheCommonDenominator} while the location (or area) considered varies over time.

For the rest of the paper, we will refer to the organizer as Alice.
We assume that the objective of Alice is to count everyone who participated at any time and in any of the locations~\cite{2016DemonstrationsInSeoul} %\sonja{commented out for
  %repetition: %\footnote{Note that the objective does not necessarily
              %align with the definition of the police whose interest
              %could be to determine the maximum crowd at any point in
              %time to deploy enough personnel for crowd-control.
%}.
% cause identifier
Formally, we define a protest as an event that is uniquely identified by its cause \(\cid\), its time interval \(t\) and its location (area) \(l\).
More specifically, we will use the following definition.
% protest
% subprotest

\begin{definition}[Subprotest, Protest]\label{DefProtest}
  A \emph{subprotest} is a tuple \((\cid, t, l)\) in which \(\cid\in \bin^{\lambda}\), for some fixed \(\lambda\in \NN\), is the identifier of the cause of the protest, \(t \subseteq \TT\) is a time period and \(l \subseteq \LL\) is the location (the topological connectedness is not necessary).
  A \emph{protest} is the set of subprotests sharing the same \(\cid\).
\end{definition}

The protests described in the previous examples can be captured using this definition by decomposing them into subprotests.
Each subprotest will then be encapsulated by our definition and to estimate the total participation to the protest we can just sum up the estimates obtained.
Similarly for marches, the marching path can be divided into subprotests with locations (or areas) that slightly overlap.

\emph{Formalizing the notion of crowd estimation.} 
Each participant who wants to be counted must submit a \emph{participation proof}.
The proof must be associated with the protest (\ie its cause identifier \(cid\)), and its time and location must coincide with one of the subprotests.

Our protocol relies on \emph{witnesses} to certify and associate the proof to the location by creating a \emph{proof share}.
A witness is only allowed to create one proof share per protester to avoid the risk of count inflation.
Note that a participant of a protest can take the role of a protester but also act as witness for other protesters.
The participation proof and its proof shares are stored as a set.
%, as there should only be unique proof shares.

% proof share
% proof shares
% protester identifier
% witness identifier

\begin{definition}[Proof share]%
  \label{DefProofShare}\label{DefProofShares}
  A \emph{proof share} \(\psh = (\cid, t, l, \pid, \wid)\) is a tuple in which: 
  \(\cid, t, l\) are as in \cref{DefProtest};
  \(\pid\) is a protester's pseudonym for the protest identified by \(\cid\); and \(\wid\) is a witness's pseudonym for a protester with pseudonym \(\pid\).
  We say that \(\psh\) is part of a subprotest \(\sprtst = (\cid', t', l')\) if and only if \(\cid = \cid', t\subseteq t', l\subseteq l'\) and denote this by 
  \(\psh \sqsubseteq \sprtst\).
  We define \(\pshs\) as the set of all proof shares.
\end{definition}

We will use the following notation to filter out a subset of \(\pshs\) with specific \(\cid\)s and \(\pid\)s:
\begin{multline*}
  \pshs_{\cid_0, \pid_0} = \\
  \{(\cid, t, l, \pid, \wid)\in \pshs\mid \cid = \cid_0 \land \pid = \pid_0\}.
\end{multline*}

A subset of the proof shares forms a participation proof for a protester.

% proof
% proofs
\begin{definition}[Participation proof]%
  \label{DefParticipationProof}\label{DefParticipationProofs}
  A \emph{participation proof} of a protester with pseudonym \(\pid\) who participates in a protest \(\prtst\) with cause identifier \(\cid\) is the 
  set
  \begin{equation}
    \nonumber
    \prf_{\pid, \prtst} =
    \left\{ \psh \in \pshs_{\cid, \pid} \mid
      \exists \sprtst\in \prtst\colon \psh\sqsubseteq \sprtst \right\},
  \end{equation}
  of all proof shares with the same protester and protest identifiers, time interval \(t\) and location \(l\) within those of the protest (\ie any subprotest).
  We denote by \(\prfs\) the set of all proofs.
\end{definition}

% strength
\NewFunction{\str}{\varsigma}

We can now define the participation count as follows.
\begin{definition}[Participation count]%
  \label{DefParticipationCount}
  We define the \emph{participation count} of a protest \(\prtst\) (as in \cref{DefParticipationProof}) as the cardinality 
  \(|\prfs_{P}^{\str,\theta}|\) of the set of participation proofs \[
    \prfs_{\prtst}^{\str,\theta} = \left\{ \prf_{i,P} \in \prfs \mid
      \str(\prf_{i,P})\geq \theta \right\}
  \] for some strength function \(\str\colon \powerset(\pshs)\to \RR_+\) and a 
  threshold \(\theta\).
\end{definition}

The strength function \(\str\) can be used to regulate the trust in the participation count estimated. In general, \(\str\) can be defined as a weighted sum of the proof shares, $\str = \sum \omega_i.s_i$, with the weights $\omega_i$ being the trust in the witness corresponding to that proof share $s_i$, and the threshold $\theta$ represents the total trust needed to accept a participant as valid.
One example would be to set all weights to 1 for \(\str\) to return the number of unique witnesses and thus let \(\theta\) to be the threshold of the number of required witnesses.
Another possibility would be to also have a particular type of witness, called \emph{trusted witness}, participating in the protest. 
For instance, the role of the trusted witness could be taken by Jane, the independent journalist. 
In this situation, the weights would be 1 for trusted witnesses and 0 for any other witness, and
setting \(\theta = 1\) would require at least one proof share issued by a trusted witness.
Finally, both approaches can be combined by giving a weight of 1 to all non-trusted witnesses and a weight of $\theta$ to trusted witnesses. This results in a participant being valid if they are witnessed either by $\theta$ non-trusted witness or by one trusted witness.

\simon{check the notation in the weights formula is consistent, check that the description in section V corresponds}