\subsection{Protest,  crowd estimation}%
\label{protest-model}

%\subsection{What is \protect\emph{a} protest?}%
%\label{WhatIsAProtest}

%\emph{Defining the concept of protest.} 
To be able to estimate the participation count for a protest, we first need to define this concept and which quantity should be counted.
Let us start by considering some examples.
During the demonstrations against the South Korean president in Seoul in 2016
\blockcquote{2016DemonstrationsInSeoul}{%
  \textins*{t}he rallies stretch\textins{ed} from midday to late night --- some people stay\textins{ed} for several hours, others just several minutes%
}.
These rallies were all in the same location in the capital and repeated every weekend for a few weeks.
The Women's Marches in 2017~\cite{2017WomensMarchesInUS}, on the other hand, occurred in parallel in many locations.
We also have the Venezuelan demonstrations in 2017 in which \blockcquote{2017VenezuelaProtestFrequency}{%
  anti-government demonstrators have staged daily protests across Venezuela%
} while
\blockcquote{AlJazeeraOnVenezuela2017}{%
  pro-government workers sang and danced as they staged a rival march to show their support for the president's controversial plan to rewrite the constitution%
}.
Generalizing from these examples, the minimal common part is the cause,\label{CauseIsTheCommonDenominator} while the location (or area) considered varies over time.

For the rest of the paper, we will refer to the organizer as Alice.
We assume that the objective of Alice is to count everyone who participated at any time and in any of the locations~\cite{2016DemonstrationsInSeoul} %\sonja{commented out for
  %repetition: %\footnote{Note that the objective does not necessarily
              %align with the definition of the police whose interest
              %could be to determine the maximum crowd at any point in
              %time to deploy enough personnel for crowd-control.
%}.
% cause identifier
Formally, we define a protest as an event that is uniquely identified by its cause \(\cid\), its time interval \(t\) and its location (area) \(l\).
More specifically, we will use the following definition.
% protest
% subprotest

\begin{definition}[Subprotest, Protest]\label{DefProtest}
  A \emph{subprotest} \(p = (\cid, t, l)\) is a tuple in which \(\cid\in 
    \bin^{\lambda}\), for some fixed \(\lambda\in \NN\), is the identifier of 
  the cause of the protest, \(t \subseteq \TT\) is a time period and \(l 
    \subseteq \LL\) is the location (topological connectedness is not 
  necessary).

  A \emph{protest} \(\prtst\) is the set of subprotests sharing the same \(\cid\).
\end{definition}

The protests described in the previous examples can be captured using this definition by decomposing them into subprotests.
Each subprotest will then be encapsulated by our definition and to estimate the total participation to the protest we can just sum up the estimates obtained.
Similarly for marches, the marching path can be divided into subprotests with locations (or areas) that slightly overlap.

%\emph{Formalizing the notion of crowd estimation.} 
Each participant who wants to be counted must submit a \emph{participation proof}.
The proof must be associated with the protest (\ie its cause identifier \(cid\)), and its time and location must coincide with one of the subprotests.

Our protocol relies on \emph{witnesses} to certify and associate the proof to the time and location by creating a \emph{proof share}.
A witness is only allowed to create one proof share per protester to avoid the risk of count inflation.
(Note that a participant of a protest can take the role of a protester but also 
act as witness for other protesters.)
Then, the set of all \emph{valid} proof shares forms the participation proof of a protester.

% proof share
% proof shares
% protester identifier
% witness identifier

\begin{definition}[Valid proof share]%
  \label{DefProofShare}\label{DefProofShares}
  A \emph{proof share} \(\psh = (\cid, t, l, \pid, \wid)\) is a tuple in which: 
  \(\cid, t, l\) are as in \cref{DefProtest};
  \(\pid\) is a protester's pseudonym for the protest identified by \(\cid\);
  \(\wid\) is a witness's pseudonym for a protester with pseudonym \(\pid\).

  Furthermore, we say that \(\psh\) is \emph{valid} for a subprotest \(\sprtst = (\cid', t', l')\) if and only if \(\cid = \cid', t\subseteq t', l\subseteq l'\) and denote this by 
  \(\psh \sqsubseteq \sprtst\).
\end{definition}

%A subset of the proof shares forms a participation proof for a protester.
We let \(\pshs\) denote the set of all proof shares.
Let \(
  \pshs_\prtst = \{ \psh \in \pshs \mid
    \exists \sprtst \in \prtst, \psh \sqsubseteq \sprtst \}
\) be the subset of proof shares related to a protest~\(\prtst\).
We denote by \(=_\pid\) the equivalence relation on \(\pshs\) where \((\cid, t, 
l, \pid, \wid) =_\pid (\cid', t', l', \pid', \wid')\) if \(\pid = \pid'\).
%and we will use the following notation to filter out a subset of \(\pshs\) with 
%specific \(\cid\)s and \(\pid\)s:
%\begin{multline*}
%  \pshs_{\cid_0, \pid_0} = \{(\cid, t, l, \pid, \wid)\in \pshs\mid \cid = 
%  \cid_0 \land \pid = \pid_0\}.
%\end{multline*}

% proof
% proofs
\begin{definition}[Participation proof]%
  \label{DefParticipationProof}\label{DefParticipationProofs}
  We denote by \(\prfs_\prtst = \pshs_\prtst/=_\pid\) the \emph{set of all 
  proofs of participation for the protest~\(\prtst\)}.
  The \emph{participation proof} of a protester with pseudonym~\(i\) who 
  participates in a protest~\(\prtst\) is the set
  \begin{equation}
    \nonumber
    \prf_{i, \prtst} =
    \left\{ (\cid, t, l, \pid, \wid) \in \pshs_\prtst \mid
      \pid = i\right\},
  \end{equation}
  of all proof shares with the same protester and valid for any subprotest of 
  \(\prtst\).
\end{definition}

% strength

We can now define the participation count as follows.
\begin{definition}[Participation count]%
  \label{DefParticipationCount}
  We define a \emph{participation count} of a protest \(\prtst\) as the cardinality 
  \(|\prfs_{\prtst}^{\str,\theta}|\) of the set of \emph{eligible} participation proofs respectively to a strength function $\str$ and a threshold $\theta$: \[
    \prfs_{\prtst}^{\str,\theta} = \left\{ \prf_{i,\prtst} \in \prfs \mid
      \str(\prf_{i,\prtst})\geq \theta \right\}
  \] with \(\str\colon \powerset(\pshs)\to \RR_+\) and  \(\theta \in \RR_+\).
\end{definition}

The strength function \(\str\) can be used to regulate the trust in the 
estimated participation count.
\daniel{%
  The strength function can take the size of the time interval into account 
  too, smaller time intervals around the event yields stronger proofs.
}
In general, \(\str\) can be defined as a weighted sum of the proof shares, 
\(\str = \sum \omega_i s_i\), with the weights \(\omega_i\) being the trust in 
the witness corresponding to the proof share \(s_i\), and the threshold 
\(\theta\) represents the total trust needed to accept a participant as valid.
One example would be to set all weights to 1 for \(\str\) to return the number of unique witnesses and thus let \(\theta\) to be the threshold of the number of required witnesses.
Another possibility would be to also have a particular type of witness, called 
\emph{trusted witness}, participating in the protest. For instance, the role of 
the trusted witness could be taken by the independent journalist Jane.
In this situation, the weights would be 1 for trusted witnesses and 
0 for any other witness, and
setting \(\theta = 1\) would require at least one proof share issued by a 
trusted witness.
Finally, both approaches can be combined by giving a weight of 1 to all 
non-trusted witnesses and a weight of \(\theta\) to trusted witnesses. This 
results in a participant being eligible if they are witnessed either by \(\theta\) 
non-trusted witness or by one trusted witness.

\simon{check that the description in section V corresponds}
