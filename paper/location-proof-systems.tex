\paragraph{Location-proof systems}%
\label{location-proof-systems}

In a nutshell, \iac{LP} is a digital certificate attesting that someone was at 
a particular location at a specific moment in time.
\Iac{LPS} is an architecture by which users can obtain \acp{LP} from 
neighboring witnesses (\eg trusted access points or other users) that can later 
be shown to verifiers who can check the validity of a particular 
proof~\cite{luo2010veriplace,zhu2011applaus}.
Most of the existing approaches to \acp{LP} require the prover and the 
witnesses to disclose their identities, thus raising many privacy issues such 
as the possibility of tracing the movements of users of the \ac{LPS}.
However, some \acp{LPS}, such as PROPS~\cite{PROPS}, exist that provide strong 
privacy guarantees along with the possibility of verifying the claim of the 
location.

Our work is very related to \acp{LPS}.
The crucial differences are how the \ac{LP} is used and the more adverse 
environment.
CROCUS must tie the cause of the protest to the location proof to designate the 
proof to the protest in which the protester participated.
Due to the adverse environment, \CROCUS cannot provide any guarantees using the 
decentralized, untrusted witnesses that, \eg, PROPS can.
The incentives to cheat are also bigger and consequently the thresholds for 
collusion are much higher.

In PASPORT~\cite{PASPORT}, the verifier is trusted and also known in advance.
This does not work with universal verifiability.
(The verifier chooses the subset of witnesses that can vouch for a prover. 
Which makes it an interactive protocol. And the verifier must know which 
witnesses are on location in advance.)

\Textcite{ProofOfWitnessPresence} does not treat the adverse setting and, 
particularly, the problem of \ac{DB} (see \cref{distance-bounding}).
