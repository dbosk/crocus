\subsection{Privacy}

We start with \cref{PseudonymUnlink}.
Given \(\pid\), \cref{in-system-adversary} cannot distinguish whether \(\pid = 
  \ACprf[_{\sk_A}][\cid]\) or \(\pid = \ACprf[_{\sk_B}][\cid]\) due to the 
properties of \(\AC\) (see~\cite{HowToWinTheCloneWars}).
This is not necessarily true for \cref{deanonymizing-storage-adversary}:
if everyone uploads their own proof shares\footnote{%
  This is probable due to the proof-of-correctness, \(\corr_\pid\), which is 
  likely computed after the protest has ended.
}, \cref{deanonymizing-storage-adversary} will link \(P\) and \(\pid\), and 
consequently \(P\) and \(\cid\).

\Cref{ProtestUnlink}, Alice's proofs must be unlinkable across protests.
This also follows from the properties of 
\(\ACprf\)~\cite{HowToWinTheCloneWars}: \(\pid =  \ACprf[_{\sk}][\cid]\) and 
\(\pid' = \ACprf[_{\sk}][\cid']\) (where \(\cid \neq \cid'\)) are unlinkable 
from the perspective of \cref{in-system-adversary}.
The argument is the same for \cref{WitnessUnlink}.
The same is not true for \cref{deanonymizing-storage-adversary}: since 
\cref{deanonymizing-storage-adversary} can link \(\pid, P\) and \(\pid', P\); 
then \cref{deanonymizing-storage-adversary} can link \(\pid, \pid'\) too, thus 
violating \cref{ProtestUnlink}.
In the same way, \cref{deanonymizing-storage-adversary} can violate 
\cref{WitnessUnlink}.

The proof shares consist of more than only \(\cid, \pid, \wid\); more 
specifically, each proof share is a tuple \((\cid, \pid, \wid, t_s, t_s', l, 
  \corr_\pid, \corr_\wid)\).
We assume that the location \(l\) is coarse enough so that many non-overlapping 
\((\pid, \wid)\)-pairs use the same location.
After all, it is the location of the protest, not the location within the 
protest that is interesting.
Thus \(l\) is not uniquely identifying any individual protester or witness.
Likewise, thanks to the constraints in the time-stamp granularity,
\(t\) is not uniquely identifying either.
