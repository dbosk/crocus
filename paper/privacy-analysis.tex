\subsection{Privacy}

We start with \cref{PseudonymUnlink}.
Given \(\pid\), \cref{in-system-adversary} cannot distinguish whether \(\pid = 
  \ACprf[_{\sk_A}][\cid]\) or \(\pid = \ACprf[_{\sk_B}][\cid]\) due to the 
properties of \(\AC\) (see~\cite{HowToWinTheCloneWars}).
This is not necessarily true for \cref{deanonymizing-storage-adversary}:
if everyone uploads their own proof shares\footnote{%
  This is probable due to the proof-of-correctness, \(\corr_\pid\), which is 
  likely computed after the protest has ended.
}, \cref{deanonymizing-storage-adversary} will link \(P\) and \(\pid\), and 
consequently \(P\) and \(\cid\).

\Cref{ProtestUnlink}, Alice's proofs must be unlinkable across protests.
This also follows from the properties of 
\(\ACprf\)~\cite{HowToWinTheCloneWars}: \(\pid =  \ACprf[_{\sk}][\cid]\) and 
\(\pid' = \ACprf[_{\sk}][\cid']\) (where \(\cid \neq \cid'\)) are unlinkable 
from the perspective of \cref{in-system-adversary}.
This is not true for \cref{deanonymizing-storage-adversary}: since 
\cref{deanonymizing-storage-adversary} can link \(\pid, P\) and \(\pid', P\); 
then \cref{deanonymizing-storage-adversary} can link \(\pid, \pid'\) too, thus 
violating \cref{ProtestUnlink}.
In the same way, \cref{deanonymizing-storage-adversary} can violate 
\cref{WitnessUnlink}.

However, the statement about \cref{ProtestUnlink,in-system-adversary} is not 
necessarily true for \cref{WitnessUnlink,in-system-adversary}.
The proof shares consist of more than only \(\cid, \pid, \wid\); more 
specifically, each proof share is a tuple \((\cid, \pid, \wid, t_s, t_s', l, 
  \corr_\pid, \corr_\wid)\).
Let \(\cid\) be fixed, then \(\pid\) is a unique identifier for \(P\).

We assume that the location \(l\) is coarse enough so that many non-overlapping 
\((\pid, \wid)\)-pairs use the same location.
Thus \(l\) is not unique for any individual (protester or witness).
This implies some degree of \(k\)-anonymity for the location \(l\).

Since \(\pid\) uniquely identifies \(P\), then if \(t_s^{(\pid)}\) is 
\enquote{unique enough}\footnote{%
  This can be verified thanks to \cref{CountOnce}.
  If there are no tuples \((\pid, t_s)\) and \((\pid, t_s')\) among the proof 
  shares such that \(\pid \neq \pid'\) but \(t_s = t_s'\), then \(t_s\) also 
  \emph{likely} uniquely identifies \(P\).
  \enquote{Likely} depends on the likelihood of there being witnesses who are 
  not protesters.
}, every \(\wid\) for which \(t_s' = t_s^{(\pid)}\) is \emph{likely} to be 
\(P\) acting as a witness.
This would allow \cref{in-system-adversary} to link \(\wid, \wid'\) for 
different protesters, thus violating \cref{WitnessUnlink}.
(The same can be said of \(\pid\) and \cref{ProtestUnlink} if there are 
multiple protests, \ie \(\cid\)'s, at the same time and location.
However, we deem this as unlikely.)

Distribution of \(t_s\) depends on the rate of \(\TSget\) and the distribution 
of when participants run the algorithm the last time.
For example, if we use a six-hour interval and a rate of one new value per 
minute, then there are potentially 21600 unique values for \(t_s\).
However, we can scale this down by a factor \(n\) by requiring \(\TSget\) to 
use only every \(n\)th block from the underlying ledger.
Thus it seems reasonable that \(t_s\) will never be \enquote{unique enough}.
