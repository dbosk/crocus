\subsection{Privacy}

We start with \cref{PseudonymUnlink}.
Given \(\pid\), \cref{in-system-adversary} cannot distinguish whether \(\pid = 
  \ACprf[_{\sk_A}][\cid]\) or \(\pid = \ACprf[_{\sk_B}][\cid]\) due to the 
properties of \(\AC\) (see~\cite{HowToWinTheCloneWars}).
This is not necessarily true for \cref{deanonymizing-storage-adversary}:
if everyone uploads their own proof shares\footnote{%
  This is probable due to the proof-of-correctness, \(\corr_\pid\), which is 
  likely computed after the protest has ended.
}, \cref{deanonymizing-storage-adversary} will link \(P\) and \(\pid\), and 
consequently \(P\) and \(\cid\).

\Cref{ProtestUnlink}, Alice's proofs must be unlinkable across protests.
This also follows from the properties of 
\(\ACprf\)~\cite{HowToWinTheCloneWars}: \(\pid =  \ACprf[_{\sk}][\cid]\) and 
\(\pid' = \ACprf[_{\sk}][\cid']\) (where \(\cid \neq \cid'\)) are unlinkable 
from the perspective of \cref{in-system-adversary}.
This is not true for \cref{deanonymizing-storage-adversary}: since 
\cref{deanonymizing-storage-adversary} can link \(\pid, P\) and \(\pid', P\); 
then \cref{deanonymizing-storage-adversary} can link \(\pid, \pid'\) too, thus 
violating \cref{ProtestUnlink}.
In the same way, \cref{deanonymizing-storage-adversary} can violate 
\cref{WitnessUnlink}.

However, the statement about \cref{ProtestUnlink,in-system-adversary} is not 
necessarily true for \cref{WitnessUnlink,in-system-adversary}.
The proof shares consist of more than only \(\cid, \pid, \wid\); more 
specifically, each proof share is a tuple \((\cid, \pid, \wid, t_s, t_s', l, 
  \corr_\pid, \corr_\wid)\).
Let \(\cid\) be fixed, then \(\pid\) is a unique identifier for \(P\).

We assume that the location \(l\) is coarse enough so that many non-overlapping 
\((\pid, \wid)\)-pairs use the same location.
Thus \(l\) is not unique for any individual (protester or witness).
This implies some degree of \(k\)-anonymity for the location \(l\).

Likewise, thanks to the constraints in the time-stamp granularity,
\(t\) is not unique.

Overall, informally, the digital trace needed for verifiability in \CROCUS could
conceivably be used to strengthen a suspicion of
participation. Given A2's (Grace's) inferences from auxiliary information that points
to Alice having participated in a protest, for example pictures at the
event, with face recognition and subsequent identification, IP address
of her home router, etc., then any remaining uncertainty from false
positives could be reduced by learning from data on the ledger and
Alice's smartphone. %This potential difference can thus be considered
                    %the price for verifiability.
For confirmation of that suspicion, Grace would have to get access to
and use Alice's private key.