\subsection{Privacy}

We start with \cref{ProtestUnlink}, that Alice's proofs must be unlinkable across protests but linkable within one protest due to \cref{CountOnce}.
It follows from the properties of \(\ACprf\) that \(\pid =  \ACprf[_{\sk_P}][\cid]\) and \(\pid' = \ACprf[_{\sk_P}][\cid']\) are unlinkable if \(\cid\neq \cid'\).

Afterwards to get the property \cref{ProofUnlink}.
Given \(\pid\), Grace cannot distinguish whether \(\pid = \ACprf[_{\sk_A}][\cid]\), \(\pid = \ACprf[_{\sk_A}][\cid']\) or \(\pid = \ACprf[_{\sk_B}][\cid]\) due to the properties of \(\AC\).

However, consider the situation in which the witness works for Grace.
This witness interacts with Alice directly and might have (physically) identified her.
The witness also learns Alice's \(\pid\), but not \(\cid\), to compute \(\wid\) and to verify the \ac{PPK}.
\daniel{This assumes that I will do the changes to hide \(\cid\) from the witness in the \ac{PPK}.}
Afterwards, Alice publishes \(\pid\) and \(\cid\) together as part of the proof 
share.
The probability of a colliding \(\pid = \ACprf[_\sk][\cid']\) for some \(\cid' 
  \neq \cid\) and \(\sk \neq \sk_P\) is \enquote{small}.
\daniel{Fix this.}
If this occurs, Grace could link \(\cid\) (and thus the opinions in the manifesto) to Alice.

\sonja{anonymity and cause unlinkability missing}
