\subsection{Zero-knowledge proofs of knowledge}%
\label{ZKPK}

Briefly describe zero-knowledge proofs and what type of properties can be proved 
using them, e.g.\ algebraic structures but not one-way functions.

We will use the notation introduced by \citet{PKnotation}:
\begin{equation}\label{PKexample}
  \PK{\alpha, \beta, \gamma}
  {y = g^\alpha h^\beta \land y' = \hat g^\gamma}
\end{equation}
which means that we prove knowledge of \(\alpha, \beta, \gamma\) ensuring that 
\(y, y'\) are of the form \(y = g^\alpha h^\beta\) and \(y' = \hat g^\gamma\), 
respectively.
Greek letters are known only to the prover and for which the prover wishes to 
prove knowledge, all other letters are known by the verifier.

When a proof of knowledge is turned into a signature using the Fiat-Shamir 
heuristic~\cite{FiatShamirHeuristic}, we will denote it as
\begin{equation*}\label{SPKexample}
  \sigma\gets \SPK{\alpha, \beta, \gamma}
  {y = g^\alpha h^\beta \land y' = \hat g^\gamma}
  {m},
\end{equation*}
which yields a signature \(\sigma\) on \(m\), ensuring that the issuer knew 
\(\alpha, \beta, \gamma\) such that \(y = g^\alpha h^\beta\) and \(y' = 
  \hat{g}^\gamma\).


\paragraph{Implementations}

There are various implementations.
We will rely on the Schnorr identification scheme~\cite{Schnorr}.

\NewAlgorithm{\PKprove}{PK.\!Prove}
\NewAlgorithm{\PKverify}{PK.\!Verify}

The notation in \cref{PKexample} could equivalently be seen as
\begin{equation*}
  \PKprove[g,h,y,y',q,\alpha,\beta,\gamma] \leftrightarrow
  \PKverify[g,h,y,y',q],
\end{equation*}
where \(\PKprove\) is run by the prover and \(\PKverify\) is run by the 
verifier.
\(\PKverify\) outputs accept (1) or reject (0).

