\subsection{Zero-knowledge proofs of knowledge}%
\label{ZKPK}

Briefly describe zero-knowledge proofs and what type of properties can be proved 
using them, e.g.\ algebraic structures but not one-way functions.

We will use the notation introduced by \citet{PKnotation}:
\begin{equation*}
  \PK{\alpha, \beta, \gamma}
  {y = g^\alpha h^\beta \land y' = \hat g^\gamma}
\end{equation*}
which means that we prove knowledge of \(\alpha, \beta, \gamma\) ensuring that 
\(y, y'\) are of the form \(y = g^\alpha h^\beta\) and \(y' = \hat g^\gamma\), 
respectively.
When a proof of knowledge is turned into a signature using the Fiat-Shamir 
heuristic~\cite{FiatShamirHeuristic}, we will denote it as
\begin{equation*}
  \SPK{\alpha, \beta, \gamma}
  {y = g^\alpha h^\beta \land y' = \hat g^\gamma}
  {m},
\end{equation*}
which yields a signature on \(m\), ensuring that the issuer knew \(\alpha, 
  \beta, \gamma\) such that \(y = g^\alpha h^\beta\) and \(y' = \hat g^\gamma\).


\paragraph{Implementations}

There are various implementations.
