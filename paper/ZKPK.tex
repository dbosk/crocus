\subsection{Zero-knowledge proofs of knowledge}%
\label{ZKPK}

Briefly describe zero-knowledge proofs and what type of properties can be proved 
using them, e.g.\ algebraic structures but not one-way functions.

We will use the notation introduced by \citet{PKnotation}:
\begin{equation}\label{PKexample}
  \PK{\alpha, \beta, \gamma}
  {y = g^\alpha h^\beta \land y' = \hat g^\gamma}
\end{equation}
which means that we prove knowledge of \(\alpha, \beta, \gamma\) ensuring that 
\(y, y'\) are of the form \(y = g^\alpha h^\beta\) and \(y' = \hat g^\gamma\), 
respectively.
Greek letters are known only to the prover and for which the prover wishes to 
prove knowledge, all other letters are known by the verifier.

When a proof of knowledge is turned into a signature using the Fiat-Shamir 
heuristic~\cite{FiatShamirHeuristic}, we will denote it as
\begin{equation*}\label{SPKexample}
  \sigma\gets \SPK{\alpha, \beta, \gamma}
  {y = g^\alpha h^\beta \land y' = \hat g^\gamma}
  {m},
\end{equation*}
which yields a signature \(\sigma\) on \(m\), ensuring that the issuer knew 
\(\alpha, \beta, \gamma\) such that \(y = g^\alpha h^\beta\) and \(y' = 
  \hat{g}^\gamma\).


\subsection{Instantiations}

\NewAlgorithm{\PKprove}{PK.\!Prove}
\NewAlgorithm{\PKverify}{PK.\!Verify}

We will rely on the Schnorr identification scheme~\cite{Schnorr}, which in its 
original form can be written as \(\PK{\alpha}{A = g^\alpha}\).
The generalized form would be \(\PK{\alpha_1, \dotsc, \alpha_n}{A = 
    \prod_{i=1}^n g_i^{\alpha_i}}\).
This could equivalently be written as
\begin{equation*}
  \PKprove[\{g_i\}_i, q, A, \{\alpha_i\}_i] \leftrightarrow
  \PKverify[\{g_i\}_i, q, A],
\end{equation*}
where \(\PKprove\) is run by the prover and \(\PKverify\) is run by the 
verifier.
\(\PKverify\) outputs accept (1) or reject (0).
We give an instance of \(\PKprove\) and \(\PKverify\) in \cref{PKalgorithms}.
We also note that
\begin{equation*}
  \PK{\alpha, \beta}{A = g^\alpha \land B = g^\beta} = \PK{\alpha}{A = g^\alpha} 
  \land \PK{\beta}{B = g^\beta},
\end{equation*}
however, the challenge \(c\) must be the same in both sub-protocols.

\begin{figure*}
  \centering
  \begin{tabular}{lcl}
    \(\PKprove[\{g_i\}_{i=1}^n, q, A, \{\alpha_i\}_{i=1}^n]\):
    &
    & \(\PKverify[\{g_i\}_{i=1}^n, q, A]\):
    \\
    \midrule

    \(\forall i, 1\leq i\leq n\colon \rho_i\rgets \Z_q\)
    &
    &
    \\

    \(R\gets \prod_{i=1}^n g_i^{\rho_i}\)
    & \(\ProtoSendRight{R}\)
    &
    \\

    % null
    & \(\ProtoSendLeft{c}\)
    & \(c\rgets \bin^k\)
    \\

    \(1\leq i\leq n\colon s_i\gets \rho_i - c\alpha_i \mod q\)
    & \(\ProtoSendRight{\{s_i\}_{i=1}^n}\)
    &
    \\

    % null
    &
    & \(R \stackrel{?}{=} A^c \prod_{i=1}^n g^{s_i}\)
    \\
    
  \end{tabular}
  \caption{%
    \(\PK{\alpha_1, \dotsc, \alpha_n}{A = \prod_{i=1}^n g_i^{\alpha_i}}\) using 
    the Schnorr identification scheme.
  }%
  \label{PKalgorithms}
\end{figure*}

The Schnorr protocol in itself only provides honest-verifier zero-knowledge.
To achieve malicious-verifier zero-knowledge, we have to choose \(k\) (size of 
the challenge, see \cref{PKalgorithms}) logarithmic in the security parameter 
\(\lambda\) and repeat the protocol sufficiently many times (also logarithmic in 
the security parameter).

\NewAlgorithm{\SPKprove}{SPK.\!Prove}
\NewAlgorithm{\SPKverify}{SPK.\!Verify}

For the non-interactive instance, \(\SPK{\alpha}{A = g^\alpha}{m}\), we use the 
Fiat-Shamir heuristic~\cite{FiatShamirHeuristic} on the above protocol, 
illustrated as \(\SPKprove\) and \(\SPKverify\) in \cref{SPKalgorithms}.

\begin{figure*}
  \begin{minipage}[t]{0.49\linewidth}
    \begin{algorithmic}
      \Function{$\SPKprove$}{$\{g_i\}_i, q, A, \{\alpha_i\}_i, m$}
        \State{\(\forall i, 1\leq i\leq n\colon \rho_i\rgets \Z_q\)}
        \State{\(R\gets \prod_{i=1}^n g_i^{\rho_i}\)}
        \State{$c \gets H(\{g_i\}_i|A|R|m)$}
        \For{$1\leq i\leq n$}
          \State{$s_i \gets \rho_i - c\alpha_i \mod q$}
        \EndFor{}
        \State{\Return{$(\{s_i\}_i, c)$}}
      \EndFunction{}
    \end{algorithmic}
  \end{minipage}
  \hfill
  \begin{minipage}[t]{0.49\linewidth}
    \begin{algorithmic}
      \Function{$\SPKverify$}{$\{g_i\}_i, q, A, m, \{s_i\}_i, c$}
        \State{$\hat R \gets A^c \prod_{i=1}^n g_i^{s_i}$}
        \If{$c = H(\{g_i\}_i|A|\hat R|m)$}
          \State{\Return{1}}
        \Else{}
          \State{\Return{0}}
        \EndIf{}
      \EndFunction{}
    \end{algorithmic}
  \end{minipage}
  \caption{%
    \(\SPK{\alpha_1, \dotsc, \alpha_n}{A = \prod_{i=1}^n g_i^{\alpha_i}}{m}\) 
    using the Fiat-Shamir heuristic on the Schnorr identification scheme.
  }%
  \label{SPKalgorithms}
\end{figure*}
