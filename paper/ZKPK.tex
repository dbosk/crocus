\subsection{Zero-knowledge proofs of knowledge}%
\label{ZKPK}

We will use the notation introduced by \citet{PKnotation}:
\begin{equation}\label{PKexample}
  \PK[\alpha, \beta, \gamma][y = g^\alpha h^\beta \land y' = \hat g^\gamma]
\end{equation}
which means that we prove knowledge of \(\alpha, \beta, \gamma\) ensuring that 
\(y, y'\) are of the form \(y = g^\alpha h^\beta\) and \(y' = \hat g^\gamma\), 
respectively.
Greek letters are known only to the prover and for which the prover wishes to 
prove knowledge, all other letters are known by the verifier.

When a proof of knowledge is turned into a signature using the Fiat-Shamir 
heuristic~\cite{FiatShamirHeuristic}, we will denote it as
\begin{equation*}\label{SPKexample}
  \sigma\gets \SPK[\alpha, \beta, \gamma]%
  [y = g^\alpha h^\beta \land y' = \hat g^\gamma]%
  [m],
\end{equation*}
which yields a signature \(\sigma\) on \(m\), ensuring that the issuer knew 
\(\alpha, \beta, \gamma\) such that \(y = g^\alpha h^\beta\) and \(y' = 
  \hat{g}^\gamma\).


