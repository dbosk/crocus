\section{Formal Verification}\label{formal-verification}

We formally verify the distance-bounding protocol security properties using the Tamarin
prover~\cite{meier2013tamarin}. Tamarin does not originally include
the possibility to reason on time and space in its model.  However,
thanks to a recent work by \textcite{TamarinDB}, the common properties
(mafia fraud, and distance fraud/distance hi-jacking) can be equivalently 
characterized with a causality-based definition relying only on the order
of messages.
Another work by the same team \textcite{TamarinDBTF} also introduces a
notion of post-collusion security that can be used to partially automate
proofs regarding terrorist fraud, provided a manual proof of a least-disclosing
message needed for collusion.


The Tamarin code we wrote\footnote{provided in \url{https://doi.org/10.5281/zenodo.3709442}}%,
 contains a specification of the distance-bounding protocol described previously, a definition 
 of the soundness and correctness properties that ensure the protocol actually produces the
 expected result, and a definition of the security properties we want to ensure.

Tamarin's auto-prover validated the soundness and correctness
properties and proved all the security
properties used in~\cite{TamarinDB} and \textcite{TamarinDBTF}: resistance to distance fraud, distance hi-jacking, mafia fraud, and terrorist fraud, all in less than 10 seconds.

Note that proofs of formal verification are always limited by how well the specification captures reality.
In particular, Tamarin only provides symbolic modeling, and as such does not work with actual values. The tool, for instance, has no notion of arithmetic, and the equations used in \cref{SchnorrFigure} cannot be modeled directly. Similarly, the details of the bit-by-bit version are lost in the abstraction, and its specification is identical to the simultaneous version; that's why Tamarin does not detect the terrorist fraud attack we mention about the bit-by-bit version.
Notable differences can be found in \cref{formal-verif-appendix}.
