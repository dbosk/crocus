\section{Formal Verification}\label{formal-verification}

We formally verify the protocol security properties using the Tamarin
prover~\cite{meier2013tamarin}. Tamarin does not originally include
the possibility to reason on time and space in its model.  However,
thanks to a recent work by \textcite{TamarinDB}, the common properties
can be equivalently characterized with a causality-based definition
relying only on the order of messages.


The Tamarin code we wrote\footnote{provided in \url{https://doi.org/10.5281/zenodo.3709442}}%, with a short discussion of the modeling choices we made. 
 contains a specification of the distance-bounding protocol described previously, a definition of the soundness and correctness properties that ensure the protocol actually produces the expected result, and a definition of the security properties we want to ensure.

Tamarin's auto-prover validated the soundness and correctness
properties (in negligible computation time) and proved the security
properties used in~\cite{TamarinDB} and
\textcite{TamarinDBTF}. The latter introduced the notion of post-collusion
security which enabled them to prove \ac{DBTF}. 

Note that proofs of formal verification are always limited by how well the specification captures reality. The abstraction does not capture the details of the bit-by-bit version that prevent \ac{DBTF} resistance. 



