\section{Current crowd counting methods}%
\label{current-crowd-counting}

The seemingly most common method for counting crowds at protests is \emph{Jacobs's method}~\cite{2016DemonstrationsInSeoul,BBCHowToCountProtestNumbers,HowWillWeKnowTrumpInauguralCrowdSize,TheXManMarch,TheCrowdNumbersGame}.
This manual method devised in the 1960s relies on aerial pictures of the event.
The verifier divides the protest venue into regions and then estimates the density of the crowd in the different regions before summing them up to get an estimate of the global count. Using pictures makes it difficult to get cumulative counts, verify that the pictures have not been manipulated, and to have both privacy and individual verifiability: either one is included in the picture (privacy problem) or not (verifiability problem). Similar limitations exist for estimating the number of persons in a picture or video (\eg the work of \cite{NNCrowdCounting} or  CrowdSize~\cite{CrowdSize}).

Another problem for all the above methods is exemplified by the demonstrations in Seoul:
\blockcquote{2016DemonstrationsInSeoul}{%
  \textins*{t}he demonstrators not only gather in open space, but also small alleys and between buildings%
}.
In this situation it is very difficult to faithfully capture the situation.
Taking pictures from different angles risks double counts. Another challenges is determining whether people near the event's perimeter are participants or simply bystanders~\cite{HowToEstimateCrowdSize}.

Counting MAC addresses, as done by a company during the protests in Seoul~\cite{2016DemonstrationsInSeoul} suffers from MAC randomization, though some tracking of smartphones could still be possible with a different method~\cite{WhyMACRandomizationIsNotEnough} or using \emph{IMSI catchers}; none of which is verifiable.

An approach that relies on a trusted infrastructure was recently deployed by a collection of media outlets to count protesters passing the line defined by a trusted sensor on marches~\cite{LeMondeProtestingSolution}. 
This solution does not offer strong verifiability guarantees and thus is complemented by micro-counts made by humans to estimate their margin of error.

CrowdCount \citet{CrowdCount} is a web service that lets Alice create an event such that anyone can submit their location to register that they are in Alice's event.
Another related approach based on devices is UrbanCount~\cite{UrbanCount}, which relies on epidemic spreading of crowd-size estimates by device-to-device communication to count crowds in dense urban environments with high node-mobility and churn.
However, there is no consideration of a potentially adversarial setting and thus no verifiability or checks on eligibility.  DiVote~\cite{DiVote}, a prior work by the same authors for polling in dense areas, avoids double counting, but again only works with honest participants and thus does not suit an adversarial setting.

