\subsection{Desired properties}%
\label{desired-properties}

We note that, in general terms, protests are petitions with a given time and 
location.
Protests, petitions and elections share that in all three many individuals 
express their opinion.
These opinions can be sensitive (\eg be a cause for discrimination or 
persecution).
For that reason we have strong requirements for verification and privacy for 
elections, it follows that we should have similar properties for protests
and petitions.

We draw inspiration from properties for voting systems as formalized
in ~\cite{VerifyingPrivacyPropertiesOfVotingProtocols}:
\daniel{Find a new ref for voting verifiability properties, Douglas commented 
  that the last one is not the origin.}
\begin{description}
  \item[Eligibility:] anyone can verify that each cast vote is legitimate.
  \item[Universal verifiability:] anyone can verify that the result is according to the cast votes.
    \daniel{Douglas: technically this is Sako-Kilian, non-technically, it's 
      old.}
  \item[Individual verifiability:] each voter can verify that their vote is included in the result.
    \daniel{Douglas: this is much earlier than Sako-Kilian, in essence 
      Fiat-Shamir (technically). Non-technically, much older.}
\end{description}
In our context, votes are translated into \emph{participation proofs}.
Universal and individual verifiability remain the same, in the sense that anyone can verify the participation count by counting the proofs and a participant can verify their proof is included.
The eligibility requirement is slightly different as for protests it must also
include temporal %\footnote{%
%   Elections also have some temporal aspects, such that one cannot vote at any 
%   time and a vote in one election should not be reusable in the next election.
%   However, the temporal relation for a protest is not as strictly defined as for 
%   elections as a protest can start at any time and last for an arbitrary period 
%   of time.%
% }
and spatial eligibility (\ie each participation proof satisfies some temporal 
and spatial relation to the protest).
In essence, the proof must bind the person to the time and location of the protest.
(This is the difference to a petition.)

In ~\cite{VerifyingPrivacyPropertiesOfVotingProtocols}, the main three
privacy properties for voting protocols are given as:
\begin{description}
  \item[Vote privacy:] the voting does not reveal any individual vote.
  \item[Receipt freeness:] the voting system does not provide any data that can be used as a proof of how the voter voted.
  \item[Coercion resistance:] a voter cannot cooperate with a coercer to prove their vote was cast in any particular way.
\end{description}

%Coercion resistance is not possible to achieve for protests.
Coercion resistance in voting typically relies on physical isolation
(\eg private voting booths), including for digital systems, 
and that is by definition not possible for public events.
For instance, someone could simply physically bring Alice to a protest against her will.
As for receipt freeness, while
desirable \emph{in itself}, it implies a conflict with verifiability in our context:
%appealing at first glance, is not desirable (probably not possible)
%in our context since this will conflict with verifiability:
in contrast to voting, receipt freeness for \emph{how} the voter voted (\ie the 
cause of the protest) here implies receipt freeness for \emph{that} the voter voted 
(\ie the protester was there), which would make verifiability impossible.

Therefore in our context, the crucial property is  vote privacy.
More precisely, for the protester we want unlinkability (from the adversary's 
perspective) between a protester's real identity \(P\) and the participation 
proof (and thus also the protest itself).
Phrased differently, given a participation proof, Grace should not be able to 
distinguish if it was Alice or Bob who participated.
Furthermore, if Grace has managed to link one proof to Alice due to some 
auxiliary knowledge, she should not be able to link it to another proof (from a different protest).

\subsection{Current crowd-counting methods}

The seemingly most common method for counting crowds at protests is \emph{Jacobs's method}~\cite{2016DemonstrationsInSeoul,BBCHowToCountProtestNumbers,HowWillWeKnowTrumpInauguralCrowdSize,TheXManMarch,TheCrowdNumbersGame}.
This manual method devised in the 1960s relies on aerial pictures of the event.
The verifier divides the protest venue into regions and then estimates the density of the crowd in the different regions before summing them up to get an estimate of the global count. Using pictures makes it difficult to get cumulative counts, verify that the pictures have not been manipulated, and to have both privacy and individual verifiability: either one is included in the picture (privacy problem) or not (verifiability problem). Similar limitations exist for estimating the number of persons in a picture or video (\eg the work of \cite{NNCrowdCounting} or  CrowdSize~\cite{CrowdSize}).

Another problem for all the above methods is exemplified by the demonstrations in Seoul:
\blockcquote{2016DemonstrationsInSeoul}{%
  \textins*{t}he demonstrators not only gather in open space, but also small alleys and between buildings%
}.
In this situation it is very difficult to faithfully capture the situation.
Taking pictures from different angles risks double counts. Another challenges is determining whether people near the event's perimeter are participants or simply bystanders~\cite{HowToEstimateCrowdSize}.

Counting MAC addresses, as done by a company during the protests in Seoul~\cite{2016DemonstrationsInSeoul} suffers from MAC randomization, though some tracking of smartphones could still be possible with a different method~\cite{WhyMACRandomizationIsNotEnough} or using \emph{IMSI catchers}; none of which is verifiable.

An approach that relies on a trusted infrastructure was recently deployed by a collection of media outlets to count protesters passing the line defined by a trusted sensor on marches~\cite{LeMondeProtestingSolution}. 
This solution does not offer strong verifiability guarantees and thus is complemented by micro-counts made by humans to estimate their margin of error.

CrowdCount \citet{CrowdCount} is a web service that lets Alice create an event such that anyone can submit their location to register that they are in Alice's event.
Another related approach based on devices is UrbanCount~\cite{UrbanCount}, which relies on epidemic spreading of crowd-size estimates by device-to-device communication to count crowds in dense urban environments with high node-mobility and churn.
However, there is no consideration of a potentially adversarial setting and thus no verifiability or checks on eligibility.  DiVote~\cite{DiVote}, a prior work by the same authors for polling in dense areas, avoids double counting, but again only works with honest participants and thus does not suit an adversarial setting.

