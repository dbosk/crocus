\section{Performance and implementation}%
\label{PerformanceAnalysis}

\subsection{Implementation on smartphones or smart-cards}
Technological progress has enabled more advanced cryptography on Smart-cards and Smartphones which could be used to implement our solution : Benchmarks~\cite{Benchmarking} have shown that Android devices are now fast enough to efficiently implement Privacy Enhancing Technologies (PETs), with a Samsung Galaxy S i9000 being able to execute Idemix in  \SI{153}{\milli\second}. However, those benchmarks also demonstrate that SmartCards remain slow to process complex protocols such as Idemix or U-Prove (taking between \SI{4}{\second} and \SI{8}{\second} to process them). While the limited processing power of many embarked systems has been a challenge, Idemix has been succesfully implemented to prove the posession of credentials on Java Cards by Bichsel et al. in 2009~\cite{Bichsel} and the IRMA project, released in 2014, aimed to achieve an implementation « suitable for real life transactions »~\cite{IRMA} while maintaining security and privacy for its users. 

\subsection{Threshold of witnesses}

Assume that every person has 5000 contacts in their contact 
book~\cite{DifficultyOfPrivateContactDiscovery}.
Then it would be reasonable to set the threshold of at least 6000 witnesses.
According to the Anon-Pass performance measures~\cite{AnonPass} each witness 
signature would require \SI{8}{\milli\second} per core on a quad-core Intel 
\SI{2.66}{\giga\hertz} Core 2 processor.
This yields \(
  \SI{8}{\milli\second}\times 6000 = \SI{48000}{\milli\second} = 
  \SI{48}{\second}.
\) of processing.
