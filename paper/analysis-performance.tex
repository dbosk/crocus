\section{Performance and implementation}
\label{PerformanceAnalysis}
\sonja{ Include evaluation environment, parameters, results, disclaimer about parts that cannot yet be implemented and evaluated because the hardware doesn't exist yet for DB on phones}

\subsection{Smartphones, smartcards}
Performance considerations are crucial during the protests due to the nature of the devices used to run \PRIVO, which are resource-constrained in terms of energy, storage and computational power, and are operating on limited network capacity. 
Recent technological progress has enabled the deployment of advanced cryptographic primitives on smartcards and smartphones that could be used to implement our solution.
For instance, benchmarks~\cite{Benchmarking} have shown that Android devices are now fast enough to efficiently implement Privacy Enhancing Technologies (PETs), with a Samsung Galaxy S i9000 being able to execute Idemix in \SI{153}{\milli\second}. 
However, those benchmarks also demonstrate that smartcards remain slow to process complex protocols such as Idemix or U-Prove (taking between \SI{4}{\second} and \SI{8}{\second} to process them). 
While the limited processing power of many embarked systems has been a challenge, Idemix has been successfully implemented to prove the possession of credentials on Java Cards by Bichsel and co-authors in 2009~\cite{Bichsel} and the IRMA project, released in 2014, aimed to achieve an implementation ``suitable for real life transactions''~\cite{IRMA} while maintaining security and privacy for its users. 

\subsection{Distance bounding anonymous credentials}
With respect to the distance-bounding protocol we have proposed, the initial prover phase can be split in two parts: (1) the setup phase during which $g$, $q$ and $\alpha$ are generated and $A$ is computed and (2) the online phase during which $\theta$ and $R$ are generated. The first step is only done once and is fairly costly in terms of ressources. 
This means that the best strategy is to precompute the corresponding results and to store them on the prover's device. 
This method is commonly found in Schnorr variants in order to save time on multiple instances of the protocol [1].
%Seb: to do add the reference to the following paper at the end of the sentence:
%[1] Chin, J. J., Tan, S. Y., Heng, S. H., & Phan, R. C. W. (2015). Twin-Schnorr: a security upgrade
%for the schnorr identity-based identification scheme. The Scientific World Journal, 2015.
Considering the entire setup of our protocol, we perform 7 random number generations (one of them being a single bit), two modular exponentiations ($A$ and $R$), two multiplications and two substractions (when computing $s_0$ and $s_1$). 
This can be reduced to 4 random number generations and one exponentiation using precomputation. The Anon-pass article [2] gives us a total time of 43,2 milliseconds of raw protocol operation time in order to create, sign and verify a given message (the registration phase), and then 21,4 mililseconds to login.
%[2] Lee, M., Dunn, A., Katz, J., Waters, B., & Witchel, E. (2013). Anon-pass: Practical anonymous
%subscriptions. IEEE Security & Privacy, (1), 1.

%\paragraph{Threshold of witnesses.}
%Assume that every person has 5000 contacts in their contact book~\cite{DifficultyOfPrivateContactDiscovery}.
%Then it would be reasonable to set the threshold of at least 6000 witnesses.
%According to the Anon-Pass performance measures~\cite{AnonPass} each witness 
%signature would require \SI{8}{\milli\second} per core on a quad-core Intel 
%\SI{2.66}{\giga\hertz} Core 2 processor.
%This yields \(
%  \SI{8}{\milli\second}\times 6000 = \SI{48000}{\milli\second} = 
%  \SI{48}{\second}.
%\) of processing.

\subsection{Ledger (blockchain) efficiency}
As an example, consider a protest with 1\,000\,000 participants.
If we use trusted witnesses, each participant only needs to acquire one proof share from a trusted witness.
Thus, there will be 1\,000\,000 proof shares submitted to the blockchain in total.
If we consider OmniLedger~\cite{OmniLedger}, which can do approximately 1500 transactions per second, it takes at least 11 minutes to process all the proof shares.

If we do not use trusted witnesses, but a threshold \(\theta = 1000\), then it will take at least 7 days before all transactions are committed to the blockchain. While this already takes longer than counting votes in national elections, the threshold is still very low. Assume that every person has 5000 contacts in their contact book~\cite{DifficultyOfPrivateContactDiscovery}.
If collusion were limited to participants that know each other, it could be reasonable to set the threshold of at least 6000 witnesses. In addition to the ledger computation, we need to consider the creation of the proof shares. \sonja{added the previous sentence to explain why anonpass shows up here. Shouldn't that be in the performance section? }
According to the Anon-Pass performance measures~\cite{AnonPass} each witness signature would require \SI{8}{\milli\second} per core on a quad-core Intel \SI{2.66}{\giga\hertz} Core 2 processors.
This yields \(
  \SI{8}{\milli\second}\times 6000 = \SI{48000}{\milli\second} = 
  \SI{48}{\second}.
\) of processing. For large groups organizing online to collectively pretend to participate at a physical protest, the threshold would have to be much higher. Considering that online forums can facilitate mass coordination, the threshold approach seems not only inefficient but also potentially ineffective for verifiability.

%\subsection{Blockchain efficiency}

\sonja{removed for now: One problem can arise in the case in which there is no trusted witnesses.
In this situation, if we set the threshold too high, we need to submit more proof shares than the blockchain can handle within a reasonable time.}


%assumptions about smartphones:
%(This means that we essentially provide a lower bound for the participation 
%count, since some participants might not have such a device.)
%Sonja says: is this still true? Given mafia fraud we can't guarantee
%that, right?