\section{Practical Considerations}%
\label{Practical}

We now briefly list some issues for deploying
CROCUS, regarding implementability and realistic adversaries\footnote{see the long version of the paper for a more thorough
discussion of these topics. }, and then analyze its performance. 

\subsection{Implementability}

Some of the assumptions that are required for implementing our proposition are 
not yet realized, however, we believe that they soon will be:
ubiquitous usage of smartphones, with an identity credential
signed by some \ac{CA}\footnote{based on \eg governmental ID on a chip as in Estonia, national
  but private as BankID in Sweden, eIDAS in the EU, or self-sovereign
  identities} ,  distance-bounding capabilities\footnote{based on \eg
  3DB-access or key-fob technology integration with phones}, and
infrastructure-independent wireless communications\footnote{based on
  \eg NFC (available but inconvenient), bluetooth or WIFI (Briar and
  FireChat have been used in protests), 5G D2D (scalable but needs
  authorization step)}.

\subsection{Realistic adversary model}
Our adversary model considers only protocol data, no auxiliary data.
Against this adversary our scheme is secure.
The question is how this adversary model maps to real
adversaries. Side-channels such as taking photos, a global passive adversary
controlling all national ISPs doing time-correlation attacks (though
may not be able to observe all ledger input or \eg Tor exit nodes), or
protocol violations in protests organized by the \ac{CA} are
not prevented.

\subsection{Performance}%
\label{PerformanceAnalysis}

\sonja{include evaluation environment, parameters, results, disclaimer about parts that cannot yet be implemented and evaluated because the hardware doesn't exist yet for DB on phones}

Performance considerations are crucial during the protests due to the nature of 
the devices used to run \CROCUS, which are resource-constrained in terms of 
energy, storage and computational power, and are operating on limited network 
capacity. 


\textbf{Smartphones and smartcards}

Recent technological progress has enabled the deployment of advanced 
cryptographic primitives on smartcards and smartphones that could be used to 
implement our solution.
For instance, benchmarks~\cite{Benchmarking} have shown that Android devices 
are now fast enough to efficiently implement \acp{PET}, with a Samsung Galaxy S 
i9000 (back in 2014) able to execute Idemix in \SI{153}{\milli\second}. 
However, those benchmarks also demonstrate that smartcards remain slow to 
process complex protocols such as Idemix or U-Prove (taking between 
\SI{4}{\second} and \SI{8}{\second} to process them). While the limited 
processing power of many embarked systems has been a challenge, Idemix has been 
successfully implemented to prove the possession of credentials on Java Cards 
by Bichsel and co-authors in 2009~\cite{Bichsel} and the IRMA project, released 
in 2014, aimed to achieve an implementation ``suitable for real life 
transactions''~\cite{IRMA} while maintaining security and privacy for its 
users. 

With respect to its implementation, \CROCUS is very similar to the 
implementation of Anon-Pass~\cite{AnonPass}, an anonymous subscription system 
in which a long-term credential can be used to derive a single login for any 
authentication window (\ie epoch) such that logins are unlinkable across 
different epochs.
In particular, the setup and registration phases are almost identical.
The evaluation conducted by \textcite{AnonPass} used as server a Dell Optiplex 
780s, which has a quad-core 2.66 GHz Intel Core 2 CPU, 8 GB of RAM and uses 
Ubuntu Linux 12.04 while the client was also simulated on a quad-core 2.66 GHz 
Intel Core 2 CPU but with only 4 GB of RAM\@.
The elliptic pairing group used in Anon-Pass is a Type A symmetric pairing group with a 160-bit group order and 512-bit base field while the ECDSA signature uses a 160-bit key.
This is typically the type of pairing that could also be used with \CROCUS\@.
Not counting the setup that is performed once by the \ac{CA}, the time reported 
for the registration on the server (\ie \ac{CA}) side is 
\SI{19.8}{\milli\second} while it is \SI{23.4}{\milli\second} on the client 
side.
With respect to joining and participation phases, with the exception of the 
distance-bounding they are actually quite similar to the login protocol of 
Anon-Pass in which the time required to create a message for a protester would 
be around \SI{13.5}{\milli\second} while the time needed for the witness would 
be only \SI{7.9}{\milli\second}.

%\subsection{Distance bounding}
%
%The distance-bounding chips~\cite{3db-access} currently available on the market 
%can already enable proofs of proximity for any range of up to 
%  200 meters.
%One of the objectives is to be able to integrate them in phones or smartcards in the near future, and even phones with off-the-shelf hardware running in RFID-emulation mode have shown promising results.
%For instance in~\cite{DBonSmartphones}, the authors have demonstrated that it is already possible to implement distance-bounding protocols as a standard Android application on existing smartphones. 
%More precisely, they have proposed three different implementations of the 
%Swiss-Knife protocol using 32 challenge-response rounds ranging from a basic 
%implementation running at the application layer to an advanced one running in 
%RFID-emulation mode. The results obtained already show that a relay attack can 
%be detected if the adversary introduces a delay of more than 1.5 ms when 
%performing such an attack (most of the existing attacks introduce delays of at 
%least tens of milliseconds).

% With respect to the distance-bounding protocol we have proposed, the initial prover phase can be split in two parts: (1) the setup phase during which $g$, $q$ and $\alpha$ are generated and $A$ is computed and (2) the online phase during which $\theta$ and $R$ are generated. 
% The first step is only done once and while being fairly costly in terms of ressources, the corresponding results can be precomputed and store them on the prover's device. 
% This method is commonly found in Schnorr variants in order to save time on multiple instances of the protocol.
%Seb: to do add the reference to the following paper at the end of the sentence:
%[1] Chin, J. J., Tan, S. Y., Heng, S. H., & Phan, R. C. W. (2015). Twin-Schnorr: a security upgrade
%for the schnorr identity-based identification scheme. The Scientific World Journal, 2015.
% Considering the entire setup of the DB protocol, it requires in its basic version to generate seven random numbers (one of them being a single bit), and also perform two modular exponentiations ($A$ and $R$), two multiplications and two substractions (when computing $s_0$ and $s_1$). 
% However, this can be reduced to four random number generations and one exponentiation using precomputation.

%\paragraph{Threshold of witnesses.}
%Assume that every person has 5000 contacts in their contact book~\cite{DifficultyOfPrivateContactDiscovery}.
%Then it would be reasonable to set the threshold of at least 6000 witnesses.
%According to the Anon-Pass performance measures~\cite{AnonPass} each witness 
%signature would require \SI{8}{\milli\second} per core on a quad-core Intel 
%\SI{2.66}{\giga\hertz} Core 2 processor.
%This yields \(
%  \SI{8}{\milli\second}\times 6000 = \SI{48000}{\milli\second} = 
%  \SI{48}{\second}.
%\) of processing.

\textbf{Witness processing}

As above, assume we have one trusted witness.
A smartphone ran Idemix in \SI{153}{\milli\second} back in 2014.
3DB-access~\cite{3db-access} says their distance-bounding part takes less than 
\SI{1}{\milli\second} and can do more than \SI{120}{\metre} (line of sight, 
maximum \SI{200}{\metre}).
Then a witness fixed in one position can cover \SI{45216}{\square\metre} 
(\SI{120}{\metre} radius), which fits 137018~protesters according to 
\textcite{2016DemonstrationsInSeoul}.
It would take this witness \SI{5.8}{\hour} to witness all of them.
Thus, to witness 1\,000\,000 protesters, it would take \SI{8}{\hour} for 5.3 
witnesses.

\textbf{Ledger (blockchain) efficiency}

For the example of 1\,000\,000 participants, with trusted witnesses, each 
participant only needs to acquire one proof share from a trusted witness.
Thus, there will be 1\,000\,000 proof shares submitted to the blockchain in total.
If we consider OmniLedger~\cite{OmniLedger}, which can do approximately 1500 transactions per second, it takes at least 11 minutes to process all the proof shares.

%If we do not use trusted witnesses, but instead a threshold \(\theta = 1000\), then it will take at least 7 days before all transactions are committed to the blockchain. 
%While this already takes longer than counting votes in national elections, the threshold is still very low. 
%% Assume that every person has 5000 contacts in their contact book~\cite{DifficultyOfPrivateContactDiscovery}.
%% If collusion were limited to participants that know each other, it could be reasonable to set the threshold of at least 6000 witnesses. 
%% In addition to the ledger computation, we need to consider the creation of the proof shares. \sonja{added the previous sentence to explain why anonpass shows up here. Shouldn't that be in the performance section? }
%% %Seb: I agree maybe we should move it to the previous subsection
%% According to the Anon-Pass performance measures~\cite{AnonPass} each witness signature would require \SI{8}{\milli\second} per core on a quad-core Intel \SI{2.66}{\giga\hertz} Core 2 processors.
%% This yields \(
%%   \SI{8}{\milli\second}\times 6000 = \SI{48000}{\milli\second} = 
%%   \SI{48}{\second}.
%% \) of processing. 
%For large groups organizing online to collectively pretend to participate at a physical protest, the threshold would have to be much higher. 
%Considering that online forums can facilitate mass coordination, the threshold approach with only untrusted witnesses seems not only inefficient but also potentially ineffective for verifiability.

%\subsection{Blockchain efficiency}

\sonja{removed for now: One problem can arise in the case in which there is no trusted witnesses.
In this situation, if we set the threshold too high, we need to submit more proof shares than the blockchain can handle within a reasonable time.}

%assumptions about smartphones:
%(This means that we essentially provide a lower bound for the participation 
%count, since some participants might not have such a device.)
%Sonja says: is this still true? Given mafia fraud we can't guarantee
%that, right?
