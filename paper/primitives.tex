\mode*
\section{Building blocks}%
\label{Primitives}\label{BuildingBlocks}

\dots

\subsection{Zero-knowledge proofs of knowledge}%
\label{ZKPK}

Briefly describe zero-knowledge proofs and what type of properties can be proved 
using them, e.g.\ algebraic structures but not one-way functions.

We will use the notation introduced by \citet{PKnotation}:
\begin{equation*}
  \PK{\alpha, \beta, \gamma}
  {y = g^\alpha h^\beta \land y' = \hat g^\gamma}
\end{equation*}
which means that we prove knowledge of \(\alpha, \beta, \gamma\) ensuring that 
\(y, y'\) are of the form \(y = g^\alpha h^\beta\) and \(y' = \hat g^\gamma\), 
respectively.
When a proof of knowledge is turned into a signature using the Fiat-Shamir 
heuristic~\cite{FiatShamirHeuristic}, we will denote it as
\begin{equation*}
  \SPK{\alpha, \beta, \gamma}
  {y = g^\alpha h^\beta \land y' = \hat g^\gamma}
  {m},
\end{equation*}
which yields a signature on \(m\), ensuring that the issuer knew \(\alpha, 
  \beta, \gamma\) such that \(y = g^\alpha h^\beta\) and \(y' = \hat g^\gamma\).


\paragraph{Implementations}

There are various implementations.

\subsection{Anonymous credentials}%
\label{ZK-anon-cred}

\NewCryptoScheme{\Idemix}{CL}
\NewFunction{\PRF}{PRF}

We will use a \(\PRF\).

\subsection{Non-interactive anonymous credentials}%
\label{NIZK-proofs}

\NewScheme{\Psig}{Psig}

\Citet{Psignatures} proposes P-signatures (\Psig), which provides a signature 
scheme and a commitment scheme with a protocol to obtain a signature on a 
committed value and two non-interactive proof systems, one for proving that a 
committed value has been signed and the other for proving that a pair of 
commitments are commitments to the same value.

\NewFunction{\PSetup}{\Psig.\!Setup}

\(params\gets \PSetup[1^\lambda]\): generates the global parameters \(params\), 
which contains parameters for the underlying algorithms.
For ease of notation we will omit \(params\) from all functions.

\NewFunction{\PCommit}{\Psig.\!Commit}

\(comm\gets \PCommit[m, open]\): takes a message \(m\) and an opening value 
\(open\) and returns a commitment.
The commitment scheme must be \emph{perfectly binding} and \emph{strongly 
  computationally hiding}.
Perfectly binding means that for every bit string \(comm\) there exists at most 
one value \(x\) such that there exists a value \(open\) with \(comm = 
  \PCommit[x, open]\).
Strongly computationally hiding means that there exists an alternate setup 
algorithm which generates indistinguishable parameters which yields 
information-theoretically hiding commitments.
I.e.\ the adversary cannot tell which setup algorithm has been used.

\NewFunction{\PObtainSig}{\Psig.\!ObtainSig}
\NewFunction{\PIssueSig}{\Psig.\!IssueSig}

\(\sigma\gets \PObtainSig[pk, m, comm, open]\leftrightarrow
  \PIssueSig[sk, comm]\): obtains a signature \(\sigma\) on the message 
\(m\) hiding it in the commitment \(comm\) with the opening value \(open\).
The function \(\PObtainSig\) is run by the user and \(\PIssueSig\) is run by the 
signature issuer.

\NewFunction{\PProve}{\Psig.\!Prove}

\((comm, \pi, open)\gets \PProve[pk, m, \sigma]\): outputs a commitment 
\(comm\) on the message \(m\) and can be opened using \(open\).
\(\pi\) is a proof of knowledge of the signature \(\sigma\) on the message \(m\) 
made by the owner of \(pk\).

\NewFunction{\PVerifyProof}{\Psig.\!VerifyProof}

\(\bin\ni \PVerifyProof[pk, comm, \pi]\): verifies the proof \(\pi\) that \(m\) 
has been signed by the owner of \(pk\).
(\(comm\) is a commitment to \(m\).)
It outputs \(1\) (accept) if \(\pi\) is a proof of knowledge of \(F(m)\) and a 
signature on \(m\).

\NewFunction{\PEqCommProve}{\Psig.\!EqCommProve}

\(\pi\gets \PEqCommProve[m, open, open']\): outputs a proof \(\pi\) saying that 
\(comm = \PCommit[m, open]\) is a commitment to the same value as \(comm' = 
  \PCommit[m, open']\).
This can be used to tie ephemeral commitments to long-term commitments.

\NewFunction{\PVerEqComm}{\Psig.\!VerEqComm}

\(\bin\ni \PVerEqComm[comm, comm', \pi]\): outputs \(1\) (accept) if the proof 
\(pi\) is a proof saying that \(comm, comm'\) are commitments to the same value.

\subsection{Distance bounding}%
\label{DistanceBounding}

\Ac{DB} protocols were first suggested by \citet{DistanceBounding} to prevent relay attacks in contactless communications in which the adversary forwards a communication between a prover and a possibly far-away verifier to authenticate. 
These attacks cannot be prevented by cryptographic means as they are independent of the semantics of the messages exchanged.
As a consequence, mechanisms ensuring the physical proximity between a verifier and a prover should be used instead.
\Ac{DB} protocols precisely enable the verifier to estimate an upper bound on his distance to the prover by measuring the time-of-flight of short challenge-response messages (or rounds) exchanged during time-critical phases. 
Time critical phases are complemented by slow phases during which the time is not taking into account. 
At the end of a \Ac{DB} protocol, the verifier should be able to determine if the prover is legitimate AND in his vicinity.
In this sense, \Ac{DB} protocols combines at the same time the classical properties of authentication protocols with the possibility of verifying the physical proximity.

The main attacks against \ac{DB} protocols can be summarized as follows:
%Seb: the following names of attacks should all have the first letters in capital
\begin{itemize}
  \item \acf{DBDF}: a legitimate but malicious prover wants to fool the verifier on the distance between them.
  \item \acf{DBMF}: the adversary illegitimately authenticates, possibly using a far-away honest prover.
  \item \acf{DBTF}: a legitimate but malicious prover helps an accomplice close to the verifier to authenticate.
%Seb: the two following properties should be defined more precisely
  \item \acf{DBDH}: similar to \ac{DBDF}, sometimes using the presence of an honest prover close to the verifier.
  \item \acf{DBIF}: the adversary plays against a simplified version of the protocol without any distance estimation.
\end{itemize}
There are two lines of attempts at formalizing the above properties: one by \citet{DB-BMV} and another by \citet{DB-DFKO}.

The majority of the existing \ac{DB} protocols are symmetric and thus requires an honest verifier.
Indeed, in this context it does not make sense to protect against the verifier as he can easily impersonate the prover as he has a knowledge of his secret key.
There has been less work done in the domain of asymmetric (or public-key) \ac{DB} protocols.
Unfortunately, our setting requires a public-key \ac{DB} protocol with a \emph{malicious verifier} who will try to \emph{impersonate the prover}.
The verifier might also try to track the provers and map their identities to their actions, thus we also require privacy.
This leads to the requirement of a \ac{DB} \ac{ZKPK}, or simply \ac{DBPK}, with resistance to all the frauds mentioned above as well as against this verifier who will attempt to impersonate the prover.
More specifically, we must combine such \iac{DB} with anonymous credentials (which we do in \cref{DB-anon-cred}).
The choice of the anonymous credential system will affect the properties we require of the \ac{DBPK} (\eg that it must be \iac{PKE} for discrete logarithms).

\subsection{Location proofs and distance bounding}

Briefly describe location proofs and distance bounding protocols.
Describe the different desirable properties for distance bounding:
\begin{itemize}
  \item Mafia-fraud resistance,
  \item terrorist-fraud resistance,
  \item impersonation-fraud resistance,
  \item distance-fraud resistance.
\end{itemize}

\subsection{Storage}%
\label{StorageProperties}

We will use the storage to provide the temporal eligibility property.
\Cref{CreatedBeforeEnd,CreatedAfterStart} requires a \emph{partially ordered 
  set}\footnote{%
  A relation \(\preceq\) which is reflexive, antisymmetric and transitive.
} of objects.
If some objects in the set relate to known points in time, then the partial 
order relates the data to the time of the event.
This allows us to \emph{verify the data temporally}.
One primitive that fulfils these requirements is a blockchain.
There are also other structures, e.g.\ a directed graph that 
converges~\cite{BlockchainFreeCryptocurrencies}, that also provides the required 
properties.
However, we will use the blockchain in our discussion for simplicity.

There are five properties that we need from the storage system, all of which 
makes a blockchain-like structure suitable.
First, we need immutability.
I.e.\ once we commit something to the blockchain it will remain there.
We will use this to ensure verifiability, the data must remain to be verifiable.

Second, we need a general form of time-stamping.
I.e.\ we must be able to relate commitments to time, e.g.\ in 
Bitcoin~\cite{Bitcoin} a new block appears approximately every 10 minutes.
This coarse-grained form of time-stamping is enough for our purposes.
We will use this property to ensure that proofs are created before a certain 
point in time.

Third, we need an unpredictable number whose time of publication can be 
verified.
For this property, we will use the head of the blockchain.
The hash value of the head of the blockchain at a given time will be difficult 
to predict ahead of time.
%(See \cref{SecurityAnalysis} for a security analysis.)
We want to use this to ensure that proofs are created after a certain point in 
time.

Fourth, once we have committed some data, we do not need to keep the data 
itself, but we can store some confirmation value instead.
E.g.\ using blockchains, we can store the hash of the block where our data was 
committed, while this block remains we are sure that our data remains committed 
--- even if we no longer remember what our data looked like.
We will use this property to approximate the property of receipt freeness.

Finally, if the storage is decentralized we do not have to trust any central 
authority, which is the usual case for blockchains.


\section{Distance-bounding \aclp*{ZKPK} for discrete logarithms}%
\label{DB-anon-cred}

We will now introduce \iac{DB} protocol which is a \ac{ZKPK} for discrete 
logarithms.
Our protocol is an adaptation of the Schnorr identification 
scheme~\cite{Schnorr}, albeit different from that of 
\textcite{DistanceBounding} in the original \ac{DB} paper\footnote{%
  The \citeauthor{DistanceBounding} protocol was shown to be prone to distance 
  hijacking~\cite{DistanceHijacking,TamarinDB} and was not secure against 
  terrorist fraud.
} and that of \textcite{Bussard-Bagga}\footnote{%
  The \citeauthor{Bussard-Bagga} protocol was also shown to be 
  vulnerable~\cite{Bussard-Bagga-attack}.
  Although this time, the protocol could not protect against \ac{DBTF} and 
  \ac{DBDF} as it claimed.
}.
We propose another way to turn the Schnorr protocol into a public-key \ac{DB} 
protocol which is \iac{ZKPK} that is secure against \ac{DBMF}, \ac{DBDF}, 
\ac{DBDH} and \ac{DBTF}.
This yields strong privacy properties and protection against a malicious, 
impersonating verifier.

We also deviate from the normal technique of one-bit challenges and responses 
otherwise used in \ac{DB} protocols.
\Textcite{UWBPR} proposed a secure encoding for the physical 
layer which removes the requirement of only using single-bit challenges and 
responses for distance bounding.
However, their protocol only works for relay attacks and cannot handle distance 
fraud.
This is due to a requirement of the protocol that the prover and verifier must 
share a secret, and thus must be mutually trusted.

We will first present our version of the \ac{DB} Schnorr protocol for the 
mutual-trust case, where the prover and verifier have a pre-shared secret.
Then we will adapt it by introducing \iac{AKE}.
The authentication in this \ac{AKE} will leverage that our protocol is \iac{DB} 
\ac{ZKPK} for discrete logarithms.
If the \ac{AKE} is based on discrete logarithms we can prove the distance bound 
while we authenticate the key.
This allows us to embed this protocol into tamper resistant hardware to provide 
security against malicious provers.


\subsection{Reattempting a distance-bounding Schnorr protocol}%
\label{DB-Schnorr}

Thanks to \textcite{UWBPR} we can keep the Schnorr protocol 
almost as is, relays (\ac{DBMF}, \ac{DBDH}) are dealt with on the physical 
layer.
The only concern, for now, is \acl{DBTF}.
(We deal with \acl{DBDF} later.)

We present the protocol in \cref{SchnorrFigure}.
The (cyclic) group with generator \(g\) and order \(q\) are system parameters.
The private key \(\alpha\) with public key \(A = g^\alpha\) are generated once by the prover in the setup phase.
Let \(\UWBPR\) be the protocol of \textcite{UWBPR}.
Further, let
\(\UWBPRsend[_k][m]\) denote sending a message~\(m\) with the shared key~\(k\),
\(m'\gets \UWBPRrecv_k\) denote receiving a message~\(m'\) with shared 
key~\(k\) and finally
\(\Delta t\gets \UWBPRtime\) denote the time between the last \(\UWBPRsend\) 
and \(\UWBPRrecv\) operations.

\begin{figure*}
  \centering
  \small
  \setlength{\ProtoArrowLength}{0.07\linewidth}
  \begin{tabular}{p{0.40\linewidth}cp{0.40\linewidth}}
    \(\DBSprove[\sk, g, q, \alpha, A = g^\alpha]\):
    & &
    \(\DBSverify[\sk, g, q, A]\):
    \\
    \midrule

    \multicolumn{3}{c}{\textbf{Setup}} \\

    Load \(\sk\) into \(\UWBPR\)
    &
    & Load \(\sk\) into \(\UWBPR\)
    \\

    \(\rho\rgets \ZZ_q, R\gets g^{\rho}\)
    & \(\ProtoSendRight{R}\)
    &
    \\

    \(s_1\gets \rho - c_1\alpha \pmod q\)
    & \(\ProtoSendLeft{c_1, \dotsc, c_k}\)
    & \(c_1\rgets \bin^k, \dotsc, c_k\rgets \bin^k\)
    \\

    \vdots
    &
    &
    \\[-1em]

    \(s_k\gets \rho - c_k\alpha \pmod q\)
    &
    &
    \\

    \midrule
    \multicolumn{3}{c}{\textbf{Distance-bounding}} \\

    \text{Ready}
    & \ProtoSendRight{\text{Ready}}
    & \\

%    % null
%    &
%    & Random delay
%    \\

    % null
    & \(\ProtoSendLeft{i}\)
    & \(i\rgets \{1, \dotsc, k\}\), start clock
    \\

    Fetch \(s_i\)
    & \(\ProtoSendRight{s_i}\)
    & Stop clock, record \(\Delta t\)
    \\

    \midrule
    \multicolumn{3}{c}{\textbf{Verification}}
    \\

    % null
    &
    & Accept if \(R = g^{s_i} A^{c_i}\) and \(\Delta t < t_{\max}\).
    \\
    
  \end{tabular}
  \caption{%
    One-round protocol instance of the \(\DBSprove\leftrightarrow \DBSverify\) 
    protocol instantiating \(\PK[\alpha][A = g^\alpha]\).
    Each transmission (arrow in the diagram) uses \(\UWBPR\).
    The protocol should be repeated \(n\) times to achieve the desired 
    soundness and distance-bounding errors.
  }%
  \label{SchnorrFigure}
\end{figure*}

During one round, in the setup phase, the prover commits to a random nonce: 
more precisely he chooses \(\rho\rgets \ZZ_q\) uniformly at random, computes 
\(R\gets g^\rho\) and sends \(R\) to the verifier.
The verifier generates \(k\) challenges \(c_1\rgets \bin^k, \dotsc, c_k\rgets 
  \bin^k\) (bit strings of length \(k\)) and sends them to the prover.
The prover computes one response per challenge, \(s_1\gets \rho - c_1\alpha, 
  \dotsc, s_k\gets \rho - c_k\alpha\).
This step is the main difference to the original Schnorr protocol: the verifier 
selects several challenges and the prover computes several responses --- but 
still only use \emph{one nonce}, \(\rho\).
This is needed for \ac{DBTF} resistance.
This is also different from the original Brands-Chaum protocol, in which the 
prover and verifier jointly construct \emph{one} challenge with \emph{one} 
response.

In the \ac{DB} phase, the prover notifies the verifier that he has computed 
\(s_1, \dotsc, s_k\).
The verifier chooses one of the challenges, \(i\), uniformly randomly and sends 
it as a challenge to the prover and starts measuring the time of flight.
The prover replies with \(s_i\).
The verifier stops the measurement and verifies that \(R = g^{s_i}A^{c_i}\) and 
that the time of flight was within \(t_{\max}\) (determined from the allowed 
  distance).

This protocol must be repeated \(n\) times.

\subsection{\Acl*{ZK} and \acl*{PK}}

The main difference between this protocol and the Schnorr protocol is that we 
have the prover compute responses for \(k\) different challenges, \(c_1, 
  \dotsc, c_k\).
However, in the authentication step, the verifier chooses only one of those 
challenges.
From the simulator's (and extractor's) perspective, there is no difference 
whether the verifier first chooses \(k\) and then chooses one of those \(k\) 
challenges, or if the verifier chooses the one challenge directly;
the distribution is the same, \(\frac{k}{n}\times \frac{1}{k} = \frac{1}{n}\).
Therefore the standard proof for Schnorr as a malicious-verifier \ac{ZKPK} 
protocol with soundness error \(2^{-kn}\) still 
holds~\cite[\eg][]{OnSigmaProtocols}.
(We note that \(k\), which is also the length of the challenge bit string, must 
be logarithmic in the security parameter, \(\lambda\), for a malicious 
verifier.)

\subsection{\acs*{DBMF}, \acs*{DBDH} and \acs*{DBTF} resistance}

The intuition behind the protocol security is as follows.
The prover must know the responses for all challenges to successfully pass the 
\ac{DB} phase.
The reason for having several challenges but only one random nonce is that 
knowing at least two responses means learning the secret~\(\alpha\).
This gives us the incentives that prevent \ac{DBTF}.
(We also need it for \ac{DBMF}.)
Bundling the authentication into the distance-bounding phase (difference from 
Brands-Chaum) prevents \ac{DBDH}.
We do not consider \ac{DBDF} here; since the prover knows the key~\(\sk\), she 
can do \iac{DBEDLC} attack to reduce the distance.
(We will solve this problem further down.)
However, the properties of \(\UWBPR\) and the unpredictability of the 
challenges ensures \ac{DBMF} resistance (since then the prover is honest).

The \(\UWBPR\) prevents \iac{DBMF} adversary from performing a 
distance-reduction through \iac{DBEDLC} attack~\cite{UWBPR}.
Thus, any relaying will be detected, \ie this justifies \cref{TaintedMF}.

\begin{theorem}[\acs*{DBMF} resistance]
  Let \(\adv\) be a \((t, \qobs, \qp, \qv)\)-MF adversary, then 
  \(\Adv_{\DBS}^{\text{MF}}(\adv) = (q+k)^n (kq)^{-n}\), where \(k, q\) are as 
  in \cref{DB-Schnorr}.
\end{theorem}

\begin{proof}
  The adversary has two options:
  \begin{enumerate*}
  \item guess which challenge the verifier will use and request the correct 
    response from the prover, this yields a success with probability 
    \(\frac{1}{k}\);
  \item guess the response, this yields success with probability 
    \(\frac{1}{q}\).
  \end{enumerate*}

  The total success probability is \(\frac{1}{k} + \frac{1}{q}\) per round, 
  thus \((q + k)^n (kq)^{-n}\) in total.
  We have that \(k \in \log(\lambda)\), \(q \in \poly[\lambda]\) and it follows 
  that the success probability is negligible,
  \[
    \left( \frac{\log(\lambda) + 
        \poly[\lambda]}{\log(\lambda)\poly[\lambda]}\right)^n \approx
    \frac{1}{\log(\lambda)^n} < \frac{1}{\poly[\lambda]}
  \] for \(n > 2\).
\end{proof}

The protocol is also secure against distance hijacking due to the fact that the 
authenticating bit string is used during the \ac{DB} phase, not the challenge 
bit string, as is the case for the original protocol of Brands-Chaum.
\Ac{DBDH} requires that the adversary finds a collision between his response 
and that of the honest prover.
Thus the probability of success is equivalent to a collision for the responses 
for the chosen challenge --- in each round.

\begin{theorem}[\acs*{DBDH} resistance]
  Let \(\adv\) be a \((t, \qobs, \qp, \qv)\)-DH adversary, then 
  \(\Adv_{DBS}^{\text{DH}}(\adv) = X\).
\end{theorem}

\begin{proof}
  The adversary must find a collision of challenges for the honest prover's 
  challenges, so that the verifying equation holds at the end.
  Let \(R, c_1, \dotsc, c_k\) be the transcript of the setup between the honest 
  verifier and the adversary.
  Let \(R', c_1', \dotsc, c_k'\) be the transcript of the setup between the 
  honest verifier and honest prover.
  The verifier will accept if \(R = g^{s_i'} A^{c_i}\), where \(s_i' = \rho' - 
    c_i'\alpha'\) is the response by the honest prover computed for his 
  session, \ie \(\rho = \rho' - c_i'\alpha' + c_i\alpha\).
  Since the verifier and prover are honest, \(\rho', c_i'\), \(\alpha'\) and 
  \(c_i\) are chosen uniformly randomly.
  \(\alpha\) is also chosen uniformly randomly and is fixed.
\end{proof}

The last property is \ac{DBTF}.

\begin{proof}[\acs*{DBTF} resistance]
  Assume that there is an algorithm~\(\adv\) that can authenticate without the 
  provers help with non-negligible advantage~\(\epsilon\), \ie the success 
  probability is \(\frac{1}{2}+\epsilon\).
  There is chance that \(\adv\) predicts the challenge is \(\frac{1}{2}\).
  Thus, there is some case for which \(\adv\) knows both responses.
  In that case \(\adv\) can simply compute the secret and impersonate with 
  probability \(1\) forever.
\end{proof}

The protocol is \ac{DBTF}-resistant.
Indeed, if the malicious prover gives both responses to the accomplice, the 
accomplice can compute his secret key.
The probability of success of \ac{DBTF} is thus reduced to guessing \(b\), \ie \(1/2\) per round or, in total, \(2^{-n}\).

\subsection{Achieving \acs*{DBDF} resistance}

\Acl{DB} protocols requires hardware implementations.
We will use a trusted hardware implementation to overcome the limitation of 
\textcite{UWBPR}, \ie that the prover and verifier be mutually trusted.
We will need \iac{PKI} for the verifier to verify the trusted hardware.
We will leverage this \ac{PKI} to overcome the shared-key requirement.
With \iac{PKI} based on discrete logarithms we will be able to perform \iac{DB} 
\ac{ZKPK} which \emph{simultaneously} proves
\begin{enumerate*}
\item that the protocol is run by trusted hardware,
\item that that trusted hardware is within proximity,
\item that some \ac{ZKPK} statement about some discrete logarithms holds, and
\item that that knowledge is within proximity.
\end{enumerate*}

The \ac{DB} phase protects against distance fraud.
Once the prover has received \(l+1\) challenges set to \(b\), it knows that the remaining challenges must also be \(b\).
\Iac{DBDF} prover must thus wait for the challenge bit \(b_i\) before responding with \(r_i\) for \(l+1\) challenges.
Thus, the probability of successfully guessing the order of the challenge bits is \(2^{-(l+1)}\) per round or, in total, \(2^{-(l+1)n}\).

\subsection{Formal verification}

We formally verified these properties with Tamarin~\cite{meier2013tamarin}.
Thanks to recent work by \textcite{TamarinDB}, all the common properties for 
\ac{DB} protocols (except \ac{DBTF}) can be characterized with a 
causality-based definition relying only on the order of messages.
The whole Tamarin specification for our protocol is provided in 
\cref{apdx:tamarin-spec}, with a short discussion of the modeling choices that 
we have made.
It contains a specification of the distance-bounding protocol described 
previously, a definition of the soundness and correctness properties that 
ensure the protocol actually produces the expected result, and a definition of 
the security properties we want to ensure.


