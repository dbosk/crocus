\mode*
\section{Building blocks}%
\label{Primitives}\label{BuildingBlocks}

\dots

\subsection{Zero-knowledge proofs of knowledge}%
\label{ZKPK}

Briefly describe zero-knowledge proofs and what type of properties can be proved 
using them, e.g.\ algebraic structures but not one-way functions.

We will use the notation introduced by \citet{PKnotation}:
\begin{equation*}
  \PK{\alpha, \beta, \gamma}
  {y = g^\alpha h^\beta \land y' = \hat g^\gamma}
\end{equation*}
which means that we prove knowledge of \(\alpha, \beta, \gamma\) ensuring that 
\(y, y'\) are of the form \(y = g^\alpha h^\beta\) and \(y' = \hat g^\gamma\), 
respectively.
When a proof of knowledge is turned into a signature using the Fiat-Shamir 
heuristic~\cite{FiatShamirHeuristic}, we will denote it as
\begin{equation*}
  \SPK{\alpha, \beta, \gamma}
  {y = g^\alpha h^\beta \land y' = \hat g^\gamma}
  {m},
\end{equation*}
which yields a signature on \(m\), ensuring that the issuer knew \(\alpha, 
  \beta, \gamma\) such that \(y = g^\alpha h^\beta\) and \(y' = \hat g^\gamma\).


\paragraph{Implementations}

There are various implementations.

\subsection{Anonymous credentials}%
\label{ZK-anon-cred}

\NewCryptoScheme{\Idemix}{CL}
\NewFunction{\PRF}{PRF}

We will use a \(\PRF\).

%\subsection{Non-interactive anonymous credentials}%
\label{NIZK-anon-cred}

\NewScheme{\Psig}{Psig}

\Citet{Psignatures} presents a class of signatures, commitments and 
non-interactive proof protocols named P-signatures (\Psig), which yields for 
creating non-interactive anonymous credentials.
P-signatures provides a signature scheme and a commitment scheme with a protocol 
to obtain a signature on a committed value and two non-interactive proof 
systems, one for proving that a committed value has been signed and the other 
for proving that a pair of commitments are commitments to the same value.

\NewAlgorithm{\PSetup}{\Psig.\!Setup}

\(params\gets \PSetup[1^\lambda]\): generates the global parameters \(params\), 
which contains parameters for the underlying algorithms.
For ease of notation we will omit \(params\) from all functions.

\NewAlgorithm{\PCommit}{\Psig.\!Commit}

\(comm\gets \PCommit[m, open]\): takes a message \(m\) and an opening value 
\(open\) and returns a commitment.
The commitment scheme must be \emph{perfectly binding} and \emph{strongly 
  computationally hiding}.
Perfectly binding means that for every bit string \(comm\) there exists at most 
one value \(x\) such that there exists a value \(open\) with \(comm = 
  \PCommit[x, open]\).
Strongly computationally hiding means that there exists an alternate setup 
algorithm which generates indistinguishable parameters which yields 
information-theoretically hiding commitments.
\Ie the adversary cannot tell which setup algorithm has been used.

\NewAlgorithm{\PObtainSig}{\Psig.\!ObtainSig}
\NewAlgorithm{\PIssueSig}{\Psig.\!IssueSig}

\(\sigma\gets \PObtainSig[pk, m, comm, open]\leftrightarrow
  \PIssueSig[sk, comm]\): obtains a signature \(\sigma\) on the message 
\(m\) hiding it in the commitment \(comm\) with the opening value \(open\).
The function \(\PObtainSig\) is run by the user and \(\PIssueSig\) is run by the 
signature issuer.

\NewAlgorithm{\PProve}{\Psig.\!Prove}

\((comm, \pi, open)\gets \PProve[pk, m, \sigma]\): outputs a commitment 
\(comm\) on the message \(m\) and can be opened using \(open\).
\(\pi\) is a proof of knowledge of the signature \(\sigma\) on the message \(m\) 
made by the owner of \(pk\).

\NewAlgorithm{\PVerifyProof}{\Psig.\!VerifyProof}

\(\bin\ni \PVerifyProof[pk, comm, \pi]\): verifies the proof \(\pi\) that \(m\) 
has been signed by the owner of \(pk\).
(\(comm\) is a commitment to \(m\).)
It outputs \(1\) (accept) if \(\pi\) is a proof of knowledge of \(F(m)\) and a 
signature on \(m\).

\NewAlgorithm{\PEqCommProve}{\Psig.\!EqCommProve}

\(\pi\gets \PEqCommProve[m, open, open']\): outputs a proof \(\pi\) saying that 
\(comm = \PCommit[m, open]\) is a commitment to the same value as \(comm' = 
  \PCommit[m, open']\).
This can be used to tie ephemeral commitments to long-term commitments.

\NewAlgorithm{\PVerEqComm}{\Psig.\!VerEqComm}

\(\bin\ni \PVerEqComm[comm, comm', \pi]\): outputs \(1\) (accept) if the proof 
\(pi\) is a proof saying that \(comm, comm'\) are commitments to the same value.

\subsection{Distance bounding}%
\label{DistanceBounding}

\Ac{DB} protocols were first suggested by \citet{DistanceBounding} to prevent relay attacks in contactless communications in which the adversary forwards a communication between a prover and a possibly far-away verifier to authenticate. 
These attacks cannot be prevented by cryptographic means as they are independent of the semantics of the messages exchanged.
As a consequence, mechanisms ensuring the physical proximity between a verifier and a prover should be used instead.
\Ac{DB} protocols precisely enable the verifier to estimate an upper bound on his distance to the prover by measuring the time-of-flight of short challenge-response messages (or rounds) exchanged during time-critical phases. 
Time critical phases are complemented by slow phases during which the time is not taking into account. 
At the end of a \Ac{DB} protocol, the verifier should be able to determine if the prover is legitimate AND in his vicinity.
In this sense, \Ac{DB} protocols combines at the same time the classical properties of authentication protocols with the possibility of verifying the physical proximity.

The main attacks against \ac{DB} protocols can be summarized as follows:
%Seb: the following names of attacks should all have the first letters in capital
\begin{itemize}
  \item \acf{DBDF}: a legitimate but malicious prover wants to fool the verifier on the distance between them.
  \item \acf{DBMF}: the adversary illegitimately authenticates, possibly using a far-away honest prover.
  \item \acf{DBTF}: a legitimate but malicious prover helps an accomplice close to the verifier to authenticate.
%Seb: the two following properties should be defined more precisely
  \item \acf{DBDH}: similar to \ac{DBDF}, sometimes using the presence of an honest prover close to the verifier.
  \item \acf{DBIF}: the adversary plays against a simplified version of the protocol without any distance estimation.
\end{itemize}
There are two lines of attempts at formalizing the above properties: one by \citet{DB-BMV} and another by \citet{DB-DFKO}.

The majority of the existing \ac{DB} protocols are symmetric and thus requires an honest verifier.
Indeed, in this context it does not make sense to protect against the verifier as he can easily impersonate the prover as he has a knowledge of his secret key.
There has been less work done in the domain of asymmetric (or public-key) \ac{DB} protocols.
Unfortunately, our setting requires a public-key \ac{DB} protocol with a \emph{malicious verifier} who will try to \emph{impersonate the prover}.
The verifier might also try to track the provers and map their identities to their actions, thus we also require privacy.
This leads to the requirement of a \ac{DB} \ac{ZKPK}, or simply \ac{DBPK}, with resistance to all the frauds mentioned above as well as against this verifier who will attempt to impersonate the prover.
More specifically, we must combine such \iac{DB} with anonymous credentials (which we do in \cref{DB-anon-cred}).
The choice of the anonymous credential system will affect the properties we require of the \ac{DBPK} (\eg that it must be \iac{PKE} for discrete logarithms).
\input{LocationsProofs.tex}

\subsection{Timestamping storage: blockchains}%
\label{StorageProperties}

We will use the storage to provide the temporal eligibility property.
\Cref{CreatedBeforeEnd,CreatedAfterStart} requires a \emph{partially ordered 
  set}\footnote{%
  A relation \(\preceq\) which is reflexive, antisymmetric and transitive.
} of objects.
If some objects in the set relate to known points in time, then the partial 
order relates the data to the time of the event.
This allows us to \emph{verify the data temporally}.
One primitive that fulfils these requirements is a blockchain.
There are also other structures, e.g.\ a directed graph that 
converges~\cite{BlockchainFreeCryptocurrencies}, that also provides the required 
properties.
However, we will use the blockchain in our discussion for simplicity.

There are five properties that we need from the storage system, all of which 
makes a blockchain-like structure suitable.
First, we need immutability.
I.e.\ once we commit something to the blockchain it will remain there.
We will use this to ensure verifiability, the data must remain to be verifiable.

Second, we need a general form of time-stamping.
I.e.\ we must be able to relate commitments to time, e.g.\ in 
Bitcoin~\cite{Bitcoin} a new block appears approximately every 10 minutes.
This coarse-grained form of time-stamping is enough for our purposes.
We will use this property to ensure that proofs are created before a certain 
point in time.

Third, we need an unpredictable number whose time of publication can be 
verified.
For this property, we will use the head of the blockchain.
The hash value of the head of the blockchain at a given time will be difficult 
to predict ahead of time.
%(See \cref{SecurityAnalysis} for a security analysis.)
We want to use this to ensure that proofs are created after a certain point in 
time.

Fourth, once we have committed some data, we do not need to keep the data 
itself, but we can store some confirmation value instead.
E.g.\ using blockchains, we can store the hash of the block where our data was 
committed, while this block remains we are sure that our data remains committed 
--- even if we no longer remember what our data looked like.
We will use this property to approximate the property of receipt freeness.

Finally, if the storage is decentralized we do not have to trust any central 
authority, which is the usual case for blockchains.
