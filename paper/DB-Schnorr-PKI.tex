\section{From mutual trust to \acs*{PKI}}%
\label{DB-Schnorr-PKI}

All \acl{DB} protocols requires hardware 
implementations~\cite{DBonSmartphones}.
We will use a \emph{trusted} hardware implementation to overcome the limitation 
of \textcite{UWBPR}, \ie that the prover and verifier must share a key.
We will provide \iac{PKI} for the verifier to verify the trusted hardware.
We will leverage this \ac{PKI} to overcome the shared-key requirement to allow 
any two devices in this \ac{PKI} to perform \acl{DB}.
With \iac{PKI} based on discrete logarithms we will be able to perform \iac{DB} 
\ac{ZKPK} which \emph{simultaneously} proves
\begin{enumerate*}
\item that the protocol is run by trusted hardware,
\item that that trusted hardware is within proximity,
\item that some \ac{ZKPK} statement about some discrete logarithms holds, and
\item that that knowledge is within proximity.
\end{enumerate*}

We will now introduce an extended protocol, \(\DBSHW\), which represents the 
secure hardware implementation.
\Cref{DBSHW-overview} presents an overview of our protocol.
For simplicity of exposition, we present a \ac{PKI} based on plain 
CL04~\cite{CLsignatures} signatures.
However, CL04 is the base for the \ac{DAA} protocol, which is used for 
\ac{TPM}, so our protocol can be modified into \iacl{DB} \ac{DAA} protocol.

In summary, the difference is as follows.
\(\DBS\) performs \(
  \PK[\alpha][A = g^\alpha]
\) with a distance bound on the user's behalf.
\(\DBSHW\) first runs \iac{DHKE} to establish a shared key~\(k =
\gDH^{\xDH \yDH}\) from \(\XDH = \gDH^{\xDH}, \YDH = \gDH^{\yDH}\), then 
performs \[
  \PK[\alpha, \xDH, \sk_P, \sigma][
    A = g^\alpha \land
    \XDH = \gDH^{\xDH} \land
    \XDHCL = \XDH^{\sk} \land
    \Blind[\sigma] = \Sign[_{\sk}][\sk_P]
  ],
\] also with a distance bound and \(A = g^\alpha\) on behalf of the user; where 
\(\sk_P\) is the prover's private key, \(\sigma\) is a signature issued with 
signing key~\(\sk\) and \(\Blind\) reblinds a signature.
Proving knowledge of \(\XDH = \gDH^{\xDH}\) and \(\XDHCL = \XDH^{\sk}\) 
prevents any \ac{MITM} attack against the \ac{DHKE}.
Proving knowledge of a signature~\(\sigma\) by \iac{CA} on the private 
key~\(\sk_P\) ensures that the prover is a device in the \ac{PKI} of the 
\ac{CA}.

\Cref{DBSHW-overview} provides an overview.
For simplicity of the exposition, we present a version of \(\DBSHW\) that 
allows the user to run \(\PK[\alpha][A = g^\alpha]\) (the Schnorr 
identification scheme) with a distance bound.
This can, of course, be generalized to any \ac{ZKPK} for discrete logarithms, 
for instance, CL04~\cite{CLsignatures}.

\begin{figure*}
  \centering
  \small
  \setlength{\ProtoArrowLength}{0.07\linewidth}
  \begin{tabular}{p{0.40\linewidth}cp{0.40\linewidth}}
    \(\HWprove[g, q, \alpha]\):
    & &
    \(\HWverify[g, q, A]\):
    \\
    \midrule

    \multicolumn{2}{l}{%
      Static \(\sk_P\in \ZZ_q, \sigma = (a, A, b, B, c)\)
    }
    & Static \(\pk = (\qCL, \gCL, \ggCL, e, \XCL, \YCL, \ZCL)\)
    \\

    \midrule
    \multicolumn{3}{c}{\textbf{Pre-setup key-agreement}} \\

    \(\xDH\rgets \ZZ_{\qDH}, \XDH\gets \gDH^{\xDH}, \XDHCL\gets \XDH^{\sk}\)
    & \(\ProtoSendRight{\XDH, \XDHCL}\)
    & \(\yDH\rgets \ZZ_{\qDH}\)
    \\

    \(k \gets \YDH^{\xDH}\)
    & \(\ProtoSendLeft{\YDH = \gDH^{\yDH}}\)
    & \(k \gets \XDH^{\yDH}\)
    \\

    \midrule
    \multicolumn{3}{c}{\textbf{Setup}} \\

    Load \(k\) into \(\UWBPR\)
    &
    & Load \(k\) into \(\UWBPR\)
    \\

    \(r_1, r_2\rgets \ZZ_{\qCL}\)
    &
    &
    \\

    \(\tilde a = a^{r_1},
      \tilde A = A^{r_1},
      \tilde b = b^{r_1},
      \tilde B = B^{r_1},
      \tilde c = (c^{r_1})^{r_2}\)
    & \(\ProtoSendRight{
        \tilde \sigma = (\tilde a, \tilde A, \tilde b, \tilde B, \tilde c)
      }\)
    &
    \\[0.5em]

    \(v_x = e(\XCL, \tilde a),
      v_{xy} = e(\XCL, \tilde b),
      V_{xy} = e(\XCL, \tilde B),
      v_s = e(\gCL, \tilde c)\)
    &
    & \(v_x = e(\XCL, \tilde a),
      v_{xy} = e(\XCL, \tilde b),
      V_{xy} = e(\XCL, \tilde B),
      v_s = e(\gCL, \tilde c)\)
    \\[0.5em]

    \(\rho\rgets \ZZ_q,
      \rhoDH \rgets \ZZ_{\qDH},
      \rhoCL_{\sk} \rgets \ZZ_{\qCL},
      \rhoCL_{r} \rgets \ZZ_{\qCL},
      \rhoCL_{r'} \rgets \ZZ_{\qCL}\)
    &
    &
    \\

    \(R\gets g^{\rho},
      \RDH\gets \gDH^{\rhoDH},
      \RDHCL\gets \XDH^{\rhoCL_{\sk}},
      \RCL\gets v_s^{\rhoCL_{r'}} v_{xy}^{\rhoCL_{\sk}} V_{xy}^{\rhoCL_r}\)
    & \(\ProtoSendRight{R, \RDH, \RDHCL, \RCL}\)
    &
    \\

    \(\forall i\in \{1, \dotsc, m\}\):
    & \(\ProtoSendLeft{c_1, \dotsc, c_k}\)
    & \(c_1\rgets \bin^l, \dotsc, c_m\rgets \bin^l\)
    \\


    \(s_i\gets \rho - c_i\alpha \pmod q\)
    &
    &
    \\

    \(\sDH_i\gets \rhoDH - c_i \xDH \pmod \qDH\)
    &
    &
    \\

    \(\sCL_i^{(r')}\gets \rhoCL_{r'} + c_i r' \pmod \qCL\)
    &
    &
    \\

    \(\sCL_i^{(\sk)}\gets \rhoCL_{\sk} - c_i \sk \pmod \qCL\)
    &
    &
    \\

    \(\sCL_i^{(r)}\gets \rhoCL_{r} - c_i r \pmod \qCL\)
    &
    &
    \\

    \midrule
    \multicolumn{3}{c}{\textbf{Distance-bounding}} \\

    \text{Ready}
    & \ProtoSendRight{\text{Ready}}
    & \\

    % null
    & \(\ProtoSendLeft{i}\)
    & \(i\rgets \{1, \dotsc, m\}\), start clock
    \\

    Fetch \(s_i, \sDH_i, \sCL_i^{(r')}, \sCL_i^{(\sk)}, \sCL_i^{(r)}\)
    & \(\ProtoSendRight{
      s_i, \sDH_i, \sCL_i^{(r')}, \sCL_i^{(\sk)}, \sCL_i^{(r)}}\)
    & Stop clock, record \(\Delta t\)
    \\

    \midrule
    \multicolumn{3}{c}{\textbf{Verification}}
    \\

    % null
    &
    & Accept if \(\Delta t < t_{\max}\) and
    \\
    
    % null
    &
    & \(R = A^{c_i} g^{s_i}\),
    \\

    % null
    &
    & \(\RDH = \XDH^{c_i} \gDH^{\sDH_i}\),
    \\

    % null
    &
    & \(\RDHCL = \XDHCL^{c_i} \XDH^{\sCL_i^{(\sk)}}\),
    \\

    % null
    &
    & \(v_x^{c_i} \RCL =
          v_s^{\sCL_i^{(r')}} v_{xy}^{\sCL_i^{(\sk)}} V_{xy}^{\sCL_i^{(r)}}\),
    \\

    %
    &
    & \(e(\tilde a, \ZCL) = e(\gCL, \tilde A)\),
    \\

    %
    &
    & \(e(\tilde a, \YCL) = e(\gCL, \tilde b)\),
    \\

    %
    &
    & \(e(\tilde A, \YCL) = e(\gCL, \tilde B)\).
    \\

  \end{tabular}
  \caption{%
    One-round protocol instance of the \(\DBSHW\) protocol instantiating 
    \(\PK[\alpha][A = g^\alpha]\).
    Each transmission (arrow in the diagram) uses \(\UWBPR\) (except in the 
    pre-setup phase).
    The protocol actually proves \(
      \PK[\alpha, \xDH, \sk, \sigma][
        A = g^\alpha \land
        \XDH = \gDH^{\xDH} \land
        \XDHCL = \XDH^{\sk} \land
        \Blind[\sigma] = \Sign[_{\sk}][\sk_P]
      ].
    \)
    This is ties \(\xDH\), and consequently \(k\), to \(\sk_P\) and \(\alpha\).
    The protocol (setup, distance-bounding and verification phases) should be 
    repeated \(n\) times to achieve the desired soundness and distance-bounding 
    errors.
  }%
  \label{DBSHW-overview}
\end{figure*}

\paragraph*{One-time initialization}

The prover's \(\DBSHW\) device must be initialized with a (static) private 
key~\(\sk_P\) and a signature~\(\sigma\) on that key by some \ac{CA} with 
public key~\(\pk\) and signing key~\(\sk\).
These are computed as follows~\cite[cf.][Sect.~4.2]{CLsignatures}.

We have public parameters \(\qCL, \gCL\in\GCL, \ggCL\in\GGCL, e\); where 
\(e\colon \GCL\times \GCL\to \GGCL\) is a bilinear map, \(\gCL\in \GCL\) and 
\(\ggCL\in \GGCL\) are generators of prime order \(\qCL\) (hence \(e(\gCL, 
\gCL) = \ggCL\)).

We let the \acg{CA} signing key be \[
  \sk = (\xCL, \yCL, \zCL)\rgets \ZZ_{\qCL}^3
\] and the public key be \[
  \pk = (\qCL, \gCL, \ggCL, e, \XCL, \YCL, \ZCL),
\] where \[
  \XCL = \gCL^{\xCL}, \quad
  \YCL = \gCL^{\yCL}, \quad
  \ZCL = \gCL^{\zCL}.
\]

The prover chooses \(\sk_P\rgets \ZZ_{\qCL}\) and computes a commitment \(M = 
\gCL^{\sk_P} Z^r\), where \(r\rgets \ZZ_{\qCL}\).
Now the prover convinces the \ac{CA} that he knows the key by running the 
following \ac{ZKPK} protocol: \(
  \PK[\sk_P, r][M = \gCL^{\sk_P} Z^r].
\)

Now the \ac{CA} computes the signature~\(\sigma = (a, A, b, B, c)\), where \[
  \alpha\rgets \ZZ_{\qCL}, \quad a\gets \gCL^\alpha, \quad
  A\gets \gCL^{\zCL}, \quad
  b\gets a^{\yCL}, \quad
  B\gets A^{\yCL}, \quad
  c\gets a^{\xCL} M^{\alpha \xCL \yCL}
\] and gives to the prover.

\paragraph*{Pre-setup key-agreement}

The pre-setup phase consists of a key agreement, the prover and verifier must 
agree on a shared key~\(k\) to use for the \(\UWBPR\) protocol on the physical 
layer.

To establish \(k\), we use the \ac{DHKE}.
The prover chooses \(\xDH \rgets \ZZ_{\qDH}\), the verifier chooses 
\(\yDH\rgets \ZZ_{\qDH}\) and they exchange \(\XDH = \gDH^{\xDH}, \YDH = 
\gDH^{\yDH}\) and both compute
\(k = \gDH^{\xDH \yDH}\), for some generator \(\gDH\) in \iac{DH} group of 
prime order \(\qDH\).
The prover also computes \(\XDHCL\rgets \XDH^{\sk}\) and sends to the verifier.

\paragraph*{Setup}

The setup phase is similar to \(\DBS\) (\cref{DB-Schnorr-UWB-figure}).
They use the agreed upon key~\(k = \gDH^{\xDH \yDH}\) as the key for 
\(\UWBPR\), instead of a static \(k\).
\(\DBS\) performs \(
  \PK[\alpha][A = g^\alpha]
\) (with a distance bound) whereas \(\DBSHW\) must perform \[
  \PK[\alpha, \xDH, \sk_P, \sigma][
    A = g^\alpha \land
    \XDH = \gDH^{\xDH} \land
    \Blind[\sigma] = \Sign[_{\sk}][\sk_P]
  ].
\] This means that the prover (and verifier) must prepare for the other proofs 
too.

The prover blinds the signature~\(\sigma\):
Choose \(r_1, r_2\rgets \ZZ_{\qCL}\) and compute \(\tilde \sigma = (\tilde a, 
\tilde A, \tilde b, \tilde B, \tilde c)\), where \[
  \tilde a = a^{r_1}, \quad
  \tilde A = A^{r_1}, \quad
  \tilde b = b^{r_1}, \quad
  \tilde B = B^{r_1}, \quad
  \tilde c = (c^{r_1})^{r_2}.
\]
The prover sends \(\tilde \sigma\) to the verifier.
Now, the prover and verifier each compute \[
  v_x\gets e(\XCL, \tilde a), \quad
  v_{xy}\gets e(\XCL, \tilde b), \quad
  V_{xy}\gets e(\XCL, \tilde B), \quad
  v_s\gets e(\gCL, \tilde c).
\]

The prover chooses \[
  \rho\rgets \ZZ_q, \quad
  \rhoDH\rgets \ZZ_{\qDH}, \quad
  \rhoCL_{\sk}\rgets \ZZ_{\qCL}, \quad
  \rhoCL_r\rgets \ZZ_{\qCL}, \quad
  \rhoCL_{r'}\rgets \ZZ_{\qCL}
\] computes \[
  R\gets g^\rho, \quad
  \RDH\gets \gDH^{\rhoDH}, \quad
  \RDHCL\gets \XDH^{\rhoCL_{\sk}}, \quad
  \RCL\gets v_s^{\rhoCL_{r'}} v_{xy}^{\rhoCL_{\sk}} V_{xy}^{\rhoCL_r}
\] and sends \(R, \RDH, \RCL\) to the verifier.

The verifier chooses \(m\) challenges of length \(l\), \[
  c_1\rgets \bin^l, \dotsc, c_m\rgets \bin^l,
\] and sends these to the prover.
The prover computes responses
\begin{multline*}
  s_i\gets \rho - c_i\alpha, \quad
  \sDH_i\gets \rhoDH - c_i \xDH, \\%\quad
  \sCL_i^{(r')}\gets \rhoCL_{r'} + c_i r', \quad
  \sCL_i^{(\sk)}\gets \rhoCL_{\sk} - c_i \sk, \quad
  \sCL_i^{(r)}\gets \rhoCL_r - c_i r
\end{multline*}
for \(i\in \{1, \dotsc, m\}\), where \(r' = r_2^{-1}\).

\paragraph*{Distance bounding}

The verifier decides which of the \(m\) challenges to use, \(i\rgets \{1, 
\dotsc, m\}\), sends its decision~\(i\) to the prover and starts its clock.
The prover instantly replies with the pre-computed \(s_i, \sDH_i, 
\sCL_i^{(r')}, \sCL_i^{(\sk)}, \sCL_i^{(r')}\).
Note that these values must be sent as a concatenation in \emph{one} \(\UWBPR\) 
reply.
The verifier stops the clock and records the round-trip time~\(\Delta t\).

\paragraph*{Verification}

The verifier checks that \[
  R = A^{c_i} g^{s_i}, \quad
  \RDH = \XDH^{c_i} g^{\sDH_i}, \quad
  \RDHCL = \XDHCL^{c_i} \XDH^{\sCL_i^{(\sk)}}, \quad
  v_x^{c_i} \RCL = v_s^{s_{r'}} v_{xy}^{s_{\sk_P}} V_{xy}^{s_r}
\] and that the round-trip time \(\Delta t < t_{\max}\) does not exceed the 
time-limit~\(t_{\max}\) (corresponding to the desired distance bound).
Finally, the verifier checks the validity of \(\tilde \sigma\): \[
  e(\tilde a, \ZCL) = e(\gCL, \tilde A), \quad
  e(\tilde a, \YCL) = e(\gCL, \tilde b), \quad
  e(\tilde A, \YCL) = e(\gCL, \tilde B).
\]


\subsection{Security analysis}

There are two questions we must answer about this protocol:
\begin{enumerate}
  \item\label{DBSHW-MITM} Can the adversary perform \iac{MITM} attack against 
    the \ac{DHKE}?
  \item\label{DBSHW-anon} Does \(\XDHCL\) break anonymity?
\end{enumerate}

To answer the first question,
by \textcite{Camenisch-phdthesis}, the above protocol is \iac{ZKPK}.
In the case of \iac{MITM}, the prover will run the protocol using \(\XDH\) and 
\(\sk\), while the verifier will use \(\XDH'\), possibly modified by the 
adversary.
The adversary's goal is to pass as knowing \(\sk\) and its signature (to 
impersonate real hardware).
The \(\sk\) must be the same \(\sk\) throughout the 
subprotocols~\cite{Camenisch-phdthesis}.
In one subprotocol, the prover will prove knowledge of \(\XDH^{\sk}\).
If \(\XDH' \neq \XDH\) used by the verifier is wrong, then by soundness the 
proof will fail.

To answer the second question,
\(\XDHCL = \XDH^{\sk}\) is simply a representation of \(\sk\) in a randomly 
chosen base~\(\XDH\).
If an adversary can use this to tell one device from another, then the 
adversary can solve the \ac{DDH} problem.
Consider the following reduction.

Assume that \(\adv\) can tell \(\XDH^{\sk}\) from \(\XDH^{\sk'}\) with 
probability \(\frac{1}{2} + \epsilon\) (\ie \(\epsilon\) better than guessing).
We get the \ac{DDH} challenge tuple \(g, X = g^x, Y = g^y, Z = g^z\) and must 
decide if \(Z = g^{xy}\) or not.
Give \(g, R_i = g^{r_i}\) and \(X^{r_i}\) to \(\adv\), for \(1\leq i\leq n\).
Finally, give \(\adv\) the challenge \(g, Y, Z\).
If \(\adv\) recognizes \(Z\), then \(Z = g^{xy}\), and we will solve the 
\ac{DDH} with the same advantage \(\epsilon\).
