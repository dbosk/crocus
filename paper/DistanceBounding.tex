\subsection{Distance bounding}%
\label{DistanceBounding}

\Ac{DB} protocols were first suggested by \citet{DistanceBounding} to prevent relay attacks contactless communications in which the adversary forwards a communication between a prover and a possibility far-away verifier to authenticate. 
These attacks cannot be prevented by cryptographic tools as they are independent of the semantics of the messages exchanged and mechanisms ensuring the physical proximity between a verifier and a prover must be used.
\Ac{DB} protocols precisely enable the verifier to estimate an upper bound on his distance to the prover by measuring the time-of-flight of short challenge-response messages (or rounds) exchanged during time-critical phases. 
Time critical phases are complemented by slow phases during which the time is not taking into account.
At the end of a \Ac{DB} protocol, the verifier should be able to determine if the prover is legitimate and in his vicinity.

The main security requirements of \ac{DB} protocols can be summarized as follows:
\begin{itemize}
  \item \ac{DBMF}: the adversary illegitimately authenticates, possibly using a 
    far-away honest prover.
  \item\ac{DBTF}: a malicious prover helps an accomplice to authenticate.
    \item \ac{DBDF}: a legitimate but malicious prover wants to fool the verifier on the distance between them.
  \item \ac{DBDH}: similar to \ac{DBDF}, sometimes using the presence of an honest prover close to the verifier.
  \item \ac{DBIF}: the adversary plays against a simplified version of the protocol without any distance estimation.
\end{itemize}
\daniel{Describe the DB frauds in more detail, i.e. explain them.}
There are two lines of attempts at formalizing the above properties: one by \citet{DB-BMV} and another by \citet{DB-DFKO}.

The majority of the existing \ac{DB} protocols are symmetric and thus requires an honest verifier.
Indeed, in this context it does not make sense to protect against the verifier as he can impersonate easily the prover as he has a knowledge of his secret key.
There has been less work done in the domain of asymmetric (or public-key) \ac{DB} protocols.
Unfortunately, our setting requires a public-key \ac{DB} protocol with a \emph{malicious verifier} who will try to \emph{impersonate the prover}.
The verifier might also try to track the provers and map their identities to their actions, thus we also require privacy.
This leads to the requirement of a \ac{DB} \ac{ZKPK}, or simply \ac{DBPK}, with resistance to all the frauds mentioned above as well as against this verifier 
who will attempt to impersonate the prover.
More specifically, we must combine such \iac{DB} with anonymous credentials, which we will describe next.
The anonymous credential system will affect what properties we require of the \ac{DBPK}, (\eg that it must be \iac{PKE} for discrete logarithms).

