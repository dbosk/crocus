\subsection{Distance bounding}%
\label{DistanceBounding}

% XXX Better intro to distance bounding
\Ac{DB} protocols were first suggested by \citet{DistanceBounding}.
Over the years the desirable properties of \ac{DB} protocols can be summarized 
as follows:
\begin{itemize}
    % XXX Describe the DB frauds in more detail, i.e. explain them.
  \item \ac{DBMF} resistance,
  \item \ac{DBDF} resistance,
  \item \ac{DBDH} resistance,
  \item \ac{DBTF} resistance,
  \item \ac{DBIF} resistance,
\end{itemize}
There are two lines of attempts at formalizing the above properties: one by 
\citet{DB-BMV} and another by \citet{DB-DFKO}.

The majority of the existing \ac{DB} protocols are symmetric and thus requires 
an honest verifier.
There has been less work done in the domain of asymmetric (or public-key) 
\ac{DB} protocols.
Unfortunately, our setting requires a public-key \ac{DB} protocol with a 
\emph{malicious verifier} who will try to \emph{impersonate the prover}.
The verifier might also try to track the provers and map their identities to 
their actions, thus we also require privacy.
This leads to the requirement of a \ac{DB} \ac{ZKPK}, or simply \ac{DBPK}, with 
resistance to all the frauds mentioned above as well as against this verifier 
who will attempt to impersonate the prover.
More specifically, we must combine such \iac{DB} with anonymous credentials, 
which we will describe next.
The anonymous credential system will affect what properties we require of the 
\ac{DBPK}, e.g.\ that it  must be \iac{PK} for discrete logarithms.
