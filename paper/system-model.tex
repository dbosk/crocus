\section{System and adversary model}%
\label{system-model}

To set the stage, we will now describe our system model including assumptions (\cref{assumptions}) and present an abstraction of a protest and how to
count the participation (\cref{protest-model}). 
Afterwards, we formulate the desired verifiability (\cref{verifiability-properties}) and privacy properties
(\cref{privacy-properties}) as well as the adversary model (\cref{adversary-model}).

\subsection{System model and physical assumptions}
\label{assumptions}

Throughout this paper, each time we refer to a participant, such as Alice, we actually mean a personal device that can perform cryptographic operations and communicate with other local devices on Alice's behalf. 
Furthermore, we assume that each participant has a digital certificate signed by the \ac{CA}.
For instance, we could use the cryptographic keys of any national electronic identity system, identity card or passport.
This assumption is needed to prevent Sybil attacks~\cite{SybilAttack}, which would have a disastrous effect on the crowd-counting estimation.

In practice, participants witness each other's participation using their smartphones (or similar devices) running the protocol described in \cref{Protocol} and uploading their testimony (\ie, proof shares) to a ledger (such as a blockchain) after the protest. 
During the protest, the devices might be limited by their batteries in terms of the computations they can perform and only have local connectivity to each other.
No connection to any global network such as the Internet is necessary at that time.  
Nonetheless, before and after the protest, we assume that the devices have global connectivity (\ie Internet connections) and are not computationally limited by any battery.
This assumption is necessary to ensure that the participants will be able to upload their proof shares to the ledger.
%\sonja{another place to put reasons? Or if here, then for all
%  assumptions. ``We need global connectivity before to read the head from the blockchain.
%We need global connectivity after to upload the proof shares to the blockchain.''}

\include*{protest-model}
\include*{verifiability-properties}
\include*{privacy-properties}
\subsection{Adversary model}
\label{adversary-model}

We consider two types of adversaries: \begin{enumerate*} \item an honest-but-curious adversary
of up to national scale representing a government that can observe any
messages sent, and \item a malicious adversary that can control a
minority of 
equipment such as participants' or witnesses' smartphones or servers
used for the ledger such as miners for a blockchain. \end{enumerate*}
