\subsection{System model and assumptions}%
\label{assumptions}

Throughout this paper, each time we refer to a participant, such as Alice, we 
actually mean a personal device that can perform cryptographic operations and 
communicate with other local devices on Alice's behalf.
Furthermore, we assume that each participant has a digital identity certificate signed 
by \iac{CA}\footnote{%
  We note that any system that prevents Sybil attacks~\cite{SybilAttack} will 
  do.
} that ensures a one-to-one mapping between an identity and a cryptographic 
key\footnote{%
  For every physical person there exists at most one valid digital certificate.
}.

In practice, participants witness each other's participation using
their smartphones (or similar devices) running the protocol described
in \cref{Protocol} and uploading their testimony (\ie, proof shares)
to a ledger (such as a blockchain) after the protest.  During the
protest, the devices might be limited by their batteries and
computations they can perform and only have local connectivity to each
other.  No connection to any global network such as the Internet is
necessary at that time.  Nonetheless, before and after the protest, we
assume that the devices have global connectivity (\ie Internet
connections) and are not computationally limited by any battery, for
the participants to be able to upload their proof shares to the
ledger.

Some of the assumptions that are required for implementing and deploying our proposition are 
not yet realized, however, we believe that they soon will
be:\footnote{see the long version of the paper for a more thorough discussion}
ubiquitous usage of smartphones, with an identity credential
signed by some \ac{CA}\footnote{based on \eg governmental ID on a chip as in Estonia, national
  but private as BankID in Sweden, eIDAS in the EU, or self-sovereign
  identities} ,  distance-bounding capabilities\footnote{based on \eg
  3DB-access or key-fob technology integration with phones}, and
infrastructure-independent wireless communications\footnote{based on
  \eg NFC (available but inconvenient), bluetooth or WIFI (Briar and
  FireChat have been used in protests), 5G D2D (scalable but needs
  authorization step)}. 

% \subsection{Adversary model}%
% \label{adversary-model}

% We consider two types of adversaries: 
% \begin{enumerate*} 
% \item an \emph{honest-but-curious adversary} of up to national scale representing a government that can observe any
% messages sent on communication links (\ie either during the protest or later during the upload of the participation), and 
% \item a \emph{malicious adversary} that can control a minority of equipments such as participants' or witnesses' smartphones or servers used for the ledger such as miners for a blockchain. 
% \end{enumerate*}
