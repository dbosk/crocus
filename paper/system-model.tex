\section{System and adversary model}%
\label{system-model}

To set the stage, we will now describe our system model including assumptions, a summary of the desired properties for crowd counting and the adversary model.

\subsection{System model and physical assumptions}%
\label{assumptions}

Throughout this paper, each time we refer to a participant, such as Alice, we actually mean a personal device that can perform cryptographic operations and communicate with other local devices on Alice's behalf. 
Furthermore, we assume that each participant has a digital certificate of her identity signed by a \ac{CA}.

In practice, participants witness each other's participation using their smartphones (or similar devices) running the protocol described in \cref{Protocol} and uploading their testimony (\ie, proof shares) to a ledger (such as a blockchain) after the protest. 
During the protest, the devices might be limited by their batteries in terms of the computations they can perform and only have local connectivity to each other.
No connection to any global network such as the Internet is necessary at that time.  
Nonetheless, before and after the protest, we assume that the devices have global connectivity (\ie Internet connections) and are not computationally limited by any battery.
This assumption is necessary to ensure that the participants will be able to upload their proof shares to the ledger.

\subsection{Desired properties}%
\label{desired-properties}

We note that, in general terms, protesting is very similar to petitions, which 
in turn are similar to voting.
Indeed, all three situations correspond to situations in which many individuals express their opinion.
These opinions can be sensitive (\eg be a cause for discrimination or persecution), hence we desire to have similar properties of verification and privacy for verifying a protest as there are for voting.

We draw inspiration from the three traditionally desired properties for verifiability for voting:
\daniel{Find a new ref for voting verifiability properties, Douglas commented 
  that the last one is not the origin.}
\begin{description}
  \item[Eligibility:] anyone can verify that each cast vote is legitimate.
  \item[Universal verifiability:] anyone can verify that the result is according to the cast votes.
    \daniel{Douglas: technically this is Sako-Kilian, non-technically, it's 
      old.}
  \item[Individual verifiability:] each voter can verify that his vote is included in the result.
    \daniel{Douglas: this is much earlier than Sako-Kilian, in essence 
      Fiat-Shamir (technically). Non-technically, much older.}
\end{description}
In our context, votes are translated into \emph{participation proofs}.
Universal and individual verifiability remain the same, in the sense that anyone can verify the participation count by counting the proofs.
The eligibility requirement is slightly different as for protests it must 
include temporal\footnote{%
  Elections also have some temporal aspects, such that one cannot vote at any 
  time and a vote in one election should not be reusable in the next election.
  However, the temporal relation for a protest is not as strictly defined as for 
  elections as a protest can start at any time and last for an arbitrary period 
  of time.%
} and spatial eligibility (\ie each participation proof satisfies some temporal 
and spatial relation to the protest).
In essence, the proof must bind the person to the time and location of the protest.

In addition to the verification requirements, we also need to define the 
privacy properties.
In voting protocols, there are mainly three levels of privacy~\cite{VerifyingPrivacyPropertiesOfVotingProtocols}:
\begin{description}
  \item[Vote privacy:] the voting does not reveal any individual vote.
  \item[Receipt freeness:] the voting system does not provide any data that can be used as a proof of how the voter voted.
  \item[Coercion resistance:] a voter cannot cooperate with a coercer to prove his vote was cast in any particular way.
\end{description}

Coercion resistance is not possible to achieve for protests.
For instance, someone could simply physically bring Alice to a protest against her will.
Receipt freeness, while appealing at first glance, is not desirable (probably 
not possible) in our context since this will conflict with verifiability: in 
contrast to voting, receipt freeness for \emph{how} the voter voted (\ie the 
cause of the protest) implies receipt freeness for \emph{that} the voter voted 
(\ie the protester was there), which conflicts with verifiability.

Thus, in our context the crucial property is therefore privacy.
More precisely, for the protester we want unlinkability (from the adversary's 
perspective) between a protester's real identity \(P\) and the participation 
proof (and thus also the protest itself).
Phrased differently, given a participation proof, Grace should not be able to 
distinguish if it was Alice or Bob who participated.
Furthermore, if Grace has managed to link one proof to Alice due to some 
auxiliary knowledge, she should not be able to link it to another proof (\eg 
from a different protest).


\subsection{Adversary model}%
\label{adversary-model}

We consider two types of adversaries: 
\begin{enumerate*} 
\item an \emph{honest-but-curious adversary} of up to national scale representing a government that can observe any
messages sent on communication links (\ie either during the protest or later during the upload of the participation), and 
\item a \emph{malicious adversary} that can control a minority of equipments such as participants' or witnesses' smartphones or servers used for the ledger such as miners for a blockchain. 
\end{enumerate*}
