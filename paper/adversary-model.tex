\subsection{Adversary models}%
\label{adversary-model-different-levels}

We provide one system model with two possible adversaries: one who tries to link a protester's real identity (\(P\)) to a protest cause (\(\cid\)) and one 
that aims at linking a protester's or a witness's real identity (\(P\), respectively \(W\)), to a location (\(l\)).
We also define varying strengths of each adversary.

\paragraph{The identity--cause linking adversary}

There are three players: the protest participant (with identity) \(P\), a witness (with identity) \(W\) and the storage \(S\).
The adversary \(A\) controls \(W\) and \(S\).

\begin{definition}[Base identity--cause adversary]%
  \label{base-identity-cause-adversary}
  The protester \(P\) and the witness \(W\) communicate.
  Each learns only the protocol data \(d_{P,W}(\cid, P)\) and when it happened 
  \(t_{P,W}\)\footnote{%
    Specifically, they do \emph{not} learn the real identities \(P\) and \(W\) 
    directly, only if those appear in the data \(d_{P,W}(\cid, P)\).
  }.
  The protester \(P\) communicates with \(S\), in which \(S\) only learns 
  \(f(d_{P,W}(\cid, P))\), for some function \(f\), and the time of the 
  communication (\(t_{P,S}\)) but not the real identities.
  The adversary controls \(W\) and \(S\) and thus learns everything that they 
  do, but can additionally correlate what he learns from \(W\) and \(S\).
\end{definition}

This definition is illustrated in \cref{fig:identity-cause-adversary}.

\begin{figure}
  \centering
  \includegraphics{base-identity-cause-adversary.tikz}
  \caption{\label{fig:identity-cause-adversary}%
    An overview of the base identity--cause adversary model.
    The protester with real identity \(P\) and witness with real identity \(W\) 
    communicate and each learn only the protocol data, \(d_{P,W}(\cid, P)\), 
    and the time it happened, \(t_{P,W}\).
    The protester submits \(f(d_{P,W}(\cid, P))\), for some function \(f\), to 
    the storage \(S\), who learns only that and the time it happened, 
    \(t_{P,S}\).
    Both the witness \(W\) and storage \(S\) are controlled by the adversary 
    \(A\).
  }
\end{figure}

We find \cref{base-identity-cause-adversary} suitable when the protester and witness both move in a crowd and there is no way for the witness to decide exactly with whom he or she communicates with.
However, in some situations this might not be the case, \eg the crowd is not dense.
In these situations the witness will likely see the face of the protester.
We capture this by the following definition.

\begin{definition}[Stronger identity--cause adversary]%
  \label{stronger-identity-cause-adversary}
  The situation is the same as in \cref{base-identity-cause-adversary}, but now the 
  witness \(W\) learns the protester \(P\)'s identity.
  (\(P\) will also learn \(W\)'s identity.)
  However, \(S\) still does not learn \(P\)'s identity.
\end{definition}

One reason to not allow \(S\) to learn \(P\)'s identity is that for this communication \(P\) has options, such as Tor~\cite{Tor}, for anonymous communication.
However, given a strong enough adversary, such anonymous communication might not be possible.
We capture such a strong adversary in the following definition.

\begin{definition}[Strongest identity--cause adversary]%
  \label{strongest-identity-cause-adversary}
  Everything is the same as in \cref{stronger-identity-cause-adversary}, except that now \(S\) also learns \(P\)'s identity.
\end{definition}

\paragraph{The identity--location adversary}

The players are still the same, but now the adversary controls other players.
In particular, the adversary controls the storage \(S\).

\begin{definition}[Base identity--location adversary]%
  \label{base-identity-location-adversary}
  The adversary controls \(S\).
  The witness \(W\) and the protester \(P\) communicate.
  They learn some protocol data \(d_{P,W}(P, W, l)\) and the time it took 
  place, \(t_{P,W}\).
  \begin{description}
    \item[The witness case:]
      The witness later communicates with \(S\), in which \(S\) only learns 
      \(f(d_{P,W}(P, W, l))\), for some function \(f\), and the time it 
      happened, \(t_{W,S}\).
      The adversary wins if he can link \(W\) and \(l\).
    \item[The protester case:]
      The protester later communicates with \(S\), in which \(S\) only learns 
      \(g(d_{P,W}(P, W, l))\), for some function \(g\), and the time it 
      happened, \(t_{P,S}\).
      The adversary wins if he can link \(P\) and \(l\).
  \end{description}
\end{definition}

\Cref{fig:identity-location-adversary} illustrates the roles and relations.

\begin{figure}
  \centering
  \includegraphics{base-identity-location-adversary.tikz}
  \caption{\label{fig:identity-location-adversary}%
    An overview of the base identity--location adversary model.
    The protester with real identity \(P\) and witness with real identity \(W\) 
    communicate and each learn only the protocol data, \(d_{P,W}(P, W, l)\), 
    and the time it happened, \(t_{P,W}\).
    The witness submits \(f(d_{P,W}(P, W, l))\), for some function \(f\), and 
    the protester submits \(g(d_{P,W}(P, W, l))\), for some function \(g\), to 
    the storage \(S\), who learns only that and the time it happened, 
    \(t_{W,S}\) respectively \(t_{P,S}\).
    Only the storage \(S\) is controlled by the adversary \(A\).
  }
\end{figure}

The base rationale is the same as for \cref{base-identity-cause-adversary}, the 
witness \(W\) can submit to \(S\) over an anonymous channel, such as Tor~\cite{Tor}.
Anyone who is in the same location will learn that \(W\) was present, we want 
to prevent anyone \emph{not present} to learn that \(W\) was there --- hence we 
do not consider the protester as part of the adversary, only \(S\).
Analogous to the identity--cause adversary, we conclude with the strongest 
identity--location adversary.

\begin{definition}[Strongest identity--location adversary]%
  \label{strongest-identity-location-adversary}
  Things are the same as in \cref{base-identity-location-adversary}, except 
  that \(S\) learns the witness' and protester's identities \(W\) and \(P\), in 
  the respective cases.
\end{definition}
