\subsection{Adversary model}%
\label{formal-adversary-model}

There are three players: the protester (with identity) \(P\), a witness (with 
identity) \(W\), and a ledger~\(\TS\) (\ie time-stamping storage, that will 
contain the set of all proof shares, \(\pshs\)).
The protester \(P\) and the witness \(W\) communicate some protocol data,
\(d_{P,W}(\cid, P)\), and record when the communication occurred \(t_{P,W}\).
The protester \(P\) communicates with \(\TS\) and \(\TS\) only learns 
some function \(f\) of the protocol data exchanged with the witness,
\(f(d_{P,W}(\cid, P))\), and the time of the 
communication (\(t_{P,\TS}\)).
This is illustrated in \cref{fig:base-adversary}.

\begin{figure}
  \centering
  \includegraphics{base-adversary.tikz}
  \caption{\label{fig:base-adversary}%
    An overview of the adversary model.
    The protester (with real identity) \(P\) and witness (with real identity) 
    \(W\) communicate.
    They exchange protocol data as a function~\(d\) of the protest and 
    protester, \(d_{P,W}(\cid, P)\), and record the time it happened, 
    \(t_{P,W}\).
    The protester submits \(f(d_{P,W}(\cid, P))\), for some function \(f\), to 
    the ledger \(\TS\), who records the time that happened, \(t_{P,\TS}\).
    %Both the witness \(W\) and the ledger \(\TS\) are controlled by the 
    %adversary.
  }
\end{figure}

The goal of the adversary is to link a protester's real identity~\(P\) to a 
protest identifier~\(\cid\).
The adversary maliciously controls \(W\).
The adversary honest-but-curiously controls \(\TS\), but can submit to \(\TS\) 
like everyone else (\eg a malicious protester~\(P\)).
The adversary only learns the protocol data, \ie what is sent over the channel 
--- no auxiliary data such as who is on the other side of the channel obtained
  as side information, \eg by face recognition or inference from address
  to identity.

