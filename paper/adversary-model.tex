\subsection{Adversary model}%
\label{formal-adversary-model}

We now provide a concretely defined system and adversary model.
We give three definitions, each defines an adversary with increasingly stronger 
capabilities (\ie more auxiliary information).

There are three players: the protester (with identity) \(P\), a witness (with 
identity) \(W\) and the time-stamping storage \(S\) (that will contain the set 
of all proof shares, \(\pshs\)).
The protester \(P\) and the witness \(W\) communicate some protocol data,
\(d_{P,W}(\cid, P)\), and records when the communication occurred \(t_{P,W}\).
The protester \(P\) communicates with \(S\), in which \(S\) only learns some 
function of the protocol data exchanged with the witness, \(f(d_{P,W}(\cid, 
  P))\) for some function \(f\), and the time of the communication 
(\(t_{P,S}\)).
This is illustrated in \cref{fig:base-adversary}.

\begin{figure}
  \centering
  \includegraphics{base-adversary.tikz}
  \caption{\label{fig:base-adversary}%
    An overview of the base adversary model.
    The protester with real identity \(P\) and witness with real identity \(W\) 
    communicate.
    They exchange protocol data, \(d_{P,W}(\cid, P)\), and record the time it 
    happened, \(t_{P,W}\).
    The protester submits \(f(d_{P,W}(\cid, P))\), for some function \(f\), to 
    the storage \(S\), who records the time it happened, \(t_{P,S}\).
    Both the witness \(W\) and storage \(S\) are controlled by the adversary 
    \(A\) (from \cref{in-system-adversary}).
  }
\end{figure}

\begin{adversaries}
\item\label{in-system-adversary}
  \emph{In-system adversary}:
  The adversary maliciously controls \(W\).
  The adversary honest-but-curiously controls \(S\).
  The adversary only learns the protocol data --- \ie what is sent over the 
  channel, no auxiliary data\footnote{%
    By auxiliary data we mean any data outside of the protocol.
    \Eg meta-data such as who is on the other side of the channel.
  }

\item\label{deanonymizing-storage-adversary}
  \emph{Global, passive adversary}:
  The situation is the same as in \cref{in-system-adversary}, but now the storage 
  \(S\) learns the protester \(P\)'s identity from an auxiliary channel.
  However, \(W\) still learns only the protocol data, no auxiliary data.

%\item\label{deanonymizing-witness-adversary}
%  \emph{Deanonymizing-witness adversary}:
%  Everything is the same as in \cref{deanonymizing-witness-adversary}, except 
%  that now \(W\) also learns \(P\)'s identity from an auxiliary channel.
\end{adversaries}

The in-system adversary (\cref{in-system-adversary}) represents an adversary 
that has no access to data from outside the system, \eg inferences that can be 
done from the communication layer (\eg IP-addresses mapped to physical 
identities), which means that it has only the protocol data at its disposal.

We find \cref{in-system-adversary} suitable when the protester and witness both 
move in a crowd and there is no way for the witness to decide exactly with whom 
he or she communicates with.
(The protesters could also simply wear masks.)
We consider this a fair model for our purposes: if the adversary can 
deanonymize the protesters \emph{at the physical location} and thus arrest 
them, the adversary can do that independently of whether the protesters use our 
protocol or not --- thus our protocol cannot make things worse in such a 
situation.

As to \(S\), \cref{in-system-adversary} is equivalent to the adversarial 
storage \(S\) receiving only perfectly anonymous connections.
Another way to see it is that the adversary cannot control \(S\) but can learn 
what \(S\) learns as soon as \(S\) learns it; \eg \(S\) is operated outside the 
adversary's jurisdiction and so the adversary cannot control \(S\), but since 
\(S\) provides public access the adversary can (possibly constantly) query it 
and learn things with little delay.
Posed like this, the adversary cannot observe the channel between \(P\) and 
\(S\), \eg when \(P\) uses tools such as Tor~\cite{Tor}.
Note that the adversary is also allowed to submit data to \(S\) for storage, 
just like everyone else.

\Cref{deanonymizing-storage-adversary} is a much stronger adversary.
Even if the adversary cannot observe the channel between \(P\) and \(S\) (as is 
the case with Tor), the adversary can still learn the identity of \(P\) through 
\eg time-correlation attacks (as is the case with Tor).
A nation state can potentially monitor the traffic of all \acp{ISP} and thus 
perform time-correlation attacks on messages on the network and events observed 
from \(S\).
We capture this in \cref{deanonymizing-storage-adversary}, where \(A\) learns 
the identity of \(P\) in the communication with \(S\).

%However, in some situations this might not be the case: \Eg if the crowd is not 
%dense the witness will likely see the face of the protester.
%If the witness is controlled by the adversary, then it is likely that the 
%witness can capture a picture of the face, which can be turned into an identity 
%through face recognition.
%There are various such scenarios leading to the adversary learning the 
%protester's identity, we capture this as 
%\cref{deanonymizing-witness-adversary}.
