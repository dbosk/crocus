\subsection{Adversary model}%
\label{formal-adversary-model}

% We now provide a concretely defined system and adversary model.
% We give three definitions, each defining an adversary with increasingly stronger 
% capabilities (\ie more auxiliary information).

There are three players: the protester (with identity) \(P\), a witness (with 
identity) \(W\) and the time-stamping storage \(\TS\) (that will contain the 
set of all proof shares, \(\pshs\)).
The protester \(P\) and the witness \(W\) communicate some protocol data,
\(d_{P,W}(\cid, P)\), and records when the communication occurred \(t_{P,W}\).
The protester \(P\) communicates with \(\TS\), in which \(\TS\) only learns 
some function of the protocol data exchanged with the witness, 
\(f(d_{P,W}(\cid, P))\) for some function \(f\), and the time of the 
communication (\(t_{P,\TS}\)).
This is illustrated in \cref{fig:base-adversary}.

\begin{figure}
  \centering
  \includegraphics{base-adversary.tikz}
  \caption{\label{fig:base-adversary}%
    An overview of the adversary model.
    The protester (with real identity) \(P\) and witness (with real identity) 
    \(W\) communicate.
    They exchange protocol data as a function~\(d\) of the protest and 
    protester, \(d_{P,W}(\cid, P)\), and record the time it happened, 
    \(t_{P,W}\).
    The protester submits \(f(d_{P,W}(\cid, P))\), for some function \(f\), to 
    the storage \(\TS\), who records the time that happened, \(t_{P,\TS}\).
    Both the witness \(W\) and the storage \(\TS\) are controlled by the 
    adversary.
  }
\end{figure}

The adversary maliciously controls \(W\).
The adversary honest-but-curiously controls \(\TS\), but can submit to \(\TS\) 
as everyone else.
The adversary only learns the protocol data --- \ie what is sent over the 
channel, no auxiliary data\footnote{%
  By auxiliary data we mean any data outside of the protocol, \ie
  meta-data such as who is on the other side of the channel obtained
  as side information, \eg by face recognition or inference from address
  to identity.
}

This adversary has no access to data from outside the system, \eg inferences 
that can be made from the communication layer (\eg IP-addresses mapped to 
physical identities or face recognition from the protest), which means that it 
has only the protocol data at its disposal.

