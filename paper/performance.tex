\mode<presentation>{%
  \begin{frame}
    \begin{question}
      \begin{itemize}
        \item How costly is this compared to non-\acs{DB} Schnorr?
      \end{itemize}
    \end{question}
  \end{frame}
}

We will now review the performance of the proposed protocols.
Since the construction is designed as a drop-in replacement for the Schnorr 
identification-scheme~\cite{Schnorr} as \iac{ZKPK} protocol, we will focus on 
how much we must \enquote{pay} for the \ac{DB} property.
We summarize the results in \cref{performance-overview}.

\begin{frame}
  \begin{table}
    \centering
    \caption{%
      A summary of how costly \acl*{DB} is in terms of randomness, arithmetic 
      operations (addition, multiplication), exponentiations and pairing 
      operations.
      \(m, l\in \mathcal{O}(\log \lambda)\) and we must repeat the protocol 
      \(n\) times to achieve desired soundness.
    }\label{performance-overview}
    \begin{tabular}{lrrrr}
      \toprule
      Protocol
      & Rand.~(\si{\bit})
      & Arith.~(\(+, \times\))
      & Exp.
      & Pairings\\
      \midrule
      Schnorr             & \((\lambda+l)n\)  & \(3n\) & \(3n\) & 0\\
      DB, mutual          & \((\lambda+ml)n\) & \(3mn\)& \(3n\) & 0\\
      DB, \acs*{PKI} \(\sum\)
                          & \(4\lambda + (5\lambda + ml)n\)
                          & \((10m+8)n\)
                          & \(9 + 16n\)
                          & \(14\)\\
      ---\acs*{DHKE}      & \(2\lambda\)      & 0    & 4      & 0\\
      ---CL04 blind       & \(2\lambda\)      & 0    & 5      & 8\\
      ---DB               & \(5\lambda + ml\) & \(10m + 2\) & 6 & 0\\
      ---Verification     & 0    & 6 & 10 & 6\\
      \bottomrule
    \end{tabular}
  \end{table}
\end{frame}

\paragraph*{Mutual trust}

We start with the mutual-trust version of the protocol in \cref{DB-Schnorr-UWB} 
(see \cref{DB-Schnorr-UWB-figure}).

In \emph{one round} of the protocol, the cost of the verifier is that of 
generating \(m\) number of \(l\)-bit strings, instead of only one.
This costs the verifier a factor \(m\) of randomness.
The cost of the prover is that of computing \(m\) number of replies (\(s\gets 
\rho - c \alpha \mod q\)), instead of only one.

To achieve security, the protocol must be repeated \(n\) times.
We have the \(l\)-parameter which controls the soundness of the \ac{ZKPK} and 
the \(m\)-parameter controls the soundness of the \acl{DB}.
If we want to achieve 80 bits of security, we can let \(l = m = 6\) (remember, 
\(l\) must be logarithmic in the security parameter) and \(n = 14\).
This makes \acl{DB} a factor \(6\) more expensive over the 
\emph{malicious-verifier secure} Schnorr protocol.

\paragraph*{\acs*{PKI}}

The \ac{PKI} version of the protocol (\cref{DB-Schnorr-PKI,DBSHW-overview}) 
performs one signature verification and four additional proofs in parallel.
The cost for the verifier is one signature verification 
(CL04~\cite{CLsignatures}) in addition to the factor \(m\) of randomness used.
On the prover side, we have the computations for the signature verification and 
four additional proofs in parallel, this changes the cost to a factor of \(5m\) 
for the prover: the prover must compute \(s_i = \rho - c_j \alpha_i\) for 
\(i\in \{1, \dotsc, 5\}, j\in \{1, \dotsc, m\}\).

The cost of the signature is the same as for Anon-Pass~\cite{AnonPass}, which 
implements a public transport pass using this signature scheme.
This signature verification introduces more additions, multiplications, 
exponentiations and pairing operations.
However, the signature will only be blinded and verified once, so this 
represents a constant factor.
It is sufficient to verify it once and reuse the same blinding for all \(n\) 
repetitions of the protocol.
(The same is true for the \ac{DHKE}.)
Thus it is only the \acl{PK} that must be rerun \(n\) times.

\paragraph*{Bit-by-bit}

The traditional bit-by-bit version of the protocol 
(\cref{DB-Schnorr-nbit,DB-Schnorr-nbit-figure}) is not as efficient.
That is due to \(m = 2\) being fixed in this case (there are only two responses 
to choose from), and consequently, we must have a larger value for \(n\) to 
achieve the desired security.

