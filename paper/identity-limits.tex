\section{Limits of current identity systems}

\Textcite{SybilAttack} points out that any identity system must be logically 
centralized to prevent Sybil attacks.
It follows that any system where users themselves can create identities cannot 
provide strong guarantees against Sybil attacks.

The most widely deployed identity system is that of nation states: in essence, 
an identity is created at birth in the form of the equivalent of a birth 
certificate\footnote{%
  Many countries issue birth certificates, but there are other, equivalent, 
  ways.
  \Eg Sweden does not issue birth certificates, it simply creates and entry in 
  the population registry.
} This birth certificate later serves as the root for many authentication 
mechanisms, \eg national identity cards and passports.

Some more \enquote{decentralized}\footnote{%
  Provably, it must be logically centralized anyway.
} methods have been suggested, \eg 
proof-of-personhood~\cite{proof-of-personhood}.
In this particular approach, people gather in a room, \emph{everyone ensures} 
that no one can enter or leave during the process.
Then \emph{everyone ensures} that everyone gets exactly one identity.

We will now discuss the limits and possibilities of these approaches.

\subsection{The \enquote{birth-certificate} system}

This system is logically centralized, it is controlled (enforced) by the 
government.
However, it is usually distributed among a country's hospitals.
It usually works without any problems: citizens --- who have an incentive to 
create more identities, \eg to accumulate various benefits --- cannot create 
new identities without babies.
Someone who has a baby does not want to give that identity to someone else, 
because then their baby will not have one --- usually the biological mechanisms 
that ensures the well-being of our offspring prevents any abuse on this side 
too.
But obviously there exists a possibility to create birth certificates without 
newborn babies, it is just that those who can either do not have any incentive 
to do so or there are other mechanisms, \eg separation of duties and 
accountability.

There is one particular situation where the incentive arises:
to influence democratic processes\footnote{%
  There are a few more, less relevant, \eg to have multiple identities for 
  foreign operatives, as seen on film.
  But those are not significant to our discussion.
}.
If the government can create identities and control them, they can use them to 
vote for themselves in upcoming elections.
In a well-designed system, however, this should be prevented.
This varies from country to country (\ie system to system).

\subsubsection{The case of Sweden}

An identity is created when an entry is made in the population registry.
Each entry in the registry has a unique identification number.
The electoral roll is created from the population 
registry~\cite{Valmyndigheten-ElectoralRoll}.

For a person to cause an entry into the population registry, they have two 
options:
they are in the population registry and give birth to a child, or
they were born abroad and apply to immigrate to Sweden.

In the case of a birth\footnote{%
  Technically, this is the procedure if the child is born abroad.
  If the child is born in Sweden, the hospital will report all these things 
  directly to the population registry without issuing any papers to the 
  parents.
} they need:
\begin{itemize}
  \item passports, to prove the identities of the parents;
  \item a birth certificate where the names of the parents are present;
  \item a pregnancy certificate, to prove who birthed the child;
  \item a certificate of who the father is~\cite{Skatteverket-RegisterBirth}.
\end{itemize}
So the security is reduced to the forgeability of these items.

Obviously the one who creates entries in the population registry can do that 
even without such documents, so there must be some separation of duty, 
accountability or transparency to prevent the agency from performing a Sybil 
attack.
In the case of Sweden, the population registry is public.
This means that in addition to any audit trail in place at the agency, we have 
these properties:
\begin{itemize}
  \item The data must fulfil basic integrity: a person must have parents, who 
    in turn must have parents, \etc.
  \item Anyone can detect extraordinary additions, \eg exceptionally many 
    births, additions of family trees, \etc.
    Individual women and men would also notice if extra children were added to 
    them.
  \item Addition of persons of foreign heritage must have corresponding 
    documentation (audit trail) at the Migration Board.
\end{itemize}
A foreign citizen can prove their identity to the Migration Board in the 
following ways:
\begin{itemize}
  \item showing a national passport in the original or
  \item showing an identity document in the original or
  \item a close relative attesting to the 
    identity~\cite{Migrationsverket-ProvenIdentity}.
\end{itemize}
Thus the security is reduced to subverting one of these methods.

\subsubsection{The case of France}

The French electoral roll is populated by citizens registering at the town 
hall.
They must then be able to authenticate their identity using an identity 
credential.

Whenever a child is born, the hospital reports this to the local town hall.
They register the birth and issues a birth certificate.
While they do not actively maintain the current whereabouts of individuals and 
do not give each an identity number, their identity consists of the tuple of 
name, date and place of birth.
Thus an identity can be verified by asking the specified town hall to verify 
the birth in their records.

This makes the audit trail more complex.
It is also difficult to detect modifications to these records, since they are 
not public.

\subsubsection{Perfectly decentralized and private birth-certificates approach}

There are no audit records, the one who can create authentic birth certificates 
can perform Sybil \dots


\subsection{Issued credentials}

In Sweden, there are several entities that are authorized to issue identity 
cards; \eg the Swedish Tax Agency, the Swedish Police, the banks.
Thus, technically, to detect a Sybil attack by them, \ie fake identities, one 
must check the ID against the population registry.

With anonymous credentials, this becomes a problem.
If two entities issues identity credentials to Alice, there is no problem: if 
she uses the first to authenticate to Bob, and later comes back and uses the 
second, Bob will that she is Alice both times.
With anonymous credentials that prevent multiple use, this is no longer the 
case.
Thus issuers of such credentials must synchronize, so that if one has issued 
such credentials to Alice, the others must not issue another.

One problem here is that there can be no audit trail.
Thus the issuers have the possibility of performing Sybil attacks.


\subsection{Proofs-of-personhood}

Even physically decentralized attempts, such as 
proof-of-personhood~\cite{proof-of-personhood}, are logically centralized.
For a nation-wide proof-of-personhood, everyone in the nation must be at the 
same place at the same time.
This is obviously not possible and to make it feasible we must split across 
multiple locations\footnote{%
  We cannot split across time.
  The base assumption is that we do not have any means of authenticating 
  people, thus we must rely on the physical impossibility of being in two 
  places at the same time.
  Obviously it is not impossible to be in the same place at two times.
} and rely on trust: if Alice is in one location, she must trust that at least 
one in every other location is also honest.

The issue with this system is scalability.
Every time that \emph{anyone} needs a new credential, we must involve 
\emph{everyone}.
Getting everyone to be at some fixed locations at the same time is even 
difficult to achieve during Election Day, it must usually be spread across the 
time of the whole day.
This would solve the problems of the issuance of anonymous credentials 
mentioned above, if it only scaled.


\subsection{Detecting bias}

%> RevA: How is a non-protester, either a voter who intends to figure out
%> their position or a concerned someone in a distant land), who is watching
%> the news supposed to tell which of the pro-government or anti-government
%> protester-counts to believe. Are we to assume that obviously any
%> pro-government activity is absolutely suspicious and all anti activity is
%> legitimate? The solution should make it impossible to nay-say the true
%> counts even if they go against our very beliefs (e.g. as for detecting bias
%> in science).

Despite potentially better audit possibilities, elections are imperfect and so are pro-government protests. Nevertheless, this has not been seen as a valid reason to not have elections or protests anymore at all. Also, if one does not trust the government with identity provision, it stands to reason that one would not trust them to count correctly either. People certainly are skeptical about massive pro-government protests in countries such as North Korea, not only for counting wrong but also because they can coerce people to physically attend a protest.

Due to the flaws inherited in currently deployed identity systems, we 
pointed out in which cases they can be used and in which they cannot: 
when there is an interest for the identity authority to perform a 
Sybil attack, the results cannot be trusted.

As we pointed out, this is similar to detecting bias in science. If the 
authors personally benefit from the results, then they are incentivized 
to bias them. The verifiability properties of our scheme serve the same 
purpose as the methodology section of a scientific paper: anyone is able 
to repeat, reproduce and verify the results --- even if they are in 
distant lands. It is detectable who the CA is for all these people and 
if that CA has incentives to perform a Sybil attack, then the results 
simply cannot be trusted.

The problem lies with the identity system: as long as the identity 
system used has these problems, these are the limits we face in our 
scheme too. If the identity system is not subject to these problems, 
then neither is our scheme.


\subsection{The need for distance bounding}

> RevA: DB does not address the issue raised by the reality of mobile phone
> farms for hire (and the ability of the centralised authority to issue
> credentials falsely) where all the witnesses, and provers could be in a
> room far away from the protests since the location is never actually
> figured out (since location can be faked anyway). How big a protest does
> one need such that the threshold would be beyond the ability of a
> government to afford enough fake mobile phones? As for the trusted witness,
> how does that scale with the size of the protest?

If the identity authority performs a Sybil attack, they 
do not need a mobile phone farm --- they can perform computations 
anywhere on any computation system they like. (Again, as pointed out, 
nothing can prevent this type of Sybil attack and (electronic) voting 
suffers from essentially the same problem.)

We point out that against this strong adversary, the 
trusted-witness approach is the only approach that will work. We provide 
the threshold approach as an alternative that might work in some 
circumstances.

Lastly, and perhaps most importantly: DB protects against exactly this. For 
the government to succeed they would need to do the computations within 
a certain distance of the (trusted) witness --- a distance determined by 
the witness themself. Obviously the government succeeds if they also 
control the trusted witness. But the reason the witness is trusted is 
that they are likely not controlled by the government. And there can be 
several trusted witnesses in parallel --- with different interests to 
prevent a bias situation.

However, in the end, the one who controls the identity system controls 
Sybil. And, unless we get representatives from several nations present 
to sign birth certificates at hospitals, we will probably be stuck in 
this situation. 

From the Meta-Review:
 
> The main concerns were about the deployment scenario which relies on a
> central entity, the complexity of the protocol, and the use of
> yet-to-be-designed primitives like distance bounding protocols while
> leaving existing primitives like functional encryption out of the model.

Distance-bounding protocols were introduced by Chaum in 1993. They are 
very much existing. What is not existing (i.e. not formally proved 
correct yet) is a distance-bounding protocol (with all the desired DB 
properties) which is a proof-of-knowledge for discrete logarithms.

Any DB protocol that is provably secure against terrorist fraud, is 
provably secure against anything along the lines of functional 
encryption. In that sense, functional encryption is part of the model. Our rebuttal response regarding functional encryption does not seem to have been taken into consideration, at least not in the reviews, we have of course no insight into any discussions.  

