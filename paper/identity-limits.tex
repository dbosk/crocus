\section{Limits of current identity systems}%
\label{identity-limits}

\Textcite{SybilAttack} points out that any identity system must be logically 
centralized to prevent Sybil attacks.
It follows that any system where users themselves can create identities cannot 
provide strong guarantees against Sybil attacks.

The most widely deployed identity system is that of nation states: in essence, 
an identity is created at birth in the form of the equivalent of a birth 
certificate\footnote{%
  Many countries issue birth certificates, but there are other, equivalent, 
  ways.
  \Eg Sweden does not issue birth certificates, it simply creates and entry in 
  the population registry.
} This birth certificate later serves as the root for many authentication 
mechanisms, \eg national identity cards and passports.
Although this seems centralized at first glance, as we will see below, it is 
indeed comparable in decentralization to distributed ledgers due to audit 
trails and separation of duties.
As we will see, the problem with this system is that its federated structure 
and reliance on audit trails fails when it comes to issuing a 
privacy-preserving attribute-based credential for identity (or physical 
person).

Some alternative technologies have been suggested, \eg 
proofs-of-personhood~\cite{proof-of-personhood}.
The idea is to ensure a one-to-one mapping between physical persons and some 
credential.
In this particular approach, people gather in a room, \emph{everyone ensures} 
that no one can enter or leave during the process.
Then \emph{everyone ensures} that everyone gets exactly one credential.
This system is independent of existing identities, \eg as established by birth 
certificates.
As we will see, the problem with this system is scalability.

We will now discuss the limits and possibilities of these approaches and in the 
end conclude with a construction that can solve the problems in practice.


\subsection{The \enquote{birth-certificate} system}

This system is logically centralized, it is controlled (enforced) by the 
government.
However, it is usually distributed among a country's hospitals (and other 
agencies).
It usually works without any problems: citizens --- who have an incentive to 
create more identities, \eg to accumulate various benefits --- cannot create 
new identities without babies.
Someone who has a baby does not want to give that identity to someone else, 
because then their baby will not have one --- usually the biological mechanisms 
that ensures the well-being of our offspring prevents any abuse on this side 
too.
But obviously there exists a possibility to create birth certificates without 
newborn babies, however, there are other mechanisms, \eg separation of duties 
and audit trails that prevent this.
This is best illustrated with a case study.

\subsubsection{The case of Sweden}

An identity is created when an entry is made in the population registry.
Each entry in the registry has a unique identification number.
The population registry serves various purposes, one, arguably among the most 
important, is the generation of the electoral 
roll~\cite{Valmyndigheten-ElectoralRoll}.

For a person to cause an entry into the population registry, they have two 
options:
they are in the population registry and give birth to a child, or
they were born abroad and apply to immigrate to Sweden.

In the case of a birth\footnote{%
  Technically, this is the procedure if the child is born abroad.
  If the child is born in Sweden, the hospital will report all these things 
  directly to the population registry without issuing any papers to the 
  parents.
} they need:
\begin{itemize}
  \item passports, to prove the identities of the parents;
  \item a birth certificate where the names of the parents are present;
  \item a pregnancy certificate, to prove who birthed the child;
  \item a certificate of who the father is~\cite{Skatteverket-RegisterBirth}.
\end{itemize}
So the security is reduced to the forgeability of these items.

Obviously the one who creates entries in the population registry can do that 
even without such documents, so there must be some separation of duty, audit 
trail or transparency to prevent the agency from performing a Sybil attack.
In the case of Sweden, the population registry is public.
This means that in addition to any audit trail in place at the agency, we have 
these properties:
\begin{itemize}
  \item The data must fulfil correctness: a person must have parents, who in 
    turn must have parents, \etc.
  \item Anyone can detect extraordinary additions, \eg exceptionally many 
    births, additions of family trees, \etc.
    Individual women and men would also notice if extra children were added to 
    them (they would receive additional child benefits).
  \item Addition of persons of foreign heritage must have a corresponding audit 
    trail at the Migration Board (separate agency).
\end{itemize}
A foreign citizen can prove their identity to the Migration Board in the 
following ways:
\begin{itemize}
  \item showing a national passport in the original or
  \item showing an identity document in the original or
  \item a close relative attesting to the 
    identity~\cite{Migrationsverket-ProvenIdentity}.
\end{itemize}
Thus the security is reduced to subverting one of these methods.

\subsubsection{The case of France}

The French electoral roll is populated by citizens registering at the town 
hall.
They must then be able to authenticate their identity using an identity 
credential.

Whenever a child is born, the hospital reports this to the local town hall.
They register the birth and issues a birth certificate.
While they do not actively maintain the current whereabouts of individuals and 
do not give each an identity number, their identity consists of the tuple of 
name, date and place of birth.
Thus an identity can be verified by asking the specified town hall to verify 
the birth in their records.

This makes the audit trail more complex, but it is still possible to follow the 
audit trail.
However, it is difficult to detect modifications to these records, since they 
are not public.

%\subsubsection{Perfectly decentralized and private birth-certificates approach}
%
%In perfectly decentralized system
%There are no audit records, the one who can create authentic birth certificates 
%can perform Sybil \dots


\subsection{Issued credentials}

In Sweden, the identity credential system is federated.
There are several entities that are authorized to issue identity cards; \eg the 
Swedish Tax Agency, the Swedish Police, the banks.
Thus, technically, to detect a Sybil attack by them, \ie fake identities, one 
must check the ID against the population registry and follow the audit trail.
With anonymous credentials, this becomes a problem.

At some point, Alice must get her first ID card.
By definition, she does not have a previous ID card that can authenticate her 
identity.
Thus, for Alice to get her first ID card, she must bring a \emph{close relative 
  who has an ID card} to vouch for her identity.
This reduces the problem of audit trail to that of the relative.
This works because close relatives have strong incentives to be honest, few 
would allow an identity theft of one of their close relatives.

If two entities issue (non-anonymous, electronic) identity credentials to 
Alice, there is no problem: if she uses the first to authenticate to Bob, and 
later comes back and uses the second, Bob will know that she is Alice both 
times.
In both cases, if Bob doubts the credentials, he can follow the audit trail.

There is one problem in the current system: it is difficult for Alice to detect 
if the \ac{CA} issues an identity credential (electronic or not) in her name 
and let someone else use it.

With privacy-preserving attribute-based credentials, we face several problems:
The first is that there can be no audit trail from the authentication, due to 
the anonymity.
For most attributes, \eg age, everything works, the problem is that individual 
issuers have the possibility to perform Sybil attacks.
However, one-to-one mappings of credentials to physical persons are not 
possible.
Alice can get one credential from \ac{CA} \(A\) and another from \(B\), if Bob 
must accept credentials issued by both \(A\) and \(B\), then Alice suddenly has 
two valid credentials instead of only one.


\subsection{Proofs-of-personhood}

Even physically decentralized attempts, such as 
proof-of-personhood~\cite{proof-of-personhood}, are logically centralized.
For a nation-wide proof-of-personhood, everyone in the nation must be at the 
same place at the same time.
This is the issue of scalability.
This is obviously not possible and to make it feasible we must split across 
multiple locations\footnote{%
  We cannot split across time.
  The base assumption is that we do not have any means of authenticating 
  people, thus we must rely on the physical impossibility of being in two 
  places at the same time.
  Obviously it is not impossible to be in the same place at two times.
} and rely on trust: if Alice is in one location, she must trust that at least 
one in every other location is also honest.

The issue with this system is scalability.
Every time that \emph{anyone} needs a new credential, we must involve 
\emph{everyone}.
Getting everyone to be at some fixed locations at the same time is even 
difficult to achieve during Election Day, it must usually be spread across 
several locations and across time of the whole day.
This would solve the problems of the issuance of anonymous credentials 
mentioned above, if it only scaled.


\subsection{Conclusions}


While the decentralized properties of proofs-of-personhood are appealing, it is 
hard to achieve such a scheme.
The birth-certificate system is currently implemented in practice, however, 
there is a need to bridge an audit trail from electronic credentials to the 
rest of the birth-certificate system.
Furthermore, due to the impossibility of an audit trail for anonymous 
credentials, we must introduce some form of separation of duties.

To bridge electronic credentials to the audit trail of non-electronic 
identities, an issuer should include a reference in the electronic credential 
to the credential it is based upon.

The separation of duties for issuing anonymous credentials can be achieved by 
collective signing~\cite{collective-signing}, \ie that all issuers (and 
possibly others) must act together to issue every credential.
This introduces dual control, which makes it harder for any one issuer to 
perform a Sybil attack.
This can be connected to the audit trail by requiring that the request for 
issuance is signed by a valid electronic identity, thus connecting to the audit 
trail of that.
Note, however, that this does not connect Alice's anonymous credentials back to 
her through any audit trail, the use of anonymous credentials is still 
inauditable.
The connection to the audit trail simply makes it harder for any individual 
issuer to cheat the other issuers into jointly issuing Sybil credentials.

We note that the issuing of non-anonymous credentials can be done either using 
an X.509-style \ac{PKI} (as is currently practiced) or using collective 
signing, as outlined for anonymous credentials above.

We do not see these requirements as impossible, \eg collective signing can be 
incorporated into the eIDAS regulation already in place to govern electronic 
identities in the EU.
