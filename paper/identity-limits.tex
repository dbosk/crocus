\section{Limits of current identity systems}

\Textcite{SybilAttack} points out that any identity system must be logically 
centralized to prevent Sybil attacks.
It follows that any system where users themselves can create identities cannot 
provide strong guarantees against Sybil attacks.

The most widely deployed identity system is that of nation states: in essence, 
an identity is created at birth in the form of the equivalent of a birth 
certificate\footnote{%
  Many countries issue birth certificates, but there are other, equivalent, 
  ways.
  \Eg Sweden does not issue birth certificates, it simply creates and entry in 
  the population registry.
} This birth certificate later serves as the root for many authentication 
mechanisms, \eg national identity cards and passports.

Some more \enquote{decentralized}\footnote{%
  Provably, it must be logically centralized anyway.
} methods have been suggested, \eg 
proof-of-personhood~\cite{proof-of-personhood}.
In this particular approach, people gather in a room, \emph{everyone ensures} 
that no one can enter or leave during the process.
Then \emph{everyone ensures} that everyone gets exactly one identity.

We will now discuss the limits and possibilities of these approaches.

\subsection{The \enquote{birth-certificate} system}

This system is logically centralized, it is controlled (enforced) by the 
government.
However, it is usually distributed among a country's hospitals.
It usually works without any problems: citizens --- who have an incentive to 
create more identities, \eg to accumulate various benefits --- cannot create 
new identities without babies.
Someone who has a baby does not want to give that identity to someone else, 
because then their baby will not have one --- usually the biological mechanisms 
that ensures the well-being of our offspring prevents any abuse on this side 
too.
But obviously there exists a possibility to create birth certificates without 
newborn babies, it is just that those who can either do not have any incentive 
to do so or there are other mechanisms, \eg separation of duties and 
accountability.

There is one particular situation where the incentive arises:
to influence democratic processes\footnote{%
  There are a few more, less relevant, \eg to have multiple identities for 
  foreign operatives, as seen on film.
  But those are not significant to our discussion.
}.
If the government can create identities and control them, they can use them to 
vote for themselves in upcoming elections.
In a well-designed system, however, this should be prevented.
This varies from country to country (\ie system to system).

\subsubsection{The case of Sweden}

An identity is created when an entry is made in the population registry.
Each entry in the registry has a unique identification number.
The electoral roll is created from the population 
registry~\cite{Valmyndigheten-ElectoralRoll}.

For a person to cause an entry into the population registry, they have two 
options:
they are in the population registry and give birth to a child, or
they were born abroad and apply to immigrate to Sweden.

In the case of a birth\footnote{%
  Technically, this is the procedure if the child is born abroad.
  If the child is born in Sweden, the hospital will report all these things 
  directly to the population registry without issuing any papers to the 
  parents.
} they need:
\begin{itemize}
  \item passports, to prove the identities of the parents;
  \item a birth certificate where the names of the parents are present;
  \item a pregnancy certificate, to prove who birthed the child;
  \item a certificate of who the father is~\cite{Skatteverket-RegisterBirth}.
\end{itemize}
So the security is reduced to the forgeability of these items.

Obviously the one who creates entries in the population registry can do that 
even without such documents, so there must be some separation of duty, 
accountability or transparency to prevent the agency from performing a Sybil 
attack.
In the case of Sweden, the population registry is public.
This means that in addition to any audit trail in place at the agency, we have 
these properties:
\begin{itemize}
  \item The data must fulfil basic integrity: a person must have parents, who 
    in turn must have parents, \etc.
  \item Anyone can detect extraordinary additions, \eg exceptionally many 
    births, additions of family trees, \etc.
    Individual women and men would also notice if extra children were added to 
    them.
  \item Addition of persons of foreign heritage must have corresponding 
    documentation (audit trail) at the Migration Board.
\end{itemize}
A foreign citizen can prove their identity to the Migration Board in the 
following ways:
\begin{itemize}
  \item showing a national passport in the original or
  \item showing an identity document in the original or
  \item a close relative attesting to the 
    identity~\cite{Migrationsverket-ProvenIdentity}.
\end{itemize}
Thus the security is reduced to subverting one of these methods.

\subsubsection{The case of France}

The French electoral roll is populated by citizens registering at the town 
hall.
They must then be able to authenticate their identity using an identity 
credential.

Whenever a child is born, the hospital reports this to the local town hall.
They register the birth and issues a birth certificate.
While they do not actively maintain the current whereabouts of individuals and 
do not give each an identity number, their identity consists of the tuple of 
name, date and place of birth.
Thus an identity can be verified by asking the specified town hall to verify 
the birth in their records.

This makes the audit trail more complex.
It is also difficult to detect modifications to these records, since they are 
not public.

\subsubsection{Perfectly decentralized and private birth-certificates approach}

There are no audit records, the one who can create authentic birth certificates 
can perform Sybil \dots


\subsection{Issued credentials}

In Sweden, there are several entities that are authorized to issue identity 
cards; \eg the Swedish Tax Agency, the Swedish Police, the banks.
Thus, technically, to detect a Sybil attack by them, \ie fake identities, one 
must check the ID against the population registry.
With anonymous credentials, this becomes a problem.

If two entities issue (non-anonymous) identity credentials to Alice, there is 
no problem: if she uses the first to authenticate to Bob, and later comes back 
and uses the second, Bob will know that she is Alice both times.
In both cases, if Bob doubts the credentials, he can follow the audit trail.
With anonymous credentials that prevent multiple use, however, this is no 
longer the case.
Thus issuers of such credentials must synchronize, so that if one has issued 
such credentials to Alice, the others must not issue another.

This problem is amplified by the fact that there can be no audit trail from the 
authentication, due to the anonymity.
Thus, individual issuers have the possibility to perform Sybil attacks.
However, this problem can be solved.
Since all individual issuers of credentials must synchronize to not issue more 
than one anonymous credential to Alice, they might just as well act together 
(\ac{MPC}) in the issuing process --- thus introducing dual control.

\subsection{Proofs-of-personhood}

Even physically decentralized attempts, such as 
proof-of-personhood~\cite{proof-of-personhood}, are logically centralized.
For a nation-wide proof-of-personhood, everyone in the nation must be at the 
same place at the same time.
This is obviously not possible and to make it feasible we must split across 
multiple locations\footnote{%
  We cannot split across time.
  The base assumption is that we do not have any means of authenticating 
  people, thus we must rely on the physical impossibility of being in two 
  places at the same time.
  Obviously it is not impossible to be in the same place at two times.
} and rely on trust: if Alice is in one location, she must trust that at least 
one in every other location is also honest.

The issue with this system is scalability.
Every time that \emph{anyone} needs a new credential, we must involve 
\emph{everyone}.
Getting everyone to be at some fixed locations at the same time is even 
difficult to achieve during Election Day, it must usually be spread across the 
time of the whole day.
This would solve the problems of the issuance of anonymous credentials 
mentioned above (it provides the same properties as \ac{MPC} did above), if it 
only scaled.


\subsection{Conclusions}

While the decentralized properties of proofs-of-personhood are appealing, it is 
hard to achieve such a scheme.
The goal is to have one identity (pseudonym) per physical person.
This goal is joint with the \enquote{birth-certificate system} which actually 
achieves the goal quite nicely in practice.
However, since the birth-certificate system relies on audit trails to function, 
it cannot deal with issuing anonymous credentials.
Thus the ideas from proof-of-personhood, where everyone must act jointly, can 
solve this problem.
To issue anonymous credentials that certify one physical person behind a 
pseudonym, all issuers must act jointly, \eg through \ac{MPC}.
We do not see this as something impossible, \eg it can be incorporated into the 
eIDAS regulation already in place in the EU.
