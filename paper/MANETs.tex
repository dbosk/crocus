We also need the participants to communicate during the protest.
Since the cellular network could be shut down to keep protesters from accessing 
the Internet, making phone calls or texting, we require a different means of 
communications between protesters. This could be accomplished by Bluetooth or 
WiFi communications, as demonstrated by Briar~\cite{Briar} and 
FireChat~\cite{FireChat}, two examples of applications for communication during 
protests via wireless mesh networking. The crowd-counting scenario likely has 
higher requirements on capacity and withstanding interference as the 
participants continuously run the protocol for witnessing each others presence; 
messages are presumably less frequent. 5G is intended to cope with billions of 
devices (IoT), and thus could help cope with the device density in crowds.
An alternative, although originally designed to work within 5G cellular networks, is device-to-device communication (D2D)~\cite{D2D}. 
Specifically, the out-of-band and autonomous version D2D would fit our
scenario thanks to using unlicensed spectrum and working without
cellular coverage. Due to the requirements of lawful intercept and the
drive for operators to identify the network users, there still is an
authorization step to communicate in this mode.
%~\footnote{ https://portal.3gpp.org/desktopmodules/Specifications/SpecificationDetails.aspx?specificationId=840}