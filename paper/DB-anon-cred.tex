\subsection{Distance bounding anonymous credentials}%
\label{DB-anon-cred}

\ProvideDocumentCommand{\DBPK}{o o}{\ensuremath{%
    \operatorname{DBPK}%
    \IfValueT{#1}{%
      \IfValueTF{#2}{%
        \mleft\{\mleft(#1\mright) : #2\mright\}%
      }{%
        \mleft\{#1\mright\}%
      }%
    }%
  }}

We pointed out in \cref{DistanceBounding} that all security properties 
concerning \ac{DB} assume an \emph{honest} verifier.
This assumption does not hold in our setting, where every verifier at some point 
also might be a (malicious) prover.
We will introduce a new property for \ac{DB} protocols, the \ac{DBIV}.
Then we will propose a protocol that is \ac{DBIV}, \ac{DBMF}, \ac{DBDF}, 
\ac{DBDH} resistant but \emph{not} \ac{DBTF} resistant.
The protocol is based on the Schnorr identification scheme and thus allows us to 
do \ac{DB} \ac{ZKPK}, which allows us to form distance-bounding anonymous 
credentials.

\subsubsection{Reattempting \iacl{DB} Schnorr protocol}

\NewCryptoScheme{\DBS}{DBS}
\NewAlgorithm{\DBSprove}{\DBS.\!Prove}
\NewAlgorithm{\DBSverify}{\DBS.\!Verify}

In the original \ac{DB} paper by \citet{DistanceBounding}, they proposed how to 
make the Schnorr identification scheme~\cite{Schnorr} distance bounding.
That \ac{DB} Schnorr protocol was shown to be prone to distance 
hijacking~\cite{DistanceHijacking}.
We now propose another way to turn the Schnorr protocol into \iac{DB} protocol 
which is secure against \ac{DBIV}, \ac{DBMF}, \ac{DBDF} and \ac{DBDH}.
It will not protect against terrorist fraud, in fact, we suspect that \ac{DBIV} 
and \ac{DBTF} resistance are mutually exclusive.

We present the protocol in \cref{SchnorrFigure}.
The difference from the original Schnorr protocol is that the prover commits to 
two random values and the verifier sends two challenges.
Then the verifier decides which commitment and challenge to use in the \ac{DB} 
phase.

\begin{figure*}
  \centering
  \begin{tabular}{lcl}
    \(\DBSprove[g, q, \alpha, A = g^\alpha]\):
    & &
    \(\DBSverify[g, q, A]\):
    \\
    \midrule

    \multicolumn{3}{c}{\textbf{Setup}} \\

    \(\rho_0\rgets \Z_q, \rho_1\rgets \Z_q\)
    &
    &
    \\

    \(R_0\gets g^{\rho_0}, R_1\gets g^{\rho_1}\)
    & \(\ProtoSendRight{R_0, R_1}\)
    &
    \\

    % null
    & \(\ProtoSendLeft{c_0, c_1}\)
    & \(c_0\rgets \bin^k, c_1\rgets \bin^k\)
    \\

    \(s_0\gets \rho_0 + c_0\alpha\)
    &
    &
    \\

    \(s_1\gets \rho_1 + c_1\alpha\)
    &
    &
    \\

    % null
    &
    & \(b\rgets \bin\)
    \\

    % null
    &
    & Prepare \(B\in \bin^{\ceil{\log_2 q}+l}\),
    \\

    % null
    &
    & with \(\ceil{\log_2 q}\) bits set to \(b\)
    \\

    \midrule
    \multicolumn{3}{c}{\textbf{Distance-bounding}: \(\forall i: 0\leq i < 
        \ceil{\log_2 q} + l\)} \\

%    % null
%    &
%    & Random delay
%    \\

    % null
    & \(\ProtoSendLeft{b_i}\)
    & \(b_i\gets B[i]\)
    \\

    \(r_i\gets s_{b_i}[i]\)
    & \(\ProtoSendRight{r_i}\)
    & Record \(\Delta t_i\)
    \\

    \midrule
    \multicolumn{3}{c}{\textbf{Verification}}
    \\

    % null
    &
    & \(r = r_1|\dotsb|r_{\ceil{\log_2 q}}\)
    \\

    % null
    &
    & \(R_b \stackrel{?}{=} g^r A^{c_b}\)
    \\
    
  \end{tabular}
  \caption{%
    One-round protocol instance of the \(\DBSprove\leftrightarrow \DBSverify\) 
    \ac{DB} Schnorr protocol for \(\PK[\alpha][A = g^\alpha]\).
    The protocol should be repeated in full to achieve the desired knowledge 
    error and distance-bounding error.
  }%
  \label{SchnorrFigure}
\end{figure*}

The intuition behind the protocol security is as follows.
To achieve malicious-verifier zero-knowledge we must choose \(k\) 
logarithmically in the security parameter \(\lambda\) and repeat the protocol 
until the knowledge error is small enough.
This will actually also decrease the success probability of \iac{DBMF} adversary 
in the \ac{DB} step.
The \ac{DBMF} adversary does not know which challenge the verifier will use, 
thus there is a \(1/2\) probability that the \ac{DBMF} adversary will guess it 
correctly.
The prover will only provide one of the responses: even if the adversary re-runs 
the protocol with the same challenges, the randomly chosen \(\rho_0, \rho_1\) 
will have changed --- which thus yields another \(R_b\) in the end of the 
protocol.
By repeating the protocol we decrease this probability.

The \ac{DB} phase protects against distance fraud.
\Iac{DBDF} prover must wait for the challenge bit \(b_i\) before responding with 
\(r_i\).

The \ac{DB} phase also ensures that \iac{DBMF} adversary will fail the \ac{DB} 
phase for at least one round.
The verifier will send \(\ceil{\log_2 q} + l < 2\ceil{\log_2 q}\) challenges.
Thus the prover reveals all the bits of either \(s_0\) or \(s_1\) but only \(l\) 
bits of the other.
Thus, the adversary can buffer \(l\) bits of both \(s_0, s_1\) by requesting 
them from the prover.
But at this point the adversary must wait for \(b_{l+1}\) from the verifier and 
then relay that challenge to the prover to receive the correct \(r_i\) --- this 
relay will be detected.

The protocol is also secure against distance hijacking.
This follows from the fact that it is the authenticating bit string that is used 
during the \ac{DB} phase.
