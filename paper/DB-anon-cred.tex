\section{Distance bounding anonymous credentials}%
\label{DB-anon-cred}

\ProvideDocumentCommand{\DBPK}{m m}{\ensuremath{%
    \operatorname{DBPK}\mleft\{\mleft(#1\mright) : #2\mright\}%
  }}

We pointed out in \cref{DistanceBounding} that all security properties 
concerning \ac{DB} assume an \emph{honest} verifier.
This assumption does not hold in our setting, where every verifier at some point 
also might be a (malicious) prover.
We will introduce a new property for \ac{DB} protocols, the \ac{DBIV}.
Then we will propose a protocol that is \ac{DBIV}, \ac{DBMF}, \ac{DBDF}, 
\ac{DBDH} resistant but \emph{not} \ac{DBTF} resistant.
The protocol is based on the Schnorr identification scheme and thus allows us to 
do \ac{DB} \ac{ZKPK}, which allows us to form distance-bounding anonymous 
credentials.

\subsection{\Acl{DBIV} distance bounding}

\dots

\subsection{Reattempting \iacl{DB} Schnorr protocol}

In the original \ac{DB} paper by \citet{DistanceBounding}, they proposed how to 
make the Schnorr identification scheme~\cite{Schnorr} distance bounding.
That \ac{DB} Schnorr protocol was shown to be prone to distance 
hijacking~\cite{DistanceHijacking}.
We now propose another way to turn the Schnorr protocol into \iac{DB} protocol 
which is secure against \ac{DBIV}, \ac{DBMF}, \ac{DBDF} and \ac{DBDH}.
It will not protect against terrorist fraud, in fact, we suspect that \ac{DBIV} 
and \ac{DBTF} resistance are mutually exclusive.
