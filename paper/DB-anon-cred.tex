\section{Distance-bounding anonymous credentials}%
\label{DB-anon-cred}

One protocol exists in the literature that provides \acp{PPK} with most of the 
desired properties described previously, ProProx~\cite{ProProx}\footnote{Note 
  that~\cite{ProProx} uses the abbreviation PoPoK, we prefer \acs{PPK} for 
  shorter notation.}.
ProProx is secure against a malicious verifier, but it provides \iac{PPK} 
protocol for quadratic residues (\ie protocols of the form \(\PPK[\alpha][a = 
  \alpha^2]\)), while in our context, we need \iac{PPK} protocol for discrete 
logarithms (\ie \(\PPK[\alpha][a = g^\alpha]\)) due to our choice of 
anonymous credentials whose desirable properties we list in
\cref{ZK-anon-cred}.%, which in turn is due to their \ac{PRF} for deriving identifiers from a long-term ID with unlinkability to the original ID but with proof of validity of this derivation. 
To realize this, we will now introduce such a protocol in order to enable us to have \ac{DB} anonymous credentials.

\subsection{Reattempting a distance-bounding Schnorr protocol}%
\label{DBSchnorr}

In the original \ac{DB} paper by \citet{DistanceBounding}, one of the distance-bounding protocol is based on the Schnorr identification scheme~\cite{Schnorr}.
However, this \ac{DB} Schnorr protocol was shown to be prone to distance hijacking~\cite{DistanceHijacking, TamarinDB}, in addition to not being secure against terrorist fraud.
We now propose another way to turn the Schnorr protocol into a
public-key \ac{DB} protocol which is secure against \ac{DBMF},
\ac{DBDF}, \ac{DBDH} and \ac{DBTF}.
It is important in our setting that the protocol is a public-key protocol, this 
property implies protection against an impersonating verifiers.

We present the protocol in \cref{SchnorrFigure}.
The (cyclic) group with generator \(g\) and order \(q\) are system parameters.
The private key \(\alpha\) with public key \(A = g^\alpha\) are generated once 
by the prover in the setup phase.

\begin{figure*}
  \centering
  \small
  \begin{tabular}{p{0.8\columnwidth}cp{0.8\columnwidth}}
    \(\DBSprove[g, q, \alpha, A = g^\alpha]\):
    & &
    \(\DBSverify[g, q, A]\):
    \\
    \midrule

    \multicolumn{3}{c}{\textbf{Setup}} \\

    \(\rho\rgets \ZZ_q, R\gets g^{\rho}\)
    & \(\ProtoSendRight{R}\)
    &
    \\

    % null
    & \(\ProtoSendLeft{c_0, c_1}\)
    & \(c_0\rgets \bin^k, c_1\rgets \bin^k\)
    \\

    \(\forall i\in\bin\colon s_i\gets \rho - c_i\alpha \mod q\)
    &
    & \(b\rgets \bin\)
    \\

    \(\forall i\in\bin\colon s_i^{(0)} \rgets \ZZ_q, s_i^{(1)} \gets s_i - 
      s_i^{(0)} \mod q\)
    &
    & Prepare \(B\in \bin^{2|q|+l}\), with \(2|q|\) bits set to \(b\).
    \\

    \midrule
    \multicolumn{3}{c}{\textbf{Distance-bounding}: \(\forall i: 0\leq i < 2|q| 
        + l, j\gets 0\)} \\

%    % null
%    &
%    & Random delay
%    \\

    % null
    & \(\ProtoSendLeft{b_i}\)
    & \(b_i\gets B[i]\)
    \\

    \(r_i\gets (s_{b_i}^{(0)}\concat s_{b_i}^{(1)})[jb_i + (i-j)(1-b_i)]\)
    & \(\ProtoSendRight{r_i}\)
    & Record \(\Delta t_i\)
    \\

    \(j \gets j + b_i\)
    &
    &
    \\

    \midrule
    \multicolumn{3}{c}{\textbf{Verification}}
    \\

    % null
    &
    & Construct \(s_b^{(0)}\concat s_b^{(1)}\) as the concatenation of 
    \(r_i\)'s for which \(b_i = b\).
    Compute \(s_b\gets s_b^{(0)} + s_b^{(1)}\).
      Accept if \(R = g^{s_b} A^{c_b}\).
    \\
    
  \end{tabular}
  \caption{%
    One-round protocol instance of the \(\DBSprove\leftrightarrow \DBSverify\) \ac{DB} Schnorr protocol for \(\PK[\alpha][A = g^\alpha]\).
    The protocol should be repeated in full to achieve the desired knowledge and distance-bounding errors.
  }%
  \label{SchnorrFigure}
\end{figure*}

During one round, the prover commits to a random nonce: \ie chooses 
\(\rho\rgets \ZZ_q\) uniformly randomly, computes \(R\gets g^\rho\) and sends 
\(R\) to the verifier.
The verifier generates two challenges \(c_0\rgets \bin^k, c_1\rgets \bin^k\) 
and sends them to the prover.
The prover computes \(s_0\gets \rho - c_0\alpha, s_1\gets \rho - c_1\alpha\).
This is the main difference to the original Schnorr protocol: the verifier 
selects two challenges and the prover \emph{computes} two responses --- but 
still only \emph{one nonce}, \(\rho\).
(This is also different from Brands-Chaum, where the prover and verifier 
jointly construct \emph{one} challenge with \emph{one} response.)

Now the prover randomly chooses
\(s_0^{(0)}, s_1^{(0)}\in \ZZ_q\)
and computes
\(s_0^{(1)} \gets s_0 - s_0^{(0)}, s_1^{(1)} \gets s_1 - s_1^{(0)}\).
\Ie \(s_b, b\in\bin\) can be reconstructed from the two shares \(s_b^{(0)}, 
  s_b^{(1)}\).

We let \(|q| = \ceil{\log_2 q}\) denote the length of \(q\) in bits.
The verifier will request all \(2|q|\) response-bits from one challenge (say 
\(s_b^{(0)}\concat s_b^{(1)}, b\in \bin\)) and only \(0 < l\leq |q|\) from the 
other (\(s_{1-b}^{(0)}\concat s_{1-b}^{(1)}\)).
Then the verifier can compute \(s_b = s_b^{(0)} + s_b^{(1)}\) and finally 
authenticate the prover by checking if \(R = g^{s_b}A^{c_b}\).

The prover must know the responses for both challenges to successfully pass the 
\ac{DB} phase.
The reason for the two challenges but only one random nonce (change from 
Schnorr) is that knowing the responses to both means learning \(\alpha\) (for
\ac{DBTF}).
Bundling the authentication into the distance bounding phase (difference from 
Brands-Chaum) was done for \ac{DBDH},
and both adaptations together aim at \ac{DBMF} resistance. %reducing success of MF according to number of repetitions.
The reason we use secret sharing and not \(s_b\) directly, is to bound the 
\(l\) bits the verifier must learn of \(s_{1-b}\):
Without secret sharing, the verifier learns \(l\) bits of \(s_{1-b}\), but with 
secret sharing the verifier learns \(l\) bits of \emph{one share} (provided 
\(l\) is small enough).
Thus we have information theoretic security although we reveal these \(l\) 
bits.

\subsection{Security analysis}

The intuition behind the protocol security is as follows.
To achieve malicious-verifier zero-knowledge, choose \(k\) logarithmic in the 
security parameter \(\lambda\) and repeat the protocol \(n\) times, such that 
the knowledge error, \(2^{-kn}\), becomes small enough.
Repeating the protocol is also needed to decrease the success probability of 
some of adversaries presented below.

The \ac{DB} phase protects against distance fraud.
Once the prover has received \(l+1\) challenges set to \(b\), it knows that the 
remaining challenges must also be \(b\).
\Iac{DBDF} prover must thus wait for the challenge bit \(b_i\) before 
responding with \(r_i\) for \(l+1\) challenges.
Thus, the probability of successfully guessing the order of the challenge bits 
is \(2^{-(l+1)}\) per round or, in total, \(2^{-(l+1)n}\).

The \ac{DB} phase also ensures that \iac{DBMF} adversary will fail the \ac{DB} 
phase for at least one round.
The verifier will send \(|q| + l < 2|q|\) challenges.
By replying, the prover reveals all the bits of either \(s_0\) or \(s_1\), but 
none of the other --- only \(l\leq |q|\) bits of \emph{one of two} information 
theoretically independent \(|q|\)-bit shares.
The adversary now has two options.
First, the adversary can buffer \(l\) bits of both \(s_0, s_1\) by requesting 
them from the prover.
However, at this point the adversary must wait for the challenge bit 
\(b_{l+1}\) from the verifier and then relay that challenge to the prover to 
receive the correct \(r_i\) --- this relay will be detected.
The (better) alternative is to guess \(b\) and extract \(s_b\) for this 
challenge from the prover.
Guessing \(b\) yields \(1/2\) probability \emph{per round} that the \ac{DBMF} 
adversary will guess it correctly.
Since we must execute the protocol \(n\) times, this is reduced to \(2^{-n}\).
The adversary can also guess the remaining \(2|q|-l\) bits, which we will 
discuss more below.
The probability of guessing these bits successfully is \(2^{-(2|q|-l)}\).
However, \(l < |q| \implies 2|q|-l > |q|\) and this might thus be more 
expensive than guessing \(s_{1-b}\) directly, which is \(2^{-|q|}\).
Thus the success probability of the \ac{DBMF} adversary is \emph{at most} 
\(2^{-n}+2^{-|q|n}\).

The protocol is \ac{DBTF}-resistant, indeed, if the malicious prover gives both 
responses to the accomplice, the accomplice can compute his secret key.
The probability of success of \ac{DBTF} is thus reduced to guessing \(b\), \ie 
\(1/2\) per round or, in total, \(2^{-n}\).

The protocol is also secure against distance hijacking due to the fact
that it is the authenticating bitstring that is used during the \ac{DB}
phase, not the challenge bitstring as in the protocol of Brands-Chaum.

Finally, it follows from the original Schnorr protocol that this is still 
\iac{PK} and \ac{ZK}.
\daniel{I should prove this for the composed protocol, I should check the 
  details with Douglas.}

\subsection{Formal Verification}

We wanted to formally verify the protocol security properties. We used the Tamarin prover~\cite{meier2013tamarin}. Tamarin does not include timestamp and location data in its model, but thanks to the work by Mauw \emph{et al.}~\cite{TamarinDB}, we know that all the common properties mentioned in the previous subsection can be equivalently characterized with a causality-based definition relying only on the order of messages.

The whole Tamarin code we wrote is provided in Appendix \ref{apdx:tamarin-spec}. It contains a specification of the distance-bounding protocol described previously, a definition of the soundness and correctness properties that ensure the protocol actually produces the expected result, and a definition of the security properties we want to ensure.

Tamarin auto-prover validated the soundness and correctness properties (in negligible computation time) but does not terminate when trying to prove the security properties, that is to say, no trace violating the security properties is found, but the prover cannot conclusively validate that the properties are true. We are currently investigating with Tamarin interactive mode to hopefully validate the security properties as well.

\sonja{\subsection{Formal Verification} 
 verification with tamarin: there is no BC Schnorr in Tamarin,  [10]
assumed equivalence with Fiat-Shamir model. Not straightforward to
implement Schnorr in Tamarin due to a lack of expressiveness for
arithmetic needed to complement Diffie-Hellman. If not done now, left
for future work. }
