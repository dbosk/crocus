\subsection{Distance bounding anonymous credentials}%
\label{DB-anon-cred}

\ProvideDocumentCommand{\DBPK}{o o}{\ensuremath{%
    \operatorname{DBPK}%
    \IfValueT{#1}{%
      \IfValueTF{#2}{%
        \mleft\{\mleft(#1\mright) : #2\mright\}%
      }{%
        \mleft\{#1\mright\}%
      }%
    }%
  }}

There is a protocol which provide \ac{DBPK} with all the desired properties 
from above, ProProx~\cite{ProProx}.
(Note that~\cite{ProProx} uses the abbreviation PoPoK, for Proof of Proximity of 
Knowledge, instead of \ac{DBPK}.)
However, ProProx provides \iac{DBPK} protocol for quadratic residues (\ie 
protocols of the form \(\DBPK[\alpha][a = \alpha^2]\)) while in our context, we 
need \iac{DBPK} protocol for discrete logarithms (\ie \(\DBPK[\alpha][a = 
  g^\alpha]\)).
To realize this, we will now introduce such a protocol that will allow us to do 
\ac{DB} anonymous credentials.

\NewScheme{\DBS}{DBS}
\NewAlgorithm{\DBSprove}{\DBS.\!Prove}
\NewAlgorithm{\DBSverify}{\DBS.\!Verify}

\emph{Reattempting \iacl{DB} Schnorr protocol.}
In the original \ac{DB} paper by \citet{DistanceBounding}, they show how to 
design a distance-bounding protocol based on the Schnorr identification 
scheme~\cite{Schnorr}.
That \ac{DB} Schnorr protocol was shown to be prone to distance 
hijacking~\cite{DistanceHijacking}.
(It was not secure against terrorist fraud either.)
We now propose another way to turn the Schnorr protocol into \iac{DB} protocol 
which is secure against \ac{DBMF}, \ac{DBDF}, \ac{DBDH}, \ac{DBTF} and an 
impersonating verifier (\ie it is public-key).

We present the protocol in \cref{SchnorrFigure}.
The difference from the original Schnorr protocol is that the prover commits to two random values and the verifier sends two challenges.
Afterwards, the verifier presents to the prover which commitment and challenge to use during the \ac{DB} phase.

\begin{figure*}
  \centering
  \begin{tabular}{lcl}
    \(\DBSprove[g, q, \alpha, A = g^\alpha]\):
    & &
    \(\DBSverify[g, q, A]\):
    \\
    \midrule

    \multicolumn{3}{c}{\textbf{Setup}} \\

    \(\rho_0\rgets \ZZ_q, \rho_1\rgets \ZZ_q\)
    &
    &
    \\

    \(R_0\gets g^{\rho_0}, R_1\gets g^{\rho_1}\)
    & \(\ProtoSendRight{R_0, R_1}\)
    &
    \\

    % null
    & \(\ProtoSendLeft{c_0, c_1}\)
    & \(c_0\rgets \bin^k, c_1\rgets \bin^k\)
    \\

    \(s_0\gets \rho_0 + c_0\alpha\)
    &
    &
    \\

    \(s_1\gets \rho_1 + c_1\alpha\)
    &
    &
    \\

    % null
    &
    & \(b\rgets \bin\)
    \\

    % null
    &
    & Prepare \(B\in \bin^{\ceil{\log_2 q}+l}\),
    \\

    % null
    &
    & with \(\ceil{\log_2 q}\) bits set to \(b\)
    \\

    \midrule
    \multicolumn{3}{c}{\textbf{Distance-bounding}: \(\forall i: 0\leq i < 
        \ceil{\log_2 q} + l\)} \\

%    % null
%    &
%    & Random delay
%    \\

    % null
    & \(\ProtoSendLeft{b_i}\)
    & \(b_i\gets B[i]\)
    \\

    \(r_i\gets s_{b_i}[i]\)
    & \(\ProtoSendRight{r_i}\)
    & Record \(\Delta t_i\)
    \\

    \midrule
    \multicolumn{3}{c}{\textbf{Verification}}
    \\

    % null
    &
    & \(r = r_1|\dotsb|r_{\ceil{\log_2 q}}\)
    \\

    % null
    &
    & \(R_b \stackrel{?}{=} g^r A^{c_b}\)
    \\
    
  \end{tabular}
  \caption{%
    One-round protocol instance of the \(\DBSprove\leftrightarrow \DBSverify\) \ac{DB} Schnorr protocol for \(\PK[\alpha][A = g^\alpha]\).
    The protocol should be repeated in full to achieve the desired knowledge and distance-bounding errors.
  }%
  \label{SchnorrFigure}
\end{figure*}

The intuition behind the protocol security is as follows.
To achieve malicious-verifier zero-knowledge we must choose \(k\) logarithmically in the security parameter \(\lambda\) and repeat the protocol until the knowledge error is small enough.
This will actually also decrease the success probability of \iac{DBMF} adversary in the \ac{DB} step.
Indeed, the \ac{DBMF} adversary does not know which challenge the verifier will use, thus there is a \(1/2\) probability that the \ac{DBMF} adversary will guess it correctly.
The prover will only provide one of the responses: even if the adversary re-runs the protocol with the same challenges, the randomly chosen \(\rho_0, \rho_1\) will have changed --- which thus yields another \(R_b\) in the end of the protocol.
A sequential repetition of protocol will decrease this probability.

The \ac{DB} phase protects against distance fraud.
\Iac{DBDF} prover must wait for the challenge bit \(b_i\) before responding with \(r_i\).

The \ac{DB} phase also ensures that \iac{DBMF} adversary will fail the \ac{DB} phase for at least one round.
The verifier will send \(\ceil{\log_2 q} + l < 2\ceil{\log_2 q}\) challenges.
Thus the prover reveals all the bits of either \(s_0\) or \(s_1\) but only \(l\) bits of the other.
Thus, the adversary can buffer \(l\) bits of both \(s_0, s_1\) by requesting them from the prover.
However at this point the adversary must wait for \(b_{l+1}\) from the verifier and then relay that challenge to the prover to receive the correct \(r_i\) --- this relay will be detected.

The protocol is \ac{DBTF} resistant, if the malicious prover gives both 
responses to the adversary, the adversary can compute the secret key.
This follows from the fact that it is \iac{ZKPK}, \ie there is an extractor that 
can extract the secret.

The protocol is also secure against distance hijacking due to the fact that it 
is the authenticating bit string that is used during the \ac{DB} phase, not the 
challenge bit string as in Brands-Chaum.
