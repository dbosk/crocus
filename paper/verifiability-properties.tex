\subsection{Verifiability requirements}%
\label{verifiability-properties}

We now try to make the properties from \cref{desired-properties} more 
specific.
We define three verifiability requirements, among which eligibility can be 
further broken up into four subproperties:
\begin{requirements}[V]
  \item\label{EligibilityVerif} \emph{Eligibility}: anyone can verify that each participation proof provides temporal and spatial eligibility and that only one participation proof is counted per individual.
    \begin{requirements}

    \item\label{TemporallyRelated} \emph{Temporal eligibility}: demonstrate that the proof was created after the start of the protest and before the end of the protest.

    \item\label{SpatiallyRelated} \emph{Spatial eligibility}: demonstrate that the proof is spatially related to the physical location or journey of the protest.

    \item\label{CountOnce} \emph{Counted only once}:
      A protester can create \emph{one and only one} pseudonym (\(\pid\) in 
      \cref{DefProofShares}) per protest (\(\cid\) in \cref{DefProofShares}), 
      this pseudonym is unique except with negligible probability.
      Analogously, a witness can create \emph{one and only one} pseudonym per 
      protester (\(\wid\) in \cref{DefProofShares}), this pseudonym is unique 
      except with negligible probability.

    \item\label{DesignatedEvent} \emph{Designated event}: prove that the proof 
      is designated for the particular protest.

    \end{requirements}

  \item\label{UniversalVerif} \emph{Universal verifiability}: anyone can verify that the result obtained match the submitted participation proofs.

  \item\label{IndividualVerif} \emph{Individual verifiability}: each participant can verify that his participation proof is included in the global count.
\end{requirements}

