\subsection{Verifiability requirements}%
\label{verifiability-properties}

To adapt these properties to the context of protests, we define three 
verifiability requirements, among which eligibility can be further broken up 
into four subproperties:
\begin{requirements}[V]
  \item\label{EligibilityVerif} \emph{Eligibility}: anyone can verify that each participation proof provides temporal and spatial eligibility and that only one participation proof is counted per individual.
    \begin{requirements}
    \item\label{TemporallyRelated} \emph{Temporal eligibility}: demonstrate that the proof was created after the start of the protest and before the end of the protest.
    \item\label{SpatiallyRelated} \emph{Spatial eligibility}: demonstrate that the proof is spatially related to the physical location or journey of the protest.
    \item\label{CountOnce} \emph{Counted only once}: prove that no individual 
      can be counted more than once for a particular protest.
      \seb{this property seems to be related to the eligibility property in the 
        sense that if a participant tries to submit a second proof this 
        directly conflicts with the eligibility}
      \daniel{Yes, that's why it's a sub-property of eligibility, we're trying 
        to make eligibility more concrete.}
    \item\label{DesignatedEvent} \emph{Designated event}: prove that the proof is associated with particular designated protest.
    \end{requirements}

  \item\label{UniversalVerif} \emph{Universal verifiability}: anyone can verify that the result obtained match the submitted participation proofs.

  \item\label{IndividualVerif} \emph{Individual verifiability}: each participant can verify that his participation proof is included in the global count.
\end{requirements}

