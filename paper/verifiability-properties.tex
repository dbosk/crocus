\subsection{Verifiability requirements}%
\label{verifiability-properties}

We note that, in general terms, protesting is very similar to petitions, which in turn are similar to voting.
Indeed, all three situations correspond to cases in which many individuals express their opinion.
These opinions can be sensitive (\eg be a cause for discrimination or persecution), hence we desire to have similar properties of verification and privacy for verifying a protest as there are for voting.
(We discuss privacy in \cref{privacy-properties}.)

We draw inspiration from the three general desirable properties for 
verifiability, as summarized by 
\textcite{VerifyingPrivacyPropertiesOfVotingProtocols}:
\begin{description}
  \item[Eligibility:] anyone can verify that each cast vote is legitimate.
  \item[Universal verifiability:] anyone can verify that the result is according to the cast votes.
  \item[Individual verifiability:] each voter can verify that his vote is included in the result.
\end{description}
In our context, votes are translated into \emph{participation proofs}.
Universal and individual verifiability remain the same, in the sense that 
anyone can verify the participation count by counting the proofs.
The eligibility requirement is slightly different as for protests it must include temporal and spatial eligibility (\ie each participation proof satisfies some temporal and spatial relation to the protest).
In essence, the proof must bind the person to the time and location of the protest.

To define these properties more in the context of protests, we define three 
verifiability requirements, among which eligibility can be further broken up 
into four subproperties:
\begin{requirements}[V]
  \item\label{EligibilityVerif} \emph{Eligibility}: anyone can verify that each participation proof provides temporal and spatial eligibility and that only one participation proof is counted per individual.
    \begin{requirements}
    \item\label{TemporallyRelated} \emph{Temporal eligibility}: prove that the proof was created after the start of the protest and before the end of the protest.
    \item\label{SpatiallyRelated} \emph{Spatial eligibility}: prove that the proof is spatially related to the physical location or journey of the protest.
    \item\label{CountOnce} \emph{One-proof-per-person}: prove that no individual can be counted more than once for a particular protest.
    \item\label{DesignatedEvent} \emph{Designated event}: prove that the proof is associated to particular designated protest.
    \end{requirements}

  \item\label{UniversalVerif} \emph{Universal verifiability}: anyone can verify that the result obtained match the submitted participation proofs.

  \item\label{IndividualVerif} \emph{Individual verifiability}: each participant can verify that his participation proof is included in the global count.
\end{requirements}

