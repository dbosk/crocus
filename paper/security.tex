\section{Security and privacy analysis}%
\label{SecurityAnalysis}

We note that Alice and Bob do not trust each other, for all they know one or the 
other could be an agent of Eve the autocrat.
\begin{itemize}
  \item Eve the evil autocrat,
  \item Rusuk the foreign nation state, and finally
\end{itemize}

\subsection{Assumptions} \label{assumptions}

We assume that every participant has a device similar to a smartphone.
Throughout this text, when we refer to a participant, say Alice, we actually 
mean a principal which can perform cryptographic operations and communicate with 
other local devices on Alice's behalf.
(This means that we essentially provide a lower bound for the participation 
count, since some participants might not have such a device.)

No connection to any global network the Internet is necessary during the 
protest.
During the protest these devices are computationally limited by their 
batteries.
Before and after the protest, we assume that the devices have Internet 
connections and are not computationally limited by any battery.

We further assume that every participant has a digital certificate signed by 
some logically centralized certificate authority.
E.g.\ that we can use the cryptographic keys stored in the chip of the national 
identity card or passport.
We need this to prevent Sybil attacks~\cite{SybilAttack}.

\subsection{Forging new proofs}

It follows from the digital signatures used that proofs are difficult to forge.

\subsection{Reusing participation proofs}

We consider two ways of reusing participation proofs:
\begin{enumerate}
  \item Alice and Bob can reuse old proofs by modifying the identifiers.
  \item Eve can reuse proofs by creating a manifesto which yields the same 
    identifier as Alice's protest, then try to create new nodes in the 
    \ac{tposet} that matches the ones used in the old proofs.
\end{enumerate}
We will now estimate the difficulty of both types of attacks.

\subsubsection{Finding second preimages}

Eve (possibly with Rusuk's aid) wants to arrange a counter-protest.
One way to get a higher participation count is to use the same identifier as 
Alice's protest.
Since the identifier for the protest is the hash value of the manifest, Eve 
must find a second preimage for the hash function.

% XXX Rephrase identifier problem as a game
\begin{proposition}
  Given \(x\) and \(H(x) = y\), the probability of finding an \(x'\) such that 
  \(H(x') = y\) is \(\frac{1}{\lambda-1}\).
\end{proposition}

\begin{question}
  What if we use a Merkle tree instead of only a hash, will that change the 
  second preimage property?
\end{question}

Alice wants to arrange another protest against Eve.
Her manifesto can remain the same, thus the protest identifier remains the 
same.
She wants to reuse some old proofs.
The head value that are included in the proofs extend them too far back into 
time.
The digital signatures prevents her from changing them.
So an alternative is to create nodes in the \ac{tposet} which has the same 
hashes.

\subsection{Spoofing locations}

\subsubsection{Trust propagation for witnesses}

\dots

\subsection{Individual verifiability}

Eve and Rusuk has an incentive to lower the confidence in the result.
One way they can do this is to say that the result does not contain all 
participation proofs.
Statements such as this should be proved: provide a valid but missing proof.
