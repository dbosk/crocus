\section{Introduction}%
\label{Introduction}

\emph{Context.} Consider the following illustrative scenario in which Alice is an activist that organizes a protest against the current regime leader Eve in some location(s).
On one hand, Alice wants to estimate the number of participants to prove a certain support for her cause.
On the other hand, Eve is an autocrat who opposes Alice's cause.
Bob, Carol and many others participate in the protest for Alice's cause against Eve.
However, they might also be willing to do so in a manner that does not leave any digital traces because of the fear of retribution against them from Eve's regime. 
To realize this objective a reliable yet privacy-preserving crowd-counting mechanism is needed, which to the best of our knowledge is a problem that has not yet been entirely solved.

Historically, there are many examples of protests in which the count estimated by police and that by the organizers 
differ significantly, sometimes in the order of hundreds of thousands.
The difference is actually quite natural as both parties have different objectives.
More precisely, the organizers want to count everyone who have participated while the police want to estimate the count at the peak of participation, due to crowd control~\cite{2016DemonstrationsInSeoul}.
Among the numerous recent examples in which it is difficult to establish the actual number of participants, we can cite the demonstrations against the president in South Korea~\cite{2016DemonstrationsInSeoul}, Trump's inauguration attendance~\cite{HowWillWeKnowTrumpInauguralCrowdSize}, the 2017 Women's March in the US~\cite{2017WomensMarchesInUS}, the demonstrations against the change of constitution in Venezuela~\cite{AlJazeeraOnVenezuela2017} or for the independence of Catalonia~\cite{CataloniaDemonstrations}.

Existing methods for crowd-counting varies significantly in terms of approaches (we will review them in details in \cref{RelatedWork}).
However, most of them lack precision (\ie they have large error margins) and they can only give an estimate for a particular snapshot in time (\eg at the peak of the event). 
In particular, the existing methods cannot estimate reliably the cumulative participation count --- at least not without counting some persons multiple times, which in turn increases the error of the estimation.
In addition, they also lack verifiability as everyone has to trust the entity responsible for performing the counting.
Finally, one important observation about crowd-counting that has been adequately addressed in the design of current crowd-counting solutions is that it is actually an adversarial setting. 

Indeed, going back to our example Alice the activist has an incentive to increase the tallied number of participants, whereas Eve (and possibly other entities) have an incentive to decrease the tallied number of participants.
In this case, we are left with two options: either we trust Alice or Eve or we have to be able to verify their claims.
In this paper, our main objective is to provide a scheme preventing both Alice and Eve from cheating (\ie their claim should be verifiable).
To solve this issue, we need a verifiable participation count that can resist to Sybil attacks while still preserving the participants'
privacy to the extent possible given their physical presence at the protest. 
More precisely, our solution will not be able to prevent a physical eavesdropper from recognizing a particular individual in a protest. 
Nonetheless, we can ensure that none of the digital traces generated by a participant in our scheme can be associated to his identity and that his actions are also unlinkable.

%Seb: I suggest that we move this part to system model
%\subsection{Combining properties of electronic voting with location proofs}
\emph{Relationship between protesting and electronic voting}. In general, protesting is very similar to petitions, which in turn are similar to voting: all three situations correspond to many individuals expressing their opinion.
These opinions can be sensitive (\eg be a cause for discrimination or persecution), hence we desire to have similar properties of verification and privacy for verifying a protest as there is for voting.

Voting has (generally) three desirable requirements for verifiability~\cite{VerifyingPrivacyPropertiesOfVotingProtocols}.
\begin{description}
  \item[Eligibility:] anyone can verify that each vote cast is legitimate.
  \item[Universal verifiability:] anyone can verify that the result is according 
    to the cast votes.
  \item[Individual verifiability:] every voter can verify that their vote is 
    included in the result.
\end{description}
We translate the votes into \emph{proofs of participation}.
Universal and individual verifiability remain the same: anyone can verify the participation count by counting the proofs.
The eligibility requirement is slightly different: for protests the eligibility requirement must include temporal and spatial 
eligibility (\ie each proof of participation satisfies some temporal and spatial relation to the protest).
In essence, the proof must bind the person to the time and location of the protest.
We will define these more formally in \cref{Verifiability}.

There are also three levels of privacy~\cite{VerifyingPrivacyPropertiesOfVotingProtocols}:
\begin{description}
  \item[Vote privacy:] the voting does not reveal the individual vote.
  \item[Receipt freeness:] the voting system does not provide any data that can 
    be used as a proof of how the voter voted.
  \item[Coercion pesistance:] a voter cannot cooperate with a coercer to prove 
    the vote was cast in any particular way.
\end{description}
We note that we cannot actually improve the privacy of participating protesters during the protest: if Alice is caught during a protest or other crowd-counting mechanisms based on photographic or video evidence are used to identify Alice, there is nothing we can do. 
%Seb: suggestion maybe we should put the following part in the discussion
However, in \PRIVO, we aim at verifiable participation count that does not add any threats to the participants' privacy. 
For instance, if there are counter-protests in the \emph{same location at the same time}, Alice could even blend into another crowd and argue that she participated in a different protest than she actually did --- and the counts would still be correct afterwards.
However, even if there is only one protest and she is not caught in the act, then using our proposed mechanism should not incur any additional risk.
Thus we need at least participation-proof privacy but receipt freeness (see \cref{Privacy} for a detailed definition) is desired.
In essence, upon completing the protocol, Eve cannot link Alice to Alice's participation proof --- even if she were to compromise Alice's device.
%Sonja says: I'm confused: can we or can we not provide this?
%Seb: I agree with Sonja, it is not clear whether we achieve it or not

%Seb: to do complete the contributions
\emph{Contributions.} In this paper, our main contribution is to propose \PRIVO, a privacy-preserving scheme providing a way to securely estimate crowd count in protests.
Additionally, this verification system does not incur any additional risk to protesters' privacy apart from the ones related to participating physically to the protest itself.

In contrast to the traditional assumption of honest verifiers in distance-bounding algorithms, we propose a way to turn the Schnorr protocol into a distance-bounding protocol that allows for a malicious verifier.
%Seb: above, we should put the reference to the Schnorr protocol 
In addition, this allows us to form distance-bounding anonymous credentials.

\emph{Limitations and assumptions.} Some of the assumptions that are required for implementing our proposition are not yet realized (see \cref{assumptions} for more detailed descriptions). 
In particular, not all (or even many) protesters, especially in countries with oppressive regimes, currently possess smartphones.
In addition, digital certificates signed by a central authority are not yet widely available and if we rely on government-issued credentials, we cannot prevent the government from performing Sybil attacks --- since it can issue arbitrarily many \emph{valid} credentials. 
%Seb: maybe to prevent the above issue, we could require the individual to generate his secret key and only use the authority to obtain a certification on his public key (in the spirit of blind signature)
In this case, it does not make sense to verify any pro-government protests in which Eve's authoritarian regime wants to organize a pro-regime protest to demonstrate its wide support~\cite[e.g.][]{AlJazeeraOnVenezuela2017,VenezuelanStateWorkersCalledToParticipate}.
Finally, achieving distance-bounding protocols on smartphones is currently not feasible within a meaningful range of distance for our scenario. 

However, we believe that some assumptions might become sufficiently realistic in the near future, due to the smartphone penetration being on the rise and the increased digitalization and government-issued credentials available in some countries already now (\eg Estonia and Germany)
In addition, the distance-bounding chips\footnote{\url{https://www.3db-access.com}} currently available can already enable proofs of proximity with a range of 200 meters.
One of the short-term objective is to be able to integrate them in phones or smartcards in the near-future, and even phones with off-the-shelf hardware running in RFID-emulation mode have shown promising results~\cite{DBonSmartphones}.

\PRIVO achieve privacy, but not receipt freeness.
This means that there is a way for Eve to efficiently verify that Alice has to the participated.
However, this requires Eve to be able to get a copy of Alice's secret key so that she can simply re-do the (deterministic) computations on the same inputs (the probability of collisions being negligible).

%Seb: to do, update the outline
\emph{Outline.} The paper is organized as follows. 
First, we define and formalize the notion of protest and discuss how is relationship with electronic voting in \cref{SystemModel}.  
We then present the desired security properties for such a system and subsequently discuss related work in terms of these properties. 
We give the relevant background on the building blocks of our solution \PRIVO and then describe our proposal for distance-bounding anonymous credentials that copes with a malicious (impersonating) verifier. 
Finally, we present the protocol constructed from the building blocks, its security and privacy analysis, and finally we conclude by a discussion. 