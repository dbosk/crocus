\section{Introduction}%
\label{Introduction}

%\emph{Context.} 

While our proposal is meant for crowd counting in general and to be applicable to any kind of event, we represent an event with a political protest against an oppressive regime throughout the paper. We choose this instantiation of an event because it is the most challenging in terms of requirements for privacy and verifiability (transparency).

Historically, there are many examples of protests in which the count estimated by police and that of the organizers differ significantly, sometimes by hundreds of thousands.
Even without foul play, the difference is quite natural as both parties have different objectives and metrics.
More precisely, the organizers want to count everyone who participated while the police want to estimate the count at the peak of participation, due to crowd control~\cite{2016DemonstrationsInSeoul}.
Among the numerous recent examples in which it is difficult to establish the actual number of participants, there are the demonstrations against the president in South Korea~\cite{2016DemonstrationsInSeoul}, Trump's inauguration~\cite{HowWillWeKnowTrumpInauguralCrowdSize}, the 2017 Women's March in the US~\cite{2017WomensMarchesInUS}, the demonstrations against the change of constitution in Venezuela~\cite{AlJazeeraOnVenezuela2017} or for the independence of Catalonia~\cite{CataloniaDemonstrations}.

Consider the following scenario.
Alice is an activist that organizes a protest in some location(s) against the current government, represented by Grace.
Alice wants to estimate the number of participants to prove a certain support for her cause against Grace in a way that is both demonstrably accurate and does not increase the participants' risk of retribution against them from Grace's regime. 
To realize this objective, a reliable yet privacy-preserving crowd-counting mechanism is needed, which to the best of our knowledge is a problem that has not yet been entirely solved.



Existing methods for crowd-counting vary significantly in terms of approaches.
% (we will review them in detail in \cref{current-crowd-counting}).
Most of them, however, lack precision (\ie they have large error margins) and can only give an estimate for a particular snapshot in time, not the cumulative participation count --- at least not without counting some persons multiple times.  In addition, they lack verifiability in the sense that one has to trust the third party responsible for implementing the counting method.

Finally, one important observation about crowd-counting that has not been adequately addressed in the design of current crowd-counting solutions is that it actually is an adversarial setting. Alice the activist has an incentive to increase the tallied number of participants, whereas Grace (and possibly other entities) has an incentive to decrease it.
%In this case, we are left with two options: either we trust Alice or Grace (or another third party) or we have to be able to verify their claims ourselves.
In this paper, our main objective is to provide a scheme preventing both Alice 
and Grace from cheating by providing verifiable participation counts that can resist Sybil attacks while still preserving the participants' privacy to the extent possible given their physical presence at the protest. While one cannot
prevent an observer (physically present or looking at photos or videos) from recognizing a particular individual at a protest, we do not want the digital traces of our protocol to increase any risk for the participants.

%we have to ensure that none of the digital traces generated by a participant due to our scheme can be associated with their identity. % and that any witnessing is also unlinkable.
%\seb{in the sentence above I am not sure that the word unlinkable will be 
%  necessary clear}

%\paragraph*{Contributions}

%Our first contribution is \CROCUS, a privacy-preserving scheme that provides a 
%way to securely estimate crowd counts in protests\footnote{%
%  If location is ignored, it can be reduced to privacy-preserving yet 
%  verifiable petitions.
%}.
%
%Our second contribution is an adaptation of the Schnorr protocol~\cite{Schnorr} for \ac{DB}, but which is different from that of \textcite{DistanceBounding}.
%Our adaptation fulfills all the requirements for modern \ac{DB} protocols, 
%including \ac{DBDH} (which was an attack against the original protocol of 
%Brands and Chaum) and \ac{DBTF}.
%More importantly, it is public-key, handles malicious impersonating verifiers.
%Since it is based on the Schnorr protocol, it provides a distance-bounded 
%\ac{PK} for discrete logarithms.
%That allows us to replace the plain Schnorr protocol in \acp{ZKPK} with our 
%distance-bounding version and thus make general privacy-preserving, 
%attribute-based credential systems distance bounding.

% The paper is organized as follows.
% First, we describe our system model and summarize the desired properties for crowd counting (\cref{system-model}). Next, we discuss related work in terms of these properties, first the crowd counting methods that are classically used and then work on location proof systems that inspired \CROCUS (\cref{current-crowd-counting}). 
% We then formalize the notion of protest and the desired verifiability and privacy properties (\cref{definitions}), and we give the relevant background on the building blocks of our solution (\cref{building-blocks}). 

\subsection{Desired properties}%
\label{desired-properties}

We note that, in general terms, protests are petitions with a given time and 
location.
Protests, petitions and elections share that in all three many individuals 
express their opinion.
To support democracy, we need strong requirements for verification and privacy 
for elections, it follows that we should have similar properties for protests
and petitions.

We draw inspiration from three properties for voting systems:
\begin{enumerate*}
  \item \emph{eligibility}, anyone can verify that each cast vote is 
    legitimate;
  \item \emph{universal verifiability}, anyone can verify that the result is 
    according to the cast votes;
  \item \emph{individual verifiability}, each voter can verify that their vote 
    is included in the result.
\end{enumerate*}
In our context, votes are translated into \emph{participation proofs}.
Universal and individual verifiability remain the same, in the sense that anyone can verify the participation count by counting the proofs and a participant can verify their proof is included.
The eligibility requirement is slightly different as for protests it must also
include temporal %\footnote{%
%   Elections also have some temporal aspects, such that one cannot vote at any 
%   time and a vote in one election should not be reusable in the next election.
%   However, the temporal relation for a protest is not as strictly defined as for 
%   elections as a protest can start at any time and last for an arbitrary period 
%   of time.%
% }
and spatial eligibility (\ie the protester must have been present at the 
protest).
In essence, the proof must bind the person to the time and location of the protest.
(This is the difference to a petition.)

The main three privacy properties for voting protocols are given as:
\begin{enumerate*}
  \item \emph{vote privacy}, the voting does not reveal any individual vote;
  \item \emph{receipt freeness}, the voting system does not provide any data 
    that can be used as a proof of how the voter voted;
  \item \emph{coercion resistance}, a voter cannot cooperate with a coercer to 
    prove their vote was cast in any particular way.
\end{enumerate*}

%Coercion resistance is not possible to achieve for protests.
Coercion resistance in voting typically relies on physical isolation
(\eg private voting booths), including for digital systems, and that is by 
definition not possible in our context.
For instance, someone could simply physically bring Alice to a protest against her will.
As for receipt freeness, while
desirable \emph{in itself}, it implies a conflict with verifiability in our context:
%appealing at first glance, is not desirable (probably not possible)
%in our context since this will conflict with verifiability:
in contrast to voting, receipt freeness for \emph{how} the voter voted (\ie the 
cause of the protest) here implies receipt freeness for \emph{that} the voter voted 
(\ie the protester was there), which would make verifiability impossible.

Privacy remains as the crucial property.
More precisely, for the protester we want unlinkability (from the adversary's 
perspective) between a protester's real identity and the participation proof 
(and thus also the protest itself).


\subsection{Related Work}
\section{Desired properties and current crowd-counting methods}%
\label{current-crowd-counting}

\subsection{Desired properties}%
\label{desired-properties}

We note that, in general terms, protests are petitions with a given time and 
location.
Protests, petitions and elections share that in all three many individuals 
express their opinion.
These opinions can be sensitive (\eg be a cause for discrimination or 
persecution).
For that reason we have strong requirements for verification and privacy for 
elections, it follows that we should have similar properties for protests
and petitions.

We draw inspiration from properties for voting systems as formalized
in ~\cite{VerifyingPrivacyPropertiesOfVotingProtocols}:
\daniel{Find a new ref for voting verifiability properties, Douglas commented 
  that the last one is not the origin.}
\begin{description}
  \item[Eligibility:] anyone can verify that each cast vote is legitimate.
  \item[Universal verifiability:] anyone can verify that the result is according to the cast votes.
    \daniel{Douglas: technically this is Sako-Kilian, non-technically, it's 
      old.}
  \item[Individual verifiability:] each voter can verify that their vote is included in the result.
    \daniel{Douglas: this is much earlier than Sako-Kilian, in essence 
      Fiat-Shamir (technically). Non-technically, much older.}
\end{description}
In our context, votes are translated into \emph{participation proofs}.
Universal and individual verifiability remain the same, in the sense that anyone can verify the participation count by counting the proofs and a participant can verify their proof is included.
The eligibility requirement is slightly different as for protests it must also
include temporal %\footnote{%
%   Elections also have some temporal aspects, such that one cannot vote at any 
%   time and a vote in one election should not be reusable in the next election.
%   However, the temporal relation for a protest is not as strictly defined as for 
%   elections as a protest can start at any time and last for an arbitrary period 
%   of time.%
% }
and spatial eligibility (\ie each participation proof satisfies some temporal 
and spatial relation to the protest).
In essence, the proof must bind the person to the time and location of the protest.
(This is the difference to a petition.)

In ~\cite{VerifyingPrivacyPropertiesOfVotingProtocols}, the main three
privacy properties for voting protocols are given as:
\begin{description}
  \item[Vote privacy:] the voting does not reveal any individual vote.
  \item[Receipt freeness:] the voting system does not provide any data that can be used as a proof of how the voter voted.
  \item[Coercion resistance:] a voter cannot cooperate with a coercer to prove their vote was cast in any particular way.
\end{description}

%Coercion resistance is not possible to achieve for protests.
Coercion resistance in voting typically relies on physical isolation
(\eg private voting booths), including for digital systems, 
and that is by definition not possible for public events.
For instance, someone could simply physically bring Alice to a protest against her will.
As for receipt freeness, while
desirable \emph{in itself}, it implies a conflict with verifiability in our context:
%appealing at first glance, is not desirable (probably not possible)
%in our context since this will conflict with verifiability:
in contrast to voting, receipt freeness for \emph{how} the voter voted (\ie the 
cause of the protest) here implies receipt freeness for \emph{that} the voter voted 
(\ie the protester was there), which would make verifiability impossible.

Therefore in our context, the crucial property is  vote privacy.
More precisely, for the protester we want unlinkability (from the adversary's 
perspective) between a protester's real identity \(P\) and the participation 
proof (and thus also the protest itself).
Phrased differently, given a participation proof, Grace should not be able to 
distinguish if it was Alice or Bob who participated.
Furthermore, if Grace has managed to link one proof to Alice due to some 
auxiliary knowledge, she should not be able to link it to another proof (from a different protest).

\subsection{Current crowd-counting methods}

The seemingly most common method for counting crowds at protests is \emph{Jacobs's method}~\cite{2016DemonstrationsInSeoul,BBCHowToCountProtestNumbers,HowWillWeKnowTrumpInauguralCrowdSize,TheXManMarch,TheCrowdNumbersGame}.
This manual method devised in the 1960s relies on aerial pictures of the event.
The verifier divides the protest venue into regions and then estimates the density of the crowd in the different regions before summing them up to get an estimate of the global count. Using pictures makes it difficult to get cumulative counts, verify that the pictures have not been manipulated, and to have both privacy and individual verifiability: either one is included in the picture (privacy problem) or not (verifiability problem). Similar limitations exist for estimating the number of persons in a picture or video (\eg the work of \cite{NNCrowdCounting} or  CrowdSize~\cite{CrowdSize}).

Another problem for all the above methods is exemplified by the demonstrations in Seoul:
\blockcquote{2016DemonstrationsInSeoul}{%
  \textins*{t}he demonstrators not only gather in open space, but also small alleys and between buildings%
}.
In this situation it is very difficult to faithfully capture the situation.
Taking pictures from different angles risks double counts. Another challenges is determining whether people near the event's perimeter are participants or simply bystanders~\cite{HowToEstimateCrowdSize}.

Counting MAC addresses, as done by a company during the protests in Seoul~\cite{2016DemonstrationsInSeoul} suffers from MAC randomization, though some tracking of smartphones could still be possible with a different method~\cite{WhyMACRandomizationIsNotEnough} or using \emph{IMSI catchers}; none of which is verifiable.

An approach that relies on a trusted infrastructure was recently deployed by a collection of media outlets to count protesters passing the line defined by a trusted sensor on marches~\cite{LeMondeProtestingSolution}. 
This solution does not offer strong verifiability guarantees and thus is complemented by micro-counts made by humans to estimate their margin of error.

CrowdCount \citet{CrowdCount} is a web service that lets Alice create an event such that anyone can submit their location to register that they are in Alice's event.
Another related approach based on devices is UrbanCount~\cite{UrbanCount}, which relies on epidemic spreading of crowd-size estimates by device-to-device communication to count crowds in dense urban environments with high node-mobility and churn.
However, there is no consideration of a potentially adversarial setting and thus no verifiability or checks on eligibility.  DiVote~\cite{DiVote}, a prior work by the same authors for polling in dense areas, avoids double counting, but again only works with honest participants and thus does not suit an adversarial setting.


\paragraph{Location-proof systems}%
\label{location-proof-systems}

In a nutshell, \iac{LP} is a digital certificate attesting that someone was at 
a particular location at a specific moment in time.
\Iac{LPS} is an architecture by which users can obtain \acp{LP} from 
neighboring witnesses (\eg trusted access points or other users) that can later 
be shown to verifiers who can check the validity of a particular 
proof~\cite{luo2010veriplace,zhu2011applaus}.
Most of the existing approaches to \acp{LP} require the prover and the 
witnesses to disclose their identities, thus raising many privacy issues such 
as the possibility of tracing the movements of users of the \ac{LPS}.
However, some \acp{LPS}, such as PROPS~\cite{PROPS}, exist that provide strong 
privacy guarantees along with the possibility of verifying the claim of the 
location.

Our work is very related to \acp{LPS}.
The crucial differences are how the \ac{LP} is used and the more adverse 
environment.
CROCUS must tie the cause of the protest to the location proof to designate the 
proof to the protest in which the protester participated.
Due to the adverse environment, \CROCUS cannot provide any guarantees using the 
decentralized, untrusted witnesses that, \eg, PROPS can.
The incentives to cheat are also bigger and consequently the thresholds for 
collusion are much higher.

In PASPORT~\cite{PASPORT}, the verifier is trusted and also known in advance.
This does not work with universal verifiability.
(The verifier chooses the subset of witnesses that can vouch for a prover. 
Which makes it an interactive protocol. And the verifier must know which 
witnesses are on location in advance.)

\Textcite{ProofOfWitnessPresence} does not treat the adverse setting and, 
particularly, the problem of \ac{DB} (see \cref{distance-bounding}).

\subsection{Sybil-free systems}%
\label{sybil-free-systems}

Our Sybil-free pseudonym system is very similar to that of 
\textcite{SybilFreePseudonyms}.
Both are based on the work of \textcite{HowToWinTheCloneWars}.
We make the same simplification of~\cite{HowToWinTheCloneWars} as 
\textcite{SybilFreePseudonyms}, but we also require an interactive version.
Furthermore, our \acp{PK} must be distance bounding.


\subsection{Outline}

The paper is organized as follows.
In \cref{definitions}, we describe our system model and formalize the notion of 
protest and the desired verifiability and privacy properties.
In \cref{building-blocks}, we give the relevant background on the building 
blocks of our solution.
%
%\Cref{DB-anon-cred}, we present a new \ac{DB} version of the Schnorr protocol, 
%different from that originally proposed by \textcite{DistanceBounding}.
%We provide a proof of its security and formally verify it using 
%Tamarin~\cite{TamarinDB}.
%
We present \CROCUS, a privacy-preserving crowd counting 
%estimation 
protocol in in \Cref{Protocol} and analyze its security in
\cref{SecurityAnalysis}.
%and discuss practical considerations including performance in \cref{Practical}.
%We compare it to related work in \cref{related-work}.
%Finally, we discuss limitations and assumptions in \cref{Discussion}
Finally, we state
our conclusions in \cref{Conclusion}.

