\section{Introduction}%
\label{Introduction}

Although there are numerous contributions in the field of \ac{DB} 
authentication protocols, the majority of them are shared-key protocols.
This means that the prover and the verifier must share a secret.
This might work for key-less entry protocols for cars, where the car and keyfob 
can share a key.
But it does not work for ubiquitous \ac{DB} authentication.

Our work present contributions in the area of \ac{DB} public-key protocols, 
particularly \ac{DB} \acp{ZKPK}.
This allows anyone to act as a verifier, unlike in the shared-key scenario 
where the verifier must be trusted.
The \ac{ZK} property also enables a variety of privacy properties, such as 
anonymity.
With our contribution, Alice can
now prove more complex statements to Bob; \eg that she is older than 18 
(without revealing her actual age), has a cinema subscription which is not 
\enquote{double spent} by someone else at the same time~\cite[\eg][]{AnonPass} 
and, at the same time, Bob can be sure that Alice is not just relaying messages 
from her older sister Carol or \ac{DBMF} relaying the subscription from some 
unsuspecting stranger also waiting in the queue to enter.


\paragraph*{Contributions}

We make three contributions.
First, we provide a version of the Schnorr protocol that is \ac{DBMF}, 
\ac{DBTF}, \ac{DBDH} and \ac{DBDF} resistant in the \ac{DFKOmodel}.
This version is based on the assumption that we can transmit all bits at once, 
not bit-by-bit as normally required for \ac{DB}.
This assumption is motivated by the results of \textcite{UWBPR}.

Second, for the assumption about sending all bits at once to hold, 
\textcite{UWBPR} requires that two devices shared a symmetric key~\(k\).
Our second contribution is \iac{PKI} and \ac{AKE} scheme for any device in 
the \ac{PKI} to convince a verifier that this is the case.
This replaces everyone having the same static key~\(k\) with everyone having a 
unique key which is signed by \iac{CA} of the \ac{PKI}, which allows for 
revocation.
The solution relies on our first contribution.

Our last contribution is another version of the protocol, one which uses 
bit-by-bit transmissions as traditionally done with \ac{DB} protocols.
We achieve all properties except (full) \ac{DBTF} resistance.
We see this contribution as merely for the sake of completion, as the previous 
contributions are more efficient.

We also formally verified both the all-bits-at-once and the bit-by-bit 
protocols.
The detailed discussion can be found in \cref{formal-verification}.

The first (and second) contribution enables us to do distance-bounding 
privacy-preserving attribute-based credentials based on discrete logarithms, 
\eg \cite{Camenisch-phdthesis} and \cite{CLsignatures}.
We can do this by simply replacing their use of the Schnorr 
protocol~\cite{Schnorr} for \acp{ZKPK} with our protocol in this paper.
To our knowledge, no other distance-bounding protocol achieves this.
The third contribution (bit-by-bit version) also enables us to do the same, 
just without (full) \ac{DBTF} resistance.

\paragraph*{Outline}

The paper is organized as follows.
\Cref{related-work} discusses some related work.
\Cref{Preliminaries} introduces the preliminaries of \ac{ZKPK} and \ac{DB}.
\Cref{DB-Schnorr} introduces the first version, where we assume that we can 
send all bits at the same time.
\Cref{DB-Schnorr-PKI} introduces the \ac{PKI} and \ac{AKE} to use the first 
version of the protocol in practice.
\Cref{DB-Schnorr-nbit} describes the bit-by-bit version of the protocol.
\Cref{formal-verification} describes our effort to formally verify the protocol 
using Tamarin.
Finally, we conclude with discussion and conclusions in 
\cref{Discussion,Conclusion}.
