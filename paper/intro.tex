\section{Introduction}%
\label{Introduction}

%\emph{Context.} 


\paragraph*{Contributions}



Our main contribution is an adaptation of the Schnorr protocol~\cite{Schnorr} for \ac{DB}, but which is different from that of \textcite{DistanceBounding}.
Our adaptation fulfills all the requirements for modern \ac{DB} protocols, 
including \ac{DBDH} (which was an attack against the original protocol of 
Brands and Chaum) and \ac{DBTF}.
More importantly, it is public-key, handles malicious impersonating verifiers.
Since it is based on the Schnorr protocol, it provides a distance-bounded 
\ac{PK} for discrete logarithms.
That allows us to replace the plain Schnorr protocol in \acp{ZKPK} with our 
distance-bounding version and thus make general privacy-preserving, 
attribute-based credential systems distance bounding.

\paragraph*{Outline}

The paper is organized as follows.
\Cref{system-model}, we describe our system model and summarize the desired 
properties 
We then discuss related work in terms of these properties in 
\cref{related-work}, and in \cref{building-blocks}, we give 
the relevant background on the building blocks of our solution.

\Cref{DB-anon-cred}, we present a new \ac{DB} version of the Schnorr protocol, 
different from that originally proposed by \textcite{DistanceBounding}.
We provide a proof of its security and formally verify it using 
Tamarin~\cite{TamarinDB}.

We analyze its security in \cref{SecurityAnalysis} and estimate its performance 
in \cref{PerformanceAnalysis}.

Finally, we discuss our assumptions and results in \cref{Discussion} and give 
our conclusions in \cref{Conclusion}.
