\subsection{Distance-bounding protocols}%
\label{distance-bounding}

\Ac{DB} protocols were first suggested by \citet{DistanceBounding} to prevent relay attacks in contactless communications in which the adversary forwards a communication between a prover and a possibly far-away verifier to authenticate. 
These attacks cannot be prevented by cryptographic means as they are independent of the semantics of the messages exchanged.
As a consequence, mechanisms ensuring the physical proximity between a verifier and a prover should be used instead.
\Ac{DB} protocols precisely enable the verifier to estimate an upper bound on his distance to the prover by measuring the time-of-flight of short challenge-response messages (or rounds) exchanged during time-critical phases. 
Time critical phases are complemented by slow phases during which the time is not taking into account. 
At the end of a \Ac{DB} protocol, the verifier should be able to determine if 
the prover is legitimate \emph{and} in his vicinity.
In this sense, \Ac{DB} protocols combine the classical properties of authentication protocols with the possibility of verifying the physical proximity.

The main attacks against \ac{DB} protocols can be summarized as follows:
\begin{itemize}
  \item \Acf{DBDF}: a legitimate but malicious prover wants to fool the verifier on the distance between them.
  \item \Acf{DBMF}: the adversary illegitimately authenticates, possibly using an honest prover far away from the verifier.
  \item \Acf{DBTF}: a legitimate but malicious prover helps an accomplice close to the verifier to authenticate.
  \item \Acf{DBDH}: similar to \ac{DBDF}, the malicious prover is far away but 
    uses an unsuspecting honest prover close to the verifier to pass as being 
    close.
\end{itemize}
There are two lines of attempts at formalizing the above properties: one by 
\citet{DB-BMV} and another by \citet{DB-DFKO}.

The majority of the existing \ac{DB} protocols are symmetric and thus require an honest verifier.
Indeed, in this context it does not make sense to protect against the verifier as he can easily impersonate the prover as he has a knowledge of his secret key.
There has been less work done in the domain of asymmetric (or public-key) \ac{DB} protocols.
Our setting requires a public-key \ac{DB} protocol with a \emph{malicious verifier} who will potentially try to \emph{impersonate the prover}.
The verifier might also try to track the provers and map their identities to their actions, thus we also require privacy.
This leads to the requirement of a \ac{DB} \ac{ZKPK}, or simply \ac{PPK}, with resistance to all the frauds mentioned above as well as against a verifier who will attempt to impersonate the prover.
More specifically, we must combine such \iac{DB} with anonymous credentials 
(which we do in \cref{DB-anon-cred}).
The choice of the anonymous credential system will affect the properties we 
require of the \ac{PPK} (\eg that it must be \iac{PK} for discrete logarithms).
