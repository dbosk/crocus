\subsection{Distance-bounding protocols}%
\label{distance-bounding}

\Ac{DB} protocols were first suggested by \citet{DistanceBounding} to prevent relay attacks in contactless communications in which the adversary forwards a communication between a prover and a possibly far-away verifier to authenticate. 
These attacks cannot be prevented by cryptographic means as they are independent of the semantics of the messages exchanged.
As a consequence, mechanisms ensuring the physical proximity between a verifier and a prover should be used instead.
\Ac{DB} protocols precisely enable the verifier to estimate an upper bound on their distance to the prover by measuring the time-of-flight of short challenge-response messages (or rounds) exchanged during time-critical phases. 
Time critical phases are complemented by slow phases during which the time is not taking into account. 
At the end of a \Ac{DB} protocol, the verifier should be able to
determine if the prover is legitimate \emph{and} in their vicinity.
In this sense, \Ac{DB} protocols combine the classical properties of authentication protocols with the possibility of verifying the physical proximity.

Our setting requires a public-key \ac{DB} protocol with a \emph{malicious 
verifier} who will potentially try to \emph{impersonate the prover}.
The verifier might also try to track the provers and map their identities to 
their actions, thus we also require strong privacy.
In fact, as the construction in \cref{Protocol} shows, we require \iac{DB} 
\ac{ZKPK}, or simply \ac{PPK}, for discrete logarithms.
There exists one such protocol in the literature, namely one by 
\textcite{DB-Schnorr}, we refer to that paper for a detailed discussion.

