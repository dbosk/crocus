\subsection{Distance-bounding protocols}%
\label{distance-bounding}

\Ac{DB} protocols were first suggested by \citet{DistanceBounding} to prevent relay attacks in contactless communications in which the adversary forwards a communication between a prover and a possibly far-away verifier to authenticate. 
These attacks cannot be prevented by cryptographic means as they are independent of the semantics of the messages exchanged.
As a consequence, mechanisms ensuring the physical proximity between a verifier and a prover should be used instead.
\Ac{DB} protocols precisely enable the verifier to estimate an upper bound on their distance to the prover by measuring the time-of-flight of short challenge-response messages (or rounds) exchanged during time-critical phases. 
Time critical phases are complemented by slow phases during which the time is not taking into account. 
At the end of a \Ac{DB} protocol, the verifier should be able to
determine if the prover is legitimate \emph{and} in their vicinity.
In this sense, \Ac{DB} protocols combine the classical properties of authentication protocols with the possibility of verifying the physical proximity.

There are four adversaries for \ac{DB} protocols established in the literature, 
each of which tries to commit a type of fraud.
These can be summarized as follows:
\begin{itemize}
  \item \Acf{DBDF}: a legitimate but malicious prover wants to fool the verifier on the distance between them.
  \item \Acf{DBMF}: the adversary illegitimately authenticates using a, 
    possibly honest, prover who is far away from the verifier.
    (Also known as relaying attack or man-in-the-middle attack.)
  \item \Acf{DBTF}: a legitimate, but malicious, prover helps an accomplice, 
    who is close to the verifier, to authenticate.
    \Ac{DBTF} resistance is a very strong property; it implies that if the 
    verifier accepts the accomplice with non-negligible probability the 
    accomplice can compute the prover's secret key\footnote{%
      This means that even things like functional encryption will not
      help the adversary.
    }.
  \item \Acf{DBDH}: similar to \ac{DBDF}, the malicious prover is far away but 
    uses an unsuspecting honest prover close to the verifier to pass as being 
    close.
    (This is different from \ac{DBMF} in that the honest prover actually tries 
    to authenticate to the verifier, but the malicious prover hijacks the 
    channel at some point(s) during the protocol.)
\end{itemize}
% There are two lines of attempts at formalizing the above properties: one by \citet{DB-BMV} and another by \citet{DB-DFKO}.

% The majority of the existing \ac{DB} protocols are symmetric and thus require an honest verifier.
% Indeed, in this context it does not make sense to protect against the
% verifier as they can easily impersonate the prover as they have knowledge of their secret key.
% There has been less work done in the domain of asymmetric (or public-key) \ac{DB} protocols.
Our setting requires a public-key \ac{DB} protocol with a \emph{malicious verifier} who will potentially try to \emph{impersonate the prover}.
The verifier might also try to track the provers and map their identities to 
their actions, thus we also require strong privacy.
In fact, as the construction in \cref{Protocol} shows, we require \iac{DB} 
\ac{ZKPK}, or simply \ac{PPK}, for discrete logarithms.
For this paper, we assume the existence of such a protocol.
There exists one such protocol in the literature, namely one by 
\textcite{DB-Schnorr}, we refer to that paper for a detailed discussion.

