\section{Security requirements}%
\label{Properties}

%\subsection{An overview of the adversary}

In our setting, we have three potential adversaries: Alice, Grace and Mallory.
\seb{I changed the name of Grace to Eve to be consistent with the introduction}
\daniel{Changed back to Grace since that's what's in the intro now.}
Alice is a participant to the protest and thus she might be also interested in manipulating the system to increase the count (\eg she might have an incentive to do so as an activist).
In contrast, the aim of Grace, the totalitarian dictator, is to decrease the 
count but also additionally to deanonymize some of the participants in order to 
arrest them or forcefully convince them to change their minds.
Finally, Mallory represents another nation state and has some interest in 
affecting the stability of Grace's regime, for Mallory's own gain, thus 
supporting either Grace or Alice as she see fits.
Thus, the objective of Mallory will also be to either increase or decrease the count.
\seb{we have to justify why Mallory is not captured either by Alice or Grace}
In addition, Mallory could also have simply as her objective to cause a 
denial-of-service attack on the architecture of \PRIVO.
In the following two subsections, we will formally define some security 
properties specifying the computational effort needed by the adversaries to 
succeed.

\subsection{Verifiability}%
\label{Verifiability}

Note that in general terms, protesting is very similar to petitions, which in 
turn are similar to voting: all three situations correspond to many individuals 
expressing their opinion.
These opinions can be sensitive (\eg be a cause for discrimination or persecution), hence we desire to have similar properties of verification and privacy for verifying a protest as there is for voting. 
To draw on this analogy, afterwards we will discuss the relationship between voting and protesting when categorizing the security requirements into verifiability and privacy properties.

Voting has (generally) three desirable requirements for verifiability~\cite{VerifyingPrivacyPropertiesOfVotingProtocols}.
\begin{description}
  \item[Eligibility:] anyone can verify that each vote cast is legitimate.
  \item[Universal verifiability:] anyone can verify that the result is according 
    to the cast votes.
  \item[Individual verifiability:] every voter can verify that their vote is 
    included in the result.
\end{description}
We translate the votes into \emph{participation proofs}.
Universal and individual verifiability remain the same: anyone can verify the participation count by counting the proofs.
The eligibility requirement is slightly different: for protests the eligibility requirement must include temporal and spatial 
eligibility (\ie each proof of participation satisfies some temporal and spatial relation to the protest).
In essence, the proof must bind the person to the time and location of the protest.

To define these properties more formally in the context of protest, we desire three verifiability requirements, among which eligibility can be further broken up in four subproperties:
\begin{requirements}[V]
  \item\label{EligibilityVerif} \emph{Eligibility}:
    anyone can verify that each participation proof provides temporal and 
    spatial eligibility and that only one participation proof is counted per 
    individual.
    \begin{requirements}
    \item\label{CreatedAfterStart} \emph{Temporal eligibility}:
      prove that the proof was created after the start of the protest and
      %\label{CreatedBeforeEnd} prove that the proof was created 
      before the end of the protest.
    \item\label{SpatiallyRelated} \emph{Spatial eligibility}:
      prove that the proof is spatially related to the physical location or 
      journey of the protest.
    \item\label{CountOnce} \emph{One-proof-per-person}:
      prove that no individual can be counted more than once for a particular 
      protest.
    \item\label{DesignatedEvent} \emph{Designated event}:
      prove that the proof is designated for the protest.
    \end{requirements}

  \item\label{UniversalVerif} \emph{Universal verifiability}:
    anyone can verify that the result obtained match the submitted participation 
    proofs.

  \item\label{IndividualVerif} \emph{Individual verifiability}:
    each participant can verify that their participation proof is included in 
    the global count.
\end{requirements}

\subsection{Privacy}%
\label{Privacy}

In addition to the verification requirements, we also need to formalize the privacy properties.
As we indirectly pointed out earlier, we focus on the privacy provided to Alice and the protesters.
In particular, our objective is that as long as Alice can conceal her identity 
at the demonstration and escape without arrest, her support is recorded in the 
data without her privacy being violated.
\seb{I think this sentence should be clarified}

In voting protocols, there are also three levels of privacy~\cite{VerifyingPrivacyPropertiesOfVotingProtocols}:
\begin{description}
\item[Vote privacy:] the voting does not reveal any individual vote.
\item\label[Receipt freeness:] the voting system does not provide any data that 
  can be used as a proof of how the voter voted.
\item[Coercion resistance:] a voter cannot cooperate with a coercer to prove the 
  vote was cast in any particular way.
\end{description}
\Citet{VerifyingPrivacyPropertiesOfVotingProtocols} showed that coercion 
resistance implies receipt freeness, which in turn implies vote privacy.
Coercion resistance is probably not possible to achieve for protests:
\eg Grace can simply physically bring Alice to a protest against her will.

In essence, receipt freeness means that upon completing the protocol, Grace 
cannot link Alice to Alice's participation proof --- even if she were to 
compromise Alice's device or Alice collaborates with Grace.
As we will see, if Grace gets Alice's device, she can re-do the computations on 
the same inputs.
Since, in our case, the algorithms are deterministic and collisions are 
negligible she will get the same result and can thus conclude that Alice is 
indeed the one she is looking for.
Thus we cannot provide receipt freeness.
(See \cref{Conclusions} for further discussion.)

We note that we cannot actually improve the privacy of participating protesters during the protest: if Alice is caught during a protest or other crowd-counting mechanisms based on photographic or video evidence are used to identify Alice, there is nothing we can do.
\seb{suggestion maybe we should put the following part in the discussion}
However, in \PRIVO, we aim at verifiable participation count that does not add any threats to the participants' privacy. 
For instance, if there are counter-protests in the \emph{same location at the same time}, Alice could even blend into another crowd and argue that she participated in a different protest than she actually did --- and the counts would still be correct afterwards.
However, even if there is only one protest and she is not caught in the act, then using our proposed mechanism should not incur any additional risk.
Thus we need at least participation-proof privacy but receipt freeness is also desired.
In essence, this means that upon completing the protocol, Eve cannot link Alice to Alice's participation proof --- even if she were to compromise Alice's device.
\sonja{I'm confused: can we or can we not provide this?}
\seb{I agree with Sonja, it is not clear whether we achieve it or not}

\subsubsection{Participation-proof privacy}

\begin{definition}[Proof indistinguishability]
  Adversary chooses two messages \(m_0, m_1\) and gives them to the challenger.
  Challenger uniformly randomly selects \(b\rgets \bin\) and runs \(c_b \gets 
    PRF_k(m_0), c_{1-b} \gets PRF_k(m_1)\).
  The challenger gives \(c_0, c_1\) to adversary.
  If the adversary outputs \(b' = b\) she wins.
  (The property should be similar to key privacy~\cite{KeyPrivacy}.)
\end{definition}

\subsubsection{Receipt freeness}

\begin{definition}[Receipt freeness/deniability]
  Adversary gets oracle access to the user?
  User provides a witness?
\end{definition}

