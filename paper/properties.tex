\section{Security requirements}%
\label{Properties}

%\subsection{An overview of the adversary}

In our setting, we have two potential adversaries: Alice and Grace.
Alice is a participant to the protest and thus she might be also interested in 
manipulating the system to increase the count (\eg she might have an incentive 
to do so as an activist).
In contrast, the aim of Grace, the totalitarian dictator, is to decrease the 
count but also additionally to deanonymize some of the participants in order to 
arrest them or forcefully convince them to change their minds.
In the following two subsections, we will define the desired security 
properties.

\subsection{Verifiability}%
\label{Verifiability}

We note that, in general terms, protesting is very similar to petitions, which 
in turn are similar to voting: all three situations correspond to many 
individuals expressing their opinion.
These opinions can be sensitive (\eg be a cause for discrimination or 
persecution), hence we desire to have similar properties of verification and 
privacy for verifying a protest as there are for voting.

Voting has (generally) three desirable requirements for verifiability~\cite{VerifyingPrivacyPropertiesOfVotingProtocols}.
\begin{description}
  \item[Eligibility:] anyone can verify that each vote cast is legitimate.
  \item[Universal verifiability:] anyone can verify that the result is according 
    to the cast votes.
  \item[Individual verifiability:] every voter can verify that their vote is 
    included in the result.
\end{description}
We translate the votes into \emph{participation proofs}.
Universal and individual verifiability remain the same: anyone can verify the participation count by counting the proofs.
The eligibility requirement is slightly different: for protests the eligibility 
requirement must include temporal and spatial eligibility (\ie each 
participation proof satisfies some temporal and spatial relation to the 
protest).
In essence, the proof must bind the person to the time and location of the protest.

To define these properties more formally in the context of protest, we desire 
three verifiability requirements, among which eligibility can be further broken 
up into four subproperties:
\begin{requirements}[V]
  \item\label{EligibilityVerif} \emph{Eligibility}:
    anyone can verify that each participation proof provides temporal and 
    spatial eligibility and that only one participation proof is counted per 
    individual.
    \begin{requirements}
    \item\label{TemporallyRelated} \emph{Temporal eligibility}:
      prove that the proof was created after the start of the protest and
      before the end of the protest.
    \item\label{SpatiallyRelated} \emph{Spatial eligibility}:
      prove that the proof is spatially related to the physical location or 
      journey of the protest.
    \item\label{CountOnce} \emph{One-proof-per-person}:
      prove that no individual can be counted more than once for a particular 
      protest.
    \item\label{DesignatedEvent} \emph{Designated event}:
      prove that the proof is designated for the protest.
    \end{requirements}

  \item\label{UniversalVerif} \emph{Universal verifiability}:
    anyone can verify that the result obtained match the submitted participation 
    proofs.

  \item\label{IndividualVerif} \emph{Individual verifiability}:
    each participant can verify that their participation proof is included in 
    the global count.
\end{requirements}

\subsection{Privacy}%
\label{Privacy}

In addition to the verification requirements, we also need to define the 
privacy properties.
In voting protocols, there are also three levels of 
privacy~\cite{VerifyingPrivacyPropertiesOfVotingProtocols}:
\begin{description}
\item[Vote privacy:] the voting does not reveal any individual vote.
\item[Receipt freeness:] the voting system does not provide any data that can 
  be used as a proof of how the voter voted.
\item[Coercion resistance:] a voter cannot cooperate with a coercer to prove the 
  vote was cast in any particular way.
\end{description}
\Citet{VerifyingPrivacyPropertiesOfVotingProtocols} showed that coercion 
resistance implies receipt freeness, which in turn implies vote privacy.
Coercion resistance is probably not possible to achieve for protests:
\eg Grace can simply physically bring Alice to a protest against her will.

In essence, receipt freeness means that upon completing the protocol, Grace 
cannot link Alice to Alice's participation proof --- even if she were to 
compromise Alice's device or Alice collaborates with Grace.
As we will see, if Grace gets Alice's device, she can re-do the computations on 
the same inputs.
Since, in our case, the algorithms are deterministic and collisions are 
negligible she will get the same result and can thus conclude that Alice is 
indeed the one she is looking for.
Thus we cannot provide receipt freeness.
(See \cref{Conclusions} for further discussion.)

This leaves us with what should correspond to vote privacy.
We need unlinkability between Alice's long-term identity and Alice's 
participation proof.
Given a participation proof, Grace will not be able to tell if it belongs to 
Alice or Bob.
Furthermore, if Grace has managed to link one proof to Alice's, she should not 
be able to link another proof as a consequence.
Hence, we can summarize our needed properties as follows:
\begin{requirements}[P]
\item\label{ProofUnlink} Participants must be unlinkable to their proofs.
\item\label{ProtestUnlink} Participants (or proofs for the same participant) 
  must be unlinkable between different protests.
\end{requirements}
Thus Alice can always argue to be part of the protest in Grace's favour instead 
of any protest against Grace, which, in fact, is a slight increase in privacy 
over ordinary protests.

