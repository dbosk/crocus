%%%%%%%%%%%%%%%%%%%%%%%%%%%%%%%%%%%%%%%%%%%%%%%%%%%%%%%%%%%%%%%%%%%%%%%%%%%%%%%%
% Template for USENIX papers.
%
% History:
%
% - TEMPLATE for Usenix papers, specifically to meet requirements of
%   USENIX '05. originally a template for producing IEEE-format
%   articles using LaTeX. written by Matthew Ward, CS Department,
%   Worcester Polytechnic Institute. adapted by David Beazley for his
%   excellent SWIG paper in Proceedings, Tcl 96. turned into a
%   smartass generic template by De Clarke, with thanks to both the
%   above pioneers. Use at your own risk. Complaints to /dev/null.
%   Make it two column with no page numbering, default is 10 point.
%
% - Munged by Fred Douglis <douglis@research.att.com> 10/97 to
%   separate the .sty file from the LaTeX source template, so that
%   people can more easily include the .sty file into an existing
%   document. Also changed to more closely follow the style guidelines
%   as represented by the Word sample file.
%
% - Note that since 2010, USENIX does not require endnotes. If you
%   want foot of page notes, don't include the endnotes package in the
%   usepackage command, below.
% - This version uses the latex2e styles, not the very ancient 2.09
%   stuff.
%
% - Updated July 2018: Text block size changed from 6.5" to 7"
%
% - Updated Dec 2018 for ATC'19:
%
%   * Revised text to pass HotCRP's auto-formatting check, with
%     hotcrp.settings.submission_form.body_font_size=10pt, and
%     hotcrp.settings.submission_form.line_height=12pt
%
%   * Switched from \endnote-s to \footnote-s to match Usenix's policy.
%
%   * \section* => \begin{abstract} ... \end{abstract}
%
%   * Make template self-contained in terms of bibtex entires, to allow
%     this file to be compiled. (And changing refs style to 'plain'.)
%
%   * Make template self-contained in terms of figures, to
%     allow this file to be compiled. 
%
%   * Added packages for hyperref, embedding fonts, and improving
%     appearance.
%   
%   * Removed outdated text.
%
%%%%%%%%%%%%%%%%%%%%%%%%%%%%%%%%%%%%%%%%%%%%%%%%%%%%%%%%%%%%%%%%%%%%%%%%%%%%%%%%

\documentclass[letterpaper,twocolumn,10pt]{article}
\usepackage{usenix2019_v3}

\usepackage[utf8]{inputenc}
\usepackage[T1]{fontenc}

\usepackage{tikz}
\usepackage{amsmath}

\usepackage[utf8]{inputenc}
\usepackage[T1]{fontenc}
\usepackage[british]{babel}
\usepackage{hyperref}
\usepackage{xparse}

\usepackage{interval}
\usepackage{amssymb}
\usepackage{amsmath}

\usepackage{csquotes}
\usepackage[natbib,style=alphabetic,maxbibnames=99]{biblatex}
\addbibresource{bibsp/crypto.bib}
\addbibresource{bibsp/location.bib}
\addbibresource{bibsp/reputation.bib}
\addbibresource{bibsp/auth.bib}
\addbibresource{bibsp/voting.bib}
\addbibresource{bibsp/protests.bib}
\addbibresource{bibsp/blockchain.bib}
\addbibresource{bibsp/anon.bib}
\addbibresource{bibsp/ecurrency.bib}

\usepackage[capitalize]{cleveref}
\crefname{properties@i}{property}{properties}
\Crefname{properties@i}{Property}{Properties}
\crefname{requirements@i}{requirement}{requirements}
\Crefname{requirements@i}{Requirement}{Requirements}

\usepackage{tikz}
\usepackage{tikz-qtree}

\usepackage{bibsp}

\DeclareDocumentCommand{\keywords}{m}{%
  \par
  \textbf{Keywords:} #1
}

\usepackage[inline]{enumitem}
\setlist[enumerate]{label*={(\arabic*)}}

\newlist{properties@}{enumerate}{5}
\setlist[properties@]{label*=P\arabic*.,ref=P\arabic*}
\newlist{requirements@}{enumerate}{5}
\setlist[requirements@]{label*=R\arabic*.,ref=R\arabic*}

\DeclareDocumentEnvironment{properties}{o}{%
  \IfValueTF{#1}{%
    \begin{properties@}[label*=#1\arabic*.,ref=#1\arabic*]%
  }{%
    \begin{properties@}%
  }
}{%
  \end{properties@}%
}
\DeclareDocumentEnvironment{requirements}{o}{%
  \IfValueTF{#1}{%
    \begin{requirements@}[label*=#1\arabic*.,ref=#1\arabic*]%
  }{%
    \begin{requirements@}%
  }
}{%
  \end{requirements@}%
}

\usepackage{mathtools}
\usepackage{amsthm}
\usepackage{thmtools}
\declaretheorem[style=dgthm]{theorem}
\declaretheorem[sibling=theorem]{lemma}
\declaretheorem[sibling=theorem]{proposition}
\declaretheorem[style=dgdef]{definition}
\declaretheorem[style=dgdef]{assumption}
\declaretheorem[style=dgdef]{example}
\declaretheorem[style=remark]{remark}
\declaretheorem[style=solution]{solution}
\declaretheorem[style=dgdef]{question}
\declaretheorem[style=dgdef]{idea}


%Seb: to do, anonymize the names of the authors for the submission
\begin{document}
%\author{%
%  {\rm Daniel Bosk}\\
%  KTH Royal Institute of Technology, Stockholm
%  \and
%  {\rm Simon Bouget}\\
%  KTH Royal Institute of Technology, Stockholm
%  \and
%  {\rm Sonja Buchegger}\\
%  KTH Royal Institute of Technology, Stockholm
%  \and
%  {\rm Sébastien Gambs}\\
%  Université du Quebec à Montréal
%  \and
%  {\rm Henri Pineau}\\
%  Université du Quebec à Montréal
%}

\title{\Large\bf%
  %Seb: why put CROCUSes in the plural?
  From grassroots to CROCUSes: privacy-preserving CROwd Counting Using 
  Smartphones 
  %and Distance-Bounding Anonymous Credentials
  %Seb: I like the title but I propose to keep it short and not mentioning anonymous credentials
%  \thanks{%
%    An initial discussion of this work appeared in 
%    \citetitle{FutureProtests}~\cite{FutureProtests}.
%  }
}

\date{}

\title{%
  Verifying Demonstrations
}
\author[D.~Bosk et al.]{%
  Daniel Bosk\inst{1}
  \and
  Sonja Buchegger\inst{1}
  \and
  Sébastien Gambs\inst{2}
}
\institute[KTH, UQAM]{%
  School of Computer Science and Communication\\
  KTH Royal Institute of Technology, Stockholm\\
  \texttt{\{dbosk,buc\}@kth.se}
  \and
  Université de Quebec à Montreal\\
  \texttt{sgambs@uqam.ca}
}

\maketitle

\mode*

\begin{abstract}
  Say Alice organizes a protest against the current regime leader Eve.
The protest is held in some location(s).
Alice wants to estimate the number of participants to prove certain support for
her cause.
Current methods have wide margins for error.
In addition, Alice's incentive is to estimate the number of participants as 
high as possible whereas the government's incentive is to estimate the number 
of participants as low as possible to support Eve.

We are interested in combining location proofs with electronic voting.
Since the parties' have incentives to cheat, we require transparency: we desire 
a decentralized scheme that provides authenticity, verifiability and privacy.
More specifically:
The scheme must be decentralized to not depend on any authority.
We must ensure that the data is time-wise related to the event (authenticity), 
which can be provided by blockchain-like structures.
The data must also be related to the location, something a location proof can 
solve.
As all things policital, privacy is critical.
Similar to e-voting schemes, we need what corresponds to vote privacy and 
receipt freeness --- Alice does not want Eve's agents to verify that she is a 
critic of Eve.
Furthermore, we need verifiability: that every participant can verify that 
their participation has been included --- individual verifiability, to prevent 
Eve from dropping data --- and that anyone can verify the correctness of the 
result --- universal and eligibility verifiability, to prevent any malicious 
actor (e.g.\ Sybil) from affecting the result.
Finally, the scheme must be scalable to handle large protests (millions of 
participants).

\keywords{%
  crowd counting;
  protesting;
  electronic voting;
  location proof;
  privacy
}

\end{abstract}


\mode*

\section{Introduction}
\label{Introduction}

Alice organizes a street protest against Eve's regime.
Bob, Carol and many others participate in the demonstration.
One problem that has not yet been entirely solved is the crowd-counting 
problem, i.e.\ to verify how many participated in Alice's protest.

\subsection<presentation>{What's the problem?}

\begin{frame}<presentation>
  \begin{block}{The crowd-counting problem}
    \begin{itemize}
      \item Alice organizes a protest against Eve's regime.
      \item Bob, Carol and others show up.
      \item We want to know how many showed up to support Alice.
    \end{itemize}
  \end{block}
\end{frame}

\begin{frame}<presentation>
  \begin{example}
    \begin{itemize}
      \item Computer vision does object recognition.
      \item Requires photos/video that cover the entire location, all the time.
      \item This will count people twice.
    \end{itemize}
  \end{example}

  \pause

  \begin{example}
    \begin{itemize}
      \item Scan active mobile phones in the area.
      \item This requires some extra equipment.
      \item This catches bystanders who are not protesting.
    \end{itemize}
  \end{example}
\end{frame}

\subsection<presentation>{Why is it a problem?}

\begin{frame}<presentation>
  \begin{example}
    \begin{itemize}
      \item South Korea 2016~\cite{2016DemonstrationsInSeoul}:
        approximate count by imagery.
        \begin{description}
          \item[Organizers] 1\,000\,000
          \item[Police] 260\,000
        \end{description}

        \pause

      \item US 2017~\cite{2017WomensMarchesInUS}:
        sum up people's approximations.
      \item This makes it even difficult to estimate the error.
    \end{itemize}
  \end{example}
\end{frame}

After many demonstrations the count by police and that by the organizers 
differ\footnote{%
  This is actually quite natural, they have different goals.
  The organizers want to count everyone who ever participated.
  Police want to estimate the count at the peak of participation, due to crowd 
  control~\cite{2016DemonstrationsInSeoul}.
}, in some instances the difference can be hundreds of thousands.
There are numerous examples, e.g.\ the demonstrations in South 
Korea~\cite{2016DemonstrationsInSeoul}, Trump's 
inauguration~\cite{HowWillWeKnowTrumpInauguralCrowdSize} or the Women's Marches 
in the US~\cite{2017WomensMarchesInUS}, where there is difficulty in 
establishing the actual number of participants.
The methods for counting the crowds vary.
Most of the available methods lack precision, i.e.\ they have large error 
margins.
They can only give an estimate for a particular snapshot in time, e.g.\ at the 
peak of the event, not the cumulative participation count --- at least not 
without counting some people multiple times, which in turn increases the error 
of the estimation.
Finally, they also lack verifiability, i.e.\ everyone must trust the entity who
does the counting.
We will discuss them in more detail in \cref{RelatedWork}.

\begin{frame}<presentation>
  \begin{block}{Verifying protest participation}
    \begin{itemize}
      \item Alice organizes a protest against Eve's regime.
      \item Bob, Carol and others show up.

        \pause

      \item {\color{green} Alice wants to show that many support her cause.}

        \pause

      \item {\color{red} Eve wants to show that few support Alice's cause.}

        \pause

      \item It's an adversarial setting!
      \item We need verifiable results.
    \end{itemize}
  \end{block}
\end{frame}

We can make one important observation about this problem that has seemingly 
been ignored in the past: it is an adversarial setting.
The protesters (organizers and participants) have an incentive to increase the 
tallied number of participants, whereas other interests might have an incentive 
to decrease the tallied number of participants, e.g.\ Eve's authoritarian 
regime in our example above.
Or the other way around, sometimes Eve's authoritarian regime wants to organize 
a pro-regime protest to demonstrate its \enquote{wide 
  support}~\cite[e.g.][]{AlJazeeraOnVenezuela2017,VenezuelanStateWorkersCalledToParticipate}.
(This makes it difficult to rely on government-issued credentials to prevent a 
Sybil attack --- government can issue arbitrarily many such credentials for its 
own protests.)
To solve this problem, we need a verifiable participation count which is 
protected from Sybil attacks.

\mode<article>{
\subsection{What is \protect\emph{a} protest?}
\label{WhatIsAProtest}

Protests can vary considerably.
To be able to estimate the participation count for one, we first need to define
what should be counted.

Let us start by considering some examples.
During the demonstrations against the South Korean president in Seoul
\blockcquote{2016DemonstrationsInSeoul}{%
  \textins*{t}he rallies stretch\textins{ed} from midday to late night --- some 
  people stay\textins{ed} for several hours, others just several minutes%
}.
These rallies were all in the same location in the capital and repeated every 
weekend for the duration of a few weeks.
The Women's Marches~\cite{2017WomensMarchesInUS}, on the other hand, occurred 
in parallel in many locations.
We also have the Venezuelan demonstrations where
\blockcquote{2017VenezuelaProtestFrequency}{%
  anti-government demonstrators have staged daily protests across Venezuela%
} while
\blockcquote{AlJazeeraOnVenezuela2017}{%
  pro-government workers sang and danced as they staged a rival march to show 
  their support for the president's controversial plan to rewrite the 
  constitution%
}.
Judging from these examples, the minimal common part is the cause.

\begin{frame}<presentation>
  \begin{example}[Seoul 2016]
    \begin{itemize}
      \item Rallies every weekend.
      \item Starting midday, lasting to late night.
      \item Some stayed for several hours, some for only minutes.
    \end{itemize}
  \end{example}

  \pause

  \begin{example}[US 2017]
    \begin{itemize}
      \item Women's Marches in several locations.
      \item Marched some distance.
      \item One-time occurrence.
    \end{itemize}
  \end{example}
\end{frame}

\begin{frame}<presentation>
  \begin{example}[Venezuela 2017]
    \begin{itemize}
      \item Daily anti-government rallies.
      \item Multiple locations.
      \item Pro-government rallies too.
    \end{itemize}
  \end{example}
\end{frame}

The organizer Alice want to count everyone who participated at any time and in 
any of the locations~\cite{2016DemonstrationsInSeoul}.
(Whereas police are only interested in the maximum crowd at any point in time, 
to deploy enough personnel for crowd-control.)
We thus settle for the following informal definition:
a protest is an event identified by its cause.

\begin{frame}<presentation>
  \begin{definition}[Informal]
    \begin{itemize}
      \item Organizers want to count anyone who participated \dots
      \item \dots at any time
      \item \dots in any location
      \item A protest is thus identified by its cause.
    \end{itemize}
  \end{definition}
\end{frame}
}

% Verifiability and privacy
\mode<all>
\section{Desired security properties}%
\label{Properties}

%\subsection{An overview of the adversary}

We have three \emph{malicious} adversaries: Alice, Eve and Rusuk.
Alice the activist (and all the protesters) try to increase the count.
Eve the totalitarian dictator tries to decrease the count.
Additionally, Eve tries to deanonymize the participants to arrest them or 
\emph{convince} them to change their minds.
Finally, we have Rusuk who is represents another nation state.
Rusuk has some interest in affecting the state of Eve's regime, for Rusuk's own 
gain, thus supporting either Eve or Alice as he see fits.
Rusuk will thus also try to either increase or decrease the count.
We will now formally define some security properties specifying the 
computational effort needed by the adversaries to succeed.
We categorize the desired properties into verifiability and privacy.

\subsection{Verifiability}%
\label{Verifiability}

We desire three verifiability requirements:

\begin{requirements}[V]
  \item\label{EligibilityVerif} Eligibility: anyone can verify that each 
    participation proof provides temporal and spatial eligibility and that it 
    has not been counted before.
    \begin{requirements}
    \item Temporal egligibility:%
      \label{CreatedAfterStart} prove that the data was created after the start of 
      the event;%
      \label{CreatedBeforeEnd} prove that the data was created before the end of 
      the event.
    \item Spatial eligibility:%
      \label{SpatiallyRelated} prove that the data is spatially related to the 
      physical location of the event.
    \item One-proof-per-person:%
      \label{CountOnce} prove that no individual can be counted more than once.
    \item Designated event:%
      \label{DesignatedEvent} prove that the data is designated for the event.
    \end{requirements}

  \item\label{UniversalVerif} Universal verifiability: anyone can verify that the 
    result is according to the submitted participation proofs.
  \item\label{IndividualVerif} Individual verifiability: every participant can 
    verify that their participation proof is included in the global count.
\end{requirements}
We will now discuss these in more detail.

\subsubsection{Temporal eligibility}

\begin{definition}[Forging temporal eligibility]
  \dots
\end{definition}

\subsubsection{Spatial eligibility}

\begin{definition}[Forging spatial eligibility]
  \dots
\end{definition}

\subsubsection{Linkability and designated protest}

\dots

\subsubsection{Individual and universal verifiability}

\dots

\subsection{Privacy}%
\label{Privacy}

We also need privacy in addition to the verification requirements.
As we indirectly pointed out earlier, we focus on the privacy provided to Alice 
and Bob by the data.
So as long as Alice and Bob can conceal their identities at the demonstration 
and escape without arrest, their support is recorded in the data while their 
privacy is not violated.
(Following this line of thinking, it can actually be beneficial for the privacy 
of the demonstrators to mix with the participants of any counter-demonstrations 
--- since the counts will still be correct.)

In voting, we have the following requirements:
\begin{requirements}[P]
\item\label{VotePrivacy} Vote privacy: the voting does not reveal any 
  individual vote.
\item\label{ReceiptFreeness} Receipt freeness: the voting system does not 
  provide any data that can be used as a proof of how the voter voted.
\item\label{CoercionResistance} Coercion resistance: a voter cannot cooperate 
  with a coercer to prove the vote was cast in any particular way.
\end{requirements}
\Citet{VerifyingPrivacyPropertiesOfVotingProtocols} showed that 
\cref{CoercionResistance} implies \cref{ReceiptFreeness}, which in turn implies
\cref{VotePrivacy}.
\Cref{CoercionResistance} is probably not possible to achieve for protests:
e.g.\ Eve can simply physically bring Alice to a protest against her will.
This leaves us with \cref{ReceiptFreeness,VotePrivacy}.

\subsubsection{Participation-proof privacy}

\begin{definition}[Proof indistinguishability]
  Adversary chooses two messages \(m_0, m_1\) and gives them to the challenger.
  Challenger uniformly randomly selects \(b\rgets \bin\) and runs \(c_b \gets 
    PRF_k(m_0), c_{1-b} \gets PRF_k(m_1)\).
  The challenger gives \(c_0, c_1\) to adversary.
  If the adversary outputs \(b' = b\) she wins.
  (The property should be similar to key privacy~\cite{KeyPrivacy}.)
\end{definition}

\subsubsection{Receipt freeness}

\begin{definition}[Receipt freeness/deniability]
  Adversary gets oracle access to the user?
  User provides a witness?
\end{definition}



\subsection<article>{Contributions}

In this paper, we combine the verifiability and privacy properties of 
electronic voting with location-proof systems to provide a verifiable 
participation count.

%In the case of the Korean demonstrations~\cite{2016DemonstrationsInSeoul}, this 
%was in one place during an entire day and then repeated for several weekends.
%In the case of the Women's Marches in the US~\cite{2017WomensMarchesInUS}, they
%were in several locations at the same time.

\blockcquote{HowToEstimateCrowdSize}{%
  Crowd size is also needed for media news reports and to historically record 
  the event%
}




\section{Data authenticity}
\label{DataAuthenticity}

The problem of authentically associating data with a physical event is 
difficult.
We can essentially divide it into the following requirements:
\begin{requirements}[A]
  \item\label{CreatedAfterStart} Prove that the data was created after the 
    start of the event.
  \item\label{CreatedBeforeEnd} Prove that the data was created before the end 
    of the event.
  \item\label{SpatiallyRelated} Prove that the data is spatially related to the 
    physical location of the event.
\end{requirements} %TODO: A4 accurate/not fabricated?
\Cref{CreatedBeforeEnd,CreatedAfterStart} together bind the data to the time of 
the event whereas \cref{SpatiallyRelated} binds the data spatially to the 
event.

Now, consider the scenario of Alice taking a photo during the demonstration and 
posting it online.
What can we say about this photo?
First, we can say that it was created before we viewed it, so 
\cref{CreatedBeforeEnd} is fulfilled if we view it in relation to the event.
If it was submitted to a service that we trust, e.g.\ a blockchain, then we can 
also trust the time-stamp of the service for fulfilling 
\cref{CreatedBeforeEnd}.
Furthermore, we can also consider it spatially related to the physical location
(\cref{SpatiallyRelated}) if we can convince ourselves that the photo is 
depicting the physical location and not any kind of \enquote{reconstruction}, 
e.g.\ it is not computer generated or a photo of a similarly looking location.

\Cref{CreatedAfterStart} is more difficult to achieve.
In the above scenario, there is nothing that prevents Alice from submitting an 
older photo which is spatially related to the event --- i.e.\ to ensure 
\cref{CreatedAfterStart} we must ensure some sort of freshness.
% XXX Verify that this can be done using Fiat-Shamir heuristic
This can (probably) be solved using the Fiat--Shamir 
heuristic~\cite{FiatShamirHeuristic} (or something similar).
The main idea is that the operation binding the unpredictable value to the data 
must be difficult to redo later to change the value.

Instead of a photo, we will use \acp{LP}, more specifically, we will use 
\ac{PROPS}.
This will allow us to fulfil \cref{SpatiallyRelated}.
Since a location proof is a cryptographic value, it is easier to adapt it to 
fulfil \cref{CreatedAfterStart} too (compared to a photo).

\input{Verification.tex}

\section{Adapting PROPS}
\label{AdaptingPROPS}

\subsection{The \acs*{CA} Problems}

One issue is the \ac{CA}.
One direction we can take to solve this is by threshold signatures, \eg 
\cite{FSThresholdSignatures}.
This way the participants can jointly issue new certificates, by signing them 
using the threshold signature scheme.
This should solve part of the problems related to Sybil attacks too.
But, if \(k > t\) participants collude, they can create multiple identities for 
themselves.
The main problem to solve, however, is that we want \(n\), and possibly \(t\) 
too, to change with the number of participants.
I'm not sure, but in what I've read about threshold schemes, they generally 
keep \(n\) and \(t\) fixed.

Another direction on solving the \ac{CA} problem, is if we can use already 
existing \acp{PKI} instead.
\Eg Sweden has a deployed national \ac{eID} system; usually these are RSA or 
DSA keys --- \((pq, e), (p, q, d): ed = 1\pmod{(p-1)(q-1)}\) --- signed by 
running everything through a hash function and the sign that value.
Due to the use of the hash function we cannot prove in zero knowledge that we 
have a signed certificate.
But maybe we can use the keys \((pq, e)\) and \((p, q, d)\) in such a way that 
they do not reveal the identity, but that we can if needed --- similar to 
a commitment which we reveal only in emergencies.

It would be interesting to solve the W--P collusion attack, where the witness 
reports false \acp{LPS}, \eg we would need to prevent that a group of 
individuals fake this information at home.
One possibility in our scenario is that \eg an external journalist on 
location can create \iac{LPS}, because a journalist is generally trusted.
This would be useful as occasionally journalists are also captured, and anyone 
that escaped should be able to show a convincing proof.
However, one problem with this is that the journalist will not necessarily have 
a key signed by the \ac{CA} of our scheme, but rather a \ac{CA} of say a UN 
\ac{PKI}.

In relation to the previous, it would be interesting to be able to tie the 
\acp{LP} (or \acp{LPS}) to a reputation system.
Then, even if no external journalist is on location, the participants can rely 
on their online reputation.
One option would be to have a reputation system in a global location-proof 
system, another would be to tie it to a reputation system already used in \eg 
\iac{OSN}.
One possible solution would be to include a commitment in the \ac{LPS} and when 
necessary use it to tie it to the online identity.

\subsection{One Proof for All}

The naïve approach would be that everyone generates \acp{LPS} for everyone 
else, but this would be computationally and communicatively expensive 
(\(O(n^2)\)).
A better approach would be \iac{MPC} scheme where several witnesses participate 
to compute a joint \ac{LPS} for one prover.
This would reduce the need for pairwise communication.
In our scenario each witness will also want to participate as a prover, if the 
jointly computed \ac{LPS} could further be used by all participants in the 
computation, this would reduce the complexity further.

To make it even more efficient, it would be desirable to make the proofs 
\enquote{transitive}.
\Ie if Alice the prover has \acp{LPS} from a set of witnesses, then she can 
act as a witness and issue \iac{LPS} which includes her previous \acp{LPS}.
Speaking in graphs, we want to be able to join sub-graphs instead of having to 
join node-by-node.



\section{Defining and formalizing protest and crowd estimation}%
\label{SystemModel}

%\subsection{What is \protect\emph{a} protest?}%
%\label{WhatIsAProtest}

\emph{Defining the concept of protest.} To be able to estimate the participation count for a protest, we first need to define this concept and what should be counted.
Let us start by considering some examples.
During the demonstrations against the South Korean president in Seoul
\blockcquote{2016DemonstrationsInSeoul}{%
  \textins*{t}he rallies stretch\textins{ed} from midday to late night --- some people stay\textins{ed} for several hours, others just several minutes%
}.
These rallies were all in the same location in the capital and repeated every 
weekend for the duration of a few weeks.
The Women's Marches~\cite{2017WomensMarchesInUS}, on the other hand, occurred in parallel in many locations.
We also have the Venezuelan demonstrations in which
\blockcquote{2017VenezuelaProtestFrequency}{%
  anti-government demonstrators have staged daily protests across Venezuela%
} while
\blockcquote{AlJazeeraOnVenezuela2017}{%
  pro-government workers sang and danced as they staged a rival march to show their support for the president's controversial plan to rewrite the constitution%
}.
Judging from these examples, the minimal common part is the cause,%
\label{CauseIsTheCommonDenominator} but there is also a location (or area) that varies over time.

For the rest of the paper, we will refer to the organizer as Alice.
Assume that the objective of Alice is to count everyone who participated at any time and in any of the locations~\cite{2016DemonstrationsInSeoul}\footnote{Note that the objective does not necessarily align with the definition of the police whose interest could be to determine the maximum crowd at any point in time, to deploy enough personnel for crowd-control.}.

\NewVariable{\cid}{cid}

Formally, we define a protest as an event that is uniquely identified by its 
cause (\(\cid\) below), its time interval (\(t\)) and its location (area \(l\)).
More specifically, we will use the following definition.

\begin{definition}[Protest]\label{DefProtest}
  A \emph{subprotest} is a tuple \((\cid, t, l)\) in which \(\cid\in 
    \ZZ_{2^\lambda}\) is the identifier of the cause of the protest,
  \(t = \interval{t_s}{t_e}\subseteq \RR\) is a time interval and \(l \subseteq 
    \RR^2\) is the location.
  A \emph{protest} is a set of subprotests sharing the same \(\cid\).
\end{definition}

The protests described in the previous examples can be captured using this definition by splitting 
them up into subprotests.
Each subprotest will then be captured by our definition and to estimate the total participation to the protest we can just sum up the estimates obtained.
Similarly for marches, the marching path can be divided into subprotests with 
locations (or areas) that slightly overlap.

\emph{Formalizing the notion of crowd estimation.} Each participant who wants to 
be counted must submit a \emph{participation proof}.
The proof must be associated with the protest (\ie its identifier \(cid\)), time and location must coincide with any of its subprotests.

We use other participants, that we called \emph{witnesses} to associate the 
proof to the location by creating a \emph{proof share}.
A witness is only allowed to create one proof share per protester to avoid the 
risk of inflation.
The participation proof and its proof shares are stored in a set, as there should be only unique proof shares.

\NewVariable{\pid}{pid}
\NewVariable{\wid}{wid}

\begin{definition}[Participation-proof share]
  A \emph{participation-proof share} \((\pid, \cid, t, l, \wid)\) is a tuple 
  where:
  \(\cid, t, l\) are as in \cref{DefProtest};
  \(\pid\) is a function of the protester's identity and \(\cid\); and
  \(\wid\) is a function of the witness' identity and 
  \(\pid\).
  We let \(W\) be the set of all proof shares.
\end{definition}
\sonja{do the same pair of people (witness-protester) get the
same wid on a different protest? The pid is unique for protester-cid
pair thus the id of a protester at a particular protest. Pid is thus
derived from the protester id (which has no term) and the protest id
(cid)? In analogy, wid is derived from pid (and thus cid) of both
participants, protesters i and j, say, where i witnesses j at cid and
simultaneously symmetrically j witnesses i?
Using wid and w is confusing since we changed the terminology to use
proof shares instead of the previous witness (as in testimony, not
person). wid sounds like it would be the id of the witness person,
pid sounds like it would be the id of the protester being witnessed,
we have defined neither and they might not be needed, but it is
confusing to read.}
\daniel{I made some minor changes, any better?}

A subset of the proof shares forms a participation proof for a protester.

\begin{definition}[Participation proof]
  A \emph{participation proof} of protester \(i\) participating in protest \(j\) 
  is the set
  \begin{multline}
    \nonumber
    p_{i, j} =
    \left\{ (\pid, \cid, t, l, \wid)\in W \mid \right. \\
    \left. \pid = i, \cid = j,
    t \subseteq t_j, l\subseteq l_j \right\},
  \end{multline}
  of all proof shares with the same protester and protest identifiers, witness 
  time interval \(t\) and location \(l\) is within those of the protest \(j\) 
  (\ie \(t_j\) and \(l_j\)).
\end{definition}

We can now define the participation count as follows.
\begin{definition}[Participation count]
  We define the \emph{participation count} of a protest \(\cid\) as the 
  cardinality \(|P_{\cid}^{s,\theta}|\) of the set of participation proofs \[
    P_{\cid}^{s,\theta} = \left\{ p_{i,j} \mid
      j = \cid, s(p_{i,j})\geq \theta \right\}
  \] for some strength function \(s\colon \powerset(W)\to \RR_+\) and a 
  threshold \(\theta\).
\end{definition}
The strength function \(s\) can be used to regulate trust in witnesses.
One example would be for \(s\) to return the number of proof shares (\ie witnesses) and thus let \(\theta\) to be the threshold of the number of required witnesses.
Another example would be to return the number of proof shares issued by trusted 
witnesses and set \(\theta = 1\) to require at least one proof share issued by a 
trusted witness.

\emph{System model and physical assumptions.}\label{assumptions}
Throughout this text, when we refer to a participant, say Alice, we actually 
mean an agent which can perform cryptographic operations and communicate with 
other local devices on Alice's behalf. We assume that every participant has a digital certificate signed by 
some logically centralized certificate authority.
\Eg that we can use the cryptographic keys of any national electronic identity 
system, identity card or passport.
We need this to prevent Sybil attacks~\cite{SybilAttack}.

In practical terms, participants witness each other's participation
using their smartphones (or similar devices) running the protocol described in
\cref{protocol} and uploading their testimony (\ie proof shares) to a blockchain 
after the protest. During the protest the devices are computationally limited by 
their batteries and need local connectivity to each other but no connection to
any global network such as the Internet is necessary. Before and
after the protest, we assume that the devices have global connectivity, \ie 
Internet  connections, and are not computationally limited by any battery.

\include*{MANETs}




%assumptions about smartphones:
%(This means that we essentially provide a lower bound for the participation 
%count, since some participants might not have such a device.)
%Sonja says: is this still true? Given mafia fraud we can't guarantee
%that, right?






\section{Building blocks}%
\label{Primitives}\label{BuildingBlocks}

\dots

\subsection{Zero-knowledge proofs}

Briefly describe zero-knowledge proofs and what type of properties can be proved 
efficiently using them, e.g.\ algebraic structures but not one-way functions.

\dots

\subsection{Location proofs and distance bounding}

Briefly describe location proofs and distance bounding protocols.
Describe the different desirable properties for distance bounding:
\begin{itemize}
  \item Mafia-fraud resistance,
  \item terrorist-fraud resistance,
  \item impersonation-fraud resistance,
  \item distance-fraud resistance.
\end{itemize}

\subsection{Storage}%
\label{StorageProperties}

We will use the storage to provide the temporal eligibility property.
\Cref{CreatedBeforeEnd,CreatedAfterStart} requires a \emph{partially ordered 
  set}\footnote{%
  A relation \(\preceq\) which is reflexive, antisymmetric and transitive.
} of objects.
If some objects in the set relate to known points in time, then the partial 
order relates the data to the time of the event.
This allows us to \emph{verify the data temporally}.
One primitive that fulfils these requirements is a blockchain.
There are also other structures, e.g.\ a directed graph that 
converges~\cite{BlockchainFreeCryptocurrencies}, that also provides the required 
properties.
However, we will use the blockchain in our discussion for simplicity.

There are five properties that we need from the storage system, all of which 
makes a blockchain-like structure suitable.
First, we need immutability.
I.e.\ once we commit something to the blockchain it will remain there.
We will use this to ensure verifiability, the data must remain to be verifiable.

Second, we need a general form of time-stamping.
I.e.\ we must be able to relate commitments to time, e.g.\ in 
Bitcoin~\cite{Bitcoin} a new block appears approximately every 10 minutes.
This coarse-grained form of time-stamping is enough for our purposes.
We will use this property to ensure that proofs are created before a certain 
point in time.

Third, we need an unpredictable number whose time of publication can be 
verified.
For this property, we will use the head of the blockchain.
The hash value of the head of the blockchain at a given time will be difficult 
to predict ahead of time.
%(See \cref{SecurityAnalysis} for a security analysis.)
We want to use this to ensure that proofs are created after a certain point in 
time.

Fourth, once we have committed some data, we do not need to keep the data 
itself, but we can store some confirmation value instead.
E.g.\ using blockchains, we can store the hash of the block where our data was 
committed, while this block remains we are sure that our data remains committed 
--- even if we no longer remember what our data looked like.
We will use this property to approximate the property of receipt freeness.

Finally, if the storage is decentralized we do not have to trust any central 
authority, which is the usual case for blockchains.




\subsubsection*{Acknowledgements}

This work was funded by the Swedish Foundation for Strategic Research grant SSF 
FFL09-0086 and the Swedish Research Council grant VR 2009-3793.


\printbibliography{}

\end{document}
