\subsection{Location proofs}
\label{location-proofs}
Some \acp{LBS} only grant access to resources to users located at a particular 
location, thus raising the issue of verifying the position claimed by a 
particular user.
One possible way to counter this threat is by having the requesting device 
formally prove that it really is at the claimed location, which gives rise to 
the concept of \acp{LP}.
In a nutshell, \iac{LP} is a digital certificate attesting that someone was at 
a particular location at a specific moment in time.
\Iac{LPS} is an architecture by which users can obtain \acp{LP} from 
neighboring witnesses (\eg trusted access points or other users) that can later 
be shown to verifiers who can check the validity of a particular 
proof~\cite{luo2010veriplace,zhu2011applaus}.
Most of the existing approaches to \acp{LP} require the prover and the 
witnesses to disclose their identities, thus raising many privacy issues such 
as the possibility of tracing the movements of users of the \ac{LPS}.
However, some \acp{LPS}, such as PROPS~\cite{PROPS}, exist that provide strong 
privacy guarantees along with the possibility of verifying the claim of the 
location.

%\CROCUS shares some similarities with PROPS, although their objective is quite 
%different as it aims at verifying a global property of the population (\ie 
%crowd estimation) in contrast to checking the location claim made by a user, 
%which is an individual property.
%
%Another difference is that \CROCUS operates in a more adverse environment.
%\CROCUS must provide \emph{universal verifiability}, this means that all proofs 
%must be available to and verifiable by anyone.
%One problem here is that we have multiple verifiers who might not trust the 
%same witnesses.
%The incentives to cheat are also bigger and consequently the thresholds for 
%collusion are much higher.
%
\sonja{add something on platin.io, details unknown but roughly relying on 
  witnesses and graph theory (unique big cluster, assumption of honest 
  majority)}
