%\subsection{Location proofs}
\section{Distance bounding and location proofs}
\label{db-and-lp}
\Ac{DB} protocols were first suggested by \citet{DistanceBounding} to prevent relay attacks in contactless communications in which the adversary forwards a communication between a prover and a possibly far-away verifier to authenticate. 
These attacks cannot be prevented by cryptographic means as they are independent of the semantics of the messages exchanged.
As a consequence, mechanisms ensuring the physical proximity between a verifier and a prover should be used instead.
\Ac{DB} protocols enable the verifier to estimate an upper bound on their distance to the prover by measuring the time-of-flight of short challenge-response messages (or rounds) exchanged during time-critical phases. 
%Time critical phases are complemented by slow phases during which the time is not taking into account. 
Slow phases, during which the time is not taken into account, complement the time critical phases.
At the end of a \Ac{DB} protocol, the verifier should be able to determine if the prover is legitimate \emph{and} in their vicinity.
In this sense, \Ac{DB} protocols combine the classical properties of authentication protocols with the possibility of verifying the physical proximity.

The main attacks against \ac{DB} protocols can be summarized as follows:
\begin{itemize}
  \item \Acf{DBDF}: a legitimate but malicious prover wants to fool the verifier on the distance between them.
  \item \Acf{DBMF}: the adversary illegitimately authenticates, possibly using an honest prover far away from the verifier.
  \item \Acf{DBTF}: a legitimate but malicious prover helps an accomplice close to the verifier to authenticate.
  \item \Acf{DBDH}: similar to \ac{DBDF}, the malicious prover is far away but uses an unsuspecting honest prover close to the verifier to pass as being close.
\end{itemize}
There are two lines of attempts at formalizing the above properties: one by \citet{DB-BMV} and another by \citet{DB-DFKO}.

The majority of the existing \ac{DB} protocols are symmetric and thus require an honest verifier.
Indeed, in this context it does not make sense to protect against the verifier as they can easily impersonate the prover as they have knowledge of their secret key.
There has been less work done in the domain of asymmetric (or public-key) \ac{DB} protocols.

Some location-based services only grant access to resources to users located at a particular location, thus raising the issue of verifying the position claimed by a particular user. 
In most of the existing schemes, the location of a user (or their device) is determined by the device itself (\eg through GPS) and forwarded to the location-based service provider. 
One of the main drawbacks of this approach is that a user can cheat by having their device transmit a false location. 
Therefore, it is possible for a user to be inappropriately granted access to a particular resource while being thousands of kilometers away.

One possible way to counter this threat is by having the requesting device formally prove that it really is at the claimed location, which gives rise to the concept of \emph{location proof}. 
In a nutshell, a location proof is a digital certificate attesting that someone was at a particular location at a specific moment in time. 
A location proof system is an architecture by which users can obtain location proofs from neighboring witnesses (\eg trusted access points or other users) that can later be shown to verifiers who can check the validity of a particular proof \cite{luo2010veriplace, zhu2011applaus}.
Most of the existing approaches to location proofs require the prover and the witnesses to disclose their identities, thus raising many privacy issues such as the possibility of tracing the movements of users of the location-proof architecture.
However, some location proofs systems, such as PROPS~\cite{PROPS}, exist that provide strong privacy guarantees along with the possibility of verifying the claim of the location.
\CROCUS shares some similarities with PROPS, although their objective is quite different as it aims at verifying a global property of the population (\ie crowd estimation) in contrast to checking the location claim made by a user, which is an individual property.
\sonja{any more differences to ours, relevance of this one?}
\simon{Yeah, there is a problem here. As it currently reads, I see no reason to not use PROPS directly as a building block for CROCUS, instead of devising our own distance-bounding. Maybe there is no authentication, it's ONLY a proof of location? I don't know, but some detail is missing.}

\sonja{add something on platin.io, details unknown but roughly relying on witnesses and graph theory (unique big cluster, assumption of honest majority)}
