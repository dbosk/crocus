\subsection{Outline}

The paper is organized as follows.
In \cref{definitions}, we describe our system model and formalize the notion of 
protest and the desired verifiability and privacy properties.
In \cref{building-blocks}, we give the relevant background on the building 
blocks of our solution.
%
%\Cref{DB-anon-cred}, we present a new \ac{DB} version of the Schnorr protocol, 
%different from that originally proposed by \textcite{DistanceBounding}.
%We provide a proof of its security and formally verify it using 
%Tamarin~\cite{TamarinDB}.
%
We present \CROCUS, a privacy-preserving crowd counting 
%estimation 
protocol in in \Cref{Protocol}, analyze its security in
\cref{SecurityAnalysis}, and discuss practical considerations including performance in \cref{Practical}.
%We compare it to related work in \cref{related-work}.
%Finally, we discuss limitations and assumptions in \cref{Discussion}
Finally, we state
our conclusions in \cref{Conclusion}.
