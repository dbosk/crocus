\section{Security and privacy analysis}%
\label{SecurityAnalysis}

\subsection{Individual and universal verifiability}%
\label{analysis-individual}%
\label{analysis-universal}

\Cref{IndividualVerif} requires that Alice and Bob, as participants, can verify 
that their participation proofs (proof shares) are indeed included.
As all proof shares (\ie \(\cid, \pid, \wid, t_s, t_s', t_e, t_e', l, \pi_\pid, 
  \pi_{\wid}\)) are committed to the blockchain and available from public 
storage (see \cref{fig:ProtocolOverview}).
Alice and Bob can simply check that all of their proof shares are indeed there 
and thus the security of individual verifiability depends on the properties of 
the blockchain and storage.

\Cref{UniversalVerif} requires that also anyone can verify the result and that 
all participation proofs counted indeed are eligible.
As the proof shares are committed and stored publicly, anyone can download 
them, verify eligibility (\ie verify \(\pi_\pid, \pi_{\wid}\)) the proofs and 
count them.
As for individual verifiability, the security of universal verifiability is 
reduced to the properties of the blockchain and storage but also the eligibility 
verification, which we will discuss next.

\subsection{Eligibility verifiability}%
\label{analysis-eligibility}

\Cref{EligibilityVerif} states that anyone must be able to determine the 
authenticity of the relevant attributes of the data.
We have several attributes that must be verifiable:
time of creation (temporal eligibility),
physical location of \(\sk_P\) at creation (spatial eligibility),
recognize two proofs from the same person (one-proof-per-person eligibility) and 
that the proof is indeed designated for the event (designated-event 
eligibility).
We will now analyse these.

\paragraph{Temporal eligibility}%
\label{analysis-temporal}

\Cref{TemporallyRelated} ensures freshness, \ie that Alice cannot simply 
resubmit an old proof as a new one.
In general, freshness requires the prover to respond to an unpredictable 
challenge.
The challenge here is to include the hash value at the head of the blockchain at 
the time of the proof's creation (included as \(t_s\) and \(t_s'\) in the proof 
share).
As per construction of blockchains, the hash value at the head depends on the 
previous blocks in the chain.
Thus the predictability of \(t_s\) depends on the predictability of the blocks, 
which depends on the predictability of the transactions and the nonce needed by 
the miner.
The transactions in turn depend on the signatures made by the signing keys 
corresponding to the used transactions.
If these were predictable, our adversary could spend arbitrary transactions in 
the blockchain --- which is designed to be hard.
Thus Alice can predict \(t_s\) with negligible probability, \ie \(t_s\) must 
have been published on the blockchain before being included in her proof share.

According to \cref{TemporallyRelated}, we must also prove that a proof share has 
not been created after a certain time.
Otherwise Grace can argue that the proof share was created after the protest.
The hash values of the proof shares are committed to the blockchain, thus there 
is a negligible probability that they were created after that.
Alice would have to choose a value \(y\) in the domain of the hash function 
\(\Hash\) and then find a preimage \(x\) such that \(y = \Hash[x]\) and \(x\) is 
a valid proof.
If \(\Hash\) is collision resistant, then finding \emph{any} preimage is hard.

\paragraph{Linkability}%
\label{analysis-linkability}

\Cref{CountOnce} is required to prevent Sybil attacks, \ie that Alice can 
provide two participation proofs and thus be counted twice.
This is prevented by the use of \(\pid\).
To be counted twice, Alice must produce a \(\pid'\neq \pid\).
Due to the deterministic property of \(\ACprf\), Alice must produce a new key 
\(\sk_P'\) and make the witness accept in the protocol
\begin{multline*}
\PK\mleft\{ (\sk_P') : \pid' = \ACprf[_{\sk_P'}][\cid] \quad \land \mright. \\
    \sigma_P'' = \mleft. \ACblind[\ACsign[_{\ssk}][\sk_P']] \mright\}
\end{multline*}
while she does not know a valid signature on \(\sk_P'\).
This is thus reduced to the security of the \(\AC\) scheme and how often Alice 
can get a valid signature on a secret key from the \ac{CA}.

\paragraph{Spatial eligibility}%
\label{analysis-spatial}

\Cref{SpatiallyRelated} is achieved by having a witness vouch that Alice was 
indeed on the location when the proof share was created.
Thus we must ensure that Alice cannot forge undeserved witness signatures, which 
is ensured using the same linkability mechanism as above.
To forge a witness signature, Alice must produce a \(\wid'\) such that
\begin{multline*}
  \pi_{\wid'}\gets \SPK\left\{ (\sk_W') : \right. \\
    \begin{aligned}
      \wid' &= \ACprf[_{\sk_W'}][\pid] \quad \land \\
      \sigma_W'' &= \left. \ACblind[\ACsign[_{\ssk}][\sk_W']] \right\}
    \end{aligned} \\
      (\pid, \wid', t_s, t_s', l),
\end{multline*}
while not knowing a signature on the secret key \(\sk_W'\).
Thus she must break the \(\AC\) scheme to succeed or get a valid signature on 
\(\sk_W'\) from the \ac{CA}.
(Note that Alice can produce one witness signature for herself, since she has 
access to her own \(\sk_P\), but no more.)

\paragraph{Designated protest}%
\label{analysis-designated}

\Cref{DesignatedEvent} is to prevent Alice from reusing the same proof (or proof 
share) for another event --- or someone else using her proof-share for another 
event.
This is prevented with the use of \(\cid\) in the proof shares.
Thus to reuse the proof share for another protest, with a different manifesto, 
one must find a second preimage \(\mfst'\) such that \(\cid = \Hash[\mfst] = 
  \Hash[\mfst']\).

There is also another case of collision that we must prevent.
Consider that Alice computes \(\pid = \ACprf[_\sk][\cid]\) for some cause 
identifier \(\cid\) and some witnesses computes \(\wid_1, \dotsc, \wid_n\), 
with \(\wid_i = \ACprf[_{\sk_i}][\pid]\).
Now, if Alice would construct a manifesto \(m\) such that \(\Hash[m] = \pid\), 
then \(\wid_1, \dotsc, \wid_n\) would be valid participant identifiers for the 
protest with manifesto \(m\).
This means that we must separate the mechanisms used for participating and 
witnessing.
The protocol achieves this by the fact that \[
  \prf_{\wid} = \SPK[\sk_i][\dotsc][\pid, \wid, t_s, t_s', l]
\]
does not include \(\cid\) whereas \[
  \prf_{\pid} = \SPK[\sk_i][\dotsc][\cid, \pid, \wid, t_s, t_s', l]
\]
does.
Thus the verification process differs for the two types of proofs.
(An alternative approach would have been to compute them differently, \eg 
\(\pid = \AC[PRF]_{\sk}(x) = g_T^{\sk+x}\) but \(\wid = g_T^{\sk+x+1}\).)

%\paragraph{Forging proof shares}
%
%\daniel{This is work-in-progress.}
%Alice can attempt to forge the full proof directly, \ie
%\begin{multline*}
%\pi_{\pid'} = \SPK\mleft\{ (\sk_P') : \mright. \\
%  \begin{aligned}
%    \pid' &= \ACprf[_{\sk_P'}][\cid], \\
%    \sigma_P'' &= \mleft. \ACblind[\ACsign[_{\ssk}][\sk_P']] \mright\}
%  \end{aligned} \\
%    (\cid, \pid', \wid', t_s, t_s', l)
%\end{multline*}
%and consequently
%and thus solve same problem twice (once for \(pid'\) and once for \(\wid'\)).
%

\subsection{Privacy}

We start with \cref{ProtestUnlink}, that Alice's proofs must be unlinkable 
across protests (but linkable within one protest due to \cref{CountOnce}).
It follows from the properties of \(\ACprf\) that \(\pid = 
  \ACprf[_{\sk_P}][\cid]\) and \(\pid' = \ACprf[_{\sk_P}][\cid']\) are 
unlinkable if \(\cid\neq \cid'\).

Now to \cref{ProofUnlink}.
Given \(\pid\), Grace cannot distinguish whether \(\pid = 
  \ACprf[_{\sk_A}][\cid]\), \(\pid = \ACprf[_{\sk_A}][\cid']\) or \(\pid = 
  \ACprf[_{\sk_B}][\cid]\) due to the properties of \(\AC\).

However, say that a witness works for Grace.
This witness interacts with Alice directly, \ie has (physically) identified her.
The witness also learns Alice's \(\pid\), but not \(\cid\), to compute \(\wid\) 
and to verify the \ac{DBPK}.
\daniel{This assumes that I will do the changes to hide \(\cid\) from the 
  witness in the \ac{DBPK}.}
Afterwards, Alice publishes \(\pid\) and \(\cid\) together as part of the proof 
share.
The probability of a colliding \(\pid = \ACprf[_\sk][\cid']\) for some \(\cid' 
  \neq \cid\) and \(\sk \neq \sk_P\) is \enquote{small}.
\daniel{Fix this.}
Now Grace can link \(\cid\) (and thus the opinions in the manifesto) to Alice.
