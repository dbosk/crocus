\section{Security and privacy analysis}%
\label{SecurityAnalysis}

\subsection{Individual and universal verifiability}%
\label{analysis-individual}%
\label{analysis-universal}

\Cref{IndividualVerif} requires that Alice and Bob, as participants, can verify 
that their participation proofs (proof shares) are indeed included.
As all proof shares (\ie \(\cid, \pid, \wid, t_s, t_s', t_e, t_e', l, \wsig, 
  \pi_\pid, \pi_{\wid, \wsig}\)) are committed to the blockchain and available 
from public storage (see \cref{fig:ProtocolOverview}).
Alice and Bob can simply check that all of their proof shares are indeed there 
and thus the security of individual verifiability depends on the properties of 
the blockchain and storage.

\Cref{UniversalVerif} requires that also anyone can verify the result and that 
all participation proofs counted indeed are eligible.
As the proof shares are committed and stored publicly, anyone can download them, 
verify eligibility (\ie verify \(\pi_\pid, \pi_{\wid, \wsig}\)) the proofs and 
count them.
As for individual verifiability, the security of universal verifiability is 
reduced to the properties of the blockchain and storage but also the eligibility 
verification, which we will discuss next.

\subsection{Eligibility verifiability}%
\label{analysis-eligibility}

\Cref{EligibilityVerif} states that anyone must be able to determine the 
authenticity of the relevant attributes of the data.
We have several attributes that must be verifiable:
time of creation (temporal eligibility),
physical location of \(\sk_P\) at creation (spatial eligibility),
recognize two proofs from the same person (one-proof-per-person eligibility) and 
that the proof is indeed designated for the event (designated-event 
eligibility).
We will now analyse these.

\paragraph{Temporal eligibility, created after start}

\daniel{Rewrite this paragraph, do proper analysis.}
Given the properties of the storage (in \cref{StorageProperties}), for 
\cref{CreatedAfterStart}, each witness must include the hash value of an 
existing block in the blockchain in each proof share (shown as \(t_s\) in 
\cref{fig:ProofFig}).
Then it is clear that the proof must have been created after that block (since 
predicting such a value is hard).
This value can either be provided to the witness by the protester or the witness 
uses the value of the last known block.

\paragraph{Location proofs and spatial eligibility}

\daniel{Discuss the two alternative approaches: trusted Jane and collusion 
  threshold.
  Do actual security analysis.}
According to \cref{SpatiallyRelated} Alice and Bob must bind their proofs 
(shares) spatially to the location, which would allow us to verify that they 
were actually in that location.
The role of the witnesses is to provide this property.

\paragraph{Temporal eligibility, created before end}

The proofs are committed to the blockchain, thus they cannot have been created 
after.
\daniel{Do a proper security analysis.}

\paragraph{Linkability and designated protest}

\Cref{CountOnce} is required to prevent Sybil attacks, \ie that one individual 
can provide two participation proofs and thus be counted twice.
This is prevented by the use of \(\pid\).
To be counted twice, Alice must produce a \(\pid'\neq \pid\) and two signatures:
\begin{multline*}
\pi_{\pid'} = \SPK\mleft\{ (\sk_P') : \mright. \\
  \begin{aligned}
    \pid' &= \ACprf[_{\sk_P'}][\cid], \\
    \sigma_P'' &= \mleft. \ACblind[\ACsign[_{\ssk}][\sk_P']] \mright\}
  \end{aligned} \\
    (\cid, \pid', \wid', t_s, t_s', l, \wsig')
\end{multline*}
and
\begin{multline*}
  \pi_{\wid', \wsig'}\gets \SPK\left\{ (\sk_W') : \right. \\
    \begin{aligned}
      \wid' &= \ACprf[_{\sk_W'}][\pid'] \quad \land \\
      \wsig' &= \ACprf[_{\sk_W'}][\wid', t_s, t_s', l] \quad \land \\
      \sigma_W'' &= \left. \ACblind[\ACsign[_{\ssk}][\sk_W']] \right\}
    \end{aligned} \\
      (\pid', \wid', t_s, t_s', l, \wsig').
\end{multline*}

\daniel{There are several approaches for Alice to do this, \eg attacking the 
  \(\SPK\)-signature, the \(\AC\)-signature, \etc.
  I think we should describe them.}

\Cref{DesignatedEvent} is to prevent Alice from reusing the same proof (or proof 
share) for another event --- or someone else using her proof-share for another 
event.
This is done with the use of \(\cid\) in the proof shares.
Thus to reuse the proof share for another protest, with a different manifesto, 
one must find a second preimage \(\mfst'\) such that \(\cid = \Hash[\mfst] = 
  \Hash[\mfst']\).

