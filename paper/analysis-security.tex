\section{Security and privacy analysis}%
\label{SecurityAnalysis}

\subsection{Constructing the participation-proof shares}

\subsubsection{Linkability and designated protest}

\NewVariable{\sk}{sk}

\Cref{CountOnce} is required to prevent Sybil attacks, \ie that one individual 
can provide two participation proofs and thus be counted twice.
This is prevented by the use of \(\pid\).
To be counted twice, Alice must produce a \(\pid'\neq \pid\) and a signature
\begin{align*}
\pi_{\pid'} &= \SPK\mleft\{ (\sk') : \mright. \\
            & \qquad pid = \ACprf[_{\sk'}][cid], \\
            & \qquad \mleft. \sigma = \ACblind[\ACsign[_{ssk}][\sk']] \mright\} 
            \\
            & \qquad\qquad (\cid, \pid', \wid', t_s, t_s', l, wsig').
\end{align*}
\daniel{There are several approaches for Alice to do this, we should describe 
  them.}

\Cref{DesignatedEvent} is to prevent Alice from reusing the same proof (or proof 
share) for another event --- or someone else using her proof-share for another 
event.
This is done with the use of \(\cid\) in the proof shares.
Thus to reuse the proof share for another protest, with a different manifesto, 
one must find a second preimage \(\mfst'\) such that \(\cid = \Hash[\mfst] = 
  \Hash[\mfst']\).

\subsubsection{Temporal eligibility, created after start}

Given the properties of the storage (in \cref{StorageProperties}), for 
\cref{CreatedAfterStart}, each witness must include the hash value of an 
existing block in the blockchain in each proof share (shown as \(t_s\) in 
\cref{fig:ProofFig}).
Then it is clear that the proof must have been created after that block (since 
predicting such a value is hard).
This value can either be provided to the witness by the protester or the witness 
uses the value of the last known block.

\subsubsection{Location proofs and spatial eligibility}

According to \cref{SpatiallyRelated} Alice and Bob must bind their proofs 
(shares) spatially to the location, which would allow us to verify that they 
were actually in that location.
The role of the witnesses, \eg Jane, is to provide this property.
Jane can do this by using a distance-bounding protocol to ensure that the secret 
associated with Alice or Bob must be within limited physical distance.

Alice will send her \(pid\) to Jane to start the witness protocol.
Then they do the distance bounding \dots
[See issue \#22 for the discussion.]
If Jane is convinced that Alice's secret is within the desired proximity, she 
will compute her witness identifier \(wid\) and witness signature \(wsig\) and 
send them to Alice.

\subsubsection{Temporal eligibility, created before end}

The proofs (shares) can be submitted to storage in two steps:
first, committing them to the blockchain; second, to submit all the required 
data.
This will allow some time for computing all the required \ac{NIZK} proofs 
(discussed in the following section).

Given the properties of the storage system in \cref{StorageProperties}, 
\cref{CreatedBeforeEnd} is straight-forward: once a proof share is created it 
should be committed to the blockchain as soon as possible.
Thus we can infer that a proof (or proof share) was created at the latest when 
it was committed.

We note that there are at least two entities who know each proof share, the 
protester and the witness.
Alice and Bob simply commit their proofs (\ie all proof shares of their 
proofs) to the blockchain as soon as possible after they have participated in 
the protest.
The witnesses do the same, they commit all the proof shares that they helped 
create.
(Shown in \cref{fig:ProtocolOverview}.)

\subsection{Verifying the participation count}

There are three parts to verifying the participation count.
First, we must verify the eligibility of each proof, \ie of all its shares.
Second, we must count these valid proofs.
Third, each protester must check that their proof is indeed registered so that 
it can be counted.

\subsubsection{Eligibility and universal verifiability}

Universal verifiability (\cref{UniversalVerif}) means that anyone should be able 
to count the valid proofs.
To do this we must be able to verify the eligibility of a proof (\ie of all 
its shares) and then count it if valid.
To verify the eligibility of a proof share, we must verify that \(pid\), \(wid\) 
and \(wsig\) are not just randomly generated numbers but they actually depend on 
the keys \(k_P\) and \(k_W\) (as illustrated in \cref{fig:ProofFig}).
We will do this with \ac{NIZK} proofs.
We note that these proofs are not needed until the verification step after the 
protest.
Thus \(pid, wid, wsig\) can be used in the proof shares (and all computations 
above) whereas their respective \ac{NIZK} proof can be computed and published 
after the protest.

Alice must provide \iac{NIZK} proof showing that \(pid = \ACprf[_{k_P}][cid]\) 
and that she knows a signature by the identity authority on \(k_P\).
\dots
[Details of how to do this, or maybe that will be in \cref{BuildingBlocks}. See 
issue \#27 for discussion.]

In the same manner, each witness must also provide \iac{NIZK} proofs for \(wid = 
  \ACprf[_{k_W}][pid]\) and \(wsig = \ACprf[_{k_W}][wid, t_s, l]\) as well as 
that the witness knows a signature by the identity authority on \(k_W\).
With these \ac{NIZK} proofs anyone can verify the validity, \ie the 
eligibility, of the proof shares and thus also count them.

\subsubsection{Individual verifiability and receipt freeness}

The purpose of individual verifiability (\cref{IndividualVerif}) is to prevent 
an adversary from dropping participation proofs.
Thus each protester must verify that their own proof is indeed included.
This can be accomplished as follows.
After Alice commited to her proof (share) in the blockchain, she stores the hash 
of the block.
At a later time, she can still verify that the block is indeed still there.
At this point, this hash value is the only thing she must store for individual 
verifiability.

More specifically, to achieve receipt freeness (\cref{ReceiptFreeness}) Alice 
must remove her secret \ac{PRF} key, \(k_A\).
With this key Grace can verify that Alice has submitted a proof by computing 
\(pid'\gets \ACprf[_{k_A}][cid]\) and compare \(pid = pid'\).


\subsection{Old}

We note that Alice and Bob do not trust each other, for all they know one or the 
other could be an agent of Grace the autocrat.
\begin{itemize}
  \item Grace the evil autocrat,
  \item Mallory the foreign nation state, and finally
\end{itemize}


\subsection{Forging new proofs}

It follows from the digital signatures used that proofs are difficult to forge.

\subsection{Reusing participation proofs}

We consider two ways of reusing participation proofs:
\begin{enumerate}
  \item Alice and Bob can reuse old proofs by modifying the identifiers.
  \item Grace can reuse proofs by creating a manifesto which yields the same 
    identifier as Alice's protest, then try to create new nodes in the 
    \ac{tposet} that matches the ones used in the old proofs.
\end{enumerate}
We will now estimate the difficulty of both types of attacks.

\subsubsection{Finding second preimages}

Grace (possibly with Mallory's aid) wants to arrange a counter-protest.
One way to get a higher participation count is to use the same identifier as 
Alice's protest.
Since the identifier for the protest is the hash value of the manifest, Grace 
must find a second preimage for the hash function.

% XXX Rephrase identifier problem as a game
\begin{proposition}
  Given \(x\) and \(H(x) = y\), the probability of finding an \(x'\) such that 
  \(H(x') = y\) is \(\frac{1}{\lambda-1}\).
\end{proposition}

\begin{question}
  What if we use a Merkle tree instead of only a hash, will that change the 
  second preimage property?
\end{question}

Alice wants to arrange another protest against Grace.
Her manifesto can remain the same, thus the protest identifier remains the 
same.
She wants to reuse some old proofs.
The head value that are included in the proofs extend them too far back into 
time.
The digital signatures prevents her from changing them.
So an alternative is to create nodes in the \ac{tposet} which has the same 
hashes.

\subsection{Spoofing locations}

\subsubsection{Trust propagation for witnesses}

\dots

\subsection{Individual verifiability}

Grace and Mallory have an incentive to lower the confidence in the result.
One way they can do this is to say that the result does not contain all 
participation proofs.
Statements such as this should be proved: provide a valid but missing proof.

