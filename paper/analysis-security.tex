\section{Security and privacy analysis}%
\label{SecurityAnalysis}

\sonja{including adversary model with different levels here for now as
  it already contains some security analysis}
\daniel{moved it to the top}
\subsection{System, communication and adversary model}%
\label{adversary-model}

We provide one system model with two possible adversaries: one who tries to 
link a protester's real identity (\(P\)) to a protest cause (\(\cid\)) and one 
which tries to link a protester's or a witness's real identity (\(P\) 
respectively \(W\)) to a location (\(l\)).
We also define varying strengths of each adversary.

\paragraph{The identity--cause linking adversary}

There are three players: the protest participant (with identity) \(P\), a 
witness (with identity) \(W\) and the storage \(S\).
The adversary \(A\) controls \(W\) and \(S\).

\begin{definition}[Base identity--cause adversary]%
  \label{base-identity-cause-adversary}
  The protester \(P\) and the witness \(W\) communicate.
  Each learn only the protocol data \(d_{P,W}(\cid, P)\) and when it happened 
  \(t_{P,W}\)\footnote{%
    Specifically, they do \emph{not} learn the real identities \(P\) and \(W\) 
    directly, only if those appear in the data \(d_{P,W}(\cid, P)\).
  }.
  The protester \(P\) communicates with \(S\), where \(S\) only learns 
  \(f(d_{P,W}(\cid, P))\), for some function \(f\), and the time of the 
  communication (\(t_{P,S}\)) but not the real identities.
  The adversary controls \(W\) and \(S\) and thus learns everything that they 
  do, but can additionally correlate what he learns from \(W\) and \(S\).
  The adversary wins if he can link \(\cid\) and \(P\).
\end{definition}

This definition is illustrated in \cref{fig:identity-cause-adversary}.

\begin{figure}
  \centering
  \includegraphics{base-identity-cause-adversary.tikz}
  \caption{\label{fig:identity-cause-adversary}%
    An overview of the base identity--cause adversary model.
    The protester with real identity \(P\) and witness with real identity \(W\) 
    communicate and each learn only the protocol data, \(d_{P,W}(\cid, P)\), 
    and the time it happened, \(t_{P,W}\).
    The protester submits \(f(d_{P,W}(\cid, P))\), for some function \(f\), to 
    the storage \(S\), who learns only that and the time it happened, 
    \(t_{P,S}\).
    Both the witness \(W\) and storage \(S\) are controlled by the adversary 
    \(A\).
  }
\end{figure}

We find \cref{base-identity-cause-adversary} suitable when the protester and 
witness both move in a crowd and there is no way for the witness to decide 
exactly with whom he or she communicates with.
However, in some situations this might not be the case, \eg the crowd is not 
dense.
In these situations the witness will likely see the face of the protester.
We capture this by the following definition.

\begin{definition}[Stonger identity--cause adversary]%
  \label{stronger-identity-cause-adversary}
  The situation is the same as in \cref{base-identity-cause-adversary}, but now the 
  witness \(W\) learns the protester \(P\)'s identity.
  (\(P\) will also learn \(W\)'s identity.)
  However, \(S\) still does not learn \(P\)'s identity.
\end{definition}

One reason to not allow \(S\) to learn \(P\)'s identity is that for this 
communication \(P\) has options, such as \ac{Tor}, for anonymous communication.
However, given a strong enough adversary, such anonymous communication might 
not be possible.
We capture such a strong adversary in the following definition.

\begin{definition}[Strongest identity--cause adversary]%
  \label{strongest-identity-cause-adversary}
  Everything is the same as in \cref{stronger-identity-cause-adversary}, except 
  that now \(S\) also learns \(P\)'s identity.
\end{definition}

\paragraph{The identity--location adversary}

The players are still the same, but now the adversary controls other players: 
in particular, the adversary controls only the storage \(S\).

\begin{definition}[Base identity--location adversary]%
  \label{base-identity-location-adversary}
  The adversary controls \(S\).
  The witness \(W\) and the protester \(P\) communicate.
  They learn some protocol data \(d_{W,P}(W, l)\) and \(d_{W,P}(P,l)\) and the 
  time it took place, \(t_{P,W}\).
  \begin{description}
    \item[The witness case:]
      The witness later communicates with \(S\), where \(S\) only learns 
      \(f(d_{W,P}(W, l))\), for some function \(f\), and the time it happened, 
      \(t_{W,S}\).
      The adversary wins if he can link \(W\) and \(l\).
    \item[The protester case:]
      The protester later communicates with \(S\), where \(S\) only learns 
      \(g(d_{W,P}(P, l))\), for some function \(g\), and the time it happened, 
      \(t_{P,S}\).
      The adversary wins if he can link \(P\) and \(l\).
  \end{description}
\end{definition}

\Cref{fig:identity-location-adversary} illustrates the roles and relations.

\begin{figure}
  \centering
  \includegraphics{base-identity-location-adversary.tikz}
  \caption{\label{fig:identity-location-adversary}%
    An overview of the base identity--location adversary model.
    The protester with real identity \(P\) and witness with real identity \(W\) 
    communicate and each learn only the protocol data, \(d_{P,W}(W, l)\), and 
    the time it happened, \(t_{P,W}\).
    The witness submits \(f(d_{P,W}(W))\), for some function \(f\), and the 
    protester submits \(g(d_{P,W}(P,l))\), for some function \(g\), to the 
    storage \(S\), who learns only that and the time it happened, \(t_{W,S}\) 
    respectively \(t_{P,S}\).
    Only the storage \(S\) is controlled by the adversary \(A\).
  }
\end{figure}

The base rationale is the same as for \cref{base-identity-cause-adversary}, the 
witness \(W\) can submit to \(S\) over an anonymous channel, such as \ac{Tor}.
Anyone who is in the same location will learn that \(W\) was present, we want 
to prevent anyone \emph{not present} to learn that \(W\) was there --- hence we 
do not consider the protester as part of the adversary, only \(S\).
Analogous to the identity--cause adversary, we conclude with the strongest 
identity--location adversary.

\begin{definition}[Strongest identity--location adversary]%
  \label{strongest-identity-location-adversary}
  Things are the same as in \cref{base-identity-location-adversary}, except 
  that \(S\) learns the witness' and protester's identities \(W\) and \(P\), in 
  the respective cases.
\end{definition}


\subsection{Individual and universal verifiability}%
\label{analysis-individual}%
\label{analysis-universal}

\Cref{IndividualVerif} requires that Alice and Bob, as participants, can verify that their participation proofs (\ie proof shares) are indeed included in the count computed.
All proof shares (\ie \(\cid, \pid, \wid, t_s, t_s', t_e, t_e', l, \pi_\pid, \pi_{\wid}\)) are committed to the blockchain and available from a public and permanent storage (see \cref{fig:ProtocolOverview}).
Thus, Alice and Bob can simply check that all of their proof shares are indeed there and the security of individual verifiability depends on the properties of the blockchain and storage.

\Cref{UniversalVerif}  includes that anyone can check the result and that all participation proofs counted are legitimate.
As the proof shares are committed and stored publicly, anyone can
download them, verify eligibility (\ie verify \(\pi_\pid,
\pi_{\wid}\)) of the proofs and count them.
As for individual verifiability, the security of universal verifiability is reduced to the properties of the blockchain and storage but also the eligibility verification, which we will discuss next.

\subsection{Eligibility verifiability}%
\label{analysis-eligibility}

\Cref{EligibilityVerif} states that anyone must be able to determine the authenticity of the relevant attributes of the data.
In \PRIVO, we have several attributes that must be verifiable: the time of creation (\ie, temporal eligibility), the physical location of \(\sk_P\) at creation (\ie, spatial eligibility), recognition of two proofs originating from the same person (\ie, one-proof-per-person eligibility) and that the proof is indeed designated for the event (\ie, designated-event eligibility).
We will now analyze these properties.

\paragraph{Temporal eligibility}%
\label{analysis-temporal}

\Cref{TemporallyRelated} ensures freshness, as Alice cannot simply resubmit an old proof as a new one.
In general, freshness requires the prover to respond to an unpredictable challenge.
The challenge here is to include the hash value at the head of the blockchain at the time of the proof's creation (included as \(t_s\) and \(t_s'\) in the proof share).
As it is the case in the construction of blockchains, the hash value at the head depends on the previous blocks in the chain.
Thus, the predictability of \(t_s\) depends on the predictability of the blocks, which depends on the predictability of the transactions and the nonce needed by the miner.
In turn, the transactions depend on the signatures made by the signing keys corresponding to the used transactions.
If these were predictable, the adversary could conduct arbitrary transactions in the blockchain --- which is assumed to be hard by design.
Thus, Alice can predict \(t_s\) only with negligible probability, as \(t_s\) must have been published on the blockchain before being included in her proof share.

According to \cref{TemporallyRelated}, we must also prove that a proof share has not been created after a certain time.
Otherwise, Grace could argue that the proof share was created after the protest, thus defeating the purpose of our protocol.
The hash values of the proof shares are committed to the blockchain, which means that there is a negligible probability that they were created after that.
Alice would have to choose a value \(y\) in the domain of the hash function \(\Hash\) and then find a pre-image \(x\) such that \(y = \Hash[x]\) and \(x\) is a valid proof.
If \(\Hash\) is collision resistant, finding \emph{any} preimage is hard.

\paragraph{Linkability}%
\label{analysis-linkability}

\Cref{CountOnce} is required to prevent Sybil attacks, in the sense
that Alice cannot provide two (or more) participation proofs for a
specific protest and thus be counted more than once.
This possibility is prevented by the use of \(\pid\).
Indeed to be counted twice, Alice must produce a \(\pid'\neq \pid\).
Due to the deterministic property of \(\ACprf\), Alice must produce a new key \(\sk_P'\) and make the witness accept in the protocol
\begin{multline*}
\PK\mleft\{ (\sk_P') : \pid' = \ACprf[_{\sk_P'}][\cid] \quad \land \mright. \\
    \sigma_P'' = \mleft. \ACblind[\ACsign[_{\ssk}][\sk_P']] \mright\}
\end{multline*}
while she does not know a valid signature on \(\sk_P'\).
As a consequence, this is reduced to the security of the \(\AC\) scheme and how often Alice can get a valid signature on a secret key from the \ac{CA}.

\paragraph{Spatial eligibility}%
\label{analysis-spatial}

\Cref{SpatiallyRelated} is achieved by having a witness vouch anonymously that Alice was indeed on the location when the proof share was created.
To realize this, we have to ensure that Alice cannot forge witness signatures, which is realized by using the same linkability mechanism as above.
To forge a witness signature, Alice must produce a \(\wid'\) such that
\begin{multline*}
  \pi_{\wid'}\gets \SPK\left\{ (\sk_W') : \right. \\
    \begin{aligned}
      \wid' &= \ACprf[_{\sk_W'}][\pid] \quad \land \\
      \sigma_W'' &= \left. \ACblind[\ACsign[_{\ssk}][\sk_W']] \right\}
    \end{aligned} \\
      (\pid, \wid', t_s, t_s', l),
\end{multline*}
while not knowing a signature on the secret key \(\sk_W'\).
Thus, she must break the \(\AC\) scheme to succeed or get a valid signature on \(\sk_W'\) from the \ac{CA}.
The latter could be done if she can find people willing to collude.
However, due to the assumptions made in \cref{ProtocolVerification}, we either assume that we can trust (dedicated) witnesses or that Alice cannot collude with more than a threshold \(\theta\) of witnesses (of course these two approaches could be combined).
Note that Alice can produce one witness signature for herself, since she has access to her own \(\sk_P\), but no more.

\paragraph{Designated protest}%
\label{analysis-designated}

\Cref{DesignatedEvent} is to prevent Alice (or someone else) from reusing the same proof (or proof share) for another event.
This possibility is prevented through the use of \(\cid\) in the proof shares.
To reuse the proof share for another protest, with a different manifesto, one must find a second preimage \(\mfst'\) such that \(\cid = \Hash[\mfst] = \Hash[\mfst']\).

There exists another case of collision that we must prevent.
Consider the situation in which Alice computes \(\pid = \ACprf[_\sk][\cid]\) for some cause identifier \(\cid\) and some witnesses computes \(\wid_1, \dotsc, \wid_n\), with \(\wid_i = \ACprf[_{\sk_i}][\pid]\).
Now, if Alice constructs a manifesto \(m\) such that \(\Hash[m] = \pid\), then \(\wid_1, \dotsc, \wid_n\) would be valid participant identifiers for the protest with manifesto \(m\).
This means that we must separate the mechanisms used for participating and witnessing.
The protocol achieves this by the fact that \[
  \prf_{\wid} = \SPK[\sk_i][\dotsc][\pid, \wid, t_s, t_s', l]
\]
does not include \(\cid\) whereas \[
  \prf_{\pid} = \SPK[\sk_i][\dotsc][\cid, \pid, \wid, t_s, t_s', l]
\]
does.
Thus, the verification process differs for the two types of proofs.
(An alternative approach would have been to compute them differently, for instance \(\pid = \AC[PRF]_{\sk}(x) = g_T^{1/(\sk+x)}\) but \(\wid = g_T^{1/(\sk+x+1)}\).)

%\paragraph{Forging proof shares}
%
%\daniel{This is work-in-progress.}
%Alice can attempt to forge the full proof directly, \ie
%\begin{multline*}
%\pi_{\pid'} = \SPK\mleft\{ (\sk_P') : \mright. \\
%  \begin{aligned}
%    \pid' &= \ACprf[_{\sk_P'}][\cid], \\
%    \sigma_P'' &= \mleft. \ACblind[\ACsign[_{\ssk}][\sk_P']] \mright\}
%  \end{aligned} \\
%    (\cid, \pid', \wid', t_s, t_s', l)
%\end{multline*}
%and consequently
%and thus solve same problem twice (once for \(pid'\) and once for \(\wid'\)).
%

\subsection{Privacy}

We start with \cref{ProtestUnlink}, that Alice's proofs must be unlinkable across protests but linkable within one protest due to \cref{CountOnce}.
It follows from the properties of \(\ACprf\) that \(\pid =  \ACprf[_{\sk_P}][\cid]\) and \(\pid' = \ACprf[_{\sk_P}][\cid']\) are unlinkable if \(\cid\neq \cid'\).

Afterwards to get the property \cref{ProofUnlink}.
Given \(\pid\), Grace cannot distinguish whether \(\pid = \ACprf[_{\sk_A}][\cid]\), \(\pid = \ACprf[_{\sk_A}][\cid']\) or \(\pid = \ACprf[_{\sk_B}][\cid]\) due to the properties of \(\AC\).

However, consider the situation in which the witness works for Grace.
This witness interacts with Alice directly and might have (physically) identified her.
The witness also learns Alice's \(\pid\), but not \(\cid\), to compute \(\wid\) and to verify the \ac{PPK}.
\daniel{This assumes that I will do the changes to hide \(\cid\) from the witness in the \ac{PPK}.}
Afterwards, Alice publishes \(\pid\) and \(\cid\) together as part of the proof 
share.
The probability of a colliding \(\pid = \ACprf[_\sk][\cid']\) for some \(\cid' 
  \neq \cid\) and \(\sk \neq \sk_P\) is \enquote{small}.
\daniel{Fix this.}
If this occurs, Grace could link \(\cid\) (and thus the opinions in the manifesto) to Alice.

\sonja{anonymity and cause unlinkability missing}
