\section{Security and privacy analysis}%
\label{SecurityAnalysis}

\paragraph{Linkability and designated protest}

\Cref{CountOnce} is required to prevent Sybil attacks, \ie that one individual 
can provide two participation proofs and thus be counted twice.
This is prevented by the use of \(\pid\).
To be counted twice, Alice must produce a \(\pid'\neq \pid\) and a signature
\begin{multline*}
\pi_{\pid'} = \SPK\mleft\{ (\sk') : \mright. \\
  \begin{aligned}
    \pid' &= \ACprf[_{\sk'}][\cid], \\
    \sigma &= \mleft. \ACblind[\ACsign[_{\ssk}][\sk']] \mright\}
  \end{aligned} \\
    (\cid, \pid', \wid', t_s, t_s', l, \wsig').
\end{multline*}
\daniel{There are several approaches for Alice to do this, \eg attacking the 
  \(\SPK\)-signature, the \(\AC\)-signature, \etc.
  I think we should describe them.}

\Cref{DesignatedEvent} is to prevent Alice from reusing the same proof (or proof 
share) for another event --- or someone else using her proof-share for another 
event.
This is done with the use of \(\cid\) in the proof shares.
Thus to reuse the proof share for another protest, with a different manifesto, 
one must find a second preimage \(\mfst'\) such that \(\cid = \Hash[\mfst] = 
  \Hash[\mfst']\).

\paragraph{Temporal eligibility, created after start}

\daniel{Rewrite this paragraph, do proper analysis.}
Given the properties of the storage (in \cref{StorageProperties}), for 
\cref{CreatedAfterStart}, each witness must include the hash value of an 
existing block in the blockchain in each proof share (shown as \(t_s\) in 
\cref{fig:ProofFig}).
Then it is clear that the proof must have been created after that block (since 
predicting such a value is hard).
This value can either be provided to the witness by the protester or the witness 
uses the value of the last known block.

\paragraph{Location proofs and spatial eligibility}

\daniel{Discuss the two alternative approaches: trusted Jane and collusion 
  threshold.
  Do actual security analysis.}
According to \cref{SpatiallyRelated} Alice and Bob must bind their proofs 
(shares) spatially to the location, which would allow us to verify that they 
were actually in that location.
The role of the witnesses is to provide this property.

\paragraph{Temporal eligibility, created before end}

The proofs are committed to the blockchain, thus they cannot have been created 
after.
\daniel{Do a proper security analysis.}

\subsection{Verifying the participation count}

\daniel{Rewrite!}
There are three parts to verifying the participation count.
First, we must verify the eligibility of each proof, \ie of all its shares.
Second, we must count these valid proofs.
Third, each protester must check that their proof is indeed registered so that 
it can be counted.

\subsubsection{Eligibility and universal verifiability}

\daniel{Do actual security analysis.}
Universal verifiability (\cref{UniversalVerif}) means that anyone should be able 
to count the valid proofs.
To do this we must be able to verify the eligibility of a proof (\ie of all 
its shares) and then count it if valid.
To verify the eligibility of a proof share, we must verify that \(\pid\), \(\wid\) 
and \(\wsig\) are not just randomly generated numbers but they actually depend on 
the keys \(k_P\) and \(k_W\) (as illustrated in \cref{fig:ProofFig}).
We will do this with \ac{NIZK} proofs.

\subsubsection{Individual verifiability}

\daniel{Do actual security analysis.
  This is reduced to the security of the blockchain.}
The purpose of individual verifiability (\cref{IndividualVerif}) is to prevent 
an adversary from dropping participation proofs.
Thus each protester must verify that their own proof is indeed included.
After Alice commited to her proof (share) in the blockchain, she stores the hash 
of the block.
At a later time, she can still verify that the block is indeed still there.
At this point, this hash value is the only thing she must store for individual 
verifiability.

