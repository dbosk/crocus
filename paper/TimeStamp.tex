\subsection{Time-stamping service -- blockchains}%
\label{StorageProperties}


We need a time-stamping service, \(\TS\), with algorithms \(\TSget\), \(\TSstamp\) and \(\TStime\) such that
\begin{itemize}
  \item \(\rho \gets \TSget\) yields a value \(\rho\) at time \(t\), \(\rho\) is difficult to guess before time \(t\) and \(\TStime[\rho] = t\);
  \item \(\pi\gets \TSstamp[x]\) yields a value \(\pi\) at time \(t\) such that \(\TSverify[x, \pi]\to 1\) and \(\TStime[\pi] = t\).
\end{itemize}

With these building blocks we can ensure that a message \(m\) was created within the time interval \(\interval{t_0}{t_1}\).
After time \(t_0\), request \(\rho_{t_0}\gets \TSget\).
Before time \(t_1\), submit \(h\gets \Hash[m, \rho_{t_0}]\) to the time-stamping service to get \(\pi_{t_1}\gets \TSstamp[h]\).

Now we can use \((\rho_{t_0}, m, \pi_{t_1})\) to prove that \(m\) was created within the time interval \(\interval{t_0}{t_1}\).
The verifier computes \(h'\gets \Hash[m, \rho_{t_0}]\) and checks whether \(\TSverify[h', \pi_{t_1}] = 1\) and \(\TStime[\rho_{t_0}] = t_0\land \TStime[\pi_{t_1}] = t_1\).

We can instantiate \(\TS\) using a blockchain such as for instance OmniLedger~\cite{OmniLedger}.
The \(\TSstamp[x]\) algorithm will simply include \(x\) in the blockchain and return the identifier of the block into which \(x\) was included.
\(\TSget\) will return the hash of the current head of the chain.
This value is difficult to predict since it depends on all submitted transactions and additional randomness (\eg nonces and miners' secrets).
The blockchain must also be continuously extended, such as in Bitcoin~\cite{Bitcoin}, in which a new block appears approximately every 10 minutes.
%Seb: why is the idea of using a dedicated blockchain not appropriate? we should give more arguments to justify it
Having a dedicated blockchain for a protest is not appropriate, and thus it is better to use a generic blockchain that is used also for other things.

There are some additional properties that we need from \(\TS\), which are implied by blockchains.
First, we need \emph{immutability}, to ensure that once we commit something (through \(\TSstamp\)) it will remain there.
Immutability will also ensure verifiability as the data must remain available to be verifiable.
Second, once we have committed some data, we do not need to keep the data itself, but we can store some confirmation value instead.
For instance in a blockchain, we can store the hash of the block in which our data was committed.
As a consequence as this block will remain, we are also sure that our data stay committed.% --- even if we no longer remember what our data looked like.
