% What's the problem?
Say Alice organizes a protest against the current regime leader Eve.
The protest is held in some location (or locations) and Alice wants to estimate 
the number of participants to prove a certain support to her cause.
% Why is it a problem?
Current methods have wide margins for error.
In addition, Alice may be tempted to cheat by estimating the number of 
participants as high as possible whereas the government's incentive is to make 
this estimation as low as possible to support Eve.
                                                                                  
% What's the approach?
In this work, we are interested in combining location proofs with some 
properties of electronic voting to provide verifiable participation counts.
Since the parties have incentives to cheat, we require transparency: in 
particular we are aiming at a decentralized scheme that provides verifiability 
and privacy.
More specifically, the scheme should be distributed, not depend on any authority 
and it must scale to handle large protests (i.e.\ millions of participants).

In our scheme each participant creates a participation proof with the help of 
other participants.
We use blockchains to commit and timestamp the participation proofs, location 
proofs to prove connection to the physical location.
% What's the result?
The scheme provides verifiability, which means that every participant can verify 
that their participation has been included --- i.e.\ individual verifiability, 
to prevent Eve from dropping data --- and that anyone can verify the correctness 
of the result --- i.e.\ universal and eligibility verifiability, to prevent any 
malicious actor (e.g. Sybil) from affecting the result.
Our scheme is privacy preserving in the sense that it adds no risk to a 
participant that was not already present when not using our scheme.

\keywords{%
  crowd counting;
  protesting;
  electronic voting;
  location proof;
  privacy
}
