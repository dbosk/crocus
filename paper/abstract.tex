\begin{abstract}
  
% % Why care?
Distance-bounding anonymous credentials could be used for any
location proofs that do not need to identify the prover and
thus could make even notoriously invasive mechanisms such as
location-based systems privacy-preserving.
% % Why is it a problem?
There is, however, no secure distance-bounding protocol with general (attribute-based) anonymous
credentials. Brands and Chaum's protocol combining distance-bounding
and Schnorr comes close but is susceptible to distance hijacking and terrorist fraud. 
% % What's the problem?
We need distance bounding with anonymity/privacy properties. 
% % What's the approach?
Our approach is another attempt toward combining distance bounding and
Schnorr, following in the footsteps of Brands and Chaum. Encouraged by
results from Capkun et al., we take advantage of lessened constraints
on how much can be sent in the fast phase of the distance-bounding
protocol and send challenges in a way that allows us to overcome the
weaknesses of Brands and Chaum's scheme. 
% % What are the findings?
We prove that our proposed scheme is resistant to the attacks
identified for distance-bounding protocols, \ie distance hijacking,
distance fraud, mafia fraud, and terrorist fraud. We also performed a
symbolic verification of security properties needed for resisting
these attacks, implemented in Tamarin. We also provide a version that does not rely on being able to send more than one bit at a time which yields the same properties except for terrorist fraud resistance. 

\end{abstract}

%\keywords{%
%  Location proof;
%  Distance-bounding;
%  Privacy;
%  Anonymous credentials.
%}
%
