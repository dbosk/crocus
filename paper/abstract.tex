\begin{abstract}
  
% % Why care?
Distance-bounding anonymous credentials could be used for any
location proofs that do not need to identify the prover and
thus could make even notoriously invasive mechanisms such as
location-based services privacy-preserving.
% % Why is it a problem?
There is, however, no secure distance-bounding protocol for general 
\emph{attribute-based} anonymous credentials.
Brands and Chaum's (EUROCRYPT'93) protocol combining distance-bounding
and Schnorr comes close, but does not fulfill the requirements of 
modern distance-bounding protocols.
% % What's the problem?
For that, we need a secure distance-bounding zero-knowledge proof-of-knowledge resisting 
mafia fraud, distance fraud, distance hijacking and terrorist fraud.

% % What's the approach?
Our approach is another attempt toward combining distance bounding and
Schnorr to construct a distance-bounding zero-knowledge proof-of-knowledge.
We construct such a protocol and prove it secure in the (extended) \ac{DFKOmodel} 
for distance bounding.
We also performed a symbolic verification of security properties needed for 
resisting these attacks, implemented in Tamarin. 

% % What are the findings?
Encouraged by results from Singh et al.~(NDSS'19), we take advantage of 
lessened constraints on how much can be sent in the fast phase of the 
distance-bounding protocol and achieve a more efficient protocol.
We also provide a version that does not rely on being able to send more than 
one bit at a time which yields the same properties except for (full) terrorist 
fraud resistance. 

\end{abstract}

%\keywords{%
%  Location proof;
%  Distance-bounding;
%  Privacy;
%  Anonymous credentials.
%}
%
