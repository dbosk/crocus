\begin{abstract}
  % Douglas: the abstract is too long
% What's the problem?
The problem of participation estimation is a fundamental democratic issue in 
the context of physical protests or demonstrations.  There, the number of 
participants reflects the strength of the support or at least the interest of 
the society in the topic of the protest.
% Why is it a problem?
Current methods for crowd counting have wide margins for error in addition to 
being privacy-invasive.
Moreover, the organizers of a protest may be tempted to cheat by inflating the 
number of participants as much as possible whereas it is in the other parties' 
interest to do the opposite if the protest is directed against it.                                                                                  
% What's the approach?
In this paper, we propose a novel approach to this problem, called \PRIVO, which combines location proofs (using distance-bounding anonymous credentials) with some properties of electronic voting to provide verifiable yet privacy-preserving participation estimation.
Since parties have incentives to cheat, transparency is a fundamental requirement that we achieve through a decentralized scheme providing both verifiability and privacy.
More specifically, the proposed scheme is distributed, does not depend on any authority and it must scale to handle large protests (\ie millions of participants).
In \PRIVO, each participant creates a participation proof with the collaboration of other participants, called witnesses, in a privacy-preserving manner (\ie without requiring to reveal their own identity or that of witnesses).
%Seb: verify in the mention of the blockchain is coherent with the rest of the paper
A public ledger is then used to commit and timestamp the participation proofs, which essentially consist of location proofs in which a privacy-preserving distance-bounding protocol is used by participants to prove their proximity to witnesses.
% What's the result?
Our scheme provides individual verifiability, which means that each participant 
can verify that their participation has been included in the final count --- to 
prevent the regime in place from dropping data --- and universal and 
eligibility verifiability, in the sense that anyone can verify the correctness 
of the result, to prevent any malicious actor (\eg Sybil attack) from affecting 
the result. 
%Seb: we need to be careful about how we formulate the following sentence
%Our scheme provides these verifiability properties in a privacy-preserving manner in the sense that it adds no risk to a participant that was not already present when not using our scheme. 
\end{abstract}

%\keywords{%
%  Crowd estimation;
%  Protest;
%  Electronic voting;
%  Location proof;
%  Distance-bounding;
%  Privacy.
%}
%
