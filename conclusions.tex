\mode*
\section{Conclusions}

\subsection<presentation>{Related work}

In general, none of the related works fulfils the eligibility verifiability 
\cref{%
  CountOnce,%
  CreatedAfterStart,CreatedBeforeEnd,SpatiallyRelated,%
}.
To some extent, some techniques do fulfil \cref{DesignatedEvent}, since people 
usually use placards in the protests.
Also, some schemes provide universal verifiability (\cref{UniversalVerif}), but 
not individual verifiability (\cref{IndividualVerif}).
If they would provide individual verifiability, they would likely violate the 
privacy requirements.
And finally, most of the current methods are estimates, with quite some error 
margins, so they cannot give the correct participation count.

\begin{frame}<presentation>
  \begin{remark}
    \begin{itemize}
      \item All current methods lack most of verifiability.
      \item Some provide universal verifiability.

        \pause

      \item Most cannot estimate the correct participation count.
    \end{itemize}
  \end{remark}
\end{frame}

\subsection{Open problems}

\begin{frame}
  \begin{question}
    How to solve Sybil, can we piggyback on existing \acp{PKI}?
    E.g.~\cite{Cinderella}.
  \end{question}

  \begin{question}
    Government can sign as many certificates as they like using their national 
    ID systems.
    How to deal with this?
  \end{question}

  \begin{question}
    Tie to something unique: phone, phone subscription?
    Register twice?
  \end{question}

  \begin{question}
    This is similar to the double-spending problem for cryptocurrencies, can 
    we reuse any such techniques?
    (Not like in Bitcoin, but more like ecash~\cite{ecash} and the like.)
  \end{question}
\end{frame}


\begin{frame}
\begin{question}
  How can we do \acp{LP} with 1000s of witnesses each?
  Can we do some preprocessing?
  Ad-hoc networks, broadcast \acp{LP}?
\end{question}
\end{frame}

\begin{frame}
\begin{question}
  Why should anyone trust these witnesses?
  What are the assumptions?
  Signed over a short period of time?\footnote{%
    In a sense, if they collude they have shown support for the cause.
    As long as they are not Sybil it should be fine?
  }
\end{question}

\pause

\begin{question}
  What if Alice acts as a witness for Eve's agents?
  Can that violate her privacy?
\end{question}

\pause

\begin{question}
  What if Eve's agents act as witnesses?
  Can they track their signatures to track users?
  Can Alice detect if this happens?
\end{question}
\end{frame}

\begin{frame}
  \begin{question}
    Malicious nodes want to provide Alice with a view that her proof is 
    included.
    Everyone else is provided a view without Alice.
    How can we ensure a consistent view for everyone?
    (Probably similar to the solution in Joakim's thesis.)
  \end{question}

  \begin{question}
    Under the protest conditions, we might have natural network splits.
    Is this a useful scenario?
  \end{question}

  \begin{question}
    How can we do blockchain branch and merge?
    That is needed to recover from the above attack.
  \end{question}
\end{frame}

\begin{frame}
  \begin{question}
    Shall we offer a platform that is run by a variety of institutions who 
    offers to verify protests?
    Or go strictly peer-to-peer?
  \end{question}
  \begin{question}
    If strictly peer-to-peer, and only run by protesters, one can infer that 
    they are protesters by running such a node.
    Can this be mitigated?
  \end{question}
\end{frame}

\begin{frame}
  \begin{question}
    How do we submit the proofs to storage?
    Voting normally uses mix-nets.
    Submit over Tor~\cite{Tor}?
  \end{question}

  \mode<presentation>{
  \begin{remark}
    Without anonymous submission we don't get vote privacy, thus no receipt 
    freeness either.
  \end{remark}
  }
\end{frame}


