\mode*
\section{Conclusions}

\subsection<presentation>{Related work}

In general, none of the related works fulfils the eligibility verifiability 
\cref{%
  CountOnce,%
  CreatedAfterStart,CreatedBeforeEnd,SpatiallyRelated,%
}.
To some extent, some techniques do fulfil \cref{DesignatedEvent}, since people 
usually use placards in the protests.
Also, some schemes provide universal verifiability (\cref{UniversalVerif}), but 
not individual verifiability (\cref{IndividualVerif}).
If they would provide individual verifiability, they would likely violate the 
privacy requirements.
And finally, most of the current methods are estimates, with quite some error 
margins, so they cannot give the correct participation count.

\begin{frame}<presentation>
  \begin{block}{Verifiability}
    \begin{itemize}
      \item All current methods lack most of verifiability.
      \item Some can provide universal verifiability.
    \end{itemize}
  \end{block}

  \pause{}

  \begin{block}{Correctness}
    \begin{itemize}
      \item Most cannot estimate the correct participation count.
      \item Some can (theoretically) give an upper bound.
    \end{itemize}
  \end{block}
\end{frame}

\subsection{Open problems}

\begin{frame}
  \frametitle{Sybil}
  \begin{question}
    How to solve Sybil, can we piggyback on existing \acp{PKI}?
    E.g.\ through~\cite{Cinderella}.
  \end{question}

  \begin{question}
    Can we otherwise bind identifiable keys to unique ring signature keys?
  \end{question}

  \begin{question}
    This is similar to the double-spending problem for cryptocurrencies, can 
    we reuse any such techniques?
    (Not like in Bitcoin, but more like ecash~\cite{ecash} and the like.)
  \end{question}
\end{frame}

\begin{frame}
  \frametitle{Distributed systems}
  \begin{question}
    \begin{itemize}
      \item Malicious nodes want to provide Alice with a view that her proof is 
        included.
      \item Everyone else is provided a view without Alice.
      \item How can we ensure a consistent view for everyone?
    \end{itemize}
  \end{question}

  \begin{question}
    How can we do blockchain branch and merge?
    That is needed to recover from the above attack.
  \end{question}
\end{frame}

\begin{frame}
  \frametitle{Anonymous communication}
  \begin{question}
    How do we submit the proofs to storage?
  \end{question}

  \mode<presentation>{%
    \begin{remark}
      \begin{itemize}
        \item Without anonymous submission we don't get proof privacy, thus no 
          receipt freeness either.
        \item Voting normally uses mix-nets.
        \item Submit over Tor~\cite{Tor}?
      \end{itemize}
    \end{remark}
  }
\end{frame}


