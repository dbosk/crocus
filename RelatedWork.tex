\mode*
\section{Related work}

\subsection<presentation>{Image-based methods}

The seemingly most commonly used method for counting crowds at protests is 
Jacob's method~\cite[c.f.][]{%
  2016DemonstrationsInSeoul,%
  BBCHowToCountProtestNumbers,%
  HowWillWeKnowTrumpInauguralCrowdSize,%
  TheXManMarch,%
  TheCrowdNumbersGame,%
}.
It is a manual method devised in the 1960s which relies on aerial photos of the 
event.
The verifier divides the protest venue into regions, then estimates the crowd 
density in the different regions and finally sums them up to get an estimate of 
the total.
This method is prone to errors, since it is based on estimates.
However, it allows for verification, since anyone can redo the counting using 
the same photos.
This provides universal verifiability (\cref{UniversalVerif}).
It is difficult to achieve individual verifiability (\cref{IndividualVerif}) 
since it is hard to verify that oneself is indeed in any of the photos.
We can argue eligibility verifiability (\cref{EligibilityVerif}) if all the 
photos are taken at the same point in time with no overlap, since then no one 
can be counted twice.
Unfortunately, we can only get the peak participation with this method.

When it comes to authenticity, we have to rely on the photos.
We can try to determine from the photos themselves when and where the they were 
taken.
However, the best we can do is to employ forensic methods to try to detect any 
modifications.

\begin{frame}<presentation>
  \begin{solution}[Jacob's method]
    \begin{itemize}
      \item Take aerial photos of the event.
      \item Divide area into regions.
      \item Estimate density in the regions.
      \item Sum up to an estimate of the total.
    \end{itemize}
  \end{solution}

  \pause

  \begin{remark}
    \begin{itemize}
      \item Prone to errors, due to estimations.
      \item Universally verifiable, anyone with the photos can redo the 
        estimate.
      \item Eligibility verifiability if photos at one snapshot in time and no 
        overlap.
    \end{itemize}
  \end{remark}
\end{frame}

There is an app CrowdSize~\cite{CrowdSize} which estimates the size of a crowd 
based on a selected space.
Then the user selects one of three pre-set density estimates: light, medium or 
dense.
This is similar to the method usually used by police, e.g.\ for the protests in
Seoul
\blockcquote{2016DemonstrationsInSeoul}{%
  \textins*{p}olice presume\textins*{d} that, when sitting, six people would 
  fill a space of 
  3.3 square meters
  \textelp{}
  The same area would hold nine or 10 people when standing%
}.
This method is not verifiable unless the user takes a picture of the crowd, 
then it inherits the verifiability properties of the previous method.
The same for authenticity.

\begin{frame}<presentation>
  \begin{solution}[CrowdSize app\footfullcite{CrowdSize}]
    \begin{itemize}
      \item User selects area on map.
      \item User selects an estimate for crowd density: light, medium or dense.
      \item App gives crowd size estimate based on size of the area and 
        density.
    \end{itemize}
  \end{solution}

  \pause

  \begin{remark}
    \begin{itemize}
      \item This is the same method as used by police.
      \item Not verifiable unless the user takes a picture of the crowd.
      \item Then it will be reduced to the previous method.
    \end{itemize}
  \end{remark}
\end{frame}

There is a body of work in the computer vision community, e.g.\ the work of 
\textcite{NNCrowdCounting}.
This class of methods requires photos or video surveillance of the protest 
location during the entire protest.
They generally include machine learning, and thus also require a training 
dataset to train the algorithms.
In the work by \textcite{NNCrowdCounting}, they actually train and evaluate 
their algorithm on different scenes, which might make this method easier to use 
for protesting.
We can provide universal verifiability (\cref{UniversalVerif}) with this 
method, since someone can always recount the participants using the recorded 
video material.
We might be able to provide some degree of individual verifiability 
(\cref{IndividualVerif}), but this might still be difficult.
Also, it is difficult (if not impossible) to not count people twice, thus we 
cannot argue for eligibility verifiability (\cref{EligibilityVerif}).
Furthermore, it is still difficult capture the entire location on surveillance 
video.
The authenticity properties are reduced to those of the video, which should be 
similar as in our discussion about the authenticity of photos.

\begin{frame}<presentation>
  \begin{solution}[Computer vision]
    \begin{itemize}
      \item Record video of the event.
      \item Train algorithms on part of the data.
      \item Run algorithm to count the total.
    \end{itemize}
  \end{solution}

  \pause

  \begin{remark}
    \begin{itemize}
      \item Requires video-recording of the entire event.
      \item Difficult to record every part.
      \item Difficult (impossible?) to not record someone twice.
      \item Is verifiable, though limited, if the video is public.
      \item Authenticity can be questioned as with the photos.
    \end{itemize}
  \end{remark}
\end{frame}

\subsection<presentation>{Signal-based methods}

During the protests in Seoul~\cite{2016DemonstrationsInSeoul} there was one 
retail analytics company that tried to estimate the number of participants 
using their technology.
They scanned for MAC addresses emitted from the Wi-Fi of participants' 
smartphones.
However, this method required many assumptions:
\blockcquote{2016DemonstrationsInSeoul}{%
  The company presumed that about half of smartphone users usually leave their 
  Wi-Fi feature on and the other half switch it off, based on a separate survey 
  on smartphone usage. It also assumed that about 20 percent of the smartphone 
  signals were repetition from the same device.
}
This method cannot provide any verifiability nor authenticity, we must trust 
the company to do the measurements correctly and to be honest about when and 
where they did them.
As MAC address randomization get wider adoption this will be even more 
difficult, although some tracking of smartphones will still be 
possible~\cite{WhyMACRandomizationIsNotEnough}.

\begin{frame}<presentation>
  \begin{solution}[Wi-Fi tracking]
    \begin{itemize}
      \item Record unique Wi-Fi MAC addresses from people's smartphones.
    \end{itemize}
  \end{solution}

  \pause

  \begin{remark}
    \begin{itemize}
      \item Imprecise, requires assumptions.
      \item Half of smartphone users turn Wi-Fi off.
      \item MAC address randomization.
      \item Cannot verify.
    \end{itemize}
  \end{remark}
\end{frame}

A better proposal would be to use \enquote{IMSI catchers} (or the real cell 
towers of the mobile network) to count unique phones at the location.
However, there are several problems with this approach.
First, it will be difficult to register only the participants' phones, many 
bystanders will also be counted.
Second, since the phone is a unique identifier, participants might be 
uncomfortable to be registered in association with the event and might thus 
turn the device off.
Finally, there will be limited verifiability and authenticity, as the data 
recorder must be trusted to record all data and do it in the relation to the 
event.

\begin{frame}<presentation>
  \begin{solution}[IMSI catchers]
    \begin{itemize}
      \item Use IMSI catchers to register unique phones.
      \item (Or use the network cell towers.)
    \end{itemize}
  \end{solution}

  \pause

  \begin{remark}
    \begin{itemize}
      \item Will include bystanders into the count.
      \item Privacy-invading due to phones being uniquely identifying.
      \item Cannot verify.
    \end{itemize}
  \end{remark}
\end{frame}

\Textcite{WifiCrowdCounting} also relies of Wi-Fi, but they instead use the 
Wi-Fi signal as a sonar to detect people.
However this method is designed for indoor use and does not scale to more than 
a crowd of 20 people.
In either case, this type of data cannot provide verifiability nor 
authenticity.

In general for all of the above methods, the more a crowd spreads out, the more
difficult it will be to determine its size:
the problem is determining whether people near the event's perimeter are 
participants of simply bystanders~\cite{HowToEstimateCrowdSize}.
These methods also have difficulty capturing the actual attendance of an event,
i.e.\ the cumulative participation, not just the count at a snapshot around the 
peak.

%\subsection<presentation>{Crowd sensing}
%
%\dots

\subsection<presentation>{More exact methods}

The most closely related work is \textcite{CrowdCount}.
This is a web service which lets Alice create an event and Bob can submit his 
location to register that he is in Alice's event.
This solves the problem that it counts everyone (who submitted), not just the 
count at the snapshot of the photos.
However, there is no verification nor authenticity, i.e.\ the service must be 
trusted to behave honestly, but even then, nothing prevents Bob from submitting 
twice (violating eligibility verification, \cref{EligibilityVerif}).
Another downside is that the service also requires an Internet connection 
during the event to register as a participant.
This can make it difficult to use if e.g.\ Alice has organized a protest 
against the government, who shuts down the cellular network or Internet 
backbone as a means to censor the protest.

\begin{frame}<presentation>
  \begin{solution}[CrowdCount\footfullcite{CrowdCount}]
    \begin{itemize}
      \item Register an upcoming event.
      \item Let people \enquote{check-in} by submitting their location.
    \end{itemize}
  \end{solution}

  \pause

  \begin{remark}
    \begin{itemize}
      \item No verification, the service must be trusted.
      \item Nothing prevents Bob from checking in twice.
      \item Requires an Internet connection during the event.
    \end{itemize}
  \end{remark}
\end{frame}
