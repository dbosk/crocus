\section{Introduction}
\label{Introduction}

There are different possibilities for a protest: e.g.\ one performed online or 
one performed in the physical reality.
We are interested in real-world protests, specifically to verify a protest.
One problem in real-world demonstrations that has not yet been entirely solved 
is the crowd-counting problem, i.e.\ the verification of the participation in 
a demonstration.
After many demonstrations the count by police and that by the organizers 
differ, in some instances the difference can be hundreds of thousands.
There are numerous examples, e.g.\ the demonstrations in South 
Korea~\cite{2016DemonstrationsInSeoul} or the women's marches in the 
US~\cite{2017WomensMarchesInUS}, where there is difficulty in establishing the 
actual number of participants.
The methods for counting the crowds vary.
There are some techniques using computer vision found in the research 
literature, e.g.\ by \textcite{CVCrowdCounting}.
These require images from the demonstration and provide no way to verify the 
authenticity of the count (except re-running the algorithm on one's own input 
data).
However, as is illustrated by 
\textcite{2016DemonstrationsInSeoul,2017WomensMarchesInUS}, the methods used in 
practice are manual and prone to errors.
In this paper, we combine electronic voting with location-proof systems.

Alice is an activist who organizes a demonstration for some cause.
Her main goal after the protest is to provide verifiable data.
Alice several problems:
\begin{itemize}
  \item She must prove that the data is related to the event, that it is not 
    reused data from previous demonstrations.
  \item She must provide data which can be used for the desired verification of 
    the event.
  \item She must also consider the privacy of the participating individuals.
\end{itemize}
Since many of the current techniques are based on photos, we will give an 
example of Alice's problems in that setting.
How can she ensure that photos from a demonstration are authentic?
We can probably recognize the place the photo is portraying, however, this 
might just as well be a reconstruction or entirely computer generated.
We cannot trust the meta-data of the photo, such as time-stamps of the file, 
because these can easily be manipulated.
So the only thing we can say for sure is that the photo was taken at the latest 
at the time of publication.
Now, if we cannot trust these photos, how can we determine the number of 
participants of a demonstration?
This is discussed further in \cref{DataAuthenticity}.

Assume that Alice can provide this authenticated data.
Alice and Bob do not want that this data ties them to the demonstration, 
because then the autocrat Eve can use this data to find out about Alice and Bob 
and arrest them.
Thus, what Alice and Bob need is a system which can provide data authenticity 
and user privacy, i.e.\ that the data can be correctly tied to the 
demonstration without outing Alice and Bob as supporters but still ensuring 
that Alice is Alice and Bob is Bob.
As a protest can be compared to voting (either voting for or against the 
cause), we will adapt the security and privacy properties of electronic 
voting.
We will discuss this further in \cref{Verification}.

It can also be compared to electronic voting --- to vote for or against the 
cause.
In this case all participants and third-parties can verify the authenticity of 
the result, this would inherit all the desirable properties of the electronic 
voting system (these properties are discussed in \cref{Verification}).
We are interested in real-world protests, i.e.\ demonstrations where 
participants gather in a physical location.
For these we need to bind all participants to the same physical location at 
a reasonably similar time (within the duration of the protest).

