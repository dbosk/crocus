\mode*

\section{Introduction}
\label{Introduction}

There are different possibilities for a protest: e.g.\ one performed online or 
one performed in the physical reality.
We are interested in real-world protests, specifically to verify a protest.
One problem in real-world demonstrations that has not yet been entirely solved 
is the crowd-counting problem, i.e.\ the verification of the participation in 
a demonstration.

\subsection<presentation>{What's the problem?}

\begin{frame}<presentation>
  \begin{block}{The crowd-counting problem}
    \begin{itemize}
      \item Alice organizes a protest against Eve's regime.
      \item Bob, Carol and others show up.
      \item We want to know how many showed up to support Alice.
    \end{itemize}
  \end{block}
\end{frame}

\begin{frame}<presentation>
  \begin{example}
    \begin{itemize}
      \item Computer Vision to do object recognition.
      \item Requires photos that cover the entire location, all the time.
    \end{itemize}
  \end{example}

  \pause

  \begin{example}
    \begin{itemize}
      \item Scan active mobile phones in the area.
      \item This requires some extra equipment.
      \item This catches bystanders who are not protesting.
    \end{itemize}
  \end{example}
\end{frame}

\begin{frame}<presentation>
  \begin{example}
    \begin{itemize}
      \item South Korea 2016~\cite{2016DemonstrationsInSeoul}:
        approximate count by imagery.

      \item US 2017~\cite{2017WomensMarchesInUS}:
        sum up people's approximations.
    \end{itemize}
  \end{example}
\end{frame}

\subsection<presentation>{Why is it a problem?}

\begin{frame}<presentation>
  \begin{block}{Verifying protest participation}
    \begin{itemize}
      \item Alice organizes a protest against Eve's regime.
      \item Bob, Carol and others show up.

        \pause

      \item {\color{green} Alice wants to show that many support her cause.}

        \pause

      \item {\color{red} Eve wants to show that few support Alice's cause.}

        \pause

      \item It's an adversarial setting!
      \item We need verifiable results.
    \end{itemize}
  \end{block}
\end{frame}

After many demonstrations the count by police and that by the organizers 
differ, in some instances the difference can be hundreds of thousands.
There are numerous examples, e.g.\ the demonstrations in South 
Korea~\cite{2016DemonstrationsInSeoul} or the women's marches in the 
US~\cite{2017WomensMarchesInUS}, where there is difficulty in establishing the 
actual number of participants.
The methods for counting the crowds vary.
There are some techniques using computer vision found in the research 
literature, e.g.\ by \textcite{CVCrowdCounting}.
These require images from the demonstration and provide no way to verify the 
authenticity of the count (except re-running the algorithm on one's own input 
data).
However, as is illustrated by 
\textcite{2016DemonstrationsInSeoul,2017WomensMarchesInUS}, the methods used in 
practice are manual and prone to errors.
Additionally, none of these techniques can estimate the cumulative 
participation count without counting some people multiple times.

\blockcquote{2016DemonstrationsInSeoul}{%
  The rallies stretch from midday to late night --- some people stay for 
  several hours, others just several minutes.%
}

\blockcquote{2016DemonstrationsInSeoul}{%
  The demonstrators not only gather in open space, but also small alleys and 
  between buildings.%
}

\blockcquote{2016DemonstrationsInSeoul}{%
  For the police, the aim is to measure the maximum crowd occupying a certain 
  space at any given time so that they could determine the size of police 
  personnel and resources to deploy%
}

\blockcquote{2016DemonstrationsInSeoul}{%
  Organizers seek to track the entire flow of people from the protest’s start 
  to its finish. They collect estimates given by counters deployed at different 
  locations to size up the crowds in each area and update the numbers 
  throughout the duration of the protest.%
}

\blockcquote{HowToEstimateCrowdSize}{%
  the more widely that a crowd is spread out, the more challenging it can be to 
  get an accurate estimate of those attending, as it can become difficult to 
  determine whether people near the event's perimeter are there for the event 
  or are in the area for other reasons,%
}

% XXX Add the 2017 Venezuelan protests

We can make one important observation about this problem that has seemingly 
been ignored in the past: it is an adversarial setting.
The protesters (organizers and participants) have an incentive to increase the 
tallied number of participants, whereas other interests might have an incentive 
to decrease the tallied number of participants, e.g.\ an authoritarian regime.
(Or the other way around, an authoritarian regime wants to organize a 
pro-regime protest to demonstrate its \enquote{wide support}.)
To solve this problem, we need a verifiable participation count.
In this paper, we combine the verifiability and privacy properties of 
electronic voting with location-proof systems to provide a verifiable 
participation count.

%In the case of the Korean demonstrations~\cite{2016DemonstrationsInSeoul}, this 
%was in one place during an entire day and then repeated for several weekends.
%In the case of the Women's Marches in the US~\cite{2017WomensMarchesInUS}, they
%were in several locations at the same time.

\blockcquote{HowToEstimateCrowdSize}{%
  Crowd size is also needed for media news reports and to historically record 
  the event%
}


