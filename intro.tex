\mode*

\section{Introduction}%
\label{Introduction}

Alice is an activist who organizes a protest for some cause.
Eve is an autocrat who opposes Alice's cause.
Bob, Carol and many others participate in the protest against Eve.
One problem that has not yet been entirely solved is the crowd-counting 
problem, i.e.\ to verify how many participated in Alice's protest.
Our main goal is to provide a tool which prevents both Alice and Eve from 
cheating.

\mode<presentation>{%
\subsection{What's the problem?}
}

\begin{frame}<presentation>
  \begin{block}{The crowd-counting problem}
    \begin{itemize}
      \item Alice organizes a protest against Eve's regime.
      \item Bob, Carol and others show up.
      \item We want to know how many showed up to support Alice.
    \end{itemize}
  \end{block}
\end{frame}

\mode<presentation>{%
\subsection{Why is it a problem?}
}

\mode<presentation>{%
\begin{frame}
  \begin{figure}
    \includegraphics[height=0.9\textheight]{fig/Jacobs-method.jpg}
  \end{figure}
\end{frame}

\begin{frame}
  \begin{example}
    \begin{itemize}
      \item Computer vision does object recognition.
      \item Requires photos/video that cover the entire location, all the time.
      \item This will count people twice.
    \end{itemize}
  \end{example}

  \pause{}

  \begin{example}
    \begin{itemize}
      \item Scan active mobile phones in the area.
      \item This requires some extra equipment.
      \item This catches bystanders who are not protesting.
    \end{itemize}
  \end{example}
\end{frame}
}

\mode<none>{%
\begin{frame}
  \begin{example}
    \begin{itemize}
      \item South Korea 2016~\cite{2016DemonstrationsInSeoul}:
        approximate count by imagery.
        \begin{description}
          \item[Organizers] 1\,000\,000
          \item[Police] 260\,000
        \end{description}

        \pause{}

      \item US 2017~\cite{2017WomensMarchesInUS}:
        sum up people's approximations.
      \item This makes it even difficult to estimate the error.
    \end{itemize}
  \end{example}
\end{frame}
}

After many demonstrations the count by police and that by the organizers 
differ\footnote{%
  This is actually quite natural, they have different goals.
  The organizers want to count everyone who ever participated.
  Police want to estimate the count at the peak of participation, due to crowd 
  control~\cite{2016DemonstrationsInSeoul}.
}, in some instances the difference can be hundreds of thousands.
There are numerous examples, e.g.\ the demonstrations in South 
Korea~\cite{2016DemonstrationsInSeoul}, Trump's 
inauguration~\cite{HowWillWeKnowTrumpInauguralCrowdSize} or the Women's Marches 
in the US~\cite{2017WomensMarchesInUS}, where there is difficulty in 
establishing the actual number of participants.
The methods for counting the crowds vary.
Most of the available methods lack precision, i.e.\ they have large error 
margins.
They can only give an estimate for a particular snapshot in time, e.g.\ at the 
peak of the event, not the cumulative participation count --- at least not 
without counting some people multiple times, which in turn increases the error 
of the estimation.
Finally, they also lack verifiability, i.e.\ everyone must trust the entity who
does the counting.
(We will discuss them in more detail in \cref{RelatedWork}.)

\begin{frame}<presentation>
  \begin{block}{Verifying protest participation}
    \begin{itemize}
      \item Alice organizes a protest against Eve's regime.
      \item Bob, Carol and others show up.

        \pause{}

      \item {\color{green} Alice wants to show that many support her cause.}

        \pause{}

      \item {\color{red} Eve wants to show that few support Alice's cause.}
      \item It's an adversarial setting!
      \item We need verifiable results.
    \end{itemize}
  \end{block}
\end{frame}

We can make one important observation about this problem that has seemingly 
been ignored in the past: it is an adversarial setting.
The protesters (organizers and participants) have an incentive to increase the 
tallied number of participants, whereas other interests might have an incentive 
to decrease the tallied number of participants, e.g.\ Eve's authoritarian 
regime in our example above.
To solve this problem, we need a verifiable participation count which is 
protected from Sybil attacks.

\subsection{Combining electronic voting and location proofs}

Electronic voting (and voting in general) require strong verifiability and 
privacy properties.
We will adapt these requirements to the setting of protests.

There are three requirements for 
verification~\cite{VerifyingPrivacyPropertiesOfVotingProtocols}:
\begin{frame}
\begin{requirements}[V]
\item\label{EligibilityVerif} Eligibility: anyone can verify that each vote 
  cast is legitimate.
\item\label{UniversalVerif} Universal verifiability: anyone can verify that the 
  result is according to the cast votes.
\item\label{IndividualVerif} Individual verifiability: every voter can verify 
  that their vote is included in the result.
\end{requirements}
\end{frame}
We translate the votes into \emph{proofs of participation}.
Both universal and individual verifiability remain the same.
The eligibility requirement is slightly different:
for protests the eligibility requirement must include temporal and spatial 
eligibility --- that each proof of participation satisfies some temporal and 
spatial relation to the protest.
We will define these more formally in \cref{Verification}.

There are also three requirements for 
privacy~\cite{VerifyingPrivacyPropertiesOfVotingProtocols}:
\begin{frame}
\begin{requirements}[P]
\item\label{VotePrivacy} Vote privacy: the voting does not reveal any 
  individual vote.
\item\label{ReceiptFreeness} Receipt freeness: the voting system does not 
  provide any data that can be used as a proof of how the voter voted.
\item\label{CoercionResistance} Coercion resistance: a voter cannot cooperate 
  with a coercer to prove the vote was cast in any particular way.
\end{requirements}
\pause{}
\mode<presentation>{%
  \begin{remark}
    P\ref{CoercionResistance} \(\implies\)
    P\ref{ReceiptFreeness} \(\implies\)
    P\ref{VotePrivacy}
  \end{remark}
}
\end{frame}
\Textcite{VerifyingPrivacyPropertiesOfVotingProtocols} showed that 
\cref{CoercionResistance} implies \cref{ReceiptFreeness}, which in turn implies
\cref{VotePrivacy}.
We note that we cannot improve the privacy of participating protesters, if they 
are caught during a protest there is nothing we can do.
But if they participate and they are not caught, then using our proposed tool 
should not incur any additional risk.
Thus we will adapt the idea of receipt freeness (see \cref{Privacy} for a 
detailed definition).
In essence, upon completing the protocol, Alice will have nothing that Eve can 
use to tie her to her participation proof.

\subsection{Contributions}

In this paper, we combine the verifiability and privacy properties of 
electronic voting with location-proof systems to provide a verifiable 
participation count.
We can only provide a verifiable participation count for non-government 
protests:
In any pro-government protest we cannot prevent the government from doing a 
Sybil attack, since the government controls the identity system.

%In the case of the Korean demonstrations~\cite{2016DemonstrationsInSeoul}, this 
%was in one place during an entire day and then repeated for several weekends.
%In the case of the Women's Marches in the US~\cite{2017WomensMarchesInUS}, they
%were in several locations at the same time.

%\blockcquote{HowToEstimateCrowdSize}{%
%  Crowd size is also needed for media news reports and to historically record 
%  the event%
%}

Sometimes Eve's authoritarian regime wants to organize a pro-regime protest to 
demonstrate its \enquote{wide 
  support}~\cite[e.g.][]{AlJazeeraOnVenezuela2017,VenezuelanStateWorkersCalledToParticipate}.
If we rely on government-issued credentials, we cannot prevent the government 
from performing Sybil attacks --- since it can issue arbitrarily many 
\enquote{valid} credentials.
This means that we cannot verify any pro-government protests.

