\section{Introduction}
\label{Introduction}

There are different possibilities for a protest: e.g.\ one performed online or 
one performed in the physical reality.
Purely online protests are essentially petitions.
This case can use digital signatures to verify the authenticity of a list of 
names.
It can also be compared to electronic voting --- to vote for or against 
something.
In this case all participants and third-parties can verify the authenticity of 
the result, but cannot verify the vote of an individual.

We are interested in real-world protests, for these we need to bind all 
participants to the same physical location at a reasonably similar time (within 
the duration of the protest).
\citeauthor{PROPS} developed a decentralized location-proof system, \ac{PROPS}, 
which provides a participant with a proof of being at the location.
However, we would like that any participant can prove the demonstration's 
authenticity to a third-party, e.g.\ the UN or a human-rights 
organization.
The reason this is an interesting property is the following scenario:
Imagine there is a demonstration, the police shows up and tries to apprehend 
every participant.
However, one demonstrator survives and manages to flee the country, here the 
demonstrator wants to prove that there was a protest of a certain size.

There are at least two reasons we are interested in authenticating a protest in 
this way:
\begin{enumerate}
  \item for a third-party can verify the authenticity of the protest, e.g.\ 
    that there were \(n\) participants;
  \item for the organizers to use the participation as feedback into 
    a reputation system.
\end{enumerate}
We want to do this in a privacy-preserving way:
We want to verify the participation of a protest, but not identifying 
individuals who participated.
If a demonstrator flees the country, that demonstrator should be able to prove 
to the e.g.\ the UN that there were \(n\) participants.
But if the demonstrator is caught by the secret police before leaving, then the 
secret police should not learn the identities of the other participants.

One problem in real-world demonstrations that has not yet been entirely solved 
is the crowd-counting problem, i.e.\ the verification of the participation in 
a demonstration.
After many demonstrations the count by police and that by the organizers 
differ, in some instances the difference can be hundreds of thousands.
There are numerous examples, e.g.\ the demonstrations in South 
Korea~\cite{2016DemonstrationsInSeoul} or the women's marches in the 
US~\cite{2017WomensMarchesInUS}, where there is difficulty in establishing the 
actual number of participants.
The methods for counting the crowds vary.
There are some techniques using computer vision found in the research 
literature, e.g.\ by~\textcite{CVCrowdCounting}.
These require images from the demonstration and provide no way to verify the 
authenticity of the count (except re-running the algorithm on one's own input 
data).
However, as is illustrated by 
\textcite{2016DemonstrationsInSeoul,2017WomensMarchesInUS}, the methods used in 
practice are manual and prone to errors.

The main goal for Alice after a protest is to provide verifiable data.
For example, how can she ensure that photos from a demonstration are authentic?
We can probably recognize the place the photo is portraying, however, this 
might just as well be a reconstruction or entirely computer generated.
We cannot trust the meta-data of the photo, such as time-stamps of the file, 
because these can easily be manipulated.
So the only thing we can say for sure is that the photo was taken at the latest 
at the time of publication.
Now, if we cannot trust these photos, how can we determine the number of 
participants of a demonstration?
Many of the techniques require photos for doing this.
Furthermore, as is shown by 
\textcite{2016DemonstrationsInSeoul,2017WomensMarchesInUS}, the numbers 
determined by these techniques have rather large margins.

Alice also has another problem.
Assume that she can provide this verifiable data.
Alice and Bob do not want that this data ties them to the demonstration, 
because then Eve can use this data to find out about Alice and Bob and arrest 
them.
Thus, what Alice and Bob need is a system which can provide data authenticity 
and user privacy, i.e.\ that the data can be correctly tied to the 
demonstration without outing Alice and Bob as supporters.

\subsection{The Problem}

Usually the count by the organizers and count by police differs, we want to 
solve this problem.

We can divide the problem into three parts:
\begin{itemize}
  \item Register participants in the system.
  \item Collect the necessary data.
  \item Perform the counting and verification.
\end{itemize}

To register the participants we might be able to piggy-back on existing 
\acp{PKI} and blind signatures --- similarly as for voting systems.

We have the verifiability properties in voting:
\begin{description}
  \item[Individual] Individual verifiability allows a voter to verify her own 
    vote was included in the election outcome.
  \item[Universal] Universal verifiability allows voters or observers to check 
    that the election outcome corresponds to the votes cast.
  \item[Eligibility] Eligibility verifiability allows voters and observers to 
    check that each vote in the election outcome was cast by a uniquely 
    registered voter.
\end{description}

There are several issues that must be treated:
\begin{itemize}
  \item Each location proof must be bound to one individual's participation.
    One protester must not be able to create two unique location proofs and be 
    able to increase the number of participants, that would violate 
    \cref{VerifEligibility} above.
    In other words, we must prevent the Sybil attack.

  \item A group of people should not be able to generate \acp{LP} to make it 
    look as if they participated in the protest but actually stayed at home.
    This means that all \acp{LP} must be linked to each other.
    If a protest has hundreds of thousands of participants, then this linking 
    must be achieved efficiently.
\end{itemize}


