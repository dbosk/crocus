\mode*

\section{Introduction}
\label{Introduction}

There are different possibilities for a protest: e.g.\ one performed online or 
one performed in the physical reality.
We are interested in real-world protests, specifically to verify a protest.
One problem in real-world demonstrations that has not yet been entirely solved 
is the crowd-counting problem, i.e.\ the verification of the participation in 
a demonstration.
After many demonstrations the count by police and that by the organizers 
differ, in some instances the difference can be hundreds of thousands.
There are numerous examples, e.g.\ the demonstrations in South 
Korea~\cite{2016DemonstrationsInSeoul} or the women's marches in the 
US~\cite{2017WomensMarchesInUS}, where there is difficulty in establishing the 
actual number of participants.
The methods for counting the crowds vary.
There are some techniques using computer vision found in the research 
literature, e.g.\ by \textcite{CVCrowdCounting}.
These require images from the demonstration and provide no way to verify the 
authenticity of the count (except re-running the algorithm on one's own input 
data).
However, as is illustrated by 
\textcite{2016DemonstrationsInSeoul,2017WomensMarchesInUS}, the methods used in 
practice are manual and prone to errors.
In this paper, we combine electronic voting with location-proof systems.

