\mode*

\section{Introduction}
\label{Introduction}

Alice organizes a street protest against Eve's regime.
Bob, Carol and many others participate in the demonstration.
One problem that has not yet been entirely solved is the crowd-counting 
problem, i.e.\ to verify how many participated in Alice's protest.

\subsection<presentation>{What's the problem?}

\begin{frame}<presentation>
  \begin{block}{The crowd-counting problem}
    \begin{itemize}
      \item Alice organizes a protest against Eve's regime.
      \item Bob, Carol and others show up.
      \item We want to know how many showed up to support Alice.
    \end{itemize}
  \end{block}
\end{frame}

\begin{frame}<presentation>
  \begin{example}
    \begin{itemize}
      \item Computer vision does object recognition.
      \item Requires photos/video that cover the entire location, all the time.
      \item This will count people twice.
    \end{itemize}
  \end{example}

  \pause

  \begin{example}
    \begin{itemize}
      \item Scan active mobile phones in the area.
      \item This requires some extra equipment.
      \item This catches bystanders who are not protesting.
    \end{itemize}
  \end{example}
\end{frame}

\subsection<presentation>{Why is it a problem?}

\begin{frame}<presentation>
  \begin{example}
    \begin{itemize}
      \item South Korea 2016~\cite{2016DemonstrationsInSeoul}:
        approximate count by imagery.
        \begin{description}
          \item[Police] X
          \item[Organizers] Y
        \end{description}

        \pause

      \item US 2017~\cite{2017WomensMarchesInUS}:
        sum up people's approximations.
      \item This makes it even difficult to estimate the error.
    \end{itemize}
  \end{example}
\end{frame}

After many demonstrations the count by police and that by the organizers 
differ\footnote{%
  This is actually quite natural, they have different goals.
  The organizers want to count everyone who ever participated.
  Police want to estimate the count at the peak of participation, due to crowd 
  control~\cite{2016DemonstrationsInSeoul}.
}, in some instances the difference can be hundreds of thousands.
There are numerous examples, e.g.\ the demonstrations in South 
Korea~\cite{2016DemonstrationsInSeoul}, Trump's 
inauguration~\cite{HowWillWeKnowTrumpInauguralCrowdSize} or the Women's Marches 
in the US~\cite{2017WomensMarchesInUS}, where there is difficulty in 
establishing the actual number of participants.
The methods for counting the crowds vary.
Most of the available methods lack precision, i.e.\ they have large error 
margins.
They can only give an estimate for a particular snapshot in time, e.g.\ at the 
peak of the event, not the cumulative participation count --- at least not 
without counting some people multiple times, which in turn increases the error 
of the estimation.
Finally, they also lack verifiability, i.e.\ everyone must trust the entity who
does the counting.
We will discuss them in more detail in \cref{RelatedWork}.

\begin{frame}<presentation>
  \begin{block}{Verifying protest participation}
    \begin{itemize}
      \item Alice organizes a protest against Eve's regime.
      \item Bob, Carol and others show up.

        \pause

      \item {\color{green} Alice wants to show that many support her cause.}

        \pause

      \item {\color{red} Eve wants to show that few support Alice's cause.}

        \pause

      \item It's an adversarial setting!
      \item We need verifiable results.
    \end{itemize}
  \end{block}
\end{frame}

We can make one important observation about this problem that has seemingly 
been ignored in the past: it is an adversarial setting.
The protesters (organizers and participants) have an incentive to increase the 
tallied number of participants, whereas other interests might have an incentive 
to decrease the tallied number of participants, e.g.\ Eve's authoritarian 
regime in our example above.
Or the other way around, sometimes Eve's authoritarian regime wants to organize 
a pro-regime protest to demonstrate its \enquote{wide 
  support}~\cite[e.g.][]{AlJazeeraOnVenezuela2017,VenezuelanStateWorkersCalledToParticipate}.
(This makes it difficult to rely on government-issued credentials to prevent a 
Sybil attack --- government can issue arbitrarily many such credentials for its 
own protests.)
To solve this problem, we need a verifiable participation count which is 
protected from Sybil attacks.

\subsection{What is \protect\emph{a} protest?}

Protests can vary considerably.
To be able to estimate the participation count for one, we first need to define
what should be counted.

Let us start by considering some examples.
During the demonstrations against the South Korean president in Seoul
\blockcquote{2016DemonstrationsInSeoul}{%
  \textins*{t}he rallies stretch\textins{ed} from midday to late night --- some 
  people stay\textins{ed} for several hours, others just several minutes%
}.
These rallies were all in the same location in the capital and repeated every 
weekend for the duration of a few weeks.
The Women's Marches~\cite{2017WomensMarchesInUS}, on the other hand, occurred 
in parallel in many locations.
We also have the Venezuelan demonstrations where
\blockcquote{2017VenezuelaProtestFrequency}{%
  anti-government demonstrators have staged daily protests across Venezuela%
} while
\blockcquote{AlJazeeraOnVenezuela2017}{%
  pro-government workers sang and danced as they staged a rival march to show 
  their support for the president's controversial plan to rewrite the 
  constitution%
}.
Judging from these examples, the minimal common part is the cause.

\begin{frame}<presentation>
  \begin{example}[Seoul 2016]
    \begin{itemize}
      \item Rallies every weekend.
      \item Starting midday, lasting to late night.
      \item Some stayed for several hours, some for only minutes.
    \end{itemize}
  \end{example}

  \pause

  \begin{example}[US 2017]
    \begin{itemize}
      \item Women's Marches in several locations.
      \item Marched some distance.
      \item One-time occurrence.
    \end{itemize}
  \end{example}
\end{frame}

\begin{frame}<presentation>
  \begin{example}[Venezuela 2017]
    \begin{itemize}
      \item Daily anti-government rallies.
      \item Multiple locations.
      \item Pro-government rallies too.
    \end{itemize}
  \end{example}
\end{frame}

The organizer Alice want to count everyone who participated at any time and in 
any of the locations~\cite{2016DemonstrationsInSeoul}.
(Whereas police are only interested in the maximum crowd at any point in time, 
to deploy enough personnel for crowd-control.)
We thus settle for the following definition.

\begin{frame}<presentation>
  \begin{remark}
    \begin{itemize}
      \item Organizers want to count anyone who participated \dots
      \item \dots at any time
      \item \dots in any location
    \end{itemize}
  \end{remark}
\end{frame}

\begin{frame}
\begin{definition}[A protest]
  A protest is an event identified by its cause.
\end{definition}
\end{frame}

% Verifiability and privacy
\mode<all>
\mode*
\section{Desired security properties}%
\label{Properties}

%\subsection{An overview of the adversary}

We have three \emph{malicious} adversaries: Alice, Eve and Rusuk.
Alice the activist (and all the protesters) try to increase the count.
Eve the totalitarian dictator tries to decrease the count.
Additionally, Eve tries to deanonymize the participants to arrest them or 
\enquote{convince} them to change their minds.
Finally, we have Rusuk who is represents another nation state.
Rusuk has some interest in affecting the state of Eve's regime, for Rusuk's own 
gain, thus supporting either Eve or Alice as he see fits.
Rusuk will thus also try to either increase or decrease the count.
We will now formally define some security properties specifying the 
computational effort needed by the adversaries to succeed.
We categorize the desired properties into verifiability and privacy.

\mode<presentation>{%
\begin{frame}
  \begin{block}{Alice the activist}
    \begin{itemize}
      \item Alice organizes the protest against Eve's regime.
      \item Alice wants to have a high participation count.
      \item She tries to increase it.
      \item So does \emph{all the protesters}.
    \end{itemize}
  \end{block}

  \pause{}

  \begin{block}{Eve the fake-news populist}
    \begin{itemize}
      \item Eve's regime is target of the protest.
      \item She wants to have a low participation count.
      \item She tries to decrease it.
    \end{itemize}
  \end{block}
\end{frame}
}

\mode<presentation>{%
\begin{frame}
  \begin{block}{Eve the totalitarian dictator}
    \begin{itemize}
      \item Eve wants to \enquote{convince} activists to support her.
      \item She wants to find all activists.
      \item She tries to identify (deanonymize) participants.
    \end{itemize}
  \end{block}
\end{frame}
}

\mode<presentation>{%
\begin{frame}
  \begin{block}{Rusuk the nation state}
    \begin{itemize}
      \item Rusuk benefits from the situation in Evestan.
      \item He wants the outcome to be the most beneficial.
      \item He will try to increase or decrease the participation count.
    \end{itemize}
  \end{block}
\end{frame}
}

\subsection{Verifiability}%
\label{Verifiability}

We desire three verifiability requirements:
\begin{requirements}[V]
\item\label{EligibilityVerif} Eligibility: anyone can verify that each 
  participation proof provides temporal and spatial eligibility and that it has 
  not been counted before.
\item\label{UniversalVerif} Universal verifiability: anyone can verify that the 
  result is according to the submitted participation proofs.
\item\label{IndividualVerif} Individual verifiability: every participant can 
  verify that their participation proof is included in the global count.
\end{requirements}
We can translate these to the case of participation in a demonstration, then 
each vote would be replaced by a proof of participation.

\Cref{EligibilityVerif} would in this case mean that anyone can verify that 
each participation proof belongs to a unique individual, i.e.\ to prevent Sybil 
attacks.
They must also be able to verify that each proof is indeed related to the 
event.
We must associate the proof to the event both spatially (the correct location) 
and temporally (it is related to the time-interval of the event).
We can essentially divide it into the following requirements:
\begin{requirements}[\ref*{EligibilityVerif}.]
  \item Temporal egligibility:%
    \label{CreatedAfterStart} prove that the data was created after the start of 
    the event;%
    \label{CreatedBeforeEnd} prove that the data was created before the end of 
    the event.
  \item Spatial eligibility:%
    \label{SpatiallyRelated} prove that the data is spatially related to the 
    physical location of the event.
  \item One-proof-per-person:%
    \label{CountOnce} prove that no individual can be counted more than once.
  \item Designated event:%
    \label{DesignatedEvent} prove that the data is designated for the event.
\end{requirements}

\mode<presentation>{%
\begin{frame}
  \begin{requirements}[V\ref*{EligibilityVerif}.]
  \item\label{CreatedAfterStart} Prove that the data was created after the 
    start of the event.
  \item\label{CreatedBeforeEnd} Prove that the data was created before the end 
    of the event.
  \item\label{SpatiallyRelated} Prove that the data is spatially related to the 
    physical location of the event.
  \item\label{CountOnce} Prove that no individual can be counted more than 
    once.
  \item\label{DesignatedEvent} Prove that the data is designated for the event.
\end{requirements}
\end{frame}
}

\subsubsection{Temporal eligibility}

There are two aspects of temporal eligibility.
The adversary can try to forge a witness for a future protest and the adversary 
can try to forge a witness for a past protest.

The first thing we want to do is to prevent Alice from collecting witnesses for 
a future protest.
For this, we introduce something we call a \emph{time-authenticator tag}.
This is an unpredictable value \(\tau_t\) which is published at time \(t\).
The idea is that it is difficult to guess \(\tau_t\) for a given \(t\) in 
advance, i.e.\ any use of \(\tau_t\) must have happened after time \(t\) with 
high probability.

\begin{definition}[Predicting the future]
  Let \(T_t = (\tau_i)_{i=0}^{t}\) be the sequence of time-authenticator tags up 
  until time \(t\), where \(\tau_i\in \Z_{2^\lambda}\) for all \(i\).
  (\(\lambda\) is the security parameter.)
  Let \A be \iac{PPT} adversary.
  The challenger gives \(T_{t-1}\) to \A.
  \A is allowed to query an oracle \(O\) which returns whether a guess is 
  correct or not.
  If \(\A^O(T_{t-1},1^\lambda)\) outputs \(\tau\) such that \(\tau = \tau_t\) 
  the adversary wins.
\end{definition}

We can see that if \(\tau_t\) is chosen uniformly randomly then the adversary's 
chance of winning will be
\begin{equation}
  \label{SuccessOfPrediction}
  \Prob{%
    \A^O(T_{t-1}, \lambda) = \tau_t
  } = \frac{p(\lambda)}{2^\lambda}
\end{equation}
for some polynomial \(p(\cdot)\).
This will be negligible for sufficiently a large \(\lambda\).

\begin{definition}[Collision]
  Given tag \(\tau\), probability of collision?
\end{definition}

\begin{definition}[Changing the past]
  Given a witness for time \(t\), change it to time \(t'\).
  \dots
\end{definition}

\begin{definition}[Forging past witnesses]
  \dots
\end{definition}

\begin{frame}
\begin{definition}[Forging temporal eligibility]
  \dots
\end{definition}
\end{frame}

\subsubsection{Spatial eligibility}

\begin{frame}
\begin{definition}[Forging spatial eligibility]
  \dots
\end{definition}
\end{frame}

\subsubsection{Linkability and designated protest}

\dots

\subsubsection{Individual and universal verifiability}

\dots

\subsection{Privacy}%
\label{Privacy}

We also need privacy in addition to the verification requirements.
As we indirectly pointed out earlier, we focus on the privacy provided to Alice 
and Bob by the data.
So as long as Alice and Bob can conceal their identities at the demonstration 
and escape without arrest, their support is recorded in the data while their 
privacy is not violated.
(Following this line of thinking, it can actually be beneficial for the privacy 
of the demonstrators to mix with the participants of any counter-demonstrations 
--- since the counts will still be correct.)

In voting, we have the following requirements:
\begin{frame}
\begin{requirements}[P]
\item\label{VotePrivacy} Vote privacy: the voting does not reveal any 
  individual vote.
\item\label{ReceiptFreeness} Receipt freeness: the voting system does not 
  provide any data that can be used as a proof of how the voter voted.
\item\label{CoercionResistance} Coercion resistance: a voter cannot cooperate 
  with a coercer to prove the vote was cast in any particular way.
\end{requirements}
\pause{}
\mode<presentation>{%
  \begin{remark}
    P\ref{CoercionResistance} \(\implies\)
    P\ref{ReceiptFreeness} \(\implies\)
    P\ref{VotePrivacy}
  \end{remark}
}
\end{frame}
\Textcite{VerifyingPrivacyPropertiesOfVotingProtocols} showed that 
\cref{CoercionResistance} implies \cref{ReceiptFreeness}, which in turn implies
\cref{VotePrivacy}.
\Cref{CoercionResistance} is probably not possible to achieve for protests:
e.g.\ Eve can simply physically bring Alice to a protest against her will.
This leaves us with \cref{ReceiptFreeness,VotePrivacy}.

\mode<none>{%
\begin{frame}
  \begin{remark}
    \begin{itemize}
      \item Coercion resistance is difficult to achieve for voting.
      \item It doesn't make much sense for protests.
      \item E.g.\ Eve physically brings Alice to a protest she doesn't want to 
        participate in.
      \item That leaves us with receipt freeness and vote privacy.
    \end{itemize}
  \end{remark}
\end{frame}
}

\subsubsection{Participation-proof privacy}

\begin{frame}
\begin{definition}[Proof indistinguishability]
  \dots
\end{definition}
\end{frame}

\subsubsection{Receipt freeness}

\begin{frame}
\begin{definition}[Receipt freeness/deniability]
  \dots
\end{definition}
\end{frame}



\subsection{Contributions}

In this paper, we combine the verifiability and privacy properties of 
electronic voting with location-proof systems to provide a verifiable 
participation count.

%In the case of the Korean demonstrations~\cite{2016DemonstrationsInSeoul}, this 
%was in one place during an entire day and then repeated for several weekends.
%In the case of the Women's Marches in the US~\cite{2017WomensMarchesInUS}, they
%were in several locations at the same time.

\blockcquote{HowToEstimateCrowdSize}{%
  Crowd size is also needed for media news reports and to historically record 
  the event%
}


