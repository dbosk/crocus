\mode*

\section{Introduction}
\label{Introduction}

Alice organizes a street protest against Eve's regime.
Bob, Carol and many others participate in the demonstration.
One problem that has not yet been entirely solved is the crowd-counting 
problem, i.e.\ to verify how many participated in Alice's protest.

\subsection<presentation>{What's the problem?}

\begin{frame}<presentation>
  \begin{block}{The crowd-counting problem}
    \begin{itemize}
      \item Alice organizes a protest against Eve's regime.
      \item Bob, Carol and others show up.
      \item We want to know how many showed up to support Alice.
    \end{itemize}
  \end{block}
\end{frame}

\begin{frame}<presentation>
  \begin{example}
    \begin{itemize}
      \item Computer vision does object recognition.
      \item Requires photos/video that cover the entire location, all the time.
      \item This will count people twice.
    \end{itemize}
  \end{example}

  \pause

  \begin{example}
    \begin{itemize}
      \item Scan active mobile phones in the area.
      \item This requires some extra equipment.
      \item This catches bystanders who are not protesting.
    \end{itemize}
  \end{example}
\end{frame}

\subsection<presentation>{Why is it a problem?}

\begin{frame}<presentation>
  \begin{example}
    \begin{itemize}
      \item South Korea 2016~\cite{2016DemonstrationsInSeoul}:
        approximate count by imagery.
        \begin{description}
          \item[Police] X
          \item[Organizers] Y
        \end{description}

        \pause

      \item US 2017~\cite{2017WomensMarchesInUS}:
        sum up people's approximations.
      \item This makes it even difficult to estimate the error.
    \end{itemize}
  \end{example}
\end{frame}

After many demonstrations the count by police and that by the organizers 
differ, in some instances the difference can be hundreds of thousands.
There are numerous examples, e.g.\ the demonstrations in South 
Korea~\cite{2016DemonstrationsInSeoul}, Trump's 
inauguration~\cite{HowWillWeKnowTrumpInauguralCrowdSize} or the Women's Marches 
in the US~\cite{2017WomensMarchesInUS}, where there is difficulty in 
establishing the actual number of participants.
The methods for counting the crowds vary.
Most of the available methods lack precision, i.e.\ they have large error 
margins.
They can only give an estimate for a particular snapshot in time, e.g.\ at the 
peak of the event, not the cumulative participation count --- at least not 
without counting some people multiple times, which in turn increases the error 
of the estimation.
Finally, they also lack verifiability, i.e.\ everyone must trust the entity who
does the counting.
We will discuss them in more detail in \cref{RelatedWork}.

\begin{frame}<presentation>
  \begin{block}{Verifying protest participation}
    \begin{itemize}
      \item Alice organizes a protest against Eve's regime.
      \item Bob, Carol and others show up.

        \pause

      \item {\color{green} Alice wants to show that many support her cause.}

        \pause

      \item {\color{red} Eve wants to show that few support Alice's cause.}

        \pause

      \item It's an adversarial setting!
      \item We need verifiable results.
    \end{itemize}
  \end{block}
\end{frame}

We can make one important observation about this problem that has seemingly 
been ignored in the past: it is an adversarial setting.
The protesters (organizers and participants) have an incentive to increase the 
tallied number of participants, whereas other interests might have an incentive 
to decrease the tallied number of participants, e.g.\ Eve's authoritarian 
regime in our example above.
Or the other way around, sometimes Eve's authoritarian regime wants to organize 
a pro-regime protest to demonstrate its \enquote{wide 
  support}~\cite[e.g.][]{AlJazeeraOnVenezuela2017,VenezuelanStateWorkersCalledToParticipate}.
(This makes it difficult to rely on government-issued credentials to prevent a 
Sybil attack --- government can issue arbitrarily many such credentials for its 
own protests.)
To solve this problem, we need a verifiable participation count which is 
protected from Sybil attacks.

\subsection{What is \protect\emph{a} protest?}

Protests can vary considerably.
To be able to estimate the participation count for one, we first need to define
what should be counted.

Let us start by considering some examples.
During the demonstrations against the South Korean president in Seoul
\blockcquote{2016DemonstrationsInSeoul}{%
  \textins*{t}he rallies stretch\textins{ed} from midday to late night --- some 
  people stay\textins{ed} for several hours, others just several minutes%
}.
These rallies were all in the same location in the capital and repeated every 
weekend for the duration of a few weeks.
The Women's Marches~\cite{2017WomensMarchesInUS}, on the other hand, occurred 
in parallel in many locations.
We also have the Venezuelan demonstrations where
\blockcquote{2017VenezuelaProtestFrequency}{%
  anti-government demonstrators have staged daily protests across Venezuela%
} while
\blockcquote{AlJazeeraOnVenezuela2017}{%
  pro-government workers sang and danced as they staged a rival march to show 
  their support for the president's controversial plan to rewrite the 
  constitution%
}.
Judging from these examples, the minimal common part is the cause.

\blockcquote{2016DemonstrationsInSeoul}{%
  For the police, the aim is to measure the maximum crowd occupying a certain 
  space at any given time so that they could determine the size of police 
  personnel and resources to deploy%
}

\blockcquote{2016DemonstrationsInSeoul}{%
  Organizers seek to track the entire flow of people from the protest’s start 
  to its finish. They collect estimates given by counters deployed at different 
  locations to size up the crowds in each area and update the numbers 
  throughout the duration of the protest.%
}

%However, to complicate things further,
%\blockcquote{VenezuelanStateWorkersCalledToParticipate}{%
%  \textins*{m}any of Venezuela’s 2.8 million state workers have reported 
%  getting text messages, phone calls and being required to attend political 
%  rallies during work hours
%}.
%
%\blockcquote{BBConVenezuelaProtestBan}{%
%  Venezuela is banning protests that could "disturb or affect" Sunday's 
%  controversial election for a new constituent assembly.
%}
%
%\blockcquote{BBConVenezuelaProtestBan}{%
%  Prison terms of between five and 10 years could be imposed on those 
%  contravening the ban,
%}
%
%\blockcquote{VenezuelanGovtBansProtesting}{%
%  Venezuelans are planning to defy a government ban on public demonstrations 
%  and risk deadly repression with marches across the country to protest against 
%  a vote Sunday that opposition forces say will mark the end of democracy
%}


\subsection{Contributions}

In this paper, we combine the verifiability and privacy properties of 
electronic voting with location-proof systems to provide a verifiable 
participation count.

%In the case of the Korean demonstrations~\cite{2016DemonstrationsInSeoul}, this 
%was in one place during an entire day and then repeated for several weekends.
%In the case of the Women's Marches in the US~\cite{2017WomensMarchesInUS}, they
%were in several locations at the same time.

\blockcquote{HowToEstimateCrowdSize}{%
  Crowd size is also needed for media news reports and to historically record 
  the event%
}


