\mode*

\section{Introduction}
\label{Introduction}

There are different possibilities for a protest: e.g.\ one performed online or 
one performed in the physical reality.
We are interested in real-world protests, specifically to verify a protest.
One problem in real-world demonstrations that has not yet been entirely solved 
is the crowd-counting problem, i.e.\ the verification of the participation in 
a demonstration.

\subsection<presentation>{What's the problem?}

\begin{frame}<presentation>
  \begin{block}{The crowd-counting problem}
    \begin{itemize}
      \item Alice organizes a protest against Eve's regime.
      \item Bob, Carol and others show up.
      \item We want to know how many showed up to support Alice.
    \end{itemize}
  \end{block}
\end{frame}

\begin{frame}<presentation>
  \begin{example}
    \begin{itemize}
      \item Computer Vision to do object recognition.
      \item Requires photos that cover the entire location, all the time.
    \end{itemize}
  \end{example}

  \pause

  \begin{example}
    \begin{itemize}
      \item Scan active mobile phones in the area.
      \item This requires some extra equipment.
      \item This catches bystanders who are not protesting.
    \end{itemize}
  \end{example}
\end{frame}

\begin{frame}<presentation>
  \begin{example}
    \begin{itemize}
      \item South Korea 2016~\cite{2016DemonstrationsInSeoul}:
        approximate count by imagery.

      \item US 2017~\cite{2017WomensMarchesInUS}:
        sum up people's approximations.
    \end{itemize}
  \end{example}
\end{frame}

\subsection<presentation>{Why is it a problem?}

\begin{frame}<presentation>
  \begin{block}{Verifying protest participation}
    \begin{itemize}
      \item Alice organizes a protest against Eve's regime.
      \item Bob, Carol and others show up.

        \pause

      \item {\color{green} Alice wants to show that many support her cause.}

        \pause

      \item {\color{red} Eve wants to show that few support Alice's cause.}

        \pause

      \item It's an adversarial setting!
      \item We need verifiable results.
    \end{itemize}
  \end{block}
\end{frame}

After many demonstrations the count by police and that by the organizers 
differ, in some instances the difference can be hundreds of thousands.
There are numerous examples, e.g.\ the demonstrations in South 
Korea~\cite{2016DemonstrationsInSeoul} or the women's marches in the 
US~\cite{2017WomensMarchesInUS}, where there is difficulty in establishing the 
actual number of participants.
The methods for counting the crowds vary.
There are some techniques using computer vision found in the research 
literature, e.g.\ by \textcite{CVCrowdCounting}.
These require images from the demonstration and provide no way to verify the 
authenticity of the count (except re-running the algorithm on one's own input 
data).
However, as is illustrated by 
\textcite{2016DemonstrationsInSeoul,2017WomensMarchesInUS}, the methods used in 
practice are manual and prone to errors.

We can make one important observation about this problem that has seemingly 
been ignored in the past: it is an adversarial setting.
The protesters (organizers and participants) have an incentive to increase the 
tallied number of participants, whereas other interests might have an incentive 
to decrease the tallied number of participants, e.g.\ an authoritarian regime.
To solve this problem, we need a verifiable participation count.
In this paper, we combine the verifiability and privacy properties of 
electronic voting with location-proof systems to provide a verifiable 
participation count.

